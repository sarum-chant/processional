\chapter{Benedictio salis et aque.}

\marginnote{2r}\lettrine[lines=7,image=true,findent=.5em]{images/2r-initial.png}{}\emph{Mnibus dominicis diebus per annum post primam et capitulum: nisi in duplicibus festis et in dominica ramis palmarum, a sacerdote ebdomadario alba et cum cappa serica induto cum diacono et subdiacono textum deferentibus cum thuribulario et duobus ceroferariis: et acolito crucem deferente omnibus albis cum amictibus indutis et in medio presbyterii ad altare conversis: et ⟨cum⟩\footnote{Barlow 7:1r; 1882:1.} duobus pueris quorum unus\footnote{`\emph{alter}', 1882:1.} scilicet puer\footnote{`\emph{quorum alter scilicet}', Barlow 7:1r.} qui ad aquam scribitur in tabula in sale\footnote{`\emph{salem}', 1882:1.} tenendo et aquam benedictam gestando: alter puer scilicet\footnote{`\emph{scilicet puer}', 1882:2.} ebdomadarius lector ad manus in libro tenendo eidem sacerdoti in superpellicio ministrent ad gradum chori fiat benedictio salis et aqua sic incipiente.\footnote{`\emph{hoc modo}', Barlow 7:1r; `\emph{incipiendo}', 1882:2.} ✠}

\begin{largeinitial}

\grecommentary{1519:2r.\footnote{In Barlow 7:1r ff. the versicles appear a fifth higher, in the C clef.}}

\gabcsnippet{(f3) Ex(hv)or(h)cí(h)zo(h) te(hv) cre(hv)a(h)tú(h)ra(h) sa(hv)lis(h) per(hv) De(hv)um(h) vi(hv)✠vum(h)}

\end{largeinitial}

\begin{lesson}
\noindent per Deum ve✠rum\footnote{In Balrow 7:1r. the three crosses are a later addition.} per Deum sanc✠tum: per Deum qui te per Helíseum prophétam in aquam mitti jussit: ut sanarétur sterílitas aque ut efficiáris sal exorcizátus\footnote{`exorcizátum', Barlow 7:1r; 1882:2.} in salútem credéntium, et sis\footnote{`ut sit', Barlow 7:1r.} ómnibus te suméntibus sánitas ánime et córporis: et effúgiat atque discédat ab eo loco quo aspérsus\footnote{`aspérsum', Barlow 7:1r; 1882:2.} fúeris: omnis fantásia et nequítia vel versútia diabólice fraudis: omnísque spíritus immúndus adjurátus. Per eum qui ventúrus est judicáre vivos et mórtuos et

\begin{noinitial}

\gabcsnippet{(f3) sé(hv)cu(h)lum(h) per(hv) ig(hv)nem.<v>\footnote{SP:2r. has no flat. 1519:2r. has no flat.}</v>(dxd) (::)}

\end{noinitial}

\end{lesson}

\emph{Chorus respondeat sic.}

\begin{noinitial}

\gabcsnippet{(f3) A(gh)men.(hv) (::)}

\emph{Et sic finiatur omnes exorcismi per totum annum.}

\emph{Sequitur oratio sine} Dóminus vobíscum. \emph{sed tantum cum}

\gabcsnippet{(f3) O(hv)ré(h)mus.(f) (::)}

\end{noinitial}

\begin{lesson}
\lettrine{I}{m}ménsam cleméntiam tuam omnípotens etérne Deus humíliter implorámus: ut hanc creatúram salis\footnote{1517: \emph{hic respiciat salem.} ⟨1882:2.⟩} quam in usum géneris humáni\footnote{`humáni géneris', Balrow 7:1r.} tribúisti, bene✠dícere et sanctificáre\footnote{1517---sancti✠ficáre. ⟨1882:2.⟩ In Barlow 7:1r. a second cross has been added here in a later hand.} tua pietáte dignéris: ut sit\footnote{`sis', Barlow 7:1r.} ómnibus suméntibus salus men\marginnote{2v}tis et córporis: ut\footnote{`et', Barlow 7:1r; 1882:3.} quicquid ex eo tactum vel respérsum fúerit cáreat omni immundícia: omníque impugnatióne spirituális \emph{⟨qui finietur hoc modo⟩\footnote{Barlow 7:1r.}}
\end{lesson}

\begin{noinitial}

\grecommentary{1519:2v.\footnote{SP:2v. has no flats. 1519--2v. has no flats.}}

\gabcsnippet{(f3) ne(hv)quí(h)ti(dxd)e.(d) (:)
Per(hv) Dó(h)mi(h)num(hv) nos(hv)trum(h) Je(h)sum(hv) Chris(h)tum(h) Fí(hv)li(h)um(h) tu(h)um(f) (,)
qui(h) te(h)cum(h) vi(h)vit(h) et(h) reg(h)nat(h) in(hv) u(h)ni(h)tá(h)te(h) Spí(h)ri(h)tus(h) Sanc(hv)ti(h) De(h)us(h) (,)
per(h) óm(h)ni(h)a(h) sé(h)cu(h)la(h)
se(h)cu(h)ló(h)rum.(dxd) (::)}

\emph{Chorus respondeat sic}

\gabcsnippet{(f3) A(gh)men.(hv) (::)}

\end{noinitial}

\emph{¶ Sub eodem tono finiantur omnes orationes sequentes ⟨et etiam oratio post aspersionem aque⟩.\footnote{‡~1517, \&c.~add---Sub eodem tono finiantur omnes orationes sequentes et etiam oratio post aspersionem aque. ⟨1882:2.⟩; also Barlow 7:1r.}}

\subsubsection[¶ Sequatur exorcismus aque ita incipiens.]{¶ Sequatur exorcismus aque ita incipiens.\footnote{`\emph{incipiendo}', 1882:2.}}

\begin{lesson}
\lettrine{E}{x}orcízo te creatúra aque in nómine Dei Patris omnipo✠téntis: et in nómine Jesu Christi ✠ fílii ejus Dómini nostri et in virtúte Spíritus ✠ Sancti\footnote{Barlow 7:1r-1v. omits the crosses.}: ut fias aqua exorcizáta ad effugándam omnem potestátem inimíci: et ipsum inimícum eradicáre et explantáre váleas cum ángelis suis apostáticis, per virtútem ejúsdem Dómini nostri Jesu Christi qui ventúrus est judicáre vivos et mórtuos et séculum per ignem. \emph{Chorus.} Amen.
\end{lesson}

\emph{Oratio sine} Dóminus vobíscum. \emph{sed tantum cum} Orémus. \emph{⟨supradicto modo.⟩\footnote{Barlow 7:1v.}}

\begin{lesson}
\subsubsection{Oratio.}

\lettrine{D}{e}us qui ad salútem humáni géneris máxima queque sacraménta in aquárum substántia condidísti: adésto propítius invocatiónibus nostris: et eleménto huic: \emph{Hic respiciat sacerdos aquam.} multímodis purificatiónibus preparáto virtútem tue bene✠dictiónis infúnde: ut creatúra tua mystériis tuis sérviens ad abjiciéndos demónes: morbósque pelléndos divíne grátie sumat efféctum: ut quicquid in dómi\marginnote{3r}bus vel in locis fidélium hec unda respérserit: cáreat omni immundícia liberétur a noxa: non illic resídeat spíritus péstilens: non aura corrúmpens: discédant omnes insídie laténtis inimíci et siquid est quod aut incolumitáti habitántium ínvidet aut quiéti: aspersióne hujus aque effúgiat: ut salúbritas per invocatiónem sancti tui nóminis expetíta: ab ómnibus sit impugnatiónibus\footnote{`impugnatiónibus sit', Barlow 7:1v.} defénsa. Per Dóminum nostrum.\footnote{`\emph{et cetera}', 1882:3. `Jesum Christum Fílium.' Barlow 7:1v.}
\end{lesson}

\emph{¶ Hic mittat sacerdos sal in aquam in modum crucis ita per se privatim dicens.\footnote{`\emph{dicens per se privatim}', 1882:3.}} Commíxtio salis et aque páriter fiat in nómine Patris et Fílii et Spíritus Sancti. Amen.

\subsubsection[¶ Sequitur benedictio salis et aque pariter hoc modo.]{¶ Sequitur benedictio salis et aque pariter hoc modo.\footnote{`\emph{hoc modo incipiens}', 1882:3.}}

\begin{noinitial}

\grecommentary{1519:3r.\footnote{Barlow 7:1v. omits `Dóminus vobíscum \ldots{} Orémus.'}}

\gabcsnippet{(f3) Dó(hv)mi(h)nus(h) vo(hv)bís(h)cum.(f) (::)
<sp>℟.</sp> Et(hv) cum(h) spí(hv)ri(h)tu(h) tu(hv)o.(f) (::Z)
<sp>℣.</sp> O(hv)ré(h)mus.(f) (::)}

\end{noinitial}

\begin{lesson}
\subsubsection[Oratio.]{Oratio.\footnote{1882:3.}}

\lettrine{D}{e}us invícte virtútis auctor et insuperábilis impérii rex ac semper magníficus triumphátor: qui advérse dominatiónis vires reprímis: qui inimíci rugiéntis sevítiam súperas: qui hóstiles nequítia poténs expúgnas, te Dómine treméntes et súpplices deprecámur ac pétimus, ut hanc creatúram \emph{Hic respiciat aquam sale mixtam.} salis et aque dignánter accípias: benígnus illústres pietátis tue more\footnote{`rore: \emph{Missale Herf.}: amore: \emph{Miss. Sar.} 1498, 1513; more: \emph{Miss. Leofr.}, \emph{Rom.} 1474, \emph{Ebor.}, \emph{Sarum} 1492, 1527, \&c.' ⟨1901:19.⟩} sanctí✠fices ut ubicúnque fúerit aspérsa per invocatiónem tui sancti nóminis omnis infestátio immúndi spíritus abjiciátur: terrórque venenósi serpéntis procul pellátur: et preséntia Sancti Spíritus nobis misericórdiam tuam poscéntibus ubíque adésse dignétur. Per Dóminum nostrum Jesum Christum Fílium tuum. \emph{et cetera.\footnote{`qui tecum vivit et regnat', 1882:3. `qui tecum \emph{\&c.}' Barlow 7:1v.}} in unitáte ejúsdem Spíritus Sancti Deus. Per ómnia sécula seculórum. \emph{⟨℟.⟩} Amen.
\end{lesson}

\emph{¶ Si fuerit duplex festum in dominica extra chorum fiat benedictio salis et aque privatim ante aliquod altare: et hora sexta\footnote{1517---hora tertia vel sexta. 1528, \&c.\emph{---}sexta. ⟨1882:3.⟩ `\emph{tertia}', 1882:3.} cantata aspergatur. In aliis vero dominicis in choro benedicatur: et ante tertiam aspergatur: nisi in dominica ra\marginnote{3v}mis palmarum tunc vero sicut in duplicibus diebus alibi benedicetur et sexta aspergatur licet duplex festum fuerit\footnote{Manual--- tunc enim extra chorum benedicatur et post sextam aspergatur, more duplicis festi, licet duplex festum non fuerit. ⟨1882:4.⟩ ⟨Barlow 7:1v.⟩ \emph{sicut in duplicibus festis observetur.} 1882:4.}: et sic\footnote{1528:3r; 1519 has `non si hic.'} peracta benedictione salis et aque accedat ipse sacerdos ad principale altare et ipsum stando circumquaque aspergat. In redeundo in primis\footnote{`\emph{imprimis}', 1882:4.} aspergat ministros ordinatim incipiendo ab accolito qui crucem defert. Deinde ad gradum chori rediens ibidem singulos ab se accedentes clericos aspergat incipiens a majoribus. Tamen si episcopus presens fuerit ad eum pertinet aspersio clericorum tantum in sede sua incipiens a majorbus. Post aspersionem clericorum laicos in presbyterio hinc\footnote{Manual---hinc inde. ⟨1882:4.⟩} stantes aspergat. Et dum aspergatur aqua benedicta, cantetur sequens antiphona a cantor ita incipiente.}

\gregorioscore{001494-asperges-me-1}

\emph{Deinde repetatur antiphona} Aspérges me. \emph{Versus.}

\begin{noinitial}

\gregorioscore{001494-asperges-me-2}

\emph{Iterum repetatur ant.} Aspérges. \emph{Versus.}

\gregorioscore{001494-asperges-me-3}

\marginnote{4r}\emph{Repetatur in hunc modum.}

\gregorioscore{001494-asperges-me-4}

\end{noinitial}

\emph{Hec antiphona ⟨predicta⟩\footnote{1882:4.} dicatur in aspersione aque supradicto modo omnibus dominicis per annum de quocunque fit servitium: et in dominica passionis ⟨Domini⟩\footnote{1882:4.} et dominica in ramis palmarum cum} Glória Patri \emph{et} Sicut erat \emph{preter quod\footnote{`\emph{preterquam}', 1882:4; Barlow 7:2r.} a pascha usque ad festum Sancte Trinitatis.}

\emph{⟨Et in hoc tempore dicatur sequens antiphona.}

\gregorioscore{005403-vidi-aquam-1}

\emph{Deinde repetatur antiphona.}

\begin{noinitial}

\gregorioscore{005403-vidi-aquam-2}

\emph{Deinde repetatur sic.}

\gregorioscore{005403-vidi-aquam-3}

\end{noinitial}

\emph{Hec antiphona predicta dicatur in aspersione aque in die pasche et omnibus dominicis a pascha usque ad festum Trinitatis.⟩\footnote{1530:4r.}}

\emph{Post aspersionem aque\footnote{`\emph{aque aspersionem}', 1882:5.} dicat sacerdos ⟨hunc versum⟩\footnote{1882:5.} ad gradum chori.}

\begin{noinitial}

\gregorioscore{008166-ostende-nobis}

\emph{sine\footnote{`\emph{Non dicatur}', 1882:5.}} Dóminus vobíscum. \emph{sed tantum cum}

\gabcsnippet{(c3) O(h)ré(hv)mus.(f) (::)}

\end{noinitial}

\begin{lesson}
\lettrine{E}{x}áudi nos Dómine sancte Pater omnípotens etérne Deus: et míttere dignáre sanctum ángelum tuum de celis: qui custódiat fóveat prótegat vísitet et deféndat omnes habitántes in hoc
\end{lesson}

\begin{noinitial}

\gabcsnippet{(c3) ha(hv)bi(h)tá(h)cu(d)lo.(d) (:)
Per(h) Chris(h)tum(h) Dó(h)mi(h)num(h) nos(hv)trum.(f) (::)}

\end{noinitial}


\emph{Chorus respondeat sic.}

\begin{noinitial}

\gabcsnippet{(c3) A(gh)men.(hv) (::)}

\end{noinitial}

\emph{Deinde eat processio hoc ordine. In primis\footnote{`\emph{Imprimis}', 1882:5.} procedat ministri virga\footnote{`\emph{virgam}', 1882:5.} manu gestans locum faciens processio.\footnote{`\emph{processioni}', 1882:5. Barlow 7:2r. omits the verger.} Deinde pueri in superpelliceis cum aqua benedicta gestante,\footnote{`\emph{puer in super pelliceo aquam benedictam gestans}', 1882:5.} deinde accolitus crucem ferens: et post ipsum duo ceroferarii pariter incedentes deinde thuribularius: \marginnote{4v}post eum subdiaconus, deinde diaconus omnes albis cum amictibus induti, et absque tunicis vel casulis et post diaconum: eat sacerdos in simili habitu cum cappa serica. Deinde sequantur pueri et clerici de secunda forma habitu non mutato non bini et bini: sed ex duabus partibus juxta ordinem quo disponuntur in choro. Et clerici\footnote{`\emph{reliqui clerici}', note, 1901:21.} de superiori gradu eodem ordine quo disponuntur in capitulo\footnote{`\emph{capite}', 1528:4r; 1544:5r; 1554:4v.} habitu non mutato: sed excellentioribus sequentur per ordinem quod in omnibus dominicis non duplicibus observetur per totum annum.}

\emph{Et exeat processio per hostium presbyterii septentrionale, circuiens presbyterium. Episcopus si presens fuerit mitram gerat et baculum in fine processionis. Sacerdos vero sive presens fuerit episcopus sive non in anteriori parte post suos ministros procedat et in eundo singula altaria aspergat.}

\newrubric
\emph{¶ In duplicibus tamen festis que in dominica contingunt in procedendo altaria non aspergat: sed eat post diaconum in habitu antedicto. Deinde ab australi parte ipsius ecclesie per fontes venientes procedant ad crucem, sacerdote cum suis ministris predictis ⟨in⟩\footnote{1882:6.} medio suo ordine stante. Ita quod puer deferens aquam et accolitus stent juxta\footnote{`\emph{ad}', 1882:6.} gradum ante crucem. Omnes autem clerici interesse possunt processioni totius anni: licet nulle hore diei precedentis interfuerint.\footnote{`\emph{Quilibet autem processionem totius anni intrare possunt, licet nulli hore diei precedenti interfuerint.}' Consuetudinary (Cap. xiv. ad fin.) ⟨1882:xv.⟩ Henderson declares that the text of the Consuetudinary is correct here.}}

\printendnotes

\chapter{⟨¶ Dominica prima adventus.⟩}

\emph{¶ Omnibus dominicis diebus per adventum ad processionem in eundo dicatur ista antiphonam hoc modo.}

\label{003792-missus-est}
\gregorioscore{003792-missus-est}

% TODO: Missing text from 5v
\marginnote{5r}\emph{¶ Quando vero pervenerit processio ante magnam crucem in ecclesia nisi fieri debeat statio non dicatur antiphona de cruce usque ad inceptionem\footnote{Barlow 7:2r--2v breaks off the rubric at this point.} historie} Deus ómnium. \emph{Sed statim post antiphonam vertat se sacerdos ad populum et dicat in lingua materna sic.\footnote{Here Barlow 7:2v includes a later marginal notation that appears to say `X. Ecclesia Salisburiensis orationes? et officium.' Then follows a form of the bidding of the bedes in English. See Appendix.}}

Orémus pro ecclésia Romána et pro papa et archiepíscopis, et epíscopis, et speciáliter pro epíscopo nostro \emph{N.} et pro decáno nostro \emph{vel} pro rectóre hujus ecclésie \emph{scilicet in ecclesiis parochialibus}: et pro terra sancta, pro pace ecclésie et terre et ⟨pro⟩\footnote{1882:6.} rege\footnote{1528:5r. omits `\emph{et rege}.'} et regína et eórum\footnote{`suis', Dickinson:37**.} líberis.\footnote{`Let us pray for the church of Rome, and ⟨for⟩ the pope and ⟨for⟩ the archbishops, and ⟨for⟩ the bishops, and especially for our bishop \emph{N.} and for the dean \emph{or} for the rector of this church, \emph{of course in parish churches}: and for the holy land, and for the peace of the church and ⟨these⟩ lands and ⟨for⟩ the king and queen and their children.' (trans. ed.)

  The form found in 1544:6r. is `Orémus pro ecclésia Anglicána et pro rege nostro et archiepíscopis, epíscopis et speciáliter pro epíscopo nostro \emph{N.} et pro decáno \emph{vel} rectóre hujus ecclésie, \emph{scilicet in ecclesiis parochialibus} et pro terra sancta pro pace ecclésie et terre et regína et suis líberis.' `Let us pray for the church of England, and ⟨for⟩ our kinge and ⟨for⟩ the archbishops, ⟨and for⟩ the bishops, and especially for our bishop \emph{N.} and for the dean \emph{or} rector of this church, \emph{of course in parish churches}: and for the holy land, ⟨and⟩ for the peace of the church and ⟨these⟩ lands and ⟨for⟩ the queen and their children.' trans. ed.

  The rubric `\emph{et cetera more solito}' suggests that particular prayers would be added here according to local circumstances, as shown in the extract from Salisbury Cathedral MS 152:12r. given below. A simple order derived from MS 152 would be: `and for \emph{N.} and \emph{N.} (benefactors), and for our friends, for our brethren and sistren, and for all our parishioners, with all those that doth any good to this church, and for all true Christian people.'} \emph{et cetera more solito.\footnote{MS 152:12r. indicates here an additional `Pater noster' before the psalm.}}

\emph{Deinde vertat se sacerdos et dicatur\footnote{`\emph{dicatur et}', 1519:5v.} iste psalmus} Deus misereátur nostri. Brev:{[}51{]}. \emph{ex utraque parte chori more solito sine nota: ex parte chori principali incipiatur. Finito psalmo cum} Glória Patri. \emph{et} Sicut ⟨erat⟩.\footnote{1882:6.} \emph{sequatur} Kyrieléyson. Christeléyson. Kyrieléyson. Pater noster. Brev:{[}5{]}. \emph{Deinde dicat sacerdos in audientia: sed sine nota}

\emph{⟨℣.⟩} Et ne nos ⟨indúcas in tentatiónem.

\emph{℟.⟩} Sed libera ⟨nos a malo.

\emph{℣.⟩} Osténde nobis Dómine ⟨misericórdiam tuam⟩.\footnote{Barlow 7:3r; 1882:7.}

\emph{⟨℟.⟩} Et salutáre ⟨tuum da nobis⟩.\footnote{1882:7.}

\emph{⟨℣.⟩} Sacerdótes tui induántur justítiam.

\emph{⟨℟.⟩} Et sancti tui exúltent.

\emph{⟨℣.⟩} Dómine salvum fac regem.

\emph{⟨℟.⟩} Et exáudi nos ⟨in die qua invocavérimus te⟩.\footnote{1882:7.}

\emph{⟨℣.⟩} Salvum fac servum ⟨tuum⟩.\footnote{1882:7. `Salvum fac servos tuos et ancíllas tuas.' Barlow 7:3r.}

\emph{⟨℟.⟩} Deus meus sperántem in te.

\emph{⟨℣.⟩} Salvum fac pópulum tuum ⟨Dómine⟩.\footnote{⟨Dómine et bénedic hereditáti tue.⟩, Barlow 7:3r.}

\emph{⟨℟.⟩} Et rege eos ⟨et extólle illos usque in etérnum⟩.\footnote{1882:7.}

\emph{⟨℣.⟩} Dómine fiat pax ⟨in virtúte tua⟩.\footnote{1882:7; Barlow 7:3r.}

\emph{⟨℟.⟩} Et abundántia ⟨in túrribus tuis⟩.\footnote{1882:7.}

\emph{⟨℣.⟩} Dómine éxáudi oratiónem ⟨meam⟩.\footnote{1882:7; Barlow 7:3r.}

\emph{⟨℟.⟩} Et clamor ⟨meus ad te véniet⟩.\footnote{1882:7.}

\emph{⟨℣.⟩} Dóminus vobíscum. \emph{⟨℟.⟩} Et cum spíritu tuo.

\emph{⟨℣.⟩} Orémus.

\begin{lesson}
\lettrine{D}{e}us qui charitátis dona per grátiam Sancti Spíritus tuórum córdibus fidélium infúndis: da fámulis et famulábus tuis: pro quibus tuam deprecámur cleméntiam salútem mentis et córporis: ut te tota virtúte díligant: et que tibi plácita sunt tota dilectióne perfíciant: et pacem tuam nostris concéde tempóribus. Per Christum Dominum nostrum. \emph{⟨℟.⟩} Amen.
\end{lesson}

\newrubric
\emph{¶ Item\footnote{`\emph{Deinde}', 1882:7.} conversus ad populum dicat sacerdos in lingua materna,} Orémus pro animábus \emph{N.} et \emph{N. more solito\footnote{The rubric `\emph{more solito}' again suggests that particular prayers would be added here according to local circumstances, as shown in the extract from Salisbury Cathedral MS 152:12r. given below. A simple order derived from MS 152 would be: `For the souls of all bishops whose bodies rest in this holy place, especially \emph{N.} and \emph{N.} which bishops have in their time honoured this church with precious vestements and many other jewels. And for the souls of all the patrons of this church, and all other lords that have honoured it with their bodies, rents, or any other jewels, especially for \emph{N.} and \emph{N.} And for the souls of all other servants of this church which have truly serve it or done any good therero in their days, especially for \emph{N.} and \emph{N.} And for all the souls whose bones rest in this church and churchyard. And for the souls of all those that have given to this church rents, vestments, or any other goods, whereby God is more worshipped in this church, and the ministers thereof better sustained. And for the souls of all our friends, our brethren and our sistren, and for the souls of all our parishioners. And for the souls of all that have done any good to this church. And for all Christian souls.'}\footnote{Ms 152:15r. again inserts and additional `Pater noster' here.}: et postea vertat se sacerdos et dicat psalmum\footnote{`\emph{dicatur iste psalmus}', 1882:7.}} De profúndis clamávi. Brev:{[}340{]}. \emph{supradicto modo sine} Glória Patri. \emph{⟨et⟩\footnote{1882:7.} cum} Kyrieléyson. Christeléyson. Kyrieléyson. Pater noster. Brev:{[}5{]}.

\emph{⟨℣.⟩} Et ne nos indúcas in temptatiónem. \emph{⟨℟.⟩} Sed líbera nos a malo.

\emph{⟨℣.⟩} Réquiem etérnam dona eis Dómine. \emph{⟨℟.⟩} Et lux perpétua lucéat eis.

\emph{⟨℣.⟩} A porta ínferi. \emph{⟨℟.⟩} Erue Dómine ⟨ánimas eórum⟩.\footnote{1882:7.}

\emph{⟨℣.⟩} Credo vidére ⟨bona Dómini⟩.\footnote{1882:7.} \emph{⟨℟.⟩} In terra vivéntium.

\emph{⟨℣.⟩} Dóminus vobíscum. \emph{⟨℟.⟩} Et cum spíritu tuo.

\emph{⟨℣.⟩} Orémus.

\begin{lesson}
\label{absolve-quesumus-domine}
\lettrine{A}{b}sólve quésumus Dómine ánimas famulórum tuórum pontíficum et sacerdótum: et ánimas famulórum familarúmque tuárum paréntum parrochianórum amicórum et\footnote{Barlow 7:3v. omits `et ánimas \ldots{} amicorum et.'} benefactórum nostrórum et ánimas\footnote{Barlow 7:3v. omits `ánimas.'} ómnium fidélium defunctórum ab omni vínculo delictórum: ut in resurrectiónis glória inter sanctos eléctos\footnote{`el eléctos', 1519:6r.} resuscitári respírent. Per Christum ⟨Dóminum nostrum⟩.\footnote{Barlow 7:3v.}

\end{lesson}

\emph{⟨℣.⟩} Requiéscant in pace. \emph{⟨℟.⟩} Amen.

\newrubric
% TODO: he/hee?
\emph{He preces predicta dicuntur supradicto modo omnibus dominicis per annum: sive de dominica: sive ⟨de⟩\footnote{1882:8.} aliquo festo fit servitium: nisi duplex festum fuerit: et nisi in sexta die a nativitate Domini et in die sancti Silvestri si in dominica evenerit et nisi in dominica ⟨in⟩\footnote{1882:8.} ramispalmarum.}

\emph{¶ Ita tamen quod in ecclesiis parochialibus non ad processionem sed post evangelium et offertorium supradicto modo dicuntur ante aliquod altare in ecclesia vel in pulpito ad hoc constituto: tunc psalmo} De profúndis. Brev:{[}340{]}. \emph{cum versiculi et oratione} Absólve quésumus Dómine. \pageref{absolve-quesumus-domine}. \emph{semper dicatur in statione ante crucem in ecclesie supradicto modo nisi in duplicibus festis et in vj. die a nativitate Domini et in die sancti Silvestri quando in dominica evenerit, et nisi in dominica palmarum ut supra diximus.}

\emph{Finitis precibus intrent chorum cantore incipiente.\footnote{`\emph{In introitu chori vel citius cantor incipiat responsorium.}' 1882:8. In Cam-Queens-MS-28 (Gradual):10.: `I\emph{n stacione ante crucem nulla dicatur antiphona usque ad incepcionem} Deus ómnium. \emph{Finiatur antiphona statim incipiatur ℟. in introitu chori hoc modo.}' Beginning at \emph{Deus ómnium} the chant for the entry into the chancel is an \emph{antiphon} to the Virgin.}}

\gregorioscore{007068-letentur-celi-1}

\emph{Et ⟨dicatur⟩\footnote{1882:8.} cum versu vel sine versu per dispositione cantoris.}

\begin{noinitial}

\gregorioscore{007068-letentur-celi-2}

\end{noinitial}

\marginnote{6v}\emph{Sed sine} Glória Patri. \emph{Quod per totum annum observetur quando dicitur responsorium in introitu chori, et quando dicatur ℣. cantetur a toto choro.}

\emph{Deinde dicat sacerdos ad gradum chori versiculum sequentem.}

\begin{noinitial}

\gregorioscore{008246-vox-clamantis}

\emph{Non dicatur} Dóminus vobíscum \emph{post processionem, similiter fiat per totum annum sed tantum cum.}

\gabcsnippet{(c3) O(hv)ré(h)mus.(f) (::)}

\begin{lesson}
\lettrine{E}{x}cita quésumus\footnote{1882:8. omits `quésumus.'} Dómine poténtiam tuam et veni: ut ab imminéntibus peccatórum nostrórum perículis: te mereámur protegénte éripi: te liberánte salvári. Qui vivis et regnas cum Deo Patre in unitáte Spíritus Sancti Deus. Per ómnia
\end{lesson}

\gabcsnippet{(c3) sé(hv)cu(h)la(h) sé(h)cu(h)ló(h)rum.(d) (::)
<sp>℟.</sp> A(gh)men.(hv) (::)}

\end{noinitial}

\emph{Deinde eat sacerdos cum suis ministris in cimiterium canonicorum, aspergendo et orando pro defunctis cum isto psalmo} De profúndis. Brev:{[}340{]}. \emph{et cum oratione.}

\begin{lesson}
\lettrine{D}{e}us cujus miseratióne ánime fidélium ⟨defunctórum⟩\footnote{1882:8.} requiéscunt animábus famulórum famularúmque tuárum: hic et ubíque in Christo quiescéntium: da propícius suórum véniam peccatórum: ut a cunctis reátibus absolúte: tecum sine fine leténtur. Per eúndem Christum Dóminum nostrum. \emph{⟨℟.⟩} Amen.

\end{lesson}

\emph{⟨℣.⟩} Requiéscant in pace. \emph{⟨℟.⟩} Amen.

\newrubric
\emph{¶ Iste modus et ordo processionis servetur generaliter omnibus dominicis non duplicibus per ⟨totum⟩\footnote{1882:9.} annum cum suis antiphonis vel responsoriis, cum oratione et ℣.\footnote{`\emph{et versibus, et orationibus ipsis dominicis diversimode intitulatis.}' 1882:9.}}

\emph{¶ In dominicis ⟨tamen⟩\footnote{1882:9.} a lxx. usque ad lx. dicatur ℣. post antiphonam in ipsa statione ad gradum ante crucem a duobus clericis de ij. forma ad populum conversis habitu non mutato.}

\emph{¶ Similiter a dominica ab octavis pasche usque ad proximam dominicam ante ascensionem Domini dicatur versus a ij. clericis de ij. forma ad populum conversus in superpelliceis coram cruce.\footnote{`\emph{ad populum conversis, ut supra.}' 1882:9.}}

\emph{¶ In proxima dominica ante ascensionem Domini dicatur ℣. a iij. de superiori gradu in pulpito conversis ad populum. Et si episcopus in hac dominica vel in quacunque alia officium \marginnote{7r}exequatur: ipse indutus in cappa serica cum mitra et baculo ac omnibus supradictis ministris ad benedictionem aque chorum solet intrare: qui dum fit benedictio salis et aque: sacerdos ut predicitur ad hoc induto in sedem recipiat episcopalem: ibique post aspersionem principali altaris a sacerdote factam tam canonicos quam ceteros clericos: ad sedem episcopalem accedente: modo et ordine prenotato aspergat: et tam versiculum quam orationem post antiphonam} Aspérges me. \emph{et post reversionem processionis in chorum dicat ubi debet: ipse similiter in dominicis per estatem si officium secutus sit misse, vel cum oratione dicatur in statione ante crucem. Sacerdos cum predictis ad processionem vestitus: preces cum orationibus sequentibus easdem dicet more solito: et post reversionem in chorum dum hore tertia et sexta cantate fuerint: pontifex induet se pro missa. Si vero episcopus executor officii non fuerit tunc in habitu clericali cum cirotecis tamen et baculo, clericos ut supra aspergat: sed in processione cum cappa serica cum mitra et baculo semper in fine procedat: et predictus sacerdos ebdomadarius super omnia exequatur que ad processionem pertinent loco et habitu consueto.}

\printendnotes

\chapter{¶ Dominica secunda adventus.}

\emph{Ad processionem ant.} Missus est ⟨ángelus⟩.\footnote{1882:9.} \emph{ut supra ⟨in prima dominica adventus Domini⟩.\footnote{1882:9.}} \pageref{003792-missus-est}.

\emph{In introitu chori dicatur istud responsorium.}

\gregorioscore{007547-rex-noster}

\emph{℣.} Vox clamántis in desérto.

\emph{℟.} Paráte viam Dómini rectas fácite sémitas Dei nostri.

\emph{⟨℣.⟩} Orémus.

\begin{lesson}
\subsubsection{Oratio.}

\lettrine{E}{x}cíta quésumus Dómine corda nostra ad preparándas Unigéniti tui vias: ut per ejus advéntum purificátis tibi méntibus servíre mereámur. Qui vivis et regnas.\footnote{`Qui tecum.' 1882:10.}
\end{lesson}

\printendnotes

\chapter[¶ Dominica tertia ⟨adventus⟩.]{¶ Dominica tertia ⟨adventus⟩.\footnote{1882:10.}}

\emph{Ad processionem ant.} Missus est ⟨ángelus⟩.\footnote{1882:10.} \emph{ut supra.} \pageref{003792-missus-est}.

\emph{In introitu chori\footnote{`\emph{In redeundo}', 1882:10.} dicatur istud responsorium.}

\gregorioscore{006606-ecce-radix}

\emph{℣.} Vox clámantis in desérto.

\emph{℟.} Paráte viam Dómini rectas fácite sémitas Dei nostri.

\emph{⟨℣.⟩} Orémus.

\begin{lesson}
\subsubsection{Oratio.}

\lettrine{A}{u}rem tuam quésumus Dómini précibus nostris accómmoda: et mentis nostre ténebras grátia tue visitatiónis illústra. Qui vivis. \emph{\&c.}
\end{lesson}

\printendnotes

\chapter[¶ Dominica quarta ⟨adventus⟩.]{¶ Dominica quarta ⟨adventus⟩.\footnote{1882:10.}}

\emph{Ad processionem ant.} Missus est. \emph{ut supra.} \pageref{003792-missus-est}.

\emph{In introitu chori\footnote{`\emph{In redeundo}', 1882:10.} responsorium.}

\label{007177-montes-israel}
\gregorioscore{007177-montes-israel}

\label{vox-clamantis}
\emph{℣.} Vox clamántis in desérto.

\emph{℟.} Paráte viam Dómini rectas fácite sémitas Dei nostri.

\emph{⟨℣.⟩} Orémus.

\begin{lesson}
\subsubsection{Oratio.}

\label{excita-quesumus-potentiam}
\lettrine{E}{x}cita quésumus Dómine poténtiam tuam et veni: et magna nobis virtúte succúrre: ut per\footnote{`per' is missing in 1519:8r. It appears in 1530:8v. and 1555:unpaged.} auxílium grátie tue quod nostra peccáta prepédiunt\footnote{`impédiunt', 1882:10. 1519:8r. has `prepediuut.'}: indulgéntia tue propitiatiónis accéleret. Qui vivis et regnas.
\end{lesson}

\printendnotes

\chapter{In vigilia nativitatis Domini.}

\emph{Si dominica fuerit ad processionem ant.} Missus est. \emph{⟨et cetera⟩\footnote{1882:11.} ut supra ⟨in dominica precedente⟩.\footnote{1882:11.}} \pageref{003792-missus-est}.

\emph{In introitu chori ℟.} Montes Israel. \pageref{007177-montes-israel}.

\emph{℣.} Vox clamántis. \pageref{vox-clamantis}.

\emph{Oratio.} Excita quésumus Dómine poténtiam tuam et veni: et magna. \pageref{excita-quesumus-potentiam}.

\printendnotes
