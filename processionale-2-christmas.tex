\chapter{¶ In die nativitatis.}

\emph{Quacunque die\footnote{`feria', 1882:11.} contigerit: dum hora prima ante missam canitur: sex pueri ad ministrandum vestiti cappis sericis, in chorum deferant quibus ceteri clerici ad processionem et ad missam donec cantatur} Agnus Dei. \emph{et} Pax Dómini. \emph{per totum chorum data fuerit induantur preter sacerdotes et ministri. Quod totiens fiat: quotiens in festis duplicibus dominicis videlicet: vel aliis festis quando fit processio causa festivitatis.}

\emph{¶ Dicta hora sexta\footnote{`\emph{tertia hora}', 1882:11.} processio per medium chori exeat per hostium occidentale circumeundo chorum ut in omnibus aliis festis duplicibus per annum quo ingredietur ecclesiam\footnote{`\emph{quando non egredietur ecclesiam}', 1882:11.}: et sic eat processio circa claustrum hoc ordine. Precedat minister virgam manu gestans\footnote{`\emph{Imprimis sacriste virgas in manibus gestantes}', 1882:11.} locum faciens processioni: deinde ⟨puer cum⟩\footnote{1882:11.} aqua benedicta: deinde tres cruces a tribus accolitis deferentibus: albis et tunicis\footnote{`\emph{albis cum amictibus indutis}', 1882:11.}: deinde ceroferarius ij. albis cum amictibus induti tantum: deinde duo thuribularii in simili habitu: deinde subdiaconus: tunc diaconus dalmatica et tunica indutus textus singulos deferat.\footnote{`\emph{textum uterque deferens}', 1882:11.} Post diaconum eat sacerdos in alba ⟨cum amictu⟩\footnote{1882:11.} et cum cappa serica: chorus itaque sequatur in cappis sericis. In primis\footnote{`\emph{Imprimis}', 1882:11.} pueri: deinde clerici de secunda forma: et clerici de superiori gradu juxta predictam ordinem ⟨videlicet in prima dominica adventus Domini prenotatum⟩\footnote{1882:11.} videlicet excellentioribus ⟨personis⟩\footnote{1882:11.} subsequentibus. Quod in omnibus duplicibus festis observetur in quibus fit processio. Ita tamen quod in festis minoribus duplicibus non habentur nisi due cruces tantum. Preterea in die ascensionis Domini: et in festo Corporis Christi precedentibus vexillis per ordinem et capsula reliquiarum qui a duobus clericis de secunda forma in cappis sericis deferantur inter subdiaconum et t⟨h⟩uribularium. ⟨ut patet in statione sequenti⟩.\footnote{1528:8r.}}

\begin{figure}
\centering
\includegraphics{images/9r-ordo-processionis-nativitas.jpg}
\caption{¶ Ordo processionis in die nativitas Domini ante missam.}
\end{figure}

\emph{¶ In eundo cantor incipiat.}

\gregorioscore{006411-descendit-de-celis}

\emph{Tres\footnote{`MS. Bodl. --- \emph{Tres clerici de superiori gradu in capis sericis dicant prosam; ita tamen dum dicti clerici santant versum, stent gradibus fixis unsa cum choro, et dum chorus prosequitur primum versum, procedant cum toto choro, quod etiam observetur per totum annum, quando versus in processione habentur.}' ⟨1882:13.⟩} clerici de superiori gradu in medio processioni in cappis sericis simul cantent in eundo prosam sequentem in hunc modum.}
% TODO: processionis?

\gregorioscore{006411Pzf-felix-maria}

\emph{Chorus respondeat.}

\begin{noinitial}

\gregorioscore{006411a-tanquam-sponsus}

\end{noinitial}

\clearpage

\emph{Clerici cantent hanc prosam sequentem in hunc modum.}

\gregorioscore{006411Pb-familiam-custodi}

\emph{Chorus ⟨cantat versum ut sequitur⟩}.\footnote{1528:9v; 1519:10r has `iui', not `tui.'}

\begin{noinitial}

\gregorioscore{006411-gloria-patri}

\end{noinitial}

\emph{Clerici aliam prosam dicatur.}

\gregorioscore{006411Pzi-te-laudant-alme-1}

\emph{Que scilicet prosa in ipsa statione ante crucem ab ipsis finiabitur: et post unamquamque ℣. respondeat chorus cantum prose super iij. vocales,} 𝔄. 𝔒. 𝔈. \emph{quod in omnibus prosis observetur.\footnote{`\emph{respondeat cantum prose more solito,} 𝔄.' 1882:13.}}

\begin{noinitial}

\gregorioscore{006411Pzi-te-laudant-alme-2}

\emph{Chorus respondeat sic ⟨dicens⟩\footnote{1882:13.}}

\gregorioscore{006411Pzi-te-laudant-alme-3}

\end{noinitial}

\emph{In introitu chori dicatur hec ⟨sequens⟩\footnote{1882:13.} antiphona cantore incipiente.}

\gregorioscore{003093-hodie-christus}

\emph{Si hec antiphona non sufficiat ad introitum chori: tunc repetitur in predicta antiphona} †~Hódie in terra canunt ángeli.

\emph{℣.} Benedíctus qui venit ⟨in nómine Dómini⟩.\footnote{1882:14.}

\emph{℟.} Deus Dóminus illúxit nobis.

\begin{lesson}
\subsubsection{Oratio.}

\lettrine{C}{o}ncéde quésumus omnípotens Deus: ut nos Unigéniti tui nova per carnem natívitas líberet: quos sub peccáti jugo vetústa sérvitus tenet. Per eúndem Christum Dóminum nostrum.
\end{lesson}

\emph{Modus et ordo processionis hujus diei locum habent in omni duplici festo per totum annum quod ex sollennitate sua processionem habet excepto quod in aliis festis non dicatur prosa excepta purificatione si episcopus presens fuerit et exequatur officium in processione omnes diaconi et subdiaconi processionem in simili habitu incedant. Sciendum est quod in omnibus majoribus festis duplicibus, tres accoliti in processione ante crucem ad tres cruces deferendas tunicis induantur in quibus ad missam subsequentem ministrent. Principalis accolitus est ille videlicet in tabula dominicalis notatus vel per ebdomadam suum exequatur officium mediam crucem defert: secundus ex altera parte chori principalis: tertius ex ea parte iiij.\footnote{`\emph{Principalis accolitus, ille videlicet in tabula dominicali notatus per ebdomadam suum exsequitur officium, mediam crucem defert; secundus ex altera parte chori principalis; tertius ex ea parte qua. primus crucem bajulat ex altera parte chori.}' ⟨1882:xv.⟩ This is Henderson's proposed suggestion.} primus crucem bajulat ex altera parte chori.}

\emph{¶ In die nativitatis Domini post vesperas finito primo} Benedicámus. \emph{a duobus de ij. forma in superpelliceis conveniant omnes diaconi in cappis sericis portantes cereos ardentes in manibus: et sic eat processio per medium chori ad altare sancti Stephani cantando hoc ℟. cantore\footnote{`\emph{diacono}', 1882:14.} incipiente hoc modo.}

\label{sancte-dei-preciose}
1519:11r; AS:60; Ant-1519:61v; Brev-1531:31v.\footnote{1519:11r has no flats. In 1519:11r `Dóminum' is set DFFG.DDCD.D; inimíco is set FGA.GF.GA.CAA; `funde' is set ABC.GA; `nos purgátos' is set GA G.B♭.AB♭C.AGF.}

\gregorioscore{missing}

\emph{Tres diaconi dicant simul hunc ℣.}

\gregorioscore{missing}

\emph{¶ Omnes\footnote{1508---tres. ⟨1882:14.⟩} diaconi dicant simul hanc prosam.}

\label{te-mundi}
1519:11v; AS:60; Ant-1519:62r; Brev-1531:31v.\footnote{In 1519:11v a flat appears only in first phrase. 1519:12r has `sequens' for sequéris'. In 1519:12r there appears to be a missing C-clef at `crímina', making the music a third higher. In 1519:12r `vénia' is set G.E.C.}

\gregorioscore{missing}

\emph{Chorus vel organa respondeat cantum prose super litteram} 𝔄. \emph{post unumquemque versum.}

\gregorioscore{missing}

\emph{Ad hanc processionem ⟨non⟩\footnote{Most editions omit `\emph{non}', but compare with the \emph{Noted breviary}, p.~334. The 1555 \emph{Processional} (unpaged) includes `\emph{non}.' `\emph{non dicitur}', 1882:15.} dicatur} Glória Patri. \emph{sed dum prosa canitur: thurificet sacerdos altare: deinde imaginem sancti Stephani: et postea dicat modesta voce ⟨versiculum⟩.\footnote{1882:15.}}

1519:12r.\footnote{1519:12r omits the response.}

\gregorioscore{missing}

\begin{lesson}
\subsubsection{Oratio.}

\lettrine{D}{a} nobis quésumus Dómine imitári quod cólimus ut discámus et inimícos dilígere: quia ejus natalícia celebrámus: qui novit étiam pro persecutóribus exoráre Dóminum nostrum Jesum Christum Fílium. \emph{⟨et cetera.⟩\footnote{1882:15.} Chorus.} Amen.
\end{lesson}

\emph{In redeundo dicatur aliqua antiphona de sancta Maria, vel responsorium.}

\gregorioscore{007709-stirps-jesse}

\emph{Non dicitur} Glória Patri. \emph{Sed sacerdos ad gradum chori dicat}

\emph{Versiculum.} Speciósus forma pre fíliis hóminum.

\emph{℟.} Diffúsa est grátia in lábiis tuis. \emph{Non dicatur ulterius.}

\emph{⟨℣.⟩} Orémus.

\begin{lesson}
\subsubsection{Oratio.}

\lettrine{D}{e}us qui salútis etérne beáte Maríe virginitáte fecúnda humáno géneri prémia prestitísti: tríbue quésumus ut ipsam pro nobis intercédere sentiámus per quam merúimus auctórem vite suscípere, Dóminum nostrum Jesum Christum Fílium tuum. Qui tecum vivit.
\end{lesson}


\chapter{¶ In die sancti Stephani.}

\emph{Si dominica fuerit eodem modo fiat processio que in ceteris diebus dominicis preter habitum et excepto quod hac die tres diaconi prosam in eundo cantent in medio procedentes que in ipsa statione ante crucem ab eisdem terminetur: cetera ut supra.}

\emph{℟.} Sancte Dei Precióse. \pageref{sancte-dei-preciose}. \emph{℣.} Ut tuo. \emph{et percantetur a toto choro: tres diaconi dicant prosam} Te mundi. \pageref{te-mundi}. \emph{Chorus} 𝔄. \emph{et sic deinceps a toto choro.} Glória Patri. \emph{et dicatur hoc modo.}

1519:13r.\footnote{1519:13r has no flat.}

\gregorioscore{missing}

\emph{In redeundo ant.} Hódie Christus natus. 23.

\emph{℣.} Benedíctus qui venit. 23.

\emph{Oratio.} Concéde quésumus omnípotens Deus. 23.

\emph{Simili modo de sancto Johanne et de Innocentibus et de sancto Thoma si in dominica festum illorum evenerit cum ix. ℟. et ℣. et prosa, et} Glória Patri. \emph{et semper in redeundo usque ad circumcisionem dicatur antiphona} Hódie Christus. \emph{cum ℣. et oratione ut supra.} 23.

\emph{¶ In die sancti Stephani ad ij. vesperas post memoriam de nativitate conveniant omnes sacerdotes in cappis sericis cum cereis ardentibus in manibus: et sic eat processio ad altare apostolorum per medium chori cantando responsorium.}

\gregorioscore{006913-in-medio-ecclesie}

\emph{⟨Tres diaconi dicant versum,} Misit. \emph{\&c.⟩\footnote{MS.~Harl.~2945---Tres diaconi dicant versum, Misit, \&c.~⟨1882:16.⟩}}

\begin{noinitial}

\gregorioscore{006913a-misit-dominus}

\end{noinitial}

\emph{Omnes\footnote{1508---Tres. MSS. and other Edd.---Omnes. ⟨1882:16.⟩} sacerdotes simul dicant prosam.}

\gregorioscore{006913Pzd-nascitur-ex-patre}

\emph{Ad hanc processionem non dicatur} Glória Patri. \emph{sed dum prosa canitur thurificet sacerdos ad altare deinde imaginem sancti Johannis: et postea dicat modesta voce versiculum.} Valde honorándus est beátus Johánnes.

\emph{℟.} Qui supra pectus Dómini in cena recúbuit.

\begin{lesson}
\subsubsection{Oratio.}

\lettrine{E}{c}clésiam tuam quésumus Dómine benígnus illústra: ut beáti Johánnis apóstoli tui et evangelíste illumináta doctrinis: ad dona pervéniat sempitérna. Per Christum Dóminum nostrum. \emph{⟨℟.⟩} Amen.
\end{lesson}

\emph{In redeundo dicitur aliqua antiphona de sancta Maria vel ℟.} Solem justície. \emph{quere in nativitate ejusdem} 369. \emph{: versus ⟨et⟩\footnote{1882:17.} oratio ut supra.} 23.


\chapter{¶ In die sancti Johannis apostoli.}

\emph{Si dominica fuerit ad processionem eodem modo fiat ut in die sancti Stephani: excepto quod hac ⟨die⟩\footnote{1882:17.} iij. sacerdotes in eundo dicant prosam in medio chori: que in ipsa statione ante crucem terminetur.}

\emph{In eundo ℟.} In médio. 29. \emph{℣.} Misit Dóminus. \emph{⟨Tres sacerdotes dicant prosam} Náscitur. \emph{ut supra et dicitur cum versu.⟩\footnote{1517. ⟨1882:17.⟩}}

\emph{In introitu chori antiphona.} Hódie Christus. 23.

\emph{Versus.} Benedíctus qui venit. 23.

\emph{Oratio ut supra.} 23.

\begin{noinitial}

\gregorioscore{006913Pzd-gloria-patri}

\end{noinitial}

\emph{¶ In die sancti Johannes ad vesperas post memoriam de sancto Stephano eat processio puerorum ad altare sancte Trinitatis: et omnium sanctorum quod dicatur} Salve.\footnote{1517 omits `\emph{quod dictur} Salve.' ⟨1882:17.⟩} \emph{in cappis cereis ardentibus in manibus cantando: episcopo puerorum incipiente hoc modo.}

\gregorioscore{006273-centum-quadraginta}

\emph{Tres pueri dicant hunc versum sequentem.}

\begin{noinitial}

\gregorioscore{006273b-hi-empti-sunt}

\end{noinitial}

\emph{Omnes pueri dicant hanc prosam. Chorus post unumquemque versum respondeat cantum prose super ultimum litteram.}

\gregorioscore{ah10068-sedentem-in-superne}

\emph{Ad hanc prosam non dicatur} Glória Patri. \emph{sed dum prosa canitur tunc episcopus puerorum thurificet altare deinde imaginem Sancte Trinitatis: et postea dicat modesta voce ℣.} Letámini in Dómino et exultáte justi.

\emph{℟.} Et gloriámini omnes recti corde.

\emph{⟨℣.⟩} Orémus.

\begin{lesson}
\subsubsection[⟨Oratio.⟩]{⟨Oratio.⟩\footnote{1882:18.}}

\lettrine{D}{e}us cujus hodiérna die precónium innocéntes mártyres non loquéndo sed moriéndo conféssi sunt, ómnia in nobis vitiórum mala mortífica: ut fidem tuum quam lingua nostra lóquitur: étiam móribus vita fateátur. Qui cum Deo Patre et Spíritu Sancto vivis et regnas Deus. Per ómnia sécula seculórum. \emph{⟨℟.⟩} Amen.
\end{lesson}

\emph{In revertendo\footnote{`\emph{redeundo}', 1882:18.} precentor\footnote{`\emph{preceptor}', 1519:15v; 1523:15v. `The editions vary between \emph{precentor puerorum} and \emph{preceptor}, as the \emph{preceptor} or \emph{terminator processionis.} The two words are elsewhere interchanged in the same kind of reference. See Ducange under \emph{Preceptor}.' ⟨1882:xv.⟩} puerorum incipiat de sancta Maria ℟.} Felix namque. \emph{Require in ⟨die⟩\footnote{1882:18.} assumptionis beata Marie} 364. \emph{: et si necesse fuerit dicatur ℣.} Ora pro pópulo. \emph{et loco} Assumptiónem. \emph{dicatur} Commemorationem.\footnote{1882:18 has `Sollennitátem.' with the following note: 1528, \&c.---Commemorationem.} \emph{et sic processio chorum intret per ostium occidentale ut supra: et omnes pueri ex utraque parte chori in superiori gradu se recipiant: et ab hac hora usque post processionem diei proximi succedentis nullus clericorum solet gradum superiorem ascendere cujuscunque conditionis fuerit.}

\emph{¶ Ad istam processionem pro dispositione puerorum scribuntur canonici ad ministrandum eisdem majores ad thuribulandum et ad librum deferendum minores ad candelabra deferenda.}

\emph{Episcopus in sede sua dicat versiculum.} Speciósus forma pre fíliis hóminum. \emph{℟.} Diffúsa est. 27. \emph{Oratio.} Deus qui salútis etérne. \emph{que terminetur sic} Qui tecum vivit et regnat. \emph{⟨et cetera.⟩\footnote{1882:18.}} 28. Pax vobis. Et cum spiritu tuo. \emph{Sequatur} Benedicámus Dómino. \emph{a duobus vicariis vel a tribus extra regulam.}

\emph{Post hec episcopus puerorum in sede sua benedicat populum in hunc modum. Cruciferarius accipiat baculum episcopi in manu conversus ad episcopum: et incipiat hanc antiphonam sequentem in hunc modum.}

\gregorioscore{000000-princeps-ecclesie-1}

\emph{Hic convertat se ad populum sic dicendo.}

\begin{noinitial}

\gregorioscore{000000-princeps-ecclesie-2}

\emph{Chorus respondent sic.}

\gregorioscore{000000-princeps-ecclesie-3}

\end{noinitial}

\emph{Deinde tradat baculum episcopo et tunc episcopus puerorum primo signando se in fronte sic dicendo.}

\gregorioscore{missing}

\emph{Chorus respondeat sic.}

\gregorioscore{missing}

\emph{Item episcopus signando se in pectore dicat sic.}

\gregorioscore{missing}

\emph{Deinde episcopus puerorum conversus ad clerum: elevet brachium suum: et dicat hanc benedictionem hoc modo.}

\gregorioscore{missing}

\emph{Hic convertat se ad populum dicendo sic.\footnote{`Nostra', 1882:19.}}

\gregorioscore{missing}

\emph{Deinde convertat se ad altare sic dicendo.\footnote{`Nos', 1882:19 and 1519. `vos' appears in the Breviary-1531.}}

\gregorioscore{missing}

\emph{Postea ad seipsum reversus ponat manum super pectus suum dicendo.}

\gregorioscore{missing}

\emph{Chorus respondeat sic.}

\gregorioscore{missing}

\emph{His itaque peractis incipiat episcopus puerorum completorium more solito et post completorium dicat episcopus puerorum ad chorum conversus sub tono supradicto modo\footnote{cf.~Breviarium, 408.}} Adjutórium nostrum. \emph{et} Sit nomen Dómini. 35. \emph{Deinde dicat episcopus.} Benedícat vos omnípotens Deus Pater et Fílius et

1519:16v.\footnote{1519:16v has no flat.}

\gregorioscore{missing}

\emph{⟨Chorus respondeat sic.⟩\footnote{1519:78r.}}

\gregorioscore{missing}⟨\gregorioscore{missing}⟩\footnote{1519:77v; 1531:37v.}


\chapter{¶ In die sanctorum innocentium.}

\emph{Si dominica fuerit eodem modo processio ⟨fiat⟩\footnote{1882:19.} ut in die sancti Stephani excepta quod hac die iij. pueri precedent in eundo dicant in medio processione\footnote{`\emph{prosam in eundo dicant in medio procedentes}', 1882:19.} que in ipsa statione ante crucem ab eisdem terminetur.}

\emph{In eundo ℟.} Centum quadragínta. 31. \emph{℣.} Hi empti. \emph{Prosa.} Sedéntem in supérne. \emph{Sequatur}

\begin{noinitial}

\gregorioscore{006273-gloria-patri}

\end{noinitial}

\emph{In introitu chori de nativitate ut supra.} 23.

\emph{¶ In die sanctorum innocentium ad vesperas post memoriam de sancto Johanne accipiat cruciferarius baculum episcopi puerorum et cantet antiphonam} Princeps ecclésie. \emph{sicut ad j. vesperas.} 35. \emph{Similiter episcopus puerorum benedicat populum supradicto modo, et sic compleatur servitium hujus diei.}

\emph{Deinde\footnote{This and the Procession for S. Thomes' day are omitted in 1544 and later edd. ⟨1882:20.⟩} eat processio ad altare sancti Thome martyris habitu non mutato absque cereis in manibus cantando ℟. cantore incipiente hoc modo.}

\gregorioscore{601260-jacet-granum}

\emph{Deinde dicatur prosa in superpelliciis ab omnibus qui voluerint si placet et chorus respondent cantum prose super literam} 𝔄.

\gregorioscore{601260Pa-clangat-pastor}

\emph{¶ In hac processione non dicatur} Glória Patri. \emph{sed dum canitur prosa thurificet sacerdos altare, deinde imaginem sancti Thome et postea dicat ℣.} Ora pro nobis beáte Thome.

\emph{℟.} Ut ⟨digni efficiámur promissiónibus Christi⟩.\footnote{1882:20.}

\begin{lesson}
\subsubsection[Oratio.]{Oratio.\footnote{In 1519:17v `gloriósus póntifex Thomas' is crossed out.}}

\lettrine{D}{e}us pro cujus ecclésia gloriósus póntifex Thomas gládiis impiórum occúbuit: presta quésumus: ut omnes qui ejus implórant auxílium petitiónis sue salutárem consequántur efféctum. Per Christum.
\end{lesson}

\emph{In revevertendo dicatur antiphona vel ℟. de sancta Maria.}


\chapter[¶ In die sancti Thome martyris.]{¶ In die sancti Thome martyris.\footnote{In 1519:17v `In die sancti Thomas' is crossed out.}}

\emph{Si dominica fuerit, ad processionem ante missam fiat eodem modo et ordine quo in die sancti Stephani: excepto quod in hac die iij. clerici de superiori gradu dicant prosam in medio processionis que in ipsa statione ante crucem finietur.}

\emph{In eundo dicatur ℟.} Jacet granum. 38. \emph{℣.} Cadit custos. \emph{Prosa.} Clangat pastor. \emph{et dicatur cum} Glória Patri.

1519:17v.\footnote{In 1519:17v `tu' is set B♭. In 1555 (unpaged), `tu' is set B♭C.}

\begin{noinitial}

\gregorioscore{601260-gloria-patri}

\end{noinitial}

\emph{In introitu chori de nativitate ut supra.} 23.


\chapter{¶ Sexta die a nativitate Domini.}

\emph{Si dominica fuerit ad processionem ℟.} Descéndit. \emph{℣.} Tanquam sponsus. 19. \emph{et dicitur sic.}

1519:18r.

\gregorioscore{missing}

\emph{\&c.~sine prosa cum} Glória Patri.\footnote{`\emph{vel ℟}. Verbum caro. \emph{℣}. In princípio. \emph{sine prosa}.' GS:18.} 21.

\emph{¶ In redeundo de nativitate ut supra.\footnote{`\emph{In redeundo} Te laudant. \emph{vel} Hódie Christus. \emph{Sequatur ℣. et or. de nat.}' GS:18.}} 23.


\chapter{¶ In die sancti Silvestri.}

\emph{Si dominica fuerit Resp.} Miles Christi.\footnote{`Christi miles', in the Processionals; however, surely `Miles Christi' is intended. `Christi miles' is proper to St.~Vincent.} 397.

\emph{In introitu chori de nativitate ut supra.} 23.


\chapter[¶ In die circuncisionis ⟨Domini⟩.]{¶ In die circuncisionis ⟨Domini⟩.\footnote{1882:21.}}

\emph{Si dominica fuerit modus et ordo processionis fiat hac die ut in die sancti Thome martyris. Dicatur responsorium sequens.}

\gregorioscore{007840-verbum-caro}

\clearpage

\emph{Tres clerici de superiori gradu in cappis sericis in medio processionis dicant ⟨hanc⟩\footnote{1882:22.} prosam ⟨sequentem⟩.\footnote{1882:22.}}

\gregorioscore{007840Pzc-quem-ethera}

\begin{noinitial}

\gregorioscore{007840-gloria-patri}

\end{noinitial}

\emph{In introitu chori dicatur ℟. ⟨sequens⟩\footnote{1882:22.} de sancta Maria.}

\gregorioscore{007756-te-laudant-angeli}

\emph{¶ Cum versu vel sine versu pro dispositione cantoris.}

\begin{noinitial}
	
\gregorioscore{007756a-ipsum-genuisti}

\end{noinitial}

\emph{℣.} Post partum ⟨virgo invioláta permansísti.

\emph{℟.} Dei génitrix intercéde pro nobis⟩.

\emph{Oratio.} Deus qui salútis. \emph{ut supra in processione sancti Stephani.} 28.


\chapter[¶ In octavis sancti Stephani et Johannis.]{¶ In octavis sancti Stephani et Johannis et sanctorum innocentium.}

\emph{Si dominica fuerit fiat processio per omnia sicut in die preter habitum et sine prosis.}

\emph{In introitu chori a circumcisione usque ad purificationem et si lxx. infra purificationem evenerit semper de sancta Maria dicatur ⟨hoc⟩\footnote{1882:22.} ℟.} Te laudant ángeli. \emph{ut supra in ⟨die⟩\footnote{1882:22.} circumcisionis} 43. \emph{excipiuntur cum\footnote{`\emph{tamen}', 1882:22.} dies epiphanie et dominica infra octavas ejusdem et in octava si ⟨in⟩\footnote{1882:22.} dominica evenerit.}


\chapter{¶ In vigilia epiphanie.}

\emph{Si ⟨in⟩\footnote{1882:22.} dominica evenerit ad processionem in eundo dicatur ℟.} Verbum caro. 41. \emph{cum ℣. et cum} Glória Patri. \emph{sed sine prosa.}

\emph{In introitu chori de sancta Maria ut supra.}

