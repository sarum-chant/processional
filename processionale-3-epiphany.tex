\chapter{¶ In die epyphanie.}

\emph{Quacunque feria contigerit modus et ordo fiat processionis per omnia sicut in die natvitatis Domini preter prosam.} 18.

\emph{In eundo dicatur.}

\label{007777-tria-sunt-munera}
\gregorioscore{007777-tria-sunt-munera}

\emph{⟨Sequitur⟩\footnote{1882:23.} aliud ℟.}

\gregorioscore{007523-reges-tharsis}

\emph{⟨℣.⟩} Glória Patri. 420. ‡~Dómino Deo.

\emph{In introitu chori dicatur ℟. sequentem.}

\label{006892-in-columbe-specie}
\gregorioscore{006892-in-columbe-specie}

\label{vox-domini}
\emph{℣.} Vox Dómini super aquas.\footnote{1519:20v; 1523:20v; 1528:19v; 1544:23v; and 1554:21r have `aquas multas'.}

\emph{℟.} Dóminus ⟨super aquas multas⟩.\footnote{1882:23.}

\emph{⟨℣.⟩} Orémus.

\begin{lesson}
\subsubsection{Oratio.}

\label{deus-qui-hodierna}
\lettrine{D}{e}us qui hodiérna die Unigénitum tuum géntibus stella duce revelásti concéde propítius: ut qui te jam\footnote{`jam te', 1882:23.} ex fide cognóvimus: usque ad contemplándam spéciem tue celsitúdinis perducámur. Per eúndem Christum ⟨Dóminum nostrum. Amen⟩.\footnote{1882:23.}
\end{lesson}


\chapter{¶ Dominica infra octavas epyphanie et in octava.}

\emph{Si dominica fuerit ad processionem ℟.} Tria sunt ⟨múnera⟩.\footnote{1882:23.} \pageref{007777-tria-sunt-munera}.

\emph{In introitu chori ℟.} In colúmbe spécie. \pageref{006892-in-columbe-specie}.

\emph{℣.} Vox Dómini. \pageref{vox-domini}.

\emph{Oratio.} Deus qui hodiérna. \pageref{deus-qui-hodierna}.


\chapter[¶ Dominica j. post octavas epiphanie.]{¶ Dominica j. post octavas epiphanie et in omnibus dominicis abhinc usque ad lxx. quando de dominica agitur.}

\emph{Modus et ordo processionis fiant ut in dominica prima adventus Domini ⟨nostri Jesu Christi⟩.\footnote{1882:23.}} 9.

\emph{Ad processionem ⟨cantetur⟩\footnote{1882:24.} ℟. hoc modo ⟨quo sequitur⟩.\footnote{1882:24.}}

\gregorioscore{006011-abscondi-tanquam-aurum}

\emph{℣.} Glória. 421. †~Miserére.

\emph{In introitu chori usque ad purificationem dicatur de sancta Maria ℟.} Te laudant. \emph{ut supra.} 43. \emph{Post purificationem vero usque ad septuagesimam dicatur in introitu chori antiphona} Ave regína ⟨celórum⟩.\footnote{1882:24 `\emph{vel} Alma redemptóris', GS:20.} 292. \emph{vel alia antiphona de sancta Maria cum hoc versu} Post partum ⟨virgo invioláta permansísti⟩.\footnote{1882:24.} 44.

Orémus.

\begin{lesson}
\subsubsection{Oratio.}

\lettrine{C}{o}ncéde quésumus miséricors Deus fragilitáti nostre presídium: ut qui sancte Dei genitrícis et vírginis Maríe commemoratiónem ágimus, intercessiónis ejus auxílio, a nostris iniquitátibus resurgámus. Per eúndem Christum Dóminum \emph{\&c.} ⟨nostrum. \emph{⟨℟.⟩} Amen⟩.\footnote{1882:24. 1519:21v has simply `\emph{\&c.}' `\emph{vel} Concéde nos fa. \emph{vel alia or. de sancta Maria.}' GS:20.}
\end{lesson}


\chapter{¶ Dominica septuagesime.}

\emph{Ad processionem ⟨dicatur antiphona que sequitur⟩.\footnote{1882:24.}}

\gregorioscore{002497-ecce-charissimi}

\emph{Duo clerici de secunda forma in statione ante crucem ad populum conversi habitu non mutato cantant ⟨versum⟩.\footnote{1882:24.}}

\begin{noinitial}

\gregorioscore{002497a-ecce-mater}

\end{noinitial}

\emph{In introitu chori ante purificationem dicatur ⟨responsorium⟩\footnote{1882:24.}} Te laudant. 22. \emph{Si post dicatur ⟨hoc responsorium sequens⟩.\footnote{1882:24.}}

\gregorioscore{007804-ubi-est-abel}

\emph{Versus.} Dómine refúgium factus es nobis.

\emph{Resp.} A generatióne et progénie.

\emph{⟨℣.⟩} Orémus.

\begin{lesson}
\subsubsection{Oratio.}

\lettrine{P}{r}eces pópuli tui quésumus Dómine cleménter exáudi ut qui juste pro peccátis nostris afflígimur: pro nóminis tui glória misericórditer liberémur. Per Christum.
\end{lesson}


\chapter{¶ Dominica in lx.}

\emph{Ad processionem ant.} Ecce charíssimi. \emph{ut supra.} 50.

\emph{In introitu chori de sancta Maria: si lx. ante purificationem evenerit ⟨ut supra⟩\footnote{Compare Dominica in l. below.}} 50.\emph{: si vero post purificationem evenerit.}

\gregorioscore{600283-benedicens}

\emph{℣.} Dómine refúgium factus es nobis.

\emph{℟.} A generatióne et progénie.

\emph{⟨℣.⟩} Orémus.

\begin{lesson}
\subsubsection{Oratio.}

\lettrine{D}{e}us qui cónspicis quia ex nulla nostra actióne confídimus: concéde propítius: ut contra ómnia advérsa doctóris géntium protectióne muniámur. Per Christum Dóminum nostrum. \emph{⟨℟.⟩} Amen.
\end{lesson}


\chapter{¶ Dominica in quinquagesima.}

\emph{Ad processionem ant.} Ecce caríssimi. 50.

\emph{In introitu chori de sancta Maria: si l. ante purificationem evenerit ut supra, si vero post purificationem evenerit dicatur hoc.}

\gregorioscore{007762-tentavit-deus}

\emph{Versus.} Dómine refúgium factus es nobis.

\emph{Resp.} A generatióne et progénie.

\emph{⟨℣.⟩} Orémus.

\begin{lesson}
\subsubsection{Oratio.}

\lettrine{P}{r}eces nostras quésumus Dómine cleménter exáudi: atque a peccatórum nostrórum vínculis absolútos ab omni nos adversitáte custódi. Per Christum Dóminum nostrum. \emph{⟨℟.⟩} Amen.
\end{lesson}

