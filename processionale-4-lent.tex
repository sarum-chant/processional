\chapter{¶ Feria quarta in capite jejunii.}

\emph{Post sextam in primis\footnote{`\emph{imprimis}', 1882:26.} fiat sermo ad populum si placuerit: deinde prosternant se clerici in choro, et dicant septem psalmos penitentiales cum} Glória Patri. \emph{et} Sicut erat. \emph{et antiphona} Ne reminiscáris.

\emph{Excellentior vero procedat sacerdos\footnote{`\emph{sacerdos procedat}', 1882:26.} indutus cappa serica rubea, cum aliis vestibus sacerdotalibus, cum diacono a dextris et subdiacono a sinistris et ceteris ministris altaris qui omnes sint albis cum amictibus induti a vestibulo ad gradum altaris procedant: et ibi dicant septem psalmos penitentiales in prostratione videlicet.}

\section[¶ Sequuntur septem psalmi penitentiales.]{¶ Sequuntur septem psalmi penitentiales.\footnote{`Dómine ne in furóre. \emph{et cetera. Require predictos psalmos in fine libri.}' 1882:26.}}

\emph{Antiphona.} Ne reminiscáris. \emph{Psalmus.\footnote{US-II:56. The Penitential Psalms were recited beginning on Ash Wednesday. The full rite appears in TUS:III-17. and in SP:22v.}}

⟨1523:24v.\footnote{1519:24v has one of the standard images of David and Bathsheba:⟩}

\begin{lesson}
\subsubsection{⟨Psalmus. vj.⟩}

\lettrine{D}{o}mine ne in furóre tuo árguas me: neque in ira tua corrípias me.

Miserére mei Dómine quóniam infírmus sum: sana me Dómine quóniam conturbáta sunt ossa mea.

Et ánima mea turbáta est valde: et tu Dómine úsquequo?

Convértere Dómine et eripe ánimam meam: salvum me fac propter misericórdiam tuam.

Quóniam non est in morte qui memor sit tui: in inférno autem quis confitébitur tibi?

Laborávi in gémitu meo: lavábo per síngulas noctes lectum meum: láchrymis meis stratum meum rigábo.

Turbátus est a furóre óculus meus: inveterávi inter omnes inimícos meos.

Discédite a me omnes qui operámini iniquitátem: quóniam exaudívit Dóminus vocem fletus mei.

Exaudívit Dóminus deprecatiónem meam: Dóminus oratiónem meam suscépit.

Erubéscant et conturbéntur veheménter omnes inimíci mei: convertántur et erubéscant valde velóciter.

Glória Patri.

Sicut erat.

\subsubsection{Psalmus. ⟨xxxj.⟩}

\lettrine{B}{e}áti quorum remísse sunt iniquitátes: et quorum tecta sunt peccáta.

Beatus vir cui non imputávit Dóminus peccátum: nec est in spíritu ejus dolus.

Quóniam tácui inveteravérunt ossa mea: dum clamárem tota die.

Quóniam die ac nocte graváta est super me manus tua: convérsus sum in erúmna mea dum confígitur spina.

Delíctum meum cógnitum tibi feci: et injustítiam meam non abscóndi.

Dixi Confitébor advérsum me injustítiam meam Dómino: et tu remisísti impietátem peccáti mei.

Pro hac orábit ad te ómnis sanctus: in témpore opportúno.

Verúntamen in dilúvio aquárum multárum: ad eum non approximábunt.

Tu es refúgium meum a tribulatióne que circundédit me: exultátio mea erue me a circundántibus me.

Intelléctum tibi dabo et ínstruam te in via hac qua gradiéris: firmábo super te óculos meos.

Nolíte fíeri sicut equus et mulus: quibus non est intelléctus.

In chamo et freno maxíllas eórum constrínge: qui non appróximant ad te.

Multa flagélla peccatóris: sperántem autem in Dómino misericórdia circundábit.

Letámini in Dómino et exultáte justi: et gloriámini omnes recti corde.

Glória.

\subsubsection{Psalmus. ⟨xxxvij.⟩}

\lettrine{D}{o}mine ne in furóre tuo árguas me: neque in ira tua corrípias me.

Quóniam sagítte tue infíxe sunt michi: et confirmásti super me manum tuam.

Non est sánitas in carne mea\footnote{In 1519:25r the punctuation differs from the Breviary. For consistency, this edition follows the Breviary. The following are the variants in 1519.

  3. Non est sánitas in carne mea: * a fácie ire tue non est pax óssibus meis a fácie peccatórum meórum.

  10. Cor meum conturbátum est derelíquit me virtus mea: * et lumen oculórum meórum, et ipsum non est mecum.} a fácie ire tue: non est pax óssibus meis a fácie peccatórum meórum.

Quóniam iniquitátes mee supergrésse sunt caput meum: et sicut onus grave graváte sunt super me.

Putruérunt et corrúpte sunt cicatríces mee: a facie insipiéntie mee.

Miser factus sum et curvátus sum usque in finem: tota die contristátus ingrediébar.

Quóniam lumbi mei impléti sunt illusiónibus: et non est sánitas in carno mea.

Afflíctus sum et humiliátus sum nimis: rugiébam a gémitu cordis mei.

Dómine ante te omne desidérium meum: et gémitus meus a te non est abscónditus.

Cor meum conturbátum est: derelíquit me virtus mea et lumen oculórum meórum, et ipsum non est mecum.

Amíci mei et próximi mei: advérsum me appropinquavérunt et stetérunt.

Et qui juxta me erant de longe stetérunt: et vim faciébant qui querébant ánimam meam.

Et qui inquirébant mala michi locúti sunt vanitátes: et dolos tota die meditabántur.

Ego autem tanquam surdus non audiébam: et sicut mutus non apériens os suum.

Et factus sum sicut homo non áudiens: et non habens in ore suo redargutiónes.

Quóniam in te Dómine sperávi: tu exáudies me\footnote{`me' is missing in 1531-P:47v, but is present in 1531-P:16v.} Dómine Deus meus.

Quia dixi Nequándo supergáudeant michi inimíci mei: et dum commovéntur pedes mei super me magna locúti sunt.

Quóniam ego in flagélla parátus sum: et dolor meus in conspéctu meo semper.

Quóniam iniquitátem meam annunciábo: et cogitábo pro peccáto meo.

Inimíci autem mei vivunt, et confirmáti sunt super me: et multiplicáti sunt qui odérunt me iníque.

Qui retríbuunt mala pro bonis detrahébant michi: quóniam sequébar bonitátem.

Ne derelínquas me Dómine: Deus meus ne discésseris a me.

Inténde in adjutórium meum Dómine: Deus salútis mee.

Glória Patri et Fílio: et Spirítui Sancto.

Sicut erat in princípio.

\subsubsection{Psalmus. ⟨l.⟩}

\lettrine{M}{i}serére mei Deus: secúndum magnam misericórdiam tuam.

Et secúndum multitúdinem miseratiónum tuárum: dele iniquitátem meam.

Amplius lava me ab iniquitáte mea: et a peccáto meo munda me.

Quóniam iniquitátem meam ego cognósco: et peccátum meum contra me est semper.

Tibi soli peccávi et malum coram te feci: ut justificéris in sermónibus tuis, et vincas cum judicáris.

Ecce enim in iniquitátibus concéptus sum: et in peccátis concépit me mater mea.

Ecce enim veritátem dilexísti: incérta et occúlta sapiéntie tue manifestásti michi.

Aspérges me ysópo et mundábor: lavábis me et super nivem dealbábor.

Audítui meo dabis gáudium et letíciam: et exultábunt ossa humiliáta.

Avérte fáciem tuam a peccátis meis: et omnes iniquitátes meas dele.

Cor mundum crea in me Deus: et spíritum rectum ínnova in viscéribus meis.

Ne projícias me a fácie tua: et spíritum sanctum tuum ne áuferas a me.

Redde michi letíciam salutáris tui: et spíritu principáli confírma me.

Docébo iníquos vias tuas: et ímpii ad te converténtur.

Líbera me de sanguínibus Deus Deus salútis mee: et exultábit lingua mea justíciam tuam.

Dómine lábia mea apéries: et os meum annunciábit laudem tuam.

Quóniam si voluísses sacrifícium dedíssem: útique holocáustis non delectáberis.

Sacrifícium Deo spíritus contribulátus: cor contrítum et humiliátum Deus non despícies.

Benígne fac Dómine in bona voluntáte tua Syon: et edificéntur muri Hierúsalem.

Tunc acceptábis sacrifícium justície\footnote{In 1519:25v the punctuation differs from the Breviary. For consistency, this edition follows the Breviary. The following is a variant in 1519.

  20. Tunc acceptábis sacrifícium justície: * oblatiónes et holocáusta tunc impónent super altáre tuum vítulos.} oblatiónes et holocáusta: tunc impónent super altáre tuum vítulos.

Glória.

Sicut.

\subsubsection{Psalmus. ⟨cj.⟩}

\lettrine{D}{o}mine exáudi oratiónem meam: et clamor meus ad te véniat.

Non avértas fáciem tuam a me: in quacúnque die tríbulor inclína ad me aurem tuam.

In quacúnque die invocávero te: velóciter exáudi me.

Quia defecérunt sicut fumus dies mei: et ossa mea sicut crémium aruérunt.

Percússus sum ut fenum et áruit cor meum: quia oblítus sum comédere panem meum.

A voce gémitus mei: adhésit os meum carni mee.

Símilis factus sum pellicáno solitúdinis: factus sum sicut nictícorax in domicílio.

Vigilávi: et factus sum\footnote{In 1519:25r the punctuation differs from the Breviary. For consistency, this edition follows the Breviary. The following are the variants in 1519.

  8. Vigilávi et factus sum: * sicut passer solitárius in tecto.} sicut passer solitárius in tecto.

Tota die exprobrábant michi inimíci mei: et qui laudábant me advérsum me jurábant.

Quia cínerem tanquam panem manducábam: et potum meum cum fletu miscébam.

A fácie ire indignatiónis tue: quia élevans allisísti me.

Dies mei sicut umbra declinavérunt: et ego sicut fenum árui.

Tu autem Dómine in etérnum pérmanes: et memoriále tuum in generatióne et generatiónem.

Tu exúrgens miseréberis Syon: quia tempus miseréndi ejus quia venit tempus.

Quóniam placuérunt servis tuis lápides ejus: et terra ejus miserebúntur.

Et timébunt gentes nomen tuum Dómine: et omnes reges terre glóriam tuam.

Quia edificávit Dóminus Syon: et vidébitur in glória sua.

Respéxit in oratiónem humílium: et non sprevit precem eórum.

Scribántur hec in generatióne áltera: et pópulus qui creábitur laudábit Dóminum.

Quia prospéxit de excélso sancto suo: Dóminus de celo in terram aspéxit.

Ut audíret gémitus compeditórum: ut sólveret fílios interemptórum.

Ut annúntient in Syon nomen Dómini: et laudem ejus in Hierúsalem.

In conveniéndo pópulos in unum: et reges ut sérviant Dómino.

Respóndit ei in via virtútis sue: Paucitátem diérum meórum núncia michi.

Ne révoces me in dimídio diérum meórum: in generatióne et generatiónem anni tui.

Inítio\footnote{`In inítio', 1531-P:34r, 48r.} tu Dómine terram fundásti: et ópera mánuum tuárum sunt celi.

Ipsi períbunt tu autem pérmanes: et omnes sicut vestiméntum veteráscent.

Et sicut opertórium mutábis eos et mutabúntur: tu autem idem ipse es, et anni tui non defícient.

Fílii servórum tuórum habitábunt: et semen eórum in séculum dirigétur.

Glória Patri.

\subsubsection{Psalmus. ⟨cxxix.⟩}

\lettrine{D}{e} profúndis clamávi ad te Dómine: Dómine exáudi vocem meam.

Fiant aures tue intendéntes: in vocem deprecatiónis mee.

Si iniquitátes observáveris Dómine: Dómine quis sustinébit?

Quia apud te propiciátio est: et propter legem tuam sustínui te Dómine.

Sustínuit ánima mea in verbo ejus: sperávit ánima mea in Dómino.

A custódia matutína usque ad noctem: speret Israel in Dómino.

Quia apud Dóminum misericórdia: et copiósa apud eum redémptio.

Et ipse rédimet Israel: ex ómnibus iniquitátibus ejus.

Glória.

\subsubsection{Psalmus. ⟨cxlij.⟩}

\lettrine{D}{o}mine exáudi oratiónem meam: áuribus pércipe obsecratiónem meam in veritáte tua\footnote{In 1519:25v the punctuation differs from the Breviary. For consistency, this edition follows the Breviary. The following is a variant in 1519.

  1. Dómine exáudi oratiónem meam áuribus pércipe obsecratiónem meam in veritáte tua: * exáudi me in tua justícia.} exáudi me in tua justícia.

Et non intres in judícium cum servo tuo:\footnote{1519:26v has `tuo Dómine'.} quia non justificábitur in conspéctu tuo omnis vivens.

Quia persecútus est inimícus ánimam meam: humiliávit in terra vitam meam.

Collocávit me in obscúris sicut mórtuos séculi: et anxiátus est super me spíritus meus in me turbátum est cor meum.

Memor fui diérum antiquórum meditátus sum in ómnibus opéribus tuis: in factis mánuum tuárum meditábar.

Expándi manus meas ad te: ánima mea sicut terra sine aqua tibi.

Velóciter exáudi me Dómine: defécit spíritus meus.

Non avértas fáciem tuam a me: et símilis ero descendéntibus in lacum.

Audítam fac michi mane misericórdiam tuam: quia in te sperávi.

Notam fac michi viam in qua ámbulem: quia ad te levávi ánimam meam.

Eripe me de inimícis meis Dómine ad te confúgi: doce me fácere voluntátem tuam, quia Deus meus es tu.

Spíritus tuus bonus dedúcet me in terram rectam: propter nomen tuum Dómine, vivificábis me in equitáte tua.

Edúces de tribulatióne ánimam meam: et in misericórdia tua dispérdes inimícos meos.

Et perdes omnes qui tríbulant ánimam meam: quóniam ego servus tuus sum.

Glória Patri.
\end{lesson}

\emph{Antiphona.} Ne reminiscáris Dómine delícta nostra vel paréntum nostrórum, neque vindíctam sumas de peccátis nostris.

Kýrie eléyson. Christe eléyson. Kýrie eléyson.

Pater noster. Brev:{[}5{]}.

\emph{Et hec omnia sine nota dicantur tam a sacerdote quam a toto choro.}

\emph{Deinde erigat se sacerdos cum dyacono et subdiacono et cum puero librum sibi administrante et stando conversus ad orientem dicat super populum cum nota hoc modo.}

\gregorioscore{a02418-et-ne-nos}

\emph{Clerici respondeant.\footnote{`\emph{Chorus respondeat}', 1882:26.}} Sed líbera nos a malo.

\emph{⟨℣.⟩} Salvos fac servos tuos et ancíllas tuas. \emph{⟨℟.⟩} Deus meus sperántes in te.

\emph{⟨℣.⟩} Mitte eis Dómine auxílium de sancto. \emph{⟨℟.⟩} Et de Syon tuére eos.

\emph{⟨℣.⟩} Convértere Domine úsquequo. \emph{⟨℟.⟩} Et deprecábilis ⟨esto super servos tuos⟩.\footnote{1882:26.}

\emph{⟨℣.⟩} Adjuva nos Deus salutáris noster. \emph{⟨℟.⟩} Et propter glóriam nóminis ⟨tui Dómine líbera nos et propítius esto peccátis nostris propter nomen tuum⟩.\footnote{1882:26.}

\emph{⟨℣.⟩} Dómine exáudi ⟨oratiónem meam⟩.\footnote{1882:26.} \emph{⟨℟.⟩} Et clamor meus ⟨ad te véniat⟩.\footnote{1882:26.}

\emph{⟨℣.⟩} Dóminus vobíscum. \emph{⟨℟.⟩} Et cum spíritu tuo.

\emph{⟨℣.⟩} Orémus.

\begin{lesson}
\subsubsection{Oratio.}

\lettrine{E}{x}áudi Dómine preces nostras et confiténtium tibi parce peccátis ut quos consciéntie reátus accúsat, indulgéntia tue miseratiónis absólvat. Per Christum Dóminum ⟨nostrum. \emph{℟.} Amen.⟩\footnote{1882:27.}
\end{lesson}

\emph{Et omnes orationes dicantur cum} Orémus. \emph{sub tono supradicto. Non dicatur} Dóminus vobíscum. \emph{nisi ante primam orationem tantum: tamen alie orationes cum} Orémus.

\begin{lesson}
\subsubsection{Oratio.}

\lettrine{A}{d}sit quésumus Dómine fámulis tuis inspirátio grátie salutáris que corda eórum flétuum ubertáte resólvat sicque macerándo confíciat: ut iracúndie tue motus: idónea satisfactióne compéscat. Per Christum Dóminum nostrum. ⟨Amen.⟩\footnote{1882:27.}
\end{lesson}

\emph{⟨℣.⟩} Orémus.

\begin{lesson}
\subsubsection{Oratio.}

\lettrine{D}{a} quésumus Dómine Deus noster his fámulis tuis contínuam purgatiónis sue observántiam peniténdo ágere: et ut hoc efficátius implére váleant grátia eos tue visitatiónis prevéniat et subsequátur. Per Christum.
\end{lesson}

\emph{⟨℣.⟩} Orémus.

\begin{lesson}
\subsubsection{Oratio.}

\lettrine{P}{r}evéniat hos fámulos ⟨tuos⟩\footnote{1882:27.} quésumus Dómine misericórdia tua: ut omnes iniquitátes eórum céleri indulgéntia deleántur. Per Christum Dóminum nostrum.
\end{lesson}

\emph{⟨℣.⟩} Orémus.

\begin{lesson}
\subsubsection{Oratio.}

\lettrine{A}{d}ésto Dómine supplicatiónibus nostris nec sit ab his fámulis tuis cleméntie tue longínqua miserátio sana vúlnera, eorúmque peccáta remítte: ut nullis a te iniquitátibus separáti tibi Dómine semper váleant adherére. Per Christum Dóminum nostrum.
\end{lesson}

\emph{⟨℣.⟩} Orémus.

\begin{lesson}
\subsubsection{Oratio.}

\lettrine{D}{o}mine Deus noster qui offensióne nostra non vínceris sed satisfactióne placáris: réspice quésumus super hos fámulus tuos qui se tibi gráviter peccásse confiténtur, tuum est enim absolutiónem críminum dare, et véniam prestáre peccántibus qui dixísti peniténtiam te male peccatórum quam mortem: concéde ergo Dómine his fámulis tuis ut tibi penténtie excúbias célebrent, et corréctis áctibus suis: conférri sibi a te sempitérna gáudia gratuléntur. Per Christum.
\end{lesson}

\emph{⟨℣.⟩} Orémus.

\begin{lesson}
\subsubsection{Oratio.}

\lettrine{D}{e}us cujus indulgéntia omnis homo índiget meménto famulórum famularúmque tuárum et quia lúbrica terrenáque córporis fragilitáte nudáti virtúte: in multis deliquérunt: quésumus ut des véniam confiténtibus parcas supplicántibus:\footnote{`supplícibus', 1882:28.} ut qui suis méritis accusántur: tua miseratióne salvéntur. Per Christum Dóminum nostrum. \emph{⟨℟.⟩} Amen.
\end{lesson}

\emph{Hic non dicatur} Dóminus vobíscum. \emph{neque} Orémus. \emph{Sed vertat se sacerdos ad populum, et dicat super eos.}

\begin{lesson}
\subsubsection{Absolutio.}

\lettrine{A}{b}sólvimus vos vice beáti Petri apostolórum príncipis cui colláta est a Dómino potéstas ligándi atque solvéndi et quantum ad vos pértinet accusátio, et ad nos remíssio sit vobis omnípotens Deus vita et salus et ómnium peccatórum vestrórum pius indúltor. Qui vivit et regnat ⟨cum Deo Patre⟩. \emph{\&c.}
\end{lesson}

\emph{Deinde surgant omnes a prostratione osculantes formulas vel terram sacerdote sic dicente} Qui vivit et regnat.

\emph{Deinde accedat sacerdos ad altare cum suis ministris et ibi super altare in dextera parte altaris ad orientem conversus fiat benedictio cinerum prius cineribus in pelvibus argenteis super altare positis.}

\emph{¶ Hic fiat benedictio cinerum sine} Dóminus vobíscum. \emph{et sine} Orémus. \emph{Hoc modo incipiat.}

\begin{lesson}
\subsubsection{Oratio.}

\lettrine{O}{m}nípotens sempitérne Deus qui miseréris ómnium et nichil odísti eórum que fecísti dissímulans peccáta hóminum propter peniténtiam, qui étiam súbvenis in necessitáte laborántibus bene✠dícere et sanctifi✠cáre hos cíneres dignáre quos causa humilitátis et sancte religionis ad emundándam delícta nostra super cápita nostra more Ninevitárum ferre constituísti: et da per invocatiónem sancti tui nóminis ut omnes qui eos ad deprecándam misericórdiam tuam super cápita sua túlerint a te mereántur ómnium delictórum suórum véniam accípere et hódie sic eórum inchoáre sancta jejúnia: ut in die resurrectiónis purificátis méntibus ad sanctum mereántur accédere pascha: et in futúro perpétuam accípere glóriam. Per Dóminum nostrum Jesum Christum Fílium tuum. Qui tecum.
\end{lesson}

\emph{Hic aspergatur aqua benedicta super cineres, deinde dicatur} Dóminus vobíscum. \emph{et} Orémus.

\begin{lesson}
\subsubsection{Oratio.}

\lettrine{D}{e}us qui non mortem sed peniténtiam desíderas peccatórum fragilitátem conditiónis humáne benigníssime réspice, et hos cíneres quos causa preferénde humilitátis atque promerénde vénie capítibus nostris impóni decrévimus: bene✠dícere pro tua pietáte dignéris: ut qui nos cíneres esse monuísti et ob pravitátis nostre méritum in púlverem reversúros cognóscimus: peccatórum ómnium véniam et prémia peniténtibus repromíssa misericórditer cónsequi mereámur. Per Dóminum nostrum Jesum Christum Fílium tuum. Qui tecum vivit et regnat Deus.
\end{lesson}

\emph{Postea distribuantur cineres.}

\emph{¶ Executor officii in sede episcopali ac duo clerici excellentiores persone, stolis amicti ex utraque parte chori in stallis suis capita quorumcunque venientium cineribus aspergent ut sic dicendo,} Meménto homo quod cinis es et in cínerem revertáris. In nómine Patris et Fílii et Spíritus Sancti. Amen.

\emph{Et interim cantentur iste sequens antiphone sequente.}

\gregorioscore{002770-exaudi-nos}

\emph{Non dicatur nisi primus ℣. psalmi sed statim sequatur} Glória Patri. 428. \emph{Deinde repetatur antiphona} Exáudi nos.\footnote{Rylands-24:82 indicates `\emph{repetatur ant.} Glória Patri'.}

\emph{Alia antiphona.}

\gregorioscore{003554-juxta-vestibulum}

\emph{⟨Alia⟩\footnote{1882:29.} antiphona.}

\gregorioscore{003193-immutemur-habitu}

\emph{¶ Peracto officio dicat sacerdos ad gradum chori sic} Dóminus vobíscum.

\emph{Chorus respondeat.} Et cum spíritu tuo.

\emph{⟨Sacerdos.⟩\footnote{1882:29.}} Orémus.

\begin{lesson}
\subsubsection{Oratio.}

\lettrine{D}{e}us qui juste irásceris et cleménter ignóscis afflícti pópuli tui lácrymas súscipe: et iram tue indignatiónis quam juste merétur ⟨propitiátus⟩\footnote{1882:29.} avérte. Per Christum Dóminum nostrum. \emph{⟨℟.⟩} Amen.

\emph{⟨℣.⟩} Orémus.

\subsubsection{Oratio.}

\lettrine{C}{o}ncéde nobis quésumus Dómine presídia milítie Christiáne sanctis\footnote{`sic sancte', 1882:29.} inchoáre jejúniis: ut contra spiritáles nequítias pugnatúri continéntie muniámur auxílio.\footnote{`auxíliis', 1882:29.} Per Dóminum nostrum Jesum Christum Fílium tuum. Qui tecum vivit. \emph{⟨et cetera.⟩\footnote{1882:29.}}
\end{lesson}

\emph{His finitis eat processio per medium chori sine cruce cum ceroferariis et thuribulariis ad ostium occidentale excellentioribus precedentibus: precedente vexillo cilicino. Deinde executor officii penitentes singillatim per manus ejiciat per ministrationem alicujus sacerdotis de choro tradentes eos per manus dextras eidem: ipsi vero penitentes osculantes manum executoris exeant, tunc si episcopis presens fuerit archidiaconus subministret ei dicto modo, et interim cantentur hec duo responsoria cum suis versiculus: sine} Glória Patri. \emph{cantore incipiente ut patet in pictura vel in statione sequente hoc modo.}

\begin{figure}
\centering
\includegraphics{images/30r-statio-die-cinerum.jpg}
\caption{¶ Statio in die cinerum dum episcopus ejiciat penitentes ⟨tunc incipiatur istud responsorium hoc modo ut sequitur⟩.}
\end{figure}

\gregorioscore{006571-ecce-adam}

\gregorioscore{006937-in-sudore}

\emph{Ejectis penitentibus claudatur ostium ecclesie et redeundo processio more solito cantore incipiente hoc modo.}

\label{006653-emendemus-in-melius}
\gregorioscore{006653-emendemus-in-melius}

\emph{Sine} Glória Patri.

\emph{Non dicatur ℣. neque oratio sed statim incipiatur missa a cantore.}


\chapter{¶ Dominica j. xl.}

\emph{Ad processionem.}

\gregorioscore{002042-cum-venerimus}

\emph{In redeundo cantatur ℟. ⟨ut sequitur⟩.}

\gregorioscore{006529-ductus-est}

\emph{℣.} Scuto circúndabit te véritas ejus.

\emph{℟.} Non timébit ⟨a timóre noctúrno⟩.\footnote{1882:31.}

\emph{⟨℣.⟩} Orémus.

\begin{lesson}
\subsubsection{Oratio.}

\lettrine{D}{e}us qui ecclésiam tuam ánnua quadragesimáli observatióne puríficas, presta famílie tue: ut quod a te obtinére abstinéndo nítitur, hoc bonis opéribus exequátur. Per Christum Dóminum nostrum. \emph{⟨℟.⟩} Amen.
\end{lesson}


\chapter[⟨Feria quarta et sexta per totam quadragesimam.⟩]{⟨Feria quarta et sexta per totam quadragesimam.⟩\footnote{1882:32.}}

\emph{¶ Sciendum est quod per totam xl. in omni feria iiij. et vj. usque ad cenam Domini fiat processio ad unum altare ecclesie per ordinem post ix. dictam ante inchoationem misse: nisi festum ix. lectionum ibidem contigerit. Exeat itaque processio per ostium presbyterii borealis ad unum altare ex ejus latere: sacerdos vero cum suis ministris albis cum amictibus induti sine cruce choro sequente habitu non mutato cantando unum istorum responsoriorum per ordinem cantore incipiente.}

\gregorioscore{006060-afflicti-pro-peccatis}

\emph{⟨Resp.} Eméndemus in mélius. \emph{et cetera ut supra in statione cinerum.} \pageref{006653-emendemus-in-melius}.

\emph{Aliud resp.⟩\footnote{1882:32.}}

\gregorioscore{007348-paradisi-portas}

\emph{Aliud ℟.}

\gregorioscore{007626-scindite-corda}

\emph{Aliud ℟.}

\gregorioscore{006012-abscondite-elemosinam}

\emph{Finito Responsorio cum ℣. ⟨et⟩\footnote{1882:32.} absque} Glória Patri. \emph{clerici eodem modo quo in processione ordinantur prostrationem faciant. Ita quod sacerdos ad gradum altaris cum diacono a dextris et subdyacono a sinistris et ceroferariis cereis interim super altare dimissis suam faciant prostrationem cum dicitur sine\footnote{`\emph{sine}', 1530:39v; 1555; 1882:32; 1901:65 `\emph{cum}', 1519:34v.} nota.}

Kyrieléyson. Christeléyson. Kyrieléyson.

Pater noster. Brev:{[}5{]}.

\emph{Deinde dicat sacerdos cum nota ⟨sic⟩.\footnote{1882:33.}}

1519:34v.

\gregorioscore{missing}

\emph{⟨℣.⟩} Osténde nobis Dómine misericórdiam tuam.

\emph{⟨℟.⟩} Et salutáre tuum da nobis.

\emph{⟨℣.⟩} Peccávimus cum pátribus nostris.

\emph{⟨℟.⟩} Injúste égimus iniquitátem fécimus.

\emph{⟨℣.⟩} Dómine non secúndum péccata nostra fácias nobis.

\emph{⟨℟.⟩} Neque secúndum iniquitátes nostras retríbuas nobis.

\emph{⟨℣.⟩} Ne memíneris iniquitátum nostrárum antiquárum.

\emph{⟨℟.⟩} Cito antícipent nos misericórdie tue, quia páuperes facti sumus nimis.

\emph{⟨℣.⟩} Adjuva nos Deus salutáris noster.

\emph{⟨℟.⟩} Et propter glóriam nóminis tui Dómine líbera nos.

\emph{⟨℣.⟩} Exáudi Dómine vocem meam qua clamávi ad te.

\emph{⟨℟.⟩} Miserére mei et exáudi me.

\emph{Ps.} Miserére mei Deus. 59. \emph{Totus psalmus dicatur sine nota sed cum} Glória Patri. \emph{et} Sicut erat. \emph{alternando ex utraque parte chori. Quo finito solus sacerdos se erigat dicendo cum nota sic.}

\emph{⟨Versus.⟩\footnote{1882:33.}}

\begin{noinitial}

\gregorioscore{008072-exurge-domine}

\end{noinitial}

\begin{lesson}
\lettrine{P}{r}eces nostras quésumus Dómine cleménter exáudi: ut qui juste pro peccátis nostris afflígimur, pro tui nóminis glória misericórditer liberémur. Per Christum Dóminum.
\end{lesson}

\emph{Et sic surgant omnes a prostratione osculantes terram vel formulas sacerdote sic dicente} Per Christum Dóminum nostrum.

\emph{Finita oratione duo clerici de ij. forma habitu non mutato dicant letaniam in revertendo, et dicatur usque ad prolationem} Sancta María. \emph{coram altare antequam processio procedat.}

\gregorioscore{909040-kyrieleyson}

Pater de celis ⟨audi nos⟩. \emph{⟨ij.⟩}

Fili redémptor mundi Deus ⟨audi nos⟩. \emph{⟨ij.⟩}

Spíritus Sancte Deus ⟨audi nos⟩. \emph{⟨ij.⟩}

Sancta Trínitas unus Deus ⟨audi nos⟩. \emph{⟨ij.⟩}

\begin{noinitial}

\gregorioscore{f35v-sancta-maria}

\emph{Et processio presbyterorum circumeundo chorum intret: et clerici predicti prosequantur de ceteris ordinibus quantum sufficit iter, scilicet tres vel quattuor: et si tantum restat iter sub tono predicto ita quod ad gradum chori finiatur sic.}

\gregorioscore{f35v-omnes-sancte}

\end{noinitial}

\emph{Hic ordo letaniarum servetur in processionibus faciendis per totam quadragesimam scilicet in quartam feriam et sextam.}


\section[¶ Feria quarta prime ebdomade ⟨quadragesime⟩.]{¶ Feria quarta prime ebdomade ⟨quadragesime⟩.\footnote{1882.34.}}

\emph{⟨Sequitur letania.⟩\footnote{1882.34.}}

Kyrieléyson. Christeléyson. Christe audi nos.

Pater de celis Deus. Miserére nobis.

Fili Redémptor mundi Deus. Miserére nobis.

Spíritus Sancte Deus. Miserére nobis.

Sancta Trínitas unus Deus. Miserére nobis.

\begin{lesson}

Sancta María. \hfill Ora pro nobis.

Sancta Dei génitrix. \hfill ora.

Sancta virgo vírginum. \hfill ora.

Sancte Míchael. \hfill ora.

Sancte Gábriel. \hfill ora.

Sancte Ráphael. \hfill ora.

Omnes sancti angeli et archángeli. \hfill Oráte pro nobis.

Omnes sancti beatórum spirítuum órdines. \hfill oráte.

Sancte Johánnes baptísta. \hfill ora.

Omnes sancti patriárche et prophéte. \hfill Oráte pro nobis.

Sancte Petre. \hfill ora.

Sancte Paule. \hfill ora.

Sancte Andréa. \hfill ora.

Omnes sancti apóstoli et evangelíste. \hfill oráte.

Omnes sancti discípuli Dómini et innocéntes. \hfill oráte.

Sancte Stéphane. \hfill ora.

Sancte Line. \hfill ora.

Sancte Clete. \hfill ora.

Omnes sancti mártyres. \hfill oráte.

Sancte Silvéster. \hfill ora.

Sancte Leo. \hfill ora.

Sancte Hierónyme. \hfill ora.

Omnes sancti confessóres. \hfill oráte.

Omnes sancti monáchi et heremíte. \hfill oráte.

Sancta María Magdaléna. \hfill ora.

Sancta María Egyptíaca. \hfill ora.

Sancta Margaréta. \hfill ora.

Omnes sancte vírgines. \hfill oráte.

Omnes sancti. \hfill oráte.

\end{lesson}


\section{¶ Feria vj. ejusdem ebdomada.}

\begin{lesson}

Sancte Johánnes. \hfill ora.

Sancte Jacóbe. \hfill ora.

Sancte Thoma. \hfill ora.

Omnes sancti apóstoli et evangelíste. \hfill oráte.

Omnes sancti discípuli Dómini et innocéntes. \hfill orate.

Sancte Clemens. \hfill ora.

Sancte Fabiáne. \hfill ora.

Sancte Sebastiáne. \hfill ora.

Omnes sancti mártyres. \hfill orate.

Sancte Augustíne. \hfill ora.

Sancte Ysidóre. \hfill ora.

Sancte Juliáne. \hfill ora.

Omnes sancti confessóres. \hfill oráte.

Omnes sancti monácbi et heremíte. \hfill Oráte\footnote{`Ora', 1519:36r.} pro nobis.

Sancta Scolástica. \hfill ora.

Sancta Petronílla. \hfill ora.

Sancta Genovéfa. \hfill ora.

Omnes sancte vírgines. \hfill oráte.

Omnes sancti. \hfill oráte pro nobis.

\end{lesson}


\section{¶ Feria iiij. ij. ebdomade.}

\begin{lesson}

Sancte Philíppe. \hfill ora.

Sancte Jacóbe. \hfill ora.

Sancte Mathée. \hfill ora.

Omnes sancti apóstoli et evangelíste. \hfill oráte.

Omnes sancti discípuli Dómini et innocéntes. \hfill oráte.

Sancte Cosma. \hfill ora.

Sancte Damiáne. \hfill ora.

Sancte Prime. \hfill ora.

Omnes sancti mártyres. \hfill oráte.

Sancte Gildárde. \hfill ora.

Sancte Medárde. \hfill ora.

Sancte Albíne. \hfill ora.

Omnes sancti confessóres. \hfill oráte.

Omnes sancti monáchi et heremíte. \hfill Oráte pro nobis.

Sancta Praxédis. \hfill ora.

Sancta Sothéris. \hfill ora.

Sancta Prisca. \hfill ora.

Omnes sancti vírgines. \hfill oráte.

Omnes sancti. \hfill oráte.

\end{lesson}


\section{¶ Feria vj. ejusdem ebdomade.}

\begin{lesson}

Sancte Bartholomée. \hfill ora.

Sancte Symon. \hfill ora.

Sancte Thadée. \hfill ora.

Omnes sancti apóstoli et evangelíste. \hfill oráte.

Omnes sancti discípuli Dómini et innocéntes. \hfill oráte.

Sancte Feliciáne. \hfill ora.

Sancte Dionísi cum sóciis tuis. \hfill oráte.

Sancte Victor cum sóciis tuis. \hfill oráte.

Omnes sancti mártyres. \hfill oráte.

Sancte Eusébi. \hfill ora.

Sancte Swythúne. \hfill ora.

Sancte Biríne \hfill ora.

Omnes sancti confessóres. \hfill oráte.

Omnes sancti monáchi et heremíte. \hfill Oráte pro nobis.

Sancta Tecla. \hfill ora.

Sancta Affra. \hfill ora.

Sancta Edítha. \hfill ora.

Omnes sancti vírgines. \hfill oráte.

Omnes sancti. \hfill oráte.

\end{lesson}


\section{¶ Feria iiij. iij. ebdomade.}

\begin{lesson}

Sancte Mathía. \hfill ora.

Sancte Bárnaba. \hfill ora.

Sancte Marce. \hfill ora.

Omnes sancti apóstoli et evangelíste. \hfill oráte.

Omnes sancti discípuli Dómini et innocéntes. \hfill oráte.

Sancte Thoma. \hfill ora.

Sancte Cornéli. \hfill ora.

Sancte Cypriáne. \hfill ora.

Omnes sancti mártyres. \hfill oráte.

Sancte Gregóri. \hfill ora.

Sancte Augustíne. \hfill ora.

Sancte Ambrósi. \hfill ora.

Omnes sancti confessóres. \hfill oráte.

Omnes sancti monáchi et heremíte. \hfill Oráte pro nobis.

Sancta Felícitas. \hfill ora.

Sancta Perpétua. \hfill ora.

Sancta Colúmba. \hfill ora.

Omnes sancte vírgines. \hfill Oráte.

Omnes sancti. \hfill Oráte.

\end{lesson}


\section{¶ Feria vj. ejusdem ebdomade.}

\begin{lesson}

Sancte Petre. \hfill ora.

Sancte Paule. \hfill ora.

Sancte Andréa. \hfill ora.

Omnes sancti apóstoli et evangelíste. \hfill oráte.

Omnes sancti discípuli Dómini et innocéntes. \hfill oráte.

Sancte Johánnes. \hfill ora.

Sancte Kenélme. \hfill ora.

Sancte Osvuálde. \hfill ora.

Omnes sancti mártyres. \hfill oráte.

Sancte Remígi. \hfill ora.

Sancte Donatiáne. \hfill ora.

Sancte Elígi. \hfill ora.

Omnes sancti confessóres. \hfill oráte.

Omnes sancti monáchi et heremíte. \hfill oráte pro nobis.

Sancta Cristiána.\footnote{`Christína', 1519:37r.} \hfill ora.

Sancta Eulália. \hfill ora.

Sancta Eufémia. \hfill ora.

Omnes sancte vírgines. \hfill oráte.

Omnes sancti. \hfill oráte pro nobis.

\end{lesson}


\section{¶ Feria iiij. iiij. ebdomade.}

\begin{lesson}

Sancte Johánnes. \hfill ora.

Sancte Jacóbe. \hfill ora.

Sancte Thoma. \hfill ora.

Omnes sancti apóstoli et evangelíste. \hfill oráte.

Omnes sancti discípuli Dómini et innocéntes. \hfill Oráte.

Sancte Laurénti. \hfill ora.

Sancte Tybúrci. \hfill ora.

Sancte Valeriáne. \hfill ora.

Omnes sancti mártyres. \hfill oráte.

Sancte Dúnstane. \hfill ora.

Sancte Edmúnde. \hfill ora.

Sancte Leonárde. \hfill ora.

Omnes sancti confessóres. \hfill oráte.

Omnes sancti monáchi et heremíte. \hfill oráte pro nobis.

Sancta Katherína. \hfill ora.

Sancta Cecília. \hfill ora.

Sancta Anastásia. \hfill ora.

Omnes sancte vírgines. \hfill oráte.

Omnes sancti. \hfill oráte.

\end{lesson}


\section{¶ Feria vj. ejusdem ebdomade.}

\begin{lesson}

Sancte Philíppe. \hfill ora.

Sancte Jacóbe. \hfill ora.

Sancte Mathée. \hfill ora.

Omnes sancti apóstoli et evangelíste. \hfill oráte.

Omnes sancti discípuli Dómini et innocéntes. \hfill oráte.

Sancte Vincénti. \hfill ora.

Sancte Géreon cum sóciis tuis. \hfill oráte.

Sancte Mauríci cum sóciis tuis. \hfill oráte pro nobis.

Omnes sancti mártyres. \hfill oráte.

Sancte Nicoláe. \hfill ora.

Sancte Richárde. \hfill ora.

Sancte Machúte. \hfill ora.

Omnes sancti confessóres. \hfill oráte.

Omnes sancti monáchi et heremíte. \hfill oráte pro nobis.

Sancta Agnes. \hfill ora.

Sancta Juliána. \hfill ora.

Sancta Cuthbúrga. \hfill ora.

Omnes sancte vírgines. \hfill oráte.

Omnes sancti. \hfill oráte.

\end{lesson}


\section{¶ Feria iiij. v. ebdomade.}

\begin{lesson}

Sancte Bartholomée. \hfill ora.

Sancte Symon. \hfill ora.

Sancte Thadée. \hfill ora.

Omnes sancti apóstoli et evangelíste. \hfill Oráte pro nobis.

Omnes sancti discípuli Dómini et innocéntes. \hfill oráte.

Sancte Albáne. \hfill ora.

Sancte Ypólite cum sóciis tuis. \hfill oráte.

Sancte Luciáne cum sóciis tuis. \hfill oráte.\footnote{1519:37v has `ora'.}

Omnes sancti mártyres. \hfill oráte.

Sancte Sanson. \hfill ora.\footnote{1519:37v has `oráte'.}

Sancte Plácide. \hfill ora.

Sancte Columbáne. \hfill ora.

Omnes sancti confessóres. \hfill oráte.\footnote{1519:37v has `ora'.}

Omnes sancti monáchi et heremíte. \hfill Oráte pro nobis.

Sancta Felícula. \hfill ora.

Sancta Susánna. \hfill ora.

Sancta Brigída. \hfill ora.

Omnes sancte vírgines. \hfill oráte.\footnote{1519:37v has `ora'.}

Omnes sancti. \hfill oráte.

\end{lesson}


\section{¶ Feria vj. ejusdem ebdomade.}

\begin{lesson}

Sancte Mathía. \hfill ora.

Sancte Bárnaba. \hfill ora.

Sancte Marce. \hfill ora.

Omnes sancti apóstoli et evangelíste. \hfill oráte.\footnote{1519:37v has `ora'.}

Omnes sancti discípuli Dómini et innocéntes. \hfill oráte.

Sancte Gervási \hfill ora.

Sancte Prothási \hfill ora.

Sancte Timothée \hfill ora.

Omnes sancti mártyres. \hfill oráte.

Sancte Benedícte. \hfill ora.

Sancte Maure. \hfill ora.

Sancte Egídi. \hfill ora.

Omnes sancti confessóres. \hfill oráte.

Omnes sancti monáchi et heremíte. \hfill Oráte pro nobis.

Sancta Scholástica. \hfill ora.

Sancta Sabína. \hfill ora.

Sancta Justína. \hfill ora.

Omnes sancte vírgines. \hfill oráte.

Omnes sancti. \hfill oráte.

\end{lesson}


\section{¶ Feria iiij. vj. ebdomade.}

\begin{lesson}

Sancte Petre. \hfill ora.

Sancte Paule. \hfill ora.

Sancte Andréa. \hfill ora.

Omnes sancti apóstoli et evangelíste. \hfill oráte.\footnote{1519:37v has `ora'.}

Omnes sancti discípuli Dómini et innocéntes. \hfill oráte.

Sancte Quintíne. \hfill ora.

Sancte Bonifáci cum sóciis tuis. \hfill oráte.\footnote{1519:37v has `ora'.}

Sancte Kyliáne cum sóciis tuis. \hfill Ora pro nobis.

Omnes sancti mártyres. \hfill oráte.

Sancte Brici. \hfill ora.

Sancte Amánde. \hfill ora.

Sancte Cuthbérte. \hfill ora.

Omnes sancti confessóres. \hfill oráte.

Omnes sancti monáchi et heremíte. \hfill Oráte pro nobis.

Sancta Margaréta. \hfill ora.

Sancta Walbúrgis. \hfill ora.

Sancta Radegúndis. \hfill ora.

Omnes sancte vírgines. \hfill oráte.

Omnes sancti. \hfill Oráte pro nobis.

\end{lesson}

\emph{Non sequatur versiculus neque oratio: sed sacerdos cum suis ministris accedat, et statim a cantore incipiatur officium misse ⟨et precedat crux lignea absque imagine crucifixi.⟩\footnote{1882:41.}}

\emph{⟨Dominica secunda quadragesime et in omnibus dominicis per totam xl. excepta prima dominica, deferatur crux lignea ante processionem sine imagine crucifixi, quando de dominica agitur. In omnibus aliis processionibus in xl. contingentibus ordinetur ut in alio tempore.⟩\footnote{Added in 1517 edition. ⟨1882:41⟩.}}


\chapter{¶ Dominica secunda xl.}

\emph{Ad processionem ant.} Cum venérimus. \emph{ut supra.} 73.

\emph{In introitu chori dicatur hoc ℟. sequens.}

\gregorioscore{007156-minor-sum}

\emph{℣.} Scuto circúndabit te véritas ejus.

\emph{℟.} Non timébis ⟨a timóre noctúrno⟩.\footnote{1882:41.}

\emph{⟨℣.⟩} Orémus.

\begin{lesson}
\subsubsection{Oratio.}

\lettrine{D}{e}us qui cónspicis omni nos virtúte destítui intérius exteriúsque custódi: ut ab ómnibus adversitátibus muniámur in córpore, et a pravis cogitatiónibus mundémur in mente. Per ⟨Dóminum nostrum. \emph{⟨℟.⟩} Amen.⟩\footnote{1882:41.}
\end{lesson}


\chapter[¶ Dominica iij. ⟨quadragesime⟩.]{¶ Dominica iij. ⟨quadragesime⟩.\footnote{1882:41.}}

\emph{Ad processionem.}

\gregorioscore{a00236-in-die}

\emph{In introitu chori dicatur ⟨hoc responsorium sequens⟩.\footnote{1882:42.}}

\gregorioscore{007102-loquens-joseph}

\emph{℣.} Scuto circúndabit te véritas ejus.

\emph{℟.} Non timébis ⟨a timóre noctúrno⟩.\footnote{1882:42.}

\emph{⟨℣.⟩} Orémus.

\begin{lesson}
\subsubsection{Oratio.}

\lettrine{Q}{u}ésumus omnípotens Deus vota humílium réspice atque ad defensiónem nostram déxteram tue majestátis exténde. Per ⟨Christum Dóminum nostrum. \emph{⟨℟.⟩} Amen.⟩\footnote{1882:42.}
\end{lesson}


\chapter{¶ Dominica medie xl.}

\emph{Ad processionem ant.} In die quando. \emph{ut supra.} 90.

\emph{In redeundo ⟨cantatur⟩\footnote{1882:42.}}

\gregorioscore{006143-audi-israel}

\emph{℣.} Scuto circúndabit te véritas ejus.

\emph{℟.} Non timébis ⟨a timóre noctúrno⟩.\footnote{1882:42.}

\emph{⟨℣.⟩} Orémus.

\begin{lesson}
\subsubsection{Oratio.}

\lettrine{C}{o}ncéde quésumus omnípotens Deus: ut qui ex mérito nostre actiónis afflígimur: tue grátie consolatióne respirémus. Per Christum ⟨Dóminum nostrum. \emph{⟨℟.⟩} Amen⟩.\footnote{1882:42.}
\end{lesson}


\chapter{¶ Dominica in passione Domini.}

\emph{Ad processionem.}

\gregorioscore{006287-circundederunt-me}

\emph{Non dicatur} Glória Patri. \emph{Si necesse fuerit repetatur ℟. cum versu.}

\emph{In introitu chori.}

\gregorioscore{006931-in-proximo-est}

\emph{℣.} Eripe me de inimícis meis Deus meus.

\emph{℟.} Et ab insurgéntibus in me líbera me.

\emph{⟨℣.⟩} Orémus.

\begin{lesson}
\subsubsection{Oratio.}

\lettrine{Q}{u}ésumus\footnote{The breviary omits `Quésumus'.} omnípotens Deus famíliam tuam propícius réspice: ut te largiénte regátur in córpore: et te servánte custodiátur in mente. Per Christum.
\end{lesson}

