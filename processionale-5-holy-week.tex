\chapter{¶ Dominica in ramis palmarum.}

\emph{Post aque benedicte aspersionem legatur hec lectio ante gradum altaris ab accolito alba induto et hoc ex parte australi super flores et frondes cum titulo suo hoc modo.}

\begin{lesson}
\subsubsection{Lectio libri Exodi. (xv. 27.)}

\lettrine{I}{n} diébus illis. Venérunt fílii Israel in Helym: ubi erant duódecim fontes aquárum\footnote{`duódecim fontes aquárum', 1882:43.} et septuagínta palme et castrametáti\footnote{`castrameáti', 1523:41v.} sunt juxta aquas. Profectíque sunt de Helym: et venit omnis multitúdo filiórum Israel in desértum Syn: quod est inter Helym et Synai quintadécima die mensis secúndi postquam egréssi sunt de terra Egýpti et murmurávit omnis congregátio filiórum Israel contra Móysen et Aaron in solitúdine. Dixerúntque ad eos fílii Israel, Utinam mórtui essémus per manum Dómini in terra Egýpti: quando sedebámus super ollas cárnium: et comedebámus panem in saturitáte. Cur induxístis nos in desértum istud: ut occiderétis omnem multitúdinem fame? Dixit autem Dóminus ad Móysen, Ecce ego pluam vobis panes de celo. Egrediátur pópulus et cólligat que suffíciunt per síngulos dies: ut temptem ⟨eum⟩\footnote{1882:43; \emph{Vulgate.}} utrum ámbulet in lege mea, an non. Die autem sexta parent quod ínferant: et sit duplum quam collígere solébant per síngulos dies. Dixerúntque Móyses et Aaron ad omnes fílios Israel, Véspere sciétis quod Dóminus edúxerit vos de terra Egýpti: et mane vidébitis glóriam Dómini. Audívi enim murmur vestrum contra Dóminum. Nos vero quid sumus: quia musitástis contra nos? Et ait Móyses, Dabit vobis Dóminus véspere carnes édere, et mane panes in saturitáte: eo quod audívit\footnote{`audíverit', 1882:43; \emph{Vulgate.}} murmuratiónes vestras quibus murmuráti estis contra eum, Nos enim quid sumus? Nec contra nos est murmur vestrum: sed contra Dóminum. Dixítque Móyses ad Aaron, Dic univérse congregatióni filiórum Israel, Accédite coram Dómino: audívit enim murmur vestrum. Cumque enim\footnote{\textit{Vulg.} omits `enim'.} loquerétur Aaron ad omnem cetum filiórum Israel: respexérunt ad solitúdinem, et ecce glória Dómini appáruit in nube.
\end{lesson}

\emph{⟨Sintque ministri processionis albis cum amictibus induti tantum sine tunicis vel casulis: executor tamen officii in capa rubea serica, choro itaque sequente habitu non mutato.⟩\footnote{1517. 1882:44.}}

\emph{Deinde accepto ⟨statim⟩\footnote{1882:44.} benedictione legatur evangelium super lectrinum ⟨ubi leguntur evangelia a diacono⟩\footnote{1882:44; `\emph{a diacono}' in 1517. 1882:44.} ad boreale\footnote{1517---ad oriéntem. 1882:44.} converso more simplicis festo ⟨post acceptam benedictionem⟩\footnote{1882:44.} cum} Dóminus vobíscum.

\begin{lesson}
\subsubsection[⟨Evangelium⟩ secundum Johannem. (xij. 12.)]{⟨Evangelium⟩\footnote{1882:44.} secundum Johannem. (xij. 12.)}

\lettrine{I}{n} illo témpore. Turba multa que convénerat\footnote{`vénerat', \emph{Vulgate.}} ad diem festum cum audíssent quia venit Jesus Hierosólimam: accepérunt ramos palmárum et processérunt óbviam ei: et clamábant, Osánna: benedíctus qui venit in nómine Dómini Rex Israel. Et invénit Jesus aséllum et sedit super eum: sicut scriptum est, Noli timére fília Syon: ecce rex tuus venit tibi mansuétus: sedens super pullum ásine. Hec non cognovérunt discípuli ejus primum: sed quando glorificátus est Jesus: tunc recordáti sunt quia hec erant scripta de eo et hec fecérunt ei. Testimónium\footnote{`Testiónium', 1519:52r.} ergo perhibébat turba que erat cum eo quando Lázarum vocávit de monuménto: et suscitávit eum a mórtuis. Proptérea et óbviam venit ei turba: quia audiérant eum fecísse hoc signum. Phariséi ergo dixérunt ad semetípsos, Vidétis quia nichil profícimus? Ecce mundus totus post eum ábiit.
\end{lesson}

\emph{Finito evangelio statim executor officii in gradu iij. ad altare positis prius palmis\footnote{`\emph{ramis}', 1882:45.} cum frondibus coram altare pro clericis pro aliis vero super gradum predictum in parte australi.}

\begin{figure}
\centering
\includegraphics{images/42r-statio-rami.jpg}
\caption[⟨Statio dum benedicuntur rami in dominica ramis palmarum.⟩]{⟨Statio dum benedicuntur rami in dominica ramis palmarum.⟩\footnote{1882:45.}}
\end{figure}

\emph{¶ Sequatur benedictio florum et frondium a sacerdote induto cappa serica rubea super gradum altaris ad austrum\footnote{1517---orientem. 1882:45.} converso ita dicens.}

\begin{lesson}
\lettrine{E}{x}orcízo te, creatúra florum vel fróndium in nómine Dei ✠ Patris omnipoténtis et in nómine Jesu Christi ✠ Fílii ejus Dómini nostri: et in virtúte ✠ Spíritus Sancti: próinde omnis virtus adversérii: omnis exércitus diáboli: omnis potéstas inimíci: omnis incúrsio démonum eradicáre et explantáre ab hac creatúra florum vel fróndium ut ad Dei grátiam festinántium vestígia \emph{Que terminetur sic.}
\end{lesson}

\begin{noinitial}

\gregorioscore{f42v-non-sequaris}

\end{noinitial}

\emph{¶ Deinde dicantur omnes orationes sine} Dóminus vobíscum. \emph{sed tantum cum} Orémus.

\begin{lesson}
\subsubsection{Oratio.}

\lettrine{O}{m}nípotens sempitérne Deus qui in dilúvii effusióne Noe fámulo tuo per os colúmbe gestántis ramum ólive pacem terris rédditam nunciásti, te súpplices deprecámur: ut hanc creatúram florum spatulásque palmárum seu frondes árborum quas ante conspéctum glórie tue offérimus verítas tua sanctí✠ficet, ut devótus pópulus in mánibus eas suscípiens\footnote{`suscípiens eas', 1882:46.} bene✠dictiónis tue grátiam cónsequi mereátur.
\end{lesson}

\emph{Et dicantur et terminentur omnes orationes sub tono isto legendo hoc modo.}

\begin{noinitial}

\gregorioscore{f42v-per-christum}

\end{noinitial}

\emph{Et omnes orationes dicantur cum} Orémus. \emph{sub tono supradicto.}

\begin{lesson}
\subsubsection{Oratio.}

\lettrine{D}{e}us cujus Fílius pro salúte géneris humáni de celo descéndit ad terras: et appropinquánte hora passiónis sue Hierosólimam in ásino sedens veníre et a turbis rex appellári et laudári vóluit: auge fidem in te sperántium: et súpplicum preces cleménter exáudi: véniat super nos quésumus Dómine misericórdia\footnote{`benedíctio', 1882:46.} tua: et hos palmárum ceterarúmque árborum ramos bene✠dícere dignáre: ut omnes qui eos latúri sunt, benedictiónis tue dono repleántur, concéde ergo ut sicut Hebreórum púeri osánna in excélsis clamántes eídem Fílio tuo Dómino nostro cum ramis palmárum occurrérunt nos ita árborum ramos gestántes cum bonis opéribus occurrámus óbviam Christo et perveniámus ad gáudium sempitérnum. Per eúndem Christum Dóminum nostrum.

⟨Orémus.

\subsubsection[Oratio.⟩]{Oratio.⟩\footnote{1882:46.}}

\lettrine{D}{e}us qui dispérsa cóngregas, et congregáta consérvas qui pópulis óbviam Christo Jesu ramos palmárum gestántibus\footnote{`portántibus', 1882:46.} benedixísti: béne✠dic\footnote{The ✠ appears in 1882:46, but not in 1519:43r.} ⟨étiam⟩\footnote{1882:46.} et hos ramos palmárum ceterarúmque árborum quos tui fámuli ad nóminis tui benedictiónem fidéliter suscípiunt: ut in quemcúnque locum introdúcti fúerint, tuam benedictiónem habitatóres illíus loci omnes consequántur: ita ut omni advérsa valitúdine effugáta: déxtera tua prótegat quos rédemit. Per eúndem Christum\footnote{1882:47 omits `Christum'.} Dóminum nostrum.
\end{lesson}

\emph{Hic asrpergatur aqua benedicta super flores et frondes et thurificentur. Deinde dicatur} Dóminus vobíscum. \emph{⟨cum⟩\footnote{1882:47.}} Orémus.

\begin{lesson}
\subsubsection{Oratio.}

\lettrine{D}{o}mine Jesu Christe mundi Cónditor et Redémptor: qui nostre liberatiónis et sanatiónis grátia ex summa celi arce descéndere, et carnem súmere, et passiónem subíre dignátus es: quique sponte própria loco ejúsdem propínquans passiónis turbis\footnote{`passiónis appropínquans a turbis', 1882:47.} cum ramis palmárum obviántibus benedíci, laudári, et rex benedíctus in nómine Dómini véniens clara voce appellári voluísti: tu nunc nostre confessiónis laudatiónem acceptáre, et hos palmárum ceterarúmque árborum ac florum ramos bene✠dícere et sancti✠ficáre dignéris: ut quicúnque ⟨in tue⟩\footnote{1882:47.} virtútis obséquio exínde áliquid túlerit celésti benedictióne sanctificátus peccatórum remissiónem et vite etérne prémia percípere mereátur. Per te Jesu Christe salvátor mundi.\footnote{1882:47 omits `mundi'.} Qui cum Patre et Spíritu Sancto vivis et regnas Deus. Per ómnia sécula seculórum. \emph{⟨℟.⟩\footnote{1882:47.}} Amen.
\end{lesson}

\emph{¶ His itaque peractis statim distribuantur palme, et interim cantentur hee sequentes antiphone cantore incipiente antiphonam hoc modo.}

\gregorioscore{004415-pueri-hebreorum-tollentes}

\emph{⟨Sequitur⟩\footnote{1882:47.} alius cantus.}

\gregorioscore{004416-pueri-hebreorum-vestimenta}

\emph{¶ Dum distribuuntur rami benedicti preparetur feretrum cum reliquiis in quo corpus Christi dependeat: et ad locum prime stationis a duobus clericis de secunda forma non tamen processionem sequendo sed ad locum prime stationis obviam veniendo habitu non mutato deferatur lumen in lanterna precedente, cum cruce et duobus vexillis precedentibus: deinde exeat processio ad locum prime stationis sintque ministri processionis albis cum amictibus ahsque tunicis vel casulis. Ita tamen quod sacerdos eat in cappa serica rubea ut supra, choro itaque sequente habitu non mutato. Si tamen assit episcopus mitram et baculum in fine proccessionis.\footnote{`\emph{Deinde per medium chori et ecclesie exeat processio per ostium chori occidentale usque ad locum prime stationis, videlicet ante crucem septentrionali cimiterio, precedente cruce sine imagine, ut in aliis dominicis quadragesime, episcopo vel excellentiore sacerdote exsequente officium processionis, sintque ministri albis cum amictibus induti absque tunicis vel casulis: ita tamen quod exsecutor officii processionis eat in capa serica rubea, ckoro itaque sequente habitu non mutato. Si tamen assit episcopus, mitram gerat et baculum in fine proccessionis.}' 1882:47.}}

\emph{In eundo ⟨cantor⟩\footnote{1882:48.} dicitur hec antiphonam ⟨sequentem⟩.\footnote{1882:48.}}

\gregorioscore{203956-prima-autem-azimorum}

\emph{Unde\footnote{1517---Unde cum hec, \&c.~1544, \&c.---Cum autem. ⟨1882:48.⟩} hec antiphona canitur cum sequentibus, exeat processio per ostium occidentale: et sic circa claustrum, et ita per portam canonicorum usque ad locum prime stationis que fit ex parte ecclesie borealis, in extrema parte orientalis tunc laicorum\footnote{`should run--- \emph{in extrema parte orientali cimiterii laicorum.}' ⟨1882:xv.⟩} legetur evangelium.\footnote{\emph{⟨cimiterii laicorum, ubi legetur evangelium modo quo sequitur.⟩} So 1517; but 1508, \&c.---in extrema parte orientalis. Tunc laicorum legetur evangelium. 1528---leaf wanting. ⟨1882:48.⟩}}

\gregorioscore{001976-cum-appropinquaret}

\emph{Si ⟨autem⟩\footnote{1882:48.} non sufficiant hee due antiphone usque ad locum prime stationis: tunc dicantur hee sequentes antiphone.}

\gregorioscore{001983-cum-audisset}

\emph{⟨Sequitur⟩ alia antiphona.}

\gregorioscore{001437-ante-sollennitatis}

\emph{⟨Item⟩\footnote{1882:49.} alia antiphona.}

\gregorioscore{001437-ante-passionis}

\emph{¶ Hic fiat prima statio ex parte ecclesie boreali in extrema parte orientalis: et legatur evangelium} Cum appropinquásset Jesus. \emph{ab ipso diacono induto ad processionem non juxta crucem sed ante sacerdotem loco parumper secus stationem illius diaconi mutato, ab ipso diacono ad boreale\footnote{1517---oriéntem. ⟨1882:49.⟩} converso. Et legatur more simplicis festi ix. lectionum post acceptam benedictionem cum} Dóminus vobíscum. \emph{et} Glória tibi Dómine.

\emph{Et tunc feretrum cum reliquiis preparatum in quo Christus\footnote{1511---Christus. ⟨1882:49.⟩} in pixide dependeat, obviam veniendo ad locum prime stationis, a duobus clericis de secunda forma deferatur habitu non mutato, ut supra: et in fine evangelii ad hec verba} Benedíctus qui venit in nómine Dómini. \emph{primo se ostendat.}

\begin{lesson}
\subsubsection{Evangelium secundum Mattheum. xxj. (1--9.)}

\lettrine{I}{n} illo témpore. Cum appropinquásset Jesus Hierosólymam, et venísset Bethpháge ad montem Olivéti: tunc misit duos discípulos dicens eis, Ite in castéllam quod contra vos est: et statim inveniétis ásinam alligátam, et pullum cum ea. Sólvite: et addúcite michi. Et si quis vobis áliquid díxerit, dícite quia Dóminus his opus habet: et conféstim dimíttet eos. Hoc autem totum factum est: ut adimplerétur quod dictum est per prophétam dicéntem, Dícite fílie Syon, Ecce rex tuus venit tibi mansuétus: sedens super ásinam, et pullum fílium subjugális. Eúntes autem discípuli: fecérunt sicut precépit illis Jesus. Et adduxérunt ásinam et pullum, et imposuérunt super eos vestiménta sua: et eum désuper sédere fecérunt. Plúrima autem turba: stravérunt vestiménta sua in via. Alii autem cedébant ramos de arbóribus: et sternébant in via. Turbe autem que precedébant et subsequebántur: clamábant dicéntes, Osánna fílio David: Benedíctus qui venit in nómine Dómini.
\end{lesson}

\emph{¶ Finito evangelio ⟨unus puer ad modum prophete indutus stans in aliquo eminenti loco cantet lectionem propheticam modo quo sequitur.}

\gregorioscore{003481-hierusalem-respice}

\emph{Tres clerici de secunda forma habitu non mutato exeuntes ab eadem processione conversi ad populum stantes ante magnam crucem ex parte occidentali simul cantent antiphonam\footnote{`\emph{hunc versum}', 1882:50.} hoc modo.}

\gregorioscore{a01485-en-rex-venit}

\emph{Post singulos\footnote{'\emph{singulas}', 1519:47v.} ℣. executor officii incipiat antiphonam} Salve. \emph{conversus ad reliquias quam prosequatur chorus cum genuflexione osculando terram ab ipso quoque executore officii primo cum flectione. Senior genuflectendo dicat sic.\footnote{`\emph{primo cum choro genufiectendo, dicatur sic}', 1882:50.}}

\gregorioscore{a01485-salve-quem-jesum}

\emph{Chorus in prostratione deosculando terram prosequatur resurgendo.}

\gregorioscore{a01485-testatur-plebs}

\emph{⟨Item propheta cantet hoc quod sequitur.}

\gregorioscore{000000-ecce-salvator-venit}\footnote{1555, unpaged.}

\emph{Iterum clerici stantes ante reliquias ⟨simul⟩\footnote{1882:51.} loco nec habitu mutato dicant ⟨hunc⟩\footnote{1882:51.} ℣. ⟨sequentem⟩.\footnote{1882:51.}}

\gregorioscore{a01485-hic-est-qui}

\emph{¶ Item senior\footnote{`\emph{executor officii}', 1882:51.} loco nec habitu mutato incipiat ⟨antiphonam⟩.\footnote{1882:51.}}

\gregorioscore{a01485-salve-lux-mundi}

\emph{Chorus resurgendo prosequatur.}

\begin{noinitial}

\gregorioscore{a01485-rex-regum}

\end{noinitial}

\emph{⟨Iterum propheta.}

\gregorioscore{000000-ecce-appropinquabit}

\emph{Postea evanescat.⟩\footnote{1882:51.}}

\emph{Item clerici ante reliquias loco nec habitu mutato ⟨dicant hunc⟩\footnote{1882:51.} ℣.\footnote{In 1882:51 the ℣. concludes `cecinerunt prophetice'.}}

\gregorioscore{a01485-hic-est-ille}

\emph{Item senior\footnote{`\emph{executor officii}', 1882:51.} loco nec habitu mutato ⟨dicat antiphonam sequentem⟩.\footnote{1882:51.}}

\gregorioscore{a01485-salve-nostra}

\emph{Chorus resurgendo prosequatur.}

\gregorioscore{a01485-pax-vera}

\footnote{1517---Tunc ministri in prima processione, videlicet accolitus cum cruce et ceroferarii procedentes aliis ministris cum feretro se associent; et statim reportata prima cruce illa, videlicet sine imagine, predictus accolitus crucem argenteam accipiat deferendam. Et tunc procedat processio usque ad locum secunde stationis. ⟨1882:51.⟩}

\emph{Deinde eat processio ad locum secunde stationis et feretrum cum capsula reliquiarum pariter cum lumine in lanternam inter subdiaconum et thuribularium deferatur cum vexillis ex utraque parte, cantore incipiente antiphonam ⟨sequentem⟩.\footnote{1882:51.}}

\gregorioscore{201244-dignus-est}

\emph{Alia antiphona.}

\gregorioscore{004107-occurrunt-turbe}

\emph{Si non sufficiunt hec due antiphone usque ad locum ij. stationis, tunc cantentur hec duo ℟. vel unum eorum cantore incipiente.}

\gregorioscore{600630-dominus-jesus}

\emph{⟨Item sequitur⟩\footnote{1882:52.} aliud ℟.}

\gregorioscore{600374-cogitaverunt}

\emph{Hic fiat secunda statio ex parte ecclesica australi: ubi septem pueri ⟨in⟩\footnote{1882:52.} eminentiores ⟨loco⟩\footnote{1882:52.} simul cantent ⟨hanc antiphonam⟩\footnote{1882:52.} ⟨hoc modo⟩.}

\gregorioscore{008310-gloria-laus}

\emph{Chorus idem repetat post unumquemque versum.}

\emph{Pueri vero dicant versum sequens.}

\gregorioscore{008310a-israel-es-tu}

\emph{Chorus idem ⟨repetat⟩\footnote{1882:52.}} Glória laus.

\emph{Septem pueri versus.}

\gregorioscore{008310b-cetus-in-excelsis}

\emph{Chorus idem ⟨repetat⟩\footnote{1882:52.}} Glória laus.

\emph{Septem pueri versus.}

\gregorioscore{008310c-plebs-hebrea}

\emph{¶ Item chorus repetatur} Glória laus et honór.

\emph{Peracta hac\footnote{`\emph{autem}', 1882:52.} statione eat processio per medium claustri a dextera manu usque ad hostium ecclesie occidentale, cantando ⟨hanc⟩ antiphonam ⟨sequentem⟩ cantore incipiente hoc modo.}

\gregorioscore{001852-collegerunt}

\emph{Hic fiat tertia statio ante predictum ostium ecclesie occidentale, ubi tres clerici de superiori gradu in ipso habitu non mutato\footnote{`For \textit{in ipso habitu non mutato,} read \textit{in ipso} ostio, \textit{habitu non mutato}'. 1882:xv.} conversi ad populum simul ⟨incipiant et⟩\footnote{1882:53.} cantent ⟨hunc⟩ ⟨sequentem⟩\footnote{1882:53.} ℣. ⟨hoc modo⟩ ⟨quo sequitur⟩.\footnote{1882:53.}}

\gregorioscore{001852a-unus-autem}

\emph{His finitis intrent ecclesiam per idem ostium sub feretro et capsula\footnote{`\emph{casula}', 1519:51v.} reliquiarum ex transverso ostii elevatis cantando cantore incipiente.}

\gregorioscore{006961-ingrediente}

\emph{Hic fiat quarta statio ante crucem in ecclesia et in ipsa statione executor officii incipiet antiphonam ⟨sequentem⟩.\footnote{1882:53.}}

\gregorioscore{001543-ave-rex-noster-1}

\emph{Cruce jam discooperta, et respondeat chorus cum genuflectione osculando terram.}

\begin{noinitial}

\gregorioscore{001543-ave-rex-noster-2}

\emph{Et sic incipiat sacerdos antiphonam ter: singulis vicibus vocem exaltando et una cum choro fiat genuflexio et post tertiam reinceptionem chorus eadem antiphonam in statione stando totam prosequatur.}

\emph{Item sacerdos vocem exaltando hoc modo incipiat ⟨antiphonam⟩.\footnote{1882:53.}}

\gregorioscore{001543-ave-rex-noster-3}

\emph{Chorus ⟨cum genuflexione osculando terram respondeat⟩\footnote{1882:53.}} Rex noster.

\emph{⟨Item sacerdos⟩\footnote{1882:53.} se modo altissimum.\footnote{1508---semen altisisimum. 1517, 1523, \&c.---modo altissim̄. 1528, 1532---et modo altissimo. 1544, \&c.---modo altissimo, probably from MS, sē. m̄. altissim̄., i.e. senior modo altissimo, \emph{i.e.,} exsecutor officii, as above, p.~51, notes. ⟨1882:53.⟩ `se', 1517:52r.}}

\gregorioscore{001543-ave-rex-noster-3}

\end{noinitial}

\emph{Chorus in eadem statione totam prosequatur antiphonam sic.}

\gregorioscore{001543-ave-rex-noster-4}

\emph{Qua finita intrent chorum et omnes cruces per ecclesiam sint discooperte usque post vesperas.}

\emph{In introitu chori ℟.} Circumdedérunt me. \emph{ut supra in dominica in passione ⟨Domini⟩.\footnote{1882:54.}} 94.

\emph{℣.} Eripe me de inimícis meis Deus meus.

\emph{℟.} Et ab insurgéntibus in me líbera me.

\emph{⟨℣.⟩} Orémus.

\begin{lesson}
\subsubsection[Oratio.⟩]{Oratio.⟩\footnote{1882:54.}}

\lettrine{O}{m}nípotens sempitérne Deus qui humáno géneri ad imitándum humilitátis exémplum: Salvatórem nostrum carnem súmere, et crucem subíre fecísti: concéde propítius, ut et patiéntie ipsíus habére documénta, et resurrectiónis consórtia mereámur. Per Christum Dóminum nostrum. \emph{⟨℟.⟩} Amen.
\end{lesson}


\chapter{¶ Feria v. in cena Domini.}

\emph{In primis\footnote{`\emph{Imprimis}', 1882:54.} fiat reconsiliatio penitentium hoc modo. ix. cantata pergat episcopus vel ejus vicarius excellentior sacerdos ad ostium ecclesie occidentale, indutus vestibus sacerdotalibus in cappa serica rubea cum duobus dyaconis cum amictibus indutis absque subdyacono et sine cruce per medium chori precedente vexillo cilicino. Sint\footnote{`\emph{Sintque}', 1882:54.} presentes in atrio ecclesie qui reconciliandi sunt, et si episcopus adest principalis archidyaconus ex parte penitentium scilicet extra ostium ecclesie in cappa serica legat hanc lectionem.} Adest tempus o veneránde. \emph{que non legatur episcopo absente.}

\begin{lesson}
\subsubsection{Lectio.}

\lettrine{A}{d}est tempus o veneránde póntifex, votívum afflíctis: cóngruum peniténtibus, optábile tribulátis. Adsunt fílii tui pater, quos Deo per Spíritum Sanctum vera mater ecclésia cum letícia péperit. Sed íterum suadénte diábolo a sua integritáte corrúptos aut míseros factos aut éxules: novis quotídie dolóribus ingemíscit. Pro hiis supplíciter orant quicúnque felíces in sinu suo remansérunt, quique divína protegénte se cleméntia fide stábiles prestitérunt. Parce hódie pater et tota bonitátis tue ímpetu fluat nobis fons ille David ⟨patens⟩\footnote{1882:54.} in ablutióne menstruáte: ⟨muliéris⟩\footnote{1544, \&c.---menstruáti panni. ⟨1882:54.⟩ `muliéris' not in 1519:53r.} nulli impróperans: nullum rejíciens nullum exclúdens. Quamvis enim a divínis supérne pietátis nichil témporis vacet: nunc tamen lárgior est per indulgéntiam remíssio peccatórum et copiósior per grátiam assúmptio renascéntium. Accepísti a Dómino ánnulum discretiónis et honóris: ut que signánda sunt signes, et que aperiénda sunt prodas: ⟨et que ligánda sunt liges: et que solvénda sunt solvas⟩:\footnote{1882:55.} atque credéntibus fidem baptismátis: lapsis autem et peniténtibus per mystérium reconciliatiónis jánuas regni celéstis apérias. Lavant aque lavant láchryme. María flevit et suscépta est. Negávit apóstolus: et quia flevit amaríssime restitútus est. Publicánus de thelóneo apostolátum méruit, latro in\footnote{`de', 1882:55.} cruce: paradísum intrávit, suscéptus a patre clementíssimo fílius ejus\footnote{`illi', 1882:55.} júnior: qui támdiu porcos pavit, qui patérne hereditátis preciósa stipéndia infelíciter dissipávit. Inde est étiam quod hódie tui súpplices póstea quam in várias formas críminum negléctum mandatórum celéstium et morum probabílium transgressióne cecidérunt humiliáti atque\footnote{`et', 1882:55.} prostráti prophética voce clamant dicéntes, Peccávimus cum pátribus nostris: injuste\footnote{`iníque', 1882:55.} égimus iniquitátem fécimus. Manducavérunt sicut scriptum est: panem dolóris láchrymis stratum rigavérunt, cor suum luctu, corpus suum afflixérunt jejúniis, ut animárum recíperent quam perdíderant sanitátem, plorat cum istis immo pro istis ploráre et oráre non desístit mater sancta ecclésia, et láchryme in maxíllis ejus quia venit tempus miseréndi ejus. Móveat pietátem tuam Pater vox fidélis et flébilis móveat gémitus et hábitus ipse miserórum.
\end{lesson}

\emph{¶ Finita lectione incipiat episcopus vel executor officii antiphonam hoc modo.}

\gregorioscore{205144-venite-1}

\emph{Scilicet infra predictum ostium conversus ad borealem signum faciendo cum manu dextra ad penitentes quasi vocando. Deinde ex parte penitentium scilicet extra ostium\footnote{`\emph{hostium}', 1519:53v.} dicat dyaconus hoc modo, quo sequitur ex parte episcopi.}

\begin{noinitial}

\gregorioscore{205144-venite-2}

\emph{Alius diaconus ex parte episcopi dicat\footnote{`\emph{respondeat}', 1882:55.} sic.}

\gregorioscore{205144-venite-3}

\end{noinitial}

\emph{Et ita respondeat dyaconus. ⟨et⟩\footnote{1882:55.} Fiat tribus vicibus ita tamen quod post tertiam repetitionem antiphone non dicatur} Flectámus génua. \emph{sed chorus prosequatur totam antiphonam cantore incipiente hoc modo sequenter.\footnote{1882:55 begins `Veníte fílii audíte'.}}

\gregorioscore{205144-venite-4}

\emph{¶ Totus psalmus dicatur sine} Glória Patri. \emph{et post unumquemque versum repetatur antiphona} Veníte. \emph{Dum psalmus canitur a toto choro cum antiphona: semper manuatim penitentes a presbyteris\footnote{`\emph{plebesanis}', Munich, Bayerische Staatsbibliothek, Clm 3909, fo. 220r. cited in Sarah Hamilton, \emph{The Practice of Penitence, 900--1050} (Woodbridge, Suffolk: Boydell and Brewer, 2001), p.~120. i.e.~by the people.} archidyacono et ab archidyacono reddantur episcopo: et ab episcopo restituantur ecclesie gremio. Tamen si episcopus presens non fuerit tunc penitentes manuatim ab aliquo presbytero de choro habitu non mutato reddantur executori officii: et ab ipso restituantur ecclesie gremio.}

\emph{Quibus expletis processio more solito in chorum redeat. Deinde prosternant se omnes clerici in choro: et dicant septem psalmos penitentiales. Sacerdos vero cum suis ministris per se ante altare dicat predictos septem psalmos cum} Glória Patri. \emph{et} Sicut erat. \emph{et antiphonam} Ne reminiscáris. \emph{psalmi et antiphone ut supra in quarta feria in capite jejunii.} \pageref{chap:capite-jejunii}. \emph{⟨Qua⟩\footnote{1882:56.} finita antiphona dicitur} Kyrieléyson. Christeléyson. Kyrieléyson. Pater noster. \emph{Et hec omnia sine nota dicantur tam a sacerdote cum suis ministris quam a toto choro.}

\emph{Deinde erigat se sacerdos cum suis ministris et dicat super populum conversus ad australe\footnote{1517---ad orientem. ⟨1882:56.⟩} coram dextro cornu altaris.} Et ne nos. \emph{⟨℟.⟩} ⟨Sed líbera.⟩\footnote{1882:56.}

\emph{℣.} Salvos fac servos tuos et ancíllas tuas. \emph{℟.} Deus meus sperántes in te.

\emph{℣.} Convértere Dómine úsquequo. \emph{℟.} Et deprecábilis esto super servos tuos.

\emph{℣.} Mitte eis Dómine auxílium de sancto. \emph{℟.} Et de Syon tuére eos.

\emph{℣.} Adjuva nos Deus salutáris noster. \emph{℟.} Et propter glóriam nóminis tui Dómine líbera nos: et propítius esto peccátis nostris propter nomen tuum.

\begin{noinitial}

\gregorioscore{f54r-domine-exaudi}

\end{noinitial}

\begin{lesson}
\lettrine{A}{d}ésto Dómine supplicatiónibus nostris et me qui étiam misericórdia tua primus indígeo cleménter exáudi: et quem non electióne mériti, sed dono grátie tue constituísti hujus óperis\footnote{`óperis hujus', 1882:56.} minístrum: da fidúciam tui múneris exequéndi: et ⟨tu⟩\footnote{1882:56.} ipse in nostro ministério quod tue pietátis\footnote{`pietátis tue', 1882:56.} est
\end{lesson}

\begin{noinitial}

\gregorioscore{f54v-operare-per-christum}

\end{noinitial}

\emph{Et omnes\footnote{`\emph{omnino}', 1882:57.} orationes dicantur cum} Orémus.\footnote{In Gough Missals 75 a blank space appears where `Orémus' is omitted.} \emph{et finiantur suh tono supradicto. Non dicatur} Dóminus vobscum. \emph{nisi ante primam orationem.}

\begin{lesson}
\subsubsection{Oratio.}

\lettrine{D}{e}us humáni géneris benigníssime Cónditor et misericordíssime Reformátor qui hóminem invídia diáboli ab eternitáte dejéctum: únici Fílii tui sánguine redemísti, vivífica hos fámulos ⟨tuos⟩\footnote{1882:57.} quos tibi nullátenus mori desíderas et qui non derelínquis dévios, assúme corréctos. Móveant pietátem tuam quésumus Dómine horum famulórum tuórum lachrymósa suspíria tuórum\footnote{`Tu eórum', 1882:57.} médere vulnéribus tu jacéntibus manum pórrige salutárem ne ecclésia tua áliqua sui córporis portióne vastétur: ne grex tuus detriméntum sustíneat: ne de famílie tue damno inimícus exúltet: ne renátos laváchro salutári mors secúnda possídeat. Tibi ergo Dómine súpplices preces offérimus: tibi cordis fletum effúndimus tu parce confiténtibus: ut sic in hac mortalitáte peccáta sua te adjuvánte défleant: quátenus in treméndi judícii die senténtiam damnatiónis evádant: et nésciant ⟨piíssime Pater⟩\footnote{1882:57.} quod terret in ténebris, quod stridet in flammis, atque ab erróris sui via, ad iter revérsi justície nequáquam ultra vulnéribus sauciéntur: sed intégrum sit eis atque perpétuum, et quod grátia tua cóntulit: et quod misericórdia tua reformávit. Per eúndem Christum Dómnum nostrum. \emph{⟨℟.⟩} Amen.

\emph{Dicatur} Orémus.

\subsubsection{Alia oratio.}

\lettrine{D}{o}mine sancte Pater omnípotens etérne Deus qui vúlnera nostra curáre dignátus es te súpplices exorámus et pétimus nos húmiles tui sacerdótes ut précibus nostris aures\footnote{`aurem', 1882:57.} tue pietátis inclináre dignéris: atque ad peniténtiam confessióne moveáris, remittéque ómnia crímina et peccáta univérsa condónes: desque his fámulis tuis Dómine pro supplíciis véniam, pro meróre letítiam, pro morte vitam: ut qui ad tantam spem celéstis ápicis devolúti sunt de tua misericórdia confidéntes, ad bona pacíferi\footnote{`pacífici', 1882.57.} prémii tui atque ad celéstia dona perveníre mereántur. Per Christum Dóminum nostrum.
\end{lesson}

\emph{Hic non dicatur} ⟨Dóminus vobíscum. \emph{necque⟩\footnote{1882:57.}} Orémus. \emph{⟨sed vertat se sacerdos ad populum, et elevata manu sua dextera sine nota prosequatur in audientia hanc absolutionem sequentem.⟩\footnote{1882:57.}}

\begin{lesson}
\subsubsection{Absolutio.}

\lettrine{A}{b}sólvimus vos vice beáti Petri apostolórum príncipis cui Dóminus dedit potestátem\footnote{`cui colláta est a Dómino potéstas', 1882:58.} ligándi atque solvéndi: et quantum ad vos pértinet accusátio: et ad nos remíssio, sit vobis omnípotens Deus vita et salus et ómnium peccatórum vestrórum pius indúltor. Qui vivit et regnat cum Deo Patre in unitáte Spíritus Sancti Deus. Per ómnia sécula seculórum. \emph{⟨℟.⟩} Amen.
\end{lesson}

\emph{Et sic surgant omnes a prostratione osculantes terram vel formulas sacerdote dicente} Qui vivit et regnat ⟨cum Deo⟩.\footnote{1882:58.}

\emph{⟨Si episcopus non celebraverit, sacerdos celebraturus post absolutionem penitentium casulam suam ad altare autenticum induat.⟩\footnote{1882:58.} Tamen si episcopus presens fuerit fiat benedictio super populum adhuc prostratione hoc modo.}

\begin{lesson}
\lettrine{B}{e}ne✠díctio Dei Patris omnipoténtis et Fílii et Spíritus Sancti descéndat super vos et máneat semper. \emph{Chorus.} Amen.
\end{lesson}

\emph{¶ Deinde incipiatur missa sollenniter a cantore incipiente sine regimine chori.}

\emph{Officium.} Nos autem ⟨gloriári⟩.\footnote{1882:58.} Missal:641. \emph{Et cum sollenne} Kýrieléyson. \emph{videlicet} Cónditor Kýrie.\footnote{\begin{verse}
  Cónditor, Kýrie, ómnium ymas creaturárum eléison.\\*
  Tu nostra delens crímina, nobis incessánter eléison.\\*
  Ne sinas períre factúram tuam, sed clemens et eléison.\\*
  Christe, Patris únice, natus de vírgine, nobis eléison.\\*
  Mundum pérditum qui tuo sánguine salvásti de morte, eléison.\\*
  Ad te nunc clamántum preces exáudiens, pius eléison.\\*
  Spíritus alme, tua nos reple grátia, eléison.\\*
  A Patre et Natu qui mansa júgiter, nobis eléison.\\*
  Trínitas una, trina Unitas, simul adoránda,\\*
  Nostrórum scélerum vincla resólve rédimens a morte.\\*
  Omnes proclamémus nunc voce dulcíflua, Deus, eléison.
  \end{verse}

  (See reprint of the Hereford Missal, p.~xxxviii--xlii. for this and other Kyries.) ⟨1882:58.⟩} \emph{absque ℣.} Missal:18*. \emph{sive episcopus celebraverit sive non.}

\emph{Tres pueri ante olei consecrationem in superpelliceis ad gradum chori simul cantent hunc hymnum.}

\gregorioscore{ah51077-o-redemptor-1}

\emph{Chorus idem repetat post unumquemque versum. Clerici prosequantur.}

\begin{noinitial}
	
\gregorioscore{ah51077-o-redemptor-2}

\emph{⟨Hic pergat episcopus ad altare. Chorus interim repetat} O redémptor.⟩\footnote{1882:59.}

\gregorioscore{ah51077-o-redemptor-3}

\end{noinitial}

\emph{⟨Hac die post} Per ómnia sécula seculórum. \emph{et post} Pax Dómini. \emph{dicta} Et cum spíritu tuo. \emph{more solito, non dicatur} Agnus Dei. \emph{nec} Pax Dei. \emph{nisi episcopus cebraverit: tunc enim dicitur} Agnus Dei. \emph{sollenniter, sicut in festis minoribus duplicibus: et sacerdos chrismatis ampullam pro pace deosculetur, et diaconus et subdiaconus dalmatica et tunica induantur propter sollennitatem cene, licet episcopus celebraverit.⟩\footnote{1882:59.}}

¶ \emph{⟨Item⟩ post prandium ⟨omnes sacerdotes atque clerici⟩ conveniant ad ecclesiam clerici ad altaria abluenda, et ad mandatum faciendum et ad completorium dicendum.}

\emph{⟨Item⟩ in primis\footnote{`\emph{Imprimis}', 1882:59.} benedicatur aqua more dominicali: extra chorum privatim scilicet in vestibulo ante altare. Deinde preparantur duo sacerdotes de excellentioribus cum dyaconis et subdyaconis\footnote{1528, \&c.---cum duobus diaconis et subdiaconis. ⟨1882:59.⟩} de secunda forma et ceroferariis de prima forma, qui omnes sint albis cum amictibus indutis ⟨absque paruris⟩,\footnote{1882:59.} et duo clerici\footnote{`\emph{et cum duobus pueris in simili habitu}', 1882:59.} vinum et aquam deferentes ⟨majoribus precedentibus et a majore altari⟩\footnote{1882:59.} incipientes abluant illud infundentes vinum et aquam ⟨super cruces in medio et in cornibus ejusdem. Postea ministri infundant aquam⟩\footnote{1882:60.} et interim cantetur responsorium sequens hoc modo.}

\emph{⟨Si episcopus presens fuerit mitram gerat et baculum in lotione altarium et similiter ad mandatum, lotisque pedibus canonicorum, postquam pedes laverint, executor ipse episcopus ultimo lavetur.⟩\footnote{1882:60.}}

% TODO: check variants with old PDF
\gregorioscore{006916-in-monte-oliveti}

\emph{Et dicatur a toto choro coram altari cum suo ℣. a cantore incipiente hoc modo.}

\gregorioscore{006916a-veruntamen}

\emph{Et dicitur sine\footnote{In Gough Missals 75, fol.~57r. `sine' is added by hand in the margin.}} Glória Patri. \emph{quo finito dicatur ℣. et oratio de sancto in cujus honore consecratum est altare: ab excellentiori sacerdote modesta voce id est sine nota que terminetur sic,} Per Christum Dóminum ⟨nostrum⟩.\footnote{1882:60.} \emph{nec precedat, nec sequatur\footnote{`\emph{subsequatur}', 1882:60.}} Dóminus vobíscum. \emph{sed tantum ⟨cum⟩\footnote{1882:60.}} Orémus. \emph{ante orationem ⟨et postea deosculentur altaria executores officii et alii clerici consequantur priusquam recedant.⟩\footnote{1882:60.} Eodem modo omnia altaria ecclesie abluantur cum responsoriis ⟨et versibus suis⟩\footnote{1882:60.} sicut ordinati sunt in ⟨predicta⟩\footnote{1882:60.} hystoria scilicet} In monte Olivéti. \emph{et cum ℣. et orationibus de sanctis ut supradictum est: ita ⟨tamen⟩\footnote{1882:60.} quod nullum ℟. incipiatur nisi tantum coram altari ubi totum cantentur ⟨ut supra diximus et semper⟩:\footnote{1882:60.} et si necesse fuerit reincipiatur hystoria. Ita in extrema ablutione cantetur ℟.} Circundedérunt me. \emph{cum suis ℣. sine} Glória Patri. 94. \emph{Finita ablutione altaria nuda usque ad sabbatum pasche: altaria conserventur\footnote{`\emph{conversentur}', 1519:57r.} preter principale altare. Si vero plura fuerint altaria in ecclesia quam ℟. et predicta hystoria tunc reincipiatur hystoria ordine predicto servatur itaque\footnote{`\emph{scilicet quod}', 1882:60.} ℟.} Circundedérunt me. \emph{semper in ultimo loco\footnote{`\emph{super ultimo loco}', 1519:57r.} cantetur.} 94.

\emph{⟨Ad altare sanctissime Trinitatis cantetur hoc responsorium sequens.⟩\footnote{1882:60.}}

\gregorioscore{007780-tristis-est}

\emph{⟨℣.} Sit nomen Dómini benedíctum.

\emph{℟.} Ex hoc nunc et usque in séculum.

\emph{⟨℣.⟩} Orémus.

\begin{lesson}
\subsubsection{Oratio.}

\lettrine{O}{m}nípotens sempitérne Deus, qui dedísti fámulis tuis in confessióne vere fidéi etérne Trinitátis glóriam agnóscere, et in poténtia majestátis tue adoráre unitátem: quésumus ut ejúsdem fidéi firmitátc ab ómnibus semper muniámur advérsis, In qua vivis et regnas Deus. Per ómnia sécula seculórum. \emph{Chorus respondeat} Amen.
\end{lesson}

\emph{Ad altare sancti Michaelis archangeli ceterorumque angelorum cantetur hoc responsorium sequens.⟩\footnote{1882:61.}}

\emph{Aliud responsorium.}

\gregorioscore{006618-ecce-vidimus}

\emph{⟨℣.} In conspéctu angelórum psallam tibi Dens meus.

\emph{℟.} Adorábo ad templum sanctum tuum et confitébor nómini tuo.

\emph{⟨℣.⟩} Orémus.

\begin{lesson}
\subsubsection{Oratio.}

\lettrine{D}{e}us, qui miro órdine angelórum mystéria hominúmque dispénsas, concéde propítius, ut quibus tibi ministrántibus in celo semper assístitur, ab his in terra vita nostra muniátur. Per.
\end{lesson}

\emph{Ad altare sanctorum apostolorum cantetur sequens responsorium.⟩\footnote{1882:61.}}

\gregorioscore{007809-unus-ex}

\emph{⟨℣.} In omnem terram exívit sonus eórum.

\emph{℟.} Et in fines. 315.

\emph{⟨℣.⟩} Orémus.

\begin{lesson}
\subsubsection{Oratio.}

\lettrine{Q}{u}ésumus omnípotens Deus ut beáti apóstoli tui Petrus et Paulus cum sóciis suis pro nobis implórent auxílium: ut a nostris reátibus absolúti a cunctis étiam perículis eruámur. Per Christum Dóminum nostrum. Amen.
\end{lesson}

\emph{Ad altare sanctorum martyrum cantatur hoc responsorium sequens modo, quo sequitur.⟩\footnote{1882:61.}}

\gregorioscore{007041-judas-mercator}

\emph{⟨℣.} Letámini in Dómino\footnote{1508:unpaged. 1882:62.has `Deo'.} et exultáte justi.

\emph{℟.} Et gloriámini. 136.

\emph{⟨℣.⟩} Orémus.

\begin{lesson}
\subsubsection{Oratio.}

\lettrine{C}{o}ncéde quésumus omnípotens Deus ut qui sanctórum mártyrum tuórum Johánnis baptíste et Thome mártyris sociorúmque éorum recólimus victórias: participémur et prémiis. Per Christum Dóminum nostrum. \emph{⟨℟.⟩} Amen.
\end{lesson}

\emph{Ad altare sanctorum confessorum cantetur istud responsorium.⟩\footnote{1882:62.}}

\gregorioscore{007807-una-hora}

\emph{⟨℣.} Letámini in Dómino et exultáte justi.

\emph{℟.} Et gloriámini omnes recti corde.

\emph{⟨℣.⟩} Orémus.

\begin{lesson}
\subsubsection{Oratio.}

\lettrine{D}{e}us qui nos sanctórum confessórum tuórum Nícolai et Edmúndi sociorúmque eórum confessiónibus gloriósis circúmdas et protégis, da nobis eórum imitatióne profícere et intercessióne gaudére. Per Christum Dóminum nostrum. Amen.
\end{lesson}

\emph{Ad altare sanctarum virginum cantetur sequens responsorium.⟩\footnote{1882:62.}}

\gregorioscore{007636-seniores-populi}

\emph{⟨℣.} Adducéntur regi vírgines.

\emph{℟.} Próxime ejus afferéntur tibi.

\emph{⟨℣.⟩} Orémus.

\begin{lesson}
\subsubsection{Oratio.}

\lettrine{I}{n}dulgéntiam nobis, Dómine, quésumus, beáte vírginas tue Katharína et Margaréta sociéque eárum semper implórent, que tibi grátie extitérunt et mérito caatitátis et tue confessióne virtútis. Per Christum Dóminum nostrum.
\end{lesson}

\emph{Ad superaltare in vestibulo resp.} Circundedérunt. \emph{Require ut supra in dominica in passione Domini.} 94.

\emph{℣.} Letámini in Dómino.

\emph{℟.} Et gloriámini. 136.

\emph{⟨℣.} Orémus.⟩

\begin{lesson}
\subsubsection{Oratio.}

\lettrine{C}{o}ncéde, quésumus, omnípotens Deus, ut intercéssio Dei genitrícis Maríe sanctarúmque ómnium celéstium virtútum et beatórum patriarchárum, prophetárum, apostolórum, evangelistárum, mártyrum, confessórum atque vírginum, et ómnium electórum tuórum nos ubíque letíficet, ut dum eórum mérita recólimus, patrocínia sentiámus. Per eúndem Dóminum nostrum. \emph{⟨℟.⟩} Amen.
\end{lesson}

\emph{Finita ablutione altarium, nuda usque ad sabbatum pasche reserventur.⟩\footnote{1882:63.}}

\emph{⟨Aliud responsorium.⟩}

\gregorioscore{007272-o-juda-qui}

\emph{⟨Sequitur⟩\footnote{1882:63.} aliud responsorium.}

\gregorioscore{007543-revelabunt-celi}

\emph{Post ablutionem altarium intrent capitulum ⟨et⟩\footnote{1882:63.} ibidem dyaconus legat evangelium sicut ad missam lectum est ⟨videlicet⟩.\footnote{1882:63.}}

\begin{lesson}
\subsubsection{Secundum Johannem. xiij. ⟨1--15.⟩}

\lettrine{A}{n}te diem festum pascbe, sciens Jesus quia venit hora ejus ut tránseat ex hoc mundo ad Patrem: cum dilexísset suos qui erant in mundo: in finem diléxit eos. Et cena facta, cum dyábolus jam se\footnote{`se' is not in \emph{Vulgate.}} misísset in cor ut tráderet eum Judas Symónis Scarióthis: sciens quia dedit ei ómnia\footnote{`ómnia dedit ei', \emph{Vulgate.}} Pater in manus et quia a Deo exívit, et ad Deum vadit: surgit a cena, et ponit vestiménta sua. Et cum accepísset líntheum: precínxit se. Deinde misit aquam in pelvim: et cepit laváre pedes discipulórum:\footnote{`discipulórum suórum', 1882:64.} et extérgere líntheo quo erat precínctus. Venit ergo ad Symónem Petrum: et dixit ei Petrus, Dómine: tu michi lavas pedes? Respóndit Jesus et dixit ei, Quod ego fácio tu nescis modo: scies autem póstea. Dixit ei Petrus, Non lavábis michi pedes in etérnum. Respóndit ei Jesus, Si non lávero te: non habébis partem mecum. Dixit ei Symon Petrus, Dómine non tantum pedes ⟨meos⟩:\footnote{1882:64., \emph{Vulgate.}} sed ⟨et⟩\footnote{1882:64., \emph{Vulgate.}} manus et caput. Dixit ei Jesus, Qui lotus est non índiget nisi ut pedes lavet: sed est mundus totus. Et vos mundi estis: sed non omnes. Sciébat enim Jesus,\footnote{`Jesue' is not in \emph{Vulgate.}} quisnam esset qui tráderet eum: proptérea dixit non estis mundi omnes. Postquam ergo lavit pedes eórum ⟨et⟩\footnote{\emph{Vulgate.}} accépit vestiménta sua: et cum recubuísset íterum dixit eis, Scitis quid fécerim vobis? Vos vocátis me Magíster et Dómine: et benedícitis.\footnote{`bene dícitis', \emph{Vulgate.}} Sum étenim. Si ergo ⟨ego⟩\footnote{1882:64., \emph{Vulgate.}} lavi pedes vestros Dóminus et Magíster: et vos debétis alter altérius pedes laváre. Exémplum enim dedi vobis: ut quemádmodum ego feci vobis: ita et vos faciátis.
\end{lesson}

\emph{¶ Deinde fiat sermo ad populum. Quo finito surgant duo predicti presbyteri: et incipient a majoribus et unus lavet omnium pedes ex una parte, et alius ex altera parte: et postmodum sibi invicem pedes lavent: et interim cantentur hee sequens antiphone cum suis psalmis a toto choro interim cantent sedentes hanc antiphonam ut sequitur.\footnote{In 1519:61r the following is added by hand `¶ Ablutio pedum'.}}

\gregorioscore{003688-mandatum-novum}

\emph{¶ Totus psalmus dicatur sine} Glória Patri. \emph{et post unumquemque versum psalmi, repetatur antiphona. Eodem modo dicantur omnes antiphone sequentes cum suis psalmis.}

\gregorioscore{002231-diligamus-nos}

\emph{⟨Cantetur psalmus usque in finem.}

\emph{Item alia ant.⟩\footnote{1882:65.}}

\gregorioscore{003224-in-diebus}

\emph{⟨Et cantetur psalmus usque in finem.}

\emph{Item⟩\footnote{1882:65.} alia ant.}

\gregorioscore{003699-maria-ergo}

\emph{⟨Et cantetur usque ad finem psalmi.}

\emph{Alia ant.⟩\footnote{1882:65.}}

\gregorioscore{004340-postquam-surrexit}

\emph{⟨He sequentes antiphone si necesse fuerit cantentur.}

\gregorioscore{005504-vos-vocatis}

\emph{Sequitur alia antiphona.}

\gregorioscore{004889-si-ego}

\emph{Item alia antiphona.}

\gregorioscore{001431-ante-sex-dies}

\emph{Sequitur alia antiphona.}

\gregorioscore{a00155-venit-ad-petrum}⟩\footnote{1882:65.}

\emph{¶ Peracta ablutione pedum dictoque sermone accipiant potum charitatis, quibus rite peractis unus sacerdos dicat has preces.}

\emph{℣.} Suscépimus Deus misericórdiam tuam. \emph{℟.} In médio templi tui.

\emph{℣.} Tu mandásti. \emph{℟.} Mandáta tua custodíri nimis.

\emph{℣.} Ecce quam bonum est quam jocúndum. \emph{℟.} Habitáre fratres in unum.

\emph{℣.} Dómine exáudi oratiónem meam. \emph{℟.} Et clamor meus ad te véniat.

\emph{⟨℣.⟩} Dóminus vobíscum. \emph{⟨℟.⟩} Et cum spíritu tuo.

\emph{⟨℣.⟩} Orémus.

\begin{lesson}
\subsubsection{Oratio.}

\lettrine{A}{d}ésto quésumus Dómine offício servitútis nostre: et quia tu pedes laváre dignátus es tuis discípulis: ne despícias ópera mánuum tuárum que nobis retinénda mandásti: et\footnote{`sed', 1882:66.} sicut hec exterióra\footnote{`exterióra hic', 1882:66.} abluúntur inquinaménta córporum sic a te ómnium nostrum interióra mundéntur peccáta: quod ipse prestáre dignéris. Qui cum Patre et Spíritu Sancto vivis.
\end{lesson}

\emph{Si vero nullus assit clericus sermonem faciens ceteris ut ⟨prius⟩\footnote{1882:66.} peractis post ablutionem pedum dicantur preces predicte. Deinde legatur evangelium secundum Johannem ⟨scilicet⟩\footnote{1882:66.}} Amen amen dico vobis. \emph{sub tono lectionis a quodam dyacono in superpelliciis, fratribus interim caritatis potum\footnote{`\emph{potum caritatis}', 1882:66.} sumentibus et legatur ⟨usque ad illum locum⟩\footnote{1882:66.}} Súrgite eámus hinc.

\begin{lesson}
\subsubsection[⟨Evangelium⟩ secundum Johannem. ⟨xiij. 16. et xiv. capitulis.⟩]{⟨Evangelium⟩\footnote{1882:66.} secundum Johannem. ⟨xiij. 16. et xiv. capitulis.⟩\footnote{1882:66.}}

\lettrine{A}{m}en amen dico vobis: non est servus major dómino suo: neque apóstolus major eo qui misit illum. Si hec scitis: beáti éritis si fecéritis ea. Non de ómnibus vobis dico. Ego scio quos elégerim.\footnote{`elégeri', 1519:62v.} Sed ut adimpleátur scriptúra, Qui mandúcat mecum\footnote{`meum', 1519:63r.} panem: levábit contra me calcáneum suum. Amen amen dico\footnote{`Amodo dico vobis', 1555, \emph{Vulgate.}} vobis: priúsquam fiat ut cum factum fúerit: credátis quia ego sum.\footnote{`priúsquam fiat ut credátis cum factum fúerit: quia ego sum', 1519:63r.} Amen amen dico vobis: quia qui áccipit si quem mísero, me áccipit. Qui autem me áccipit: áccipit eum qui me misit. ⟨Et⟩\footnote{1519:63r.} cum hec dixísset Jesus turbátus est spíritu: et protestátus est et dixit, Amen amen dico vobis: quia unus ex vobis me tradet. Aspiciébant ergo discípuli ad ínvicem:\footnote{`ad ínvicem discípuli', \textit{Vulg.}} hesitántes de quo díceret. Erat autem\footnote{`ergo', \textit{Vulg.}} recúmbens unus ex discípulis ejus in sinu Jesu: quem diligébat Jesus. Innuit ergo discípulo\footnote{\textit{Vulg.} omits `discípulo'.} huic Symon Petrus: et dixit ei, Quis est de quo dixit? Itaque cum recubuísset ille supra pectus Jesu: dixit ei, Dómine, quis est? Respóndit Jesus, Ille est: cui ego intínctum panem porréxero: et cum intinxísset panem, dedit Jude Symónis Scarióthis. Et post bucéllam: introívit in illum Sáthanas. Et dixit ei Jesus, Quod facis: fac cítius. Hoc autem nemo scivit discumbéntium: ad quid díxerit ei. Quidam autem putábant quia lóculos habébat Judas: quod dixísset ei Jesus, Eme ea que opus sunt nobis\footnote{`nobis opus sunt', 1882:67.} ad diem festum: aut egénis ut áliquid daret. Cum ergo accepísset ille bucéllam: exívit contínuo. Erat autem nox. Cum ergo exísset dixit Jesus, Nunc clarificátus est Fílius Hóminis: et Deus clarificávit\footnote{`clarificátus', 1519:63r.} est in eo. Si Deus clarificátus est in eo: et Deus clarificávit eum in semetípso: et contínuo clarificábit eum. Filióli~Gdhuc módicum vobíscum sum. Quéretis me: et sicut dixi Judéis quo ego vado: vos non potéstis veníre. Et vobis dico modo. Mandátum novum do vobis: ut diligátis ínvicem, sicut ego\footnote{\textit{Vulg.} omits `ego'.} diléxi vos ⟨ut et vos diligátis ínvicem⟩.\footnote{1882:67., \emph{Vulgate.}} In hoc agnóscent\footnote{`cognóscent', \textit{Vulg.}} omnes quia discípuli mei estis: si dilectiónem habuéritis ad ínvicem. Dixit ei Symon Petrus, Dómine: quo vadis? Respóndit Jesus, Quo ego vado non potes me modo sequi: sequéris autem póstea. Dixit ei Petrus, Quare non possum te sequi modo? Animam meam pro te ponam. Respóndit ⟨ei⟩\footnote{1882:67., \emph{Vulgate.}} Jesus, Animam tuam pro me pones? Amen amen dico tibi: non cantábit gallus, donec ter negábis me. Et ait discípulis suis,\footnote{`Et ait discípulis suis', not in \emph{Vulgate.}} Non turbétur cor vestrum neque fórmidet. Créditis in Deum: et in me crédite. In domo Patris mei mansiónes multe sunt. Si quo minus\footnote{`quóminus', \textit{Vulg.}} dixíssem vobis: quia vado paráre vobis locum. Et si abíero et preparávero vobis locum: íterum véniam et accípiam vos ad meípsum: ut ubi ego sum, et vos sitis. Et quo ego vado scitis: et viam scitis. Dixit ei Thomas, Dómine: néscimus quo vadis. Et quómodo póssumus viam scire? Dixit ei Jesus, Ego sum via, véritas, et vita. Nemo venit ad Patrem nisi per me. Si cognovissétis me: et Patrem meum útique cognovissétis. Et ámodo cognoscétis eum: et vidístis eum. Dixit ei Philíppus, Dómine: osténde nobis Patrem: et súfficit nobis. Dixit ei Jesus, Tanto témpore vobícum sum: et non cognovítis me? Philíppe qui videt me: videt et Patrem ⟨meum⟩.\footnote{1882:68.} Et quómodo tu dicis, Osténde nobis Patrem? Non credis quia ego in Patre: et Pater in me est? Verba que ego loquor vobis: a meípso non loquor. Pater autem in me manens ipse facit ópera. Non créditis quia ego in Patre: et Pater in me est? Alíoquin propter ópera ipsa crédite. Amen amen dico vobis: qui credit in me ópera que ego fácio: et ipse fáciet. Et majóra horum fáciet: quia ego ad Patrem vado. Et quocúmque petiéritis Patrem in nómine meo hoc fáciam: ut glorificétur Pater in Fílio. Si quid petiéritis Patrem in nómine meo hoc faciam. Si dilígitis me: mandáta mea serváte. Et ego rogábo Patrem, et álium Paráclytum dabit vobis: ut máneat vobíscum in etérnum. Spíritum veritátis quem mundus non potest cápere: quia non videt eum, nec scit eum. Vos autem cognoscétis eum: quia apud vos manébit, et in vobis erit. Non relínquam vos órphanos: véniam ad vos. Adhuc módicum: et mundus me jam non vidébit.\footnote{`videt', \textit{Vulg.}} Vos autem vidébitis me quia ego vivo: et vos vivétis. In illo die vos cognoscétis quia ego sum in Patre meo: et vos in me, et ego in vobis. Qui habet mandáta mea et servat ea: ille est qui díligit me. Qui autem díligit me: diligétur a Patre meo. Et ego díligam illum:\footnote{`eum', \textit{Vulg.}} et manifestábo ei meípsum. Dixit\footnote{`Dicit', \textit{Vulg.}} ei Judas non ille Scarióthis, Dómine: quid factum est quia manifestatúrus es nobis teípsum: et non mundo? Respóndit Jesus: et dixit ei, Si quis díligit me: sermónem meum servábit. Et Pater meus díliget eum: et ad eum veniémus, et mansiónem apud eum faciémus. Qui non díligit me: sermónes meos non servat. Et sermónem quem audístis, non est meus: sed ejus qui misit me Patris. Hec locútus sum vobis: apud vos manens. Paráclytus autem Spíritus Sanctus quem mittet Pater in nómine meo: ille vos docébit ómnia, et súggeret vobis ómnia quecúmque díxero vobis. Pacem relínquo vobis: pacem meam do vobis. Non quómodo mundus dat ego do vobis. Non turbétur cor vestrum neque fórmidet. Audístis quia ⟨ego⟩\footnote{\emph{Vulgate.}} dixi vobis, Vado et vénio ad vos. Si diligerétis me gauderétis útique: quia vado ad Patrem: quia Pater major me est. Et nunc dixi vobis priúsquam fiat: ut cum factum fúerit credátis. Jam non multa loquar vobíscum. Venit enim princeps mundi hujus: et in me non habet quicquam. Sed ut cognóscat mundus quia díligo Patrem: et sicut mandátum dedit michi Pater, sic fácio. Súrgite, eámus hinc.
\end{lesson}

\emph{Et ita recedant euntes in ecclesiam: et ibidem dicant completorium privatim.}


\chapter{¶ Feria sexta in die parasceves.}

\emph{Dicta hora nona accedat sacerdos ad altare indutus vestibus sacerdotalibus, et casula rubea cum dyacono et subdyacono, et ceteris ministris altaris: qui omnes sint albis cum amictibus tantum sine paruris ⟨induti⟩\footnote{1882:69.} \&c.\footnote{`(\emph{i.e., ut in missali continetur})', 1882:69.}}

\emph{Finitis orationibus, exuat sacerdos casulam post passionem ⟨et illam super altare dimittat⟩:\footnote{1517 ⟨1882:69.⟩} et in sede sua juxta altare se ponat\footnote{`\emph{reponat}', 1882:69.} cum dyacono et subdyacono: et interim alii duo presbyteri de superiori gradu nudis pedibus albis indutis absque paruris tenentes crucem coopertam super eorum brachia sollemniter inter eos retro magnum altare ex dextera parte retro dextrum cornu stantes cantent hunc versum.}

\gregorioscore{008451-popule-meus}

\filbreak
\emph{¶ ⟨Finito versu,⟩ duo dyaconi de secunda forma ⟨indutis⟩ in cappis nigris stantes ad gradum ⟨chori⟩\footnote{1555, unpaged.} conversi ad altare simul dicant sic.\footnote{`\emph{habitu non mutato, sparsim stantes simul dicant}', 1882:69. 1517---Duo diaconi de ii. forma stantes ad gradum chori conversi ad altare simul dicant. ⟨1882:69.⟩ `\emph{simul incipiant hoc modo sequentur}', 1523:64v.}}

\gregorioscore{008450-agyos-o-theos-1}

\emph{Chorus respondeat cum genuflexione osculando tribus vicibus in una responsione et resurgat cum canitur} Deus. \emph{Iterum genuflectat cum dicitur} Sanctus. \emph{et resurgat cum dicatur} Fortis. \emph{Tertio chorus genuflectat cum dicatur} Sanctus. \emph{et resurgat cum} Et immortális miserére ⟨nobis⟩.\footnote{1882:69.}

\gregorioscore{008450-agyos-o-theos-2}

\emph{Hoc est ad quemlibet} Sanctus. \emph{genuflectere debet chorus. Sacerdotes vero tenentes crucem retro magnum altare et dyaconi ad gradum chori cantantes} Agyos. \emph{semper sint stantes.}\footnote{`\emph{sparsim}', 1882:70. 1544, \&c,---semper sint. ⟨1882:70.⟩}

\emph{Item sacerdotes loco non mutato simul cantent hanc antiphonam sequentem.\footnote{`\emph{modo quo sequitur}', 1882:70. In 1882:70 `Quia edúxi' is labelled `\emph{Ant.}'}}

\gregorioscore{008452-quia-eduxi}

\emph{Dyaconi.} Agyos. \emph{Chorus.} Sanctus Deus.

\emph{Sacerdotes loco non mutato dicat.}

\gregorioscore{008453-quid-ultra}

\emph{Dyaconi} Agyos. \emph{Chorus} Sanctus Deus. \emph{\&c.}

\emph{Deinde sacerdotes stantes juxta altare ex dextra parte discooperientes crucem simul cantent hanc sequentem antiphonam hoc modo.}

\gregorioscore{002522-ecce-lignum}

\emph{Chorus cum genuflectione osculando terram vel formulas ⟨cantent antiphonam⟩.\footnote{1882:70.}}

\gregorioscore{001953-crucem-tuam-adoramus}

\emph{Et repetatur antiphona post unumquemque versum istius psalmi.}

\begin{lesson}

Ut cognoscámus in terra viam tuam: in ómnibus géntibus salutáre tuum.

Confiteántur tibi pópuli Deus: confiteántur tibi pópuli omnes.

Leténtur et exúltent gentes quóniam júdicas pópulos in equitáte: et gentes in terra dírigis.

Confiteántur tibi pópuli Deus: confiteántur tibi pópuli omnes terra dedit fructum suum.

Benedícat nos Deus Deus noster benedícat nos Deus: et métuant eum omnes fines terre.

\end{lesson}

\emph{⟨Totus psalmus dicatur⟩\footnote{1882:71.} sine} Glória Patri. \emph{a toto choro cum genuflexione continue et interim deportetur crux sollenniter super gradum ⟨tertium⟩\footnote{1882:71.} altaris juxta quam sedeant predicti ⟨duo⟩\footnote{1882:71.} sacerdotes, unus a dextris et alius a sinistris. Deinde procedant clerici ad crucem adorandam nudis\footnote{`\emph{nudatis}', 1882:71.} pedibus incipientes a majoribus. Finito psalmo cum antiphona cantetur hymnus a predictis ⟨duobus⟩\footnote{1882:71.} sacerdotibus interim sedentibus juxta crucem hoc modo hymnus sequens.}

\gregorioscore{008367g-crux-fidelis}

\emph{Chorus idem repetat post unumquemque versum, interim sedendo cantent versum hoc modo.\footnote{`\emph{interim sedentes sacerdotes cantant versum}', 1882:71. 1517---Chorus idem repetat post unumquemque versum interim stando: sacerdotes dicant versum. ⟨1882:71.⟩}}

\gregorioscore{008367-pange-lingua-gloriosi}

\emph{His finitis deportetur crux per medium chori solenniter a duobus predictis sacerdotibus precedente ceroferario et reponatur ante aliquod altare ubi a populo adoretur, et interim cantetur hec sequens antiphona cum suo versu a tota choro interim sedente\footnote{1517---stando. ⟨1882:72.⟩} cantore incipiente ⟨hanc⟩\footnote{1882:72.} antiphonam ⟨sequentem⟩.\footnote{1882:72.}}

\gregorioscore{002453-dum-fabricator}

\emph{Revertendo ceroferariis ad vesbibulum precedant sacerdotes in superpelliceis discalceatis sine almico\footnote{Presumably and Amice `\emph{amictus}' or Almuce, `\emph{almucia}'.} deferentes Corpus Christi in pixide ad altare authenticum dum predicta antiphona cantatur. Adorata cruce et finita predicta antiphona cum suo ℣. predicti duo sacerdotes ea reverentia qua crucem asportaverunt usque ad summum altare iterum reportent: tunc conveniant omnes clerici de choro ad altare et reinduat ibidem sacerdos casulam qua prius exuerat et accedat ad gradum altaris cum dyacono et subdyacono et dicant} Confíteor. Misereátur. \emph{et} Absolutiónem. \emph{cum precibus et oratione} Aufer a nobis. \emph{more solito.} Missale:1142.

\emph{Finitis vesperis exuat sacerdos casulum et sumens secum unum de prelatis in superpelliceis discalciati reponant crucem cum corpore dominico in sepulcrum incipiens ipse solus hoc ℟.} Estimátus sum. \emph{genuflectendo cum socio suo: quo incepto statim surgat: similiter fiat in responsorio} Sepúlto Dómino. \emph{Chorus totum ℟. prosequatur cum suo ℣. genuflectendo per totum tempus usque ad finem servitii. Responsoria ut sic.}

\gregorioscore{006057-estimatus-sum-1}

\emph{Chorus prosequatur.}

\begin{noinitial}

\gregorioscore{006057-estimatus-sum-2}

\end{noinitial}

\emph{Dum predictum ℟. canitur cum suo ℣. predicti duo sacerdotes thurificent sepulchrum quo facto et clauso hostio incipiat idem sacerdos sequens.}

\gregorioscore{007640-sepulto-domino-1}

\emph{Chorus prosequatur.}

\begin{noinitial}

\gregorioscore{007640-sepulto-domino-2}

\end{noinitial}

\emph{⟨Item predicti duo⟩ sacerdotes dicat ⟨istas⟩\footnote{1882:73.} iij. antiphonas sequentes ⟨genuflectendo continue.⟩}

\gregorioscore{003265-in-pace-1}

\emph{Chorus prosequatur.}

\begin{noinitial}

\gregorioscore{003265-in-pace-2}

\end{noinitial}

\emph{⟨Iterum⟩\footnote{1882:73.} sacerdotes.}

\gregorioscore{003264-in-pace-1}

\emph{Chorus prosequatur.}

\begin{noinitial}

\gregorioscore{003264-in-pace-2}

\end{noinitial}

\emph{Item sacerdotes dicat sic.}

\gregorioscore{001175-caro-mea-1}

\emph{Chorus prosequatur.}

\begin{noinitial}

\gregorioscore{001175-caro-mea-2}

\end{noinitial}

\emph{Ad istas iij. antiphonas genuflectent continue predicti ij. sacerdotes.}

\emph{His finitis ordine servato reinduat sacerdos casulam: et eodem modo quo accesserit in principio servitii cum dyacono et subdyacono et ceteris ministris abscedat, dictis prius orationibus ad placitum. Secrete ab omnibus cum genuflexione omnibus aliis ad libitum recedentibus.}

\emph{Exinde continuo ardebit unus cereus ad minus ante sepulchrum usque ad processionem que fit in resurrectione Domini in die pasche, ita tamen quod dum psalmus} Benedíctus. \emph{cantetur \&c.~que sequuntur in sequenti nocte extinguatur: similiter et extinguatur in vigilia pasche dum benedicitur novus ignis usque accendatur cereus paschalis ⟨triginta et sex pedes continens in longitudine⟩.\footnote{1517. ⟨}1882:73.⟩}


\chapter{¶ Sabbato in vigilia pasche.}

\emph{Congregatis clericis in choro: dictaque hora ix. executor officii indutus vestibus sacerdotalibus cum cappa serica rubea: dyaconus dalmatica et subdyaconus tunica induantur cum ceteris ministris altaris albis cum amictibus indutis sine lumine ⟨in⟩\footnote{1882:74.} cereis et sine cruce et sine igne in thuribulo et quidam de prima forma scilicet acolitus in superpellicio cereum extinctum de tribus candelis tortis in unum in yma parte conjuncti et insuper abinvicem divisis super quandam hastam deferens et processio precedat\footnote{`\emph{procedat}', 1882:74.} post portitorem aque benedicte per medium chori ad fontes et ad novum ignem benedicendum processionaliter eant: choro sequente habitu non mutato excellentioribus precedentibus et ad columnam ex parte australi juxta fontem ubi sacerdos executor officii illius diei: ignem benedicat qui accendatur ibidem videlicet inter ij. columnas. In eundo vero dicatur a toto choro alternatim sine nota iste psalmus sequens.}

\begin{lesson}
\subsubsection{⟨Psalmus xxvj.⟩}

\lettrine{D}{o}minus illuminátio mea: et salus mea quem timébo.

Dóminus protéctor vite mee: a quo trepidábo.

Dum apprópiant super me nocéntes: ut edant carnes meas.

Qui tríbulant me\footnote{1882:74. In 1519:70v the punctuation of the following verses differs from the Breviary.

  4. Qui tríbulant me:* inimíci mei ipsi infirmáti sunt et cecidérunt.

  10. In petra exaltávit me et nunc exaltávit caput meum:* super inimícos meos.

  11. Circuívi et immolávi in tabernáculo ejus hóstiam vociferatiónis cantábo: * et psalmum dicam Dómino.} inimíci mei: ipsi infirmáti sunt et cecidérunt.

Si consístant advérsum me castra: non timébit cor meum.

Si exúrgat advérsum me prélium: in hoc ego sperábo.

Unam pétii a Dómino hanc requíram: ut inhábitem in domo Dómini ómnibus diébus vite mee.

Ut vidéam voluntátem Dómini: et vísitem templum ejus.

Quóniam abscóndit me in tabernáculo suo in die malórum: protéxit me in abscóndito tabernáculi sui.

In petra exaltávit me: et nunc exaltávit caput meum super inimícos meos.

Circuívi et immolávi in tabernáculo ejus hóstiam vociferatiónis: cantábo et psalmum dicam Dómino.

Exáudi Dómine vocem meam qua clamávi ad te: miserére mei et exáudi me.

Tibi dixit cor meum exquisívit te fácies mea: fáciem tuam Dómine requíram.

Ne avértas fáciem tuam a me: ne declínes in ira a servo tuo.

Adjútor meus esto Dómine\footnote{1882:77 omits `Dómine'.} ne derelínquas me: neque despícias me Deus\footnote{1882:77 omits `Deus'.} salutáris meus.

Quóniam pater meus et mater mea dereliquérunt me: Dóminus autem assúmpsit me.

Legem pone michi Dómine in via tua: et dírige me in sémita recta propter inimícos meos.

Ne tradidéris me in ánimas tribulántium me: quóniam insurrexérunt in me testes iníqui, et mentíta est iníquitas sibi.

Credo vidére bona Dómini: in terra vivéntium.

Expécta Dóminum viríliter age et confortétur cor tuum: et sústine Dóminum.

\emph{Et sine} Glória Patri. \emph{neque} Sicut erat. \emph{\&c.}

\end{lesson}

\emph{¶ Hoc modo fiat statio ad ignem benedicendum: sacerdos juxta ignem stet ad orientem conversus: et ad sinistrum ejus stet dyaconus, subdyaconus vero ad sinistrum dyaconi, unus ceroferarius stet sacerdoti oppositus, ad dextram cujus stet puer ferens librum proximior sacerdoti, alius autem ceroferarius stet retro sacerdotem, ad dextrum cujus stet portitor aque proximior sacerdote et ultimo loco ultra omnes a parte occidentali: stet portitor haste cum cereo ex alia parte ignis videlicet ex parte australi stet thuribularius ad accipiendum ignem in thuribulo post benedictionem omnibus istis ministris ad sacerdotem conversis, choro interim circumstante videlicet ex parte boreali: ut patet ⟨ubi protrahitur⟩\footnote{1882:75.} in statione sequente hoc modo.}

\begin{figure}
\centering
\includegraphics{images/71r-statio-ignis.jpg}
\caption{¶ Statio dum benedicitur ignis in vigilia pasche.}
\end{figure}

\emph{¶ Sequatur benedictio ignis ⟨in vigilia pasche⟩\footnote{1882:76.} solenniter a sacerdote dicente.}

\begin{noinitial}

\gregorioscore{f71v-dominus-vobiscum}

\end{noinitial}

\begin{lesson}
\subsubsection{Oratio.}

\lettrine{D}{o}mine Deus noster Pater omnípotens: lumen indefíciens: Cónditor ómnium lúminum, exáudi nos fámulos tuos et béne✠dic hunc ignem qui tua sanctificatióne atque\footnote{`ac', 1882:76.} benedictióne consecrátur, tu illúminas omnem hóminem veniéntem in hunc mundum: illúmina consciéntias cordis nostri igne tue claritátis, ut tuo igne ígniti, tuo lúmine illumináti, expúlsis a córdibus nostris peccatórum ténebris ad vitam te illustránte perveníre mereámur etérnam: et sicut illuminásti ignem Móysi fámulo tuo per colúmnam ignis ambulánti\footnote{`ambulántem', 1882:77.} in mari rubro: ita illústra nostrum lumen ut candéla que de eo fúerit accénsa: in hónore majestátis tue semper perséveret benedícta, ut quicúnque ex eo lúmine\footnote{`lumen ex eo', 1882:77.} portáverit sit illuminátus misericórdia grátie spiri-
\end{lesson}

\begin{noinitial}

\gregorioscore{f72v-per-dominum}

\emph{Omnes orationes dicuntur cum} Orémus. \emph{sub tono predicto. Hic aspergatur aqua benedicta super ignem. Deinde cantatur ut sequitur.}

\gregorioscore{f71v-dominus-vobiscum}

\end{noinitial}

\begin{lesson}
\subsubsection[⟨Oratio.⟩]{⟨Oratio.⟩\footnote{1882:77.}}

\lettrine{D}{o}mine sancte Pater omnípotens etérne Deus bene✠dícere et sancti✠ficáre dignéris ignem istum\footnote{`istum ignem', 1882:77.} quem nos indígni per invocatiónem unigéniti Fílii tui Dómini nostri Jesu Christi benedícere presúmimus: tu clementíssime Pater eum tua benedictióne sanctífica: et ad proféctum humáni géneris perveníre concéde. Per eúndem Christum Dóminum nostrum. \emph{⟨℟.⟩} Amen.

Orémus.

\subsubsection[⟨Oratio.⟩]{⟨Oratio.⟩\footnote{1882:77.}}

\lettrine{C}{e}lésti lúmine quésumus Dómine semper hic et ubíque nos prevéni ut mystérium cujus nos partícipes esse voluísti: et puro cernámus intúitu, et digno percipiámus efféctu. Per Dóminum nostrum.

\subsubsection{¶ Sequatur benedictio thimiamatis sive incensi sub tono predicti.}

\lettrine{E}{x}orcízo te immundíssime spíritus et omne phantásma inimíci in nómine Dei Patris omnipoténtis et in nómine Jesu Christi Fílii ejus et Spíritus Sancti: ut éxeas et recédas ab hac creatúra thimiamátis \emph{sive} incénsi cum omni fallácia atque nequítia tua, ut sit hec creatúra sancti✠ficáta\footnote{`✠' is in 1882:77., but not in 1519:72r.} in nómine Dómini nostri Jesu Christi: ut omnes gustántes, tangéntes, sive odorántes eam virtútem et auxílium percípiant Spíritus Sancti: ita ut ubicúnque hoc incénsum \emph{vel} thimiáma fúerit: íbidem nullátenus appropinquáre áudeas nec advérsa inférre presúmas, sed quicúnque spíritus immúnde es: cum omni versútia tua procul inde fúgias atque discédas adjurátus per nomen et virtútem Dei Patris omnipoténtis et Fílii ejus Dómini nostri Jesu Christi qui ventúrus est ⟨cum (\emph{sic}) Spíritu Sancto⟩\footnote{1882:77.} judicáre vivos et mórtuos et te prevaricatórem et séculum per ignem.

Orémus.

\subsubsection[⟨Oratio.⟩]{⟨Oratio.⟩\footnote{1882:78.}}

\lettrine{E}{t}érnam ac justíssimam pietátem tuam deprecámur Dómine sanctíssime\footnote{`sancte', 1882:78.} Pater omnípotens etérne Deus ut bene✠dícere dignéris hanc thimiamátis \emph{vel} incénsi spéciem ut sit incénsum majestáti tue in odórem suavitátis accéptum sit a te spécies hec bene✠dícta, per invocatiónem sancti nóminis tui sanctificáta: ita ut ubicúmque fumus ejus pervénerit: extricétur et effugétur omne genus demoniórum sicut incénsum jécoris piscis quam\footnote{`quem', 1882:78.} Ráphael archángelus Thobíam fámulum tuum dócuit, cum ascéndit ad Sare liberatiónem. Per Christum Dóminum nostrum.

Orémus.

\subsubsection[⟨Oratio.⟩]{⟨Oratio.⟩\footnote{1882:78.}}

\lettrine{D}{e}scéndat bene✠díctio\footnote{`✠' is in 1882:78., but not in 1519:72v.} ⟨tua Dómine⟩\footnote{1882:78.} super hanc spéciem\footnote{1882:78. 1519:72v has `hunc spíritum'.} incénsi, \emph{vel\footnote{`et', 1882:78.}} thimiamátis sicut in illo de quo David prophéta tuus cécinit dicens, Dirigátur orátio mea sicut incénsum in conspéctu tuo sit nobis odor consolatiónis suavitátis et grátie, ut fumo istíus\footnote{`isto', 1882:78.} effugétur omne fantásma inimíci mentis et córporis: ut simus Pauli apóstoli voce bonus odor Deo: effúgiant\footnote{`effúgiat', 1882:78.} a fácie incénsi hujus \emph{sive\footnote{1882:78; `et', 1519:72v.}} thimiamátis omnes\footnote{`omnis', 1882:78.} demónum incúrsus sicut pulvis a fácie venti, et sicut fumus a fácie ignis: ⟨et⟩\footnote{1882:78.} presta hoc piíssime Pater boni odóris incénsum ad opus ecclésie tue ob causam religiónis júgiter permanére: ut mýstica nobis significatióne spirituálium virtútum fragrans osténdit\footnote{`osténdat', 1882:78.} odor suavitátem. Tu\footnote{`tua', 1882:78.} ergo quésumus omnípotens\footnote{1882:78; `omnípotens quésumus', 1519:72v.} Deus imménse majestátis déxtera hanc creatúram benedícere ex diversárum rerum commixtióne conféctam\footnote{1882:78; `inféctam', 1519:72v.} dignáre: ut in virtúte sancti tui nóminis, omnes immundórum spírituum fantásticos incúrsus effugáre, omnésque morbos réddita sanitáte expéllere, ut ubicúmque fumus arómatum ejus affláverit mirabíliter possit atque in odóre fragrantíssimo tibi perpétua suavitáte redolére. Per Dóminum nostrum Jesum Christum Fílium tuum. ⟨Qui tecum vivit et regnat Deus. Per ómnia sécula seculórum. \emph{⟨℟.⟩} Amen.⟩
\end{lesson}

\emph{¶ Post benedictionum incensi ponatur de ipsis carbonibus in thuribulo cum incenso et incensetur novus ignis: postea accendatur cereus super hastam solus de novo igne ceteris luminariis ⟨ecclesie prius extinctis et deferatur in processione ad locum ubi cereus paschalis benedicendus est⟩.\footnote{1882:78.} In redeundo duo\footnote{1882:78; `\emph{secundi}', 1519:72v.} clerici de secunda forma, in superpelliceis post sacerdotem incedentes cantent hos versus sequentes hoc modo.}

\gregorioscore{ah50031-inventor-rutili-1}

\emph{Chorus idem repetat post unumquemque versum ⟨ita videlicet ut dum dicti clerici cantent versum stent gradibus fixis, et dum chorus prosequatur primum versum procedant. Quod similiter est observandum dum canitur} Rex sanctórum angelórum. \emph{cum suis versibus⟩.\footnote{1882:79.}}

\begin{noinitial}

\gregorioscore{ah50031-inventor-rutili-2}

\emph{Chorus repetat} Invéntor ⟨rútili⟩.\footnote{1882:79.}

\emph{Clerici ⟨dicant⟩\footnote{1882:79.}}

\gregorioscore{ah50031-inventor-rutili-3}

\emph{Chorus repetat} Invéntor ⟨rútili⟩.

\emph{Clerici ⟨dicant⟩\footnote{1882:79.}}

\gregorioscore{ah50031-inventor-rutili-4}

\emph{¶ Chorus repetat} Invéntor rútili.

\emph{Clerici ⟨dicant⟩\footnote{1882:79.} versum.}

\gregorioscore{ah50031-inventor-rutili-5}

\end{noinitial}

\emph{Deinde sequatur benedictio cerei paschalis ab ipso dyacono indutus ad processionem accepta ⟨prius⟩\footnote{1882:79.} benedictione ab executore officii ad borealem converso ad gradum presbyterii ceroferarii dyaconi assistentibus uno a dextris, reliquo a sinistris ad eum conversis cereis extinctis nisi cereus super hastam. Subdyaconus vero textum tenens stet ei oppositus juxta quem stet portitor haste ex una parte et ceroferarius ejusdem parvi cerei\footnote{`The \emph{parvus cereus} is explained by the Ordinal quoted by Dr.~Rock, iv. 246: \emph{Finita benedicitione ignis et incensi \ldots{} Cereus super hastam illuminetur, et alia candela incendatur, unde idem cereus super hastam si forte extinguatur possit reaccendi.}' ⟨1882:xv.⟩} ut patet ⟨in ubi protrahitur⟩\footnote{1882:80.} in statione ⟨sequente⟩\footnote{1882:80.} hoc modo.}

\begin{figure}
\centering
\includegraphics{images/74r-statio-cereus.jpg}
\caption{¶ Statio dum benedicitur cereus paschalis in vigilia pasche.}
\end{figure}

\emph{Ad fontes benedicendas ex altera eodem modo conversus et post dyaconum dyaconus, dyacono canente.}

\gregorioscore{850202-exultet-1}

\gregorioscore{850202-exultet-2}

\gregorioscore{850202-exultet-3}

\emph{Hic dyaconus ponat incensum in cereum in modum crucis ⟨cantando hoc quod sequitur hoc modo⟩.\footnote{1882:82.}}

\begin{noinitial}

\gregorioscore{850202-exultet-4}

\emph{Hic accendatur cereus paschalis de novo igne, nec extinguatur ante completorium diei sequentis. Hic accendantur cetera luminaria per ecclesiam.}

\gregorioscore{850202-exultet-5}

\end{noinitial}

\emph{Finita benedictione cerei paschalis sacerdos completurus officium indutus casula ad altare authenticum assumpta: cum ministris suis ad altare accedat confessione jam non dicta, sed tantum} Pater noster. \emph{et osculando altare cum suis ministris eat sessum. Accedat cereus super hastam: minister vero qui alium cereum defert ad sinistrum cornu altaris stet super gradum ⟨ad⟩\footnote{1882:83.} australe conversus quousque finiatur septiformis letania, postea legantur lectiones sine titulo a dignioribus personis, \&c.} Missale:731.

\emph{Quibus peractis sequatur septiformis letania, que in medio chori a septem pueris in superpelliceis dicatur et interim exuat sacerdos casulam et super altare reponat, et sumat cappam rubeam adhuc stando ante altare donec cantetur letania sequens, \&c.}

\emph{Sequitur letania.}

\gregorioscore{909041.1-kyrie-vigilia-pasche-1}

\begin{lesson}

Sancta virgo vírginum. \hfill Ora pro nobis.

Sancte Míchael. \hfill Ora.

Sancte Gábriel. \hfill Ora.

Sancto Ráphael. \hfill Ora.

Omnes sancti ángeli et archángeli Dei. \hfill Oráte pro nobis.

Sancte Johánnes baptísta. \hfill Ora.

Omnes sancti patriárche et prophéte. \hfill Oráte.

Sancte Petre. \hfill Ora.

Sancte Andréa. \hfill Ora.

Sancte Johánnes. \hfill Ora.

Sancte Jacóbe. \hfill Ora.

Sancte Philíppe. \hfill Ora.

Sancte Bartholomée. \hfill Ora.

Sancte Mathée. \hfill Ora.

Omnes sancti apóstoli et evangelíste. \hfill Oráte ⟨pro nobis⟩.\footnote{1882:83.}

Sancte Stéphane. \hfill Ora.

Sancte Line. \hfill Ora.

Sancte Clete. \hfill Ora.

Sancte Laurénti. \hfill Ora.

Sancte Vincénti. \hfill Ora.

Sancte Sixte. \hfill Ora.

Sancte Dionýsi cum sociis tuis. \hfill Oráte\footnote{Usually `Ora pro nobis'.} ⟨pro nobis⟩.\footnote{1882:83.}

Omnes sancti mártyres. Orate.

Sancte Silvéster. \hfill Ora.

Sancte Gregóri. \hfill Ora.

Sancte Hylári. \hfill Ora.

Sancte Martíne. \hfill Ora.

Sancte Remígi. \hfill Ora.

Sancte Audoéne. \hfill Ora.

Sancte Augustíne. \hfill Ora.

Omnes sancti confessóres. \hfill Oráte ⟨pro nobis⟩.\footnote{1882:83.}

Sancta María Magdaléna. \hfill Ora.

Sancta Felícitas. \hfill Ora.

Sancta Perpétua. \hfill Ora.

Sancta Agatha. \hfill Ora.

Sancta Agnes. \hfill Ora.

Sancta Cecília. \hfill Ora.

Sancta Scolástica. \hfill Ora.

\end{lesson}

\begin{noinitial}

\gregorioscore{909041.1-kyrie-vigilia-pasche-2}

\end{noinitial}

\emph{¶ Si episcopus presens fuerit indutus cappa serica stet in sede sua dum predicta letania canitur.}

\emph{Finita hac letania statim incipiatur quinta partita letania que a v. dyacono similiter in medio chori in superpellicio de ij. forma dicatur et finiatur sub tono predicto.}

\emph{Cumque perventum fuerit ad hanc prolationem,} Sancta María. \emph{statim eat processio ad fontes benedicendas hoc ordine. In primis accolitus crucem ferens alba et tunica indutus: post eum vero duo ceroferarii in albis cum amictibus deinde thuribularius in simili habitu, post eum vero duo pueri in superpelliceis pariter incedentes unus ferens librum, alius a dextris ejus ferens cereum ad fontes benedicendas: deinde duo dyaconi de secunda forma albis cum amictis induti, pariter incedentes unus ferens oleum, alius a dextris ejus ferens chrisma: deinde subdyaconus tunica: et deinde dyaconus in dalmatica: deinde sacerdos in cappa serica rubea, clero itaque sequente habitu non mutato, ex australi latere ecclesie procedendo ad fontes veniant, predicti dyaconi letaniam canentibus de singulis ordinibus, v. in medio clericorum de secunda forma post executorem officii hoc modo.}

\begin{lesson}
\subsubsection{⟨¶ Sequitur letania.⟩}

Kyrieléyson. Christeléyson. Christe audi nos.

Sancta María. \hfill Ora.

Sancta Dei génitrix. \hfill Ora.

Sancta virgo vírginum. \hfill Ora.

Sancte Míchael. \hfill Ora.

Sancte Gábriel. \hfill Ora.

Sancte Ráphael. \hfill Ora.

Omnes sancti ángeli et archángeli Dei. \hfill Oráte.

Sancte Johánnes baptísta. \hfill Ora.

Omnes sancti patriárche et prophéte. \hfill Oráte ⟨pro nobis⟩.\footnote{1882:85.}

Sancte Paule. \hfill Ora.

Sancte Jacóbe. \hfill Ora.

Sancte Thoma. \hfill Ora.

Sancte Symon. \hfill Ora.

Sancte Thadée. \hfill Ora.

Omnes sancti apóstoli et evangelíste. \hfill Oráte.

Sancte Clemens. \hfill Ora.

Sancte Cornéli. \hfill Ora.

Sancte Cypriáne. \hfill Ora.

Sancte Sebastiáne. \hfill Ora.

Sancte Mauríci cum sóciis tuis. \hfill Oráte\footnote{`Ora', 1882:85; usually `Ora pro nobis'.} ⟨pro nobis⟩.\footnote{1882:85.}

Onmes sancti mártyres. \hfill Oráte.

Sancte Benedícte. \hfill Ora.

Sancte Nicoláe. \hfill Ora.

Sancte Germáne. \hfill Ora.

Sancte Románe. \hfill Ora.

Sancte Aldélme. \hfill Ora.

Sancte Augustíne. Ora.\footnote{This line does not appear in 1555 (unpaged).}

Omnes sancti confessóres. \hfill Oráte ⟨pro nobis⟩.\footnote{1882:85.}

Sancta Lúcia. \hfill Ora.

Sancta Petronílla. \hfill Ora.

Sancta Katherína. \hfill Ora.

Sancta Christína. \hfill Ora.

Sancta Brigída. \hfill Ora.

Omnes sancte vírgines. \hfill Oráte.

Omnes sancti. \hfill Oráte.
\end{lesson}

\emph{In his duabus letaniis non dicatur} Pater de celis. \emph{⟨neque⟩\footnote{1882:85.}} Fili Redémptor mundi Deus. \emph{neque} Spíritus Sancte Deus. \emph{neque} Sancta Trínitas unus Deus.\footnote{In 1882:85 this rubric appears at the head of the second Litany.}

¶ \emph{⟨Item⟩ Gelasius papa\footnote{In 1519:81r `papa' is crossed out.} ostendit dicens quia ipse qui Pater et Filius et Spiritus Sanctus una persona in Trinitate,\footnote{`\emph{Unitate}', 1519:81r.} et tres persone in Unitate et in sepulchro se custodiri mittitur\footnote{`\emph{promittitur}', 1882:86.} omnino de\footnote{`\emph{mittitur omnino dicit}', 1519:81r.} adhuc surrexerat a mortuis qui voluit prophetiam implere, sed jacuit in sepulchro usque ad tertium diem quod bene iste predicte iiij. clausule in his letaniis possunt pretermittere.\footnote{`The first paragraph of the rubric, the quotation from Gelasius, I have not been able to correct', 1882:xv.}}

\emph{¶ Hoc modo fiat statio ad fontes ex parte occidentali donec percantetur letania scilicet ad gradum fontis ex parte occidentali stet sacerdos, retro quem stent quinque dyaconi letaniam cantantes: deinde ad alium gradum fontis ex parte orientali puer librum ferens, deinde dyaconus, deinde subdyaconus, deinde oleum et crisma: deinde portitor cerei fontis, deinde thuribularius, deinde oleum: ceroferarius, exinde duo accoliti crucem ferentes omnibus ad orientem conversis.}

\emph{Executor officii conversus ad orientem fontes benedicendo, assistat minister juxta ad fontem circumstantibus ordinate scilicet a dextris juxta sacerdotem stet dyaconus, subdiaconus vero a sinistris, qui fert chrisma stet juxta dyaconum, qui vero fert crucem stet sacerdoti opposito ad eum conversus, juxta quem eodem modo stent ceroferarii ij. post ceroferarium et thuribularium, qui vero fert cereum inter dyaconum et chrisma: puer autem ferens librum stet inter subdyaconum et oleum ⟨ut patet in pictura sequente⟩.\footnote{1882:86.} Episcopus tamen si presens fuerit a tergo canentium letaniam ut in aliis processionibus semper in fine ultimum locum tenet.}

\begin{figure}
\centering
\includegraphics{images/81v-statio-fontes.jpg}
\caption{¶ Statio dum cantatur letania ad fontes in vigilia pasche.}
\end{figure}

\emph{Deinde executor officii ad fontes dicat sic.}

\begin{noinitial}

\gregorioscore{909041.1-kyrie-vigilia-pasche-3}

\begin{lesson}
\lettrine{O}{m}nípotens sempitérne Deus adésto magne pietátis mystériis, adésto sacraméntis et ad recreándos novos pópulos quos tibi fons baptismátis párturit Spíritum adoptiónis emítte: ut quod nostre humilitátis geréndum est, ministério tue virtútis impleátur
\end{lesson}

\emph{⟨Prosequatur ut sequitur.⟩}

\gregorioscore{909041.1-kyrie-vigilia-pasche-4}

\gregorioscore{g02767-vere-dignum}

\emph{Hic dividat aquam manu sua in modum crucis dicenda ⟨hoc quod sequitur hoc modo⟩\footnote{1882:88.} sic.}

\gregorioscore{a01384-qui-hanc-aquam-1}

\emph{Hic sacerdos ejiciat sua manu aquam de fonte in modum crucis in iiij. partes ⟨dicendo hoc modo⟩.\footnote{1882:89.}}

\gregorioscore{a01384-qui-hanc-aquam-2}

\emph{Hic mutet vocem suam quasi legendo.} Hec nobis precépta servántibus tu Deus omnípotens et\footnote{not in 1882:89.} clemens: adésto tu benígnus aspíra.

\emph{Hic aspiret in fontem ter in modum crucis.} Tu has símplices aquas tuo ore benedícito ut preter naturálem emundatiónem quam lavándis possunt adhibére corpóribus sint étiam purificándis méntibus efficáces.

\emph{Hic stillet de cereo in fontem in modum crucis postea dicat.}

\gregorioscore{a01384-qui-hanc-aquam-3}

\emph{Hic mittat cereum in medio\footnote{`\emph{infra}', 1882:89.} fontis crucem faciens, ⟨et postea prosequatur⟩.\footnote{1523:85r. `\emph{dicendo hoc quod sequitur hoc modo}', 1882:89. 1519:85r has after faciens `\emph{\&c.}'}}

\gregorioscore{a01385-hic-omnium-1}

\emph{Hic extrahatur cereus a\footnote{`\emph{extra}', 1882:89.} fontes dicendo ⟨sic⟩.\footnote{1882:89.}}

\gregorioscore{a01385-hic-omnium-2}

Per Dóminum nostrum Jesum Christum Fílium tuum: qui tecum vivit et regnat in unitáte Spíritus Sancti Deus. Per ómnia sécula seculórum. \emph{⟨℟.⟩} Amen.

\end{noinitial}

\emph{¶ Consecratis fontibus non infundetur oleum neque crisma nisi fuerit aliquis baptizandus. Completo fontium ministerio tres clerici de superiori gradu in cappis sericis videlicet duo in cappis rubeis et tertius in cappa alba\footnote{`\emph{alma}', 1519:85v.} in medio processionis in eundo cantent hanc letaniam simul: ita quod primus ℣. cantetur a predictis clericis antequam procedat processio hoc modo sequente.}

% TODO: Mode III. mundum notes together.

\label{a00176-rex-sanctorum-1}
\gregorioscore{a00176-rex-sanctorum-1}

\emph{Chorus idem repetat post unumquemque versum.}

\begin{noinitial}

\gregorioscore{a00176-rex-sanctorum-2}

\end{noinitial}

\emph{Deinde incipiatur missa sine regimine chori immediate cantore incipiente.} Missale:743.

