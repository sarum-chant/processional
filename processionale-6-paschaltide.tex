\chapter{¶ In die pasche.}

\emph{Ante matutinas\footnote{1528, \&c.---Ante missam. ⟨1882:91.⟩ Also 1519:86v. `Ante matutinas' appears in the Breviary.} et ante campanarum pulsationem conveniant clerici ad ecclesiam et accendantur omnia luminaria per ecclesiam. ij. excellentiores cum ceroferariis et thuribulario et clero circumstante ad sepulcrum accedant, et incensato prius sepulchro cum magna veneratione statim post thurificationem videlicet genuflectendo corpus Domini privatim super altare deponant. Iterum accipiant crucem de sepulchro et incipiat excellentior persona} Christus resúrgens. \pageref{001796-christus-resurgens-1}. \emph{cum quo eat processio ⟨per ostium⟩\footnote{1882:92.} presbyterii australe incedendo per medium chori et regrediens cum predicta cruce de sepulchro assumpta, inter duos sacerdotes predictos, super eorum brachia venerabiliter portata cum thuribulario et ceroferariis precedentibus per ostium presbyterii boreale, exeundo ad unum altare ex parte boreali ecclesie choro sequente habitu non mutato excellentioribus precedentibus, corpore vero Dominico super altare in pixide dimisso in subthesaurarii custodia, qui illud in predicta pixide in tabernaculo dependat, ut patet in ⟨ista⟩\footnote{1882:92.} statione sequenti ⟨precedente⟩.\footnote{1882:92.}}

\begin{figure}
\centering
\includegraphics{images/87r-matutinas-cum-cruce.png}
\caption{¶ Statio et ordo processionis in die pasche ante matutinas cum cruce.}
\end{figure}

\emph{Et tunc pulsentur omnes campane in classicum, et cantetur antiphona.}

\label{001796-christus-resurgens-1}
\gregorioscore{001796-christus-resurgens-1}

\emph{Chorus respondeat ⟨hoc modo⟩\footnote{1882:92.} sic ⟨quo sequitur⟩.\footnote{1882:92.}}

\begin{noinitial}

\gregorioscore{001796-christus-resurgens-2}

\end{noinitial}

\emph{Finita antiphona cum sua versu a toto choro dicat\footnote{`\emph{incipiat}', 1882:92.} excellentior persona in ipsa statione conversus ad altare versiculum.}

\emph{⟨℣.⟩} Surréxit Dóminus de sepúlchro.

\emph{⟨℟.} Qui pro nobis pepéndit in ligno.⟩\footnote{1882:92.}

Orémus.

\begin{lesson}
\subsubsection{⟨Oratio.⟩}

\label{deus-qui-pro-nobis}
\lettrine{D}{e}us qui pro nobis Fílium tuum crucis patíbulum subíre voluísti: ut inimíci a nobis expélleres potestátem, concéde nobis fámulis tuis: ut in resurrectiónis ejus gáudiis semper vivámus. Per eúndem Christum Dóminum nostrum. \emph{⟨℟.⟩} Amen. \emph{Nec precedat nec subsequatur} Dóminus vobíscum.
\end{lesson}

\emph{¶ Finita oratione omnes cum gaudio genuflectant ibidem, et ipsam crucem adorent.\footnote{1882:92. 1519:88r has `\emph{odorent}'.} In primis\footnote{`\emph{imprimis}', 1882:92.} digniores persone et secrete sine processione in chorum redeant. His itaque gestis discooperiantur omnes cruces per ecclesiam et omnes ymagines: et pulsentur campane ad matutinas more solito. Hac die ad matutinas nulla fiat processio ad crucem sicut in ceteris diebus.}

\emph{¶ In die pasche et\footnote{`\emph{atque}', 1882:93.} in omnibus domincis diebus ab hinc ad festum Trinitatis dicetur hec antiphona in aspersione aque benedicte cantore incipiente hoc modo.}

\grecommentary{1502:2r; 1530:4r; 1519:88r; GS:116; Grad-1508:110r; Missal-1513:86r\footnote{In the processionals `salvi' is st GBD.C. ⟨In 1508:110r `pervénit' is set Ba.AC.C; `salvi' is set GCCD.C; at `dicent' the letter `e' is missing; the first `allelúya' is set AG.G.GAABa.A. GS:116 gives only beginning of Psalm-Verse, plus the word `Amen' along with the Psalm-Tone ending.⟩}}

\gregorioscore{005403-vidi-aquam-1}

Vidi aquam. Glória Patri. \pageref{005403-vidi-aquam-2}. \emph{Et omnes ℣. et orationes ut supra in prima dominica adventus Domini.} \pageref{008166-ostende-nobis}.

\emph{Hac die hora vj. cantata et aqua benedicta aspersa ordinetur processio ad gradum chori sicut in die natalis Domini cum aque bajulo tribus crucibus, duobus ceroferariis et duobus thuribulariis. Subdyacono et dyacono utroque codicem evangelii deferentes, et eat processio per medium chori et ecclesie, circueundo ecclesiam et claustrum: tres clerici de superiore gradu prius in choro incipiant prosam modo sequente.}

\gregorioscore{a00177.5-salve-qua-deus-1}

\emph{Chorus idem repetat post unumquemque versum cantore incipiente} ⟨Salve festa dies⟩.\footnote{1882:93.} \emph{Quotienscunque dicitur} Salve festa dies. \emph{percantetur primus ℣. a predictis clericis in medio chori antequam procedat processio.}

\gregorioscore{a00177.5-salve-qua-deus-2}

\emph{In revertendo usque ad crucem in ecclesia per idem ostium qua egressa est dicatur hec antiphona cantore incipiente ⟨hoc modo⟩.\footnote{1882:94.}}

\label{004858-sedit-angelus}
\gregorioscore{004858-sedit-angelus}

\emph{Tres clerici de superiori gradu in cappis sericis ad populum conversi in pulpito dicant ⟨hunc⟩\footnote{1882:94.} versum ⟨modo quo sequitur⟩.\footnote{1882:94.}}

\gregorioscore{004858a-crucifixum-in-carne}

\emph{In introitu chori cantor incipiat antiphonam} Christus resúrgens. \emph{cum suo versu ⟨et percantetur⟩\footnote{1882:94.} a toto choro ut supra.} \pageref{001796-christus-resurgens-1}.

\emph{℣.} Surréxit Dóminus de sepúlchro.

\emph{℟.} Qui pro ⟨nobis pepéndit in ligno⟩.

\emph{⟨℣.⟩} ⟨Orémus.⟩\footnote{1901:88.}

\begin{lesson}
\subsubsection{Oratio.}

\lettrine{D}{e}us qui hodiérna die per Unigénitum tuum eternitátis nobis áditum devícta morte reserásti, vota nostra que preveniéndo aspíras: étiam adjuvándo proséquere. Per eúndem Christum ⟨Dóminum nostrum⟩. \emph{⟨℟.⟩} ⟨Amen⟩.\footnote{1882:94.}
\end{lesson}

\emph{¶ In die pasche ad vesperas fiat processio ad fontes per ostium australe presbyterii cum oleo et chrismate, ordinata processione cum cruce et ceroferariis et thuribulariis, exinde oleum et crisma a duobus dyaconis de secunda forma qui induti sint albis, deinde puer librum ferens superpelliceo induto: deinde executor officii: post illum rectores secundarii: deinde rectores principales. Nulla vero die per hanc hebdomadam precedat cereus paschalis processioni nec subsequatur secundum usum Sarum ecclesie ⟨nec⟩\footnote{1882:94.} ad vesperas nec ad matutinas.}

\emph{Rectores chori in eundo et redeundo incipiant in primis incipiant antiphonam ⟨in chorum⟩\footnote{1882:94.} ⟨incipiant antiphonam sequentem hoc modo.⟩ ⟨Imprimis vero incipiant antiphonam hoc modo, antequam procedat processio.⟩}

\gregorioscore{920112-alleluya-laudate-pueri-1}

\emph{Chorus ⟨totam⟩\footnote{1882:94.} prosequatur antiphonam antequam processio procedat sic.}

\begin{noinitial}

\gregorioscore{920112-alleluya-laudate-pueri-2}

\end{noinitial}

\emph{Qua finita rectores ex parte chori incipiant psalmum sequentem sic.\footnote{`\emph{hoc modo}', 1882:94.}}

\gregorioscore{920112-alleluya-laudate-pueri-3}

\emph{Et percantetur ab illa parte hoc modo ut sequitur hic.}

\begin{noinitial}

\gregorioscore{920112-alleluya-laudate-pueri-4}

\emph{Hic procedat processio: deinde dicatur alius versus ex alia parte chori hoc modo ⟨quo sequitur⟩.\footnote{1882:95.}}

\gregorioscore{920112-alleluya-laudate-pueri-5}

\emph{⟨Et⟩\footnote{1882:95.} sic dicatur totus psalmus cum} Glória Patri. \emph{et} Sicut erat. \emph{repetatur j.} allelúya. \emph{post suum versiculum semel dicendo non alternando ⟨sed modo superius notato⟩\footnote{1882:95.} ut patet.\footnote{`\emph{ut prius}', 1882:95.}}

\gregorioscore{920112-alleluya-laudate-pueri-6}

\end{noinitial}

\emph{Quo finito reincipiatur antiphona a rectoribus chori, et percantetur a toto choro.}

\emph{Hoc modo fiat statio ad fontes. In primis\footnote{`\emph{Imprimis}', 1882:95.} cruciferarius, deinde ceroferarius, deinde thuribularius, deinde oleum et crisma: deinde rectores secundarii, post ipsos vero tres pueri cantant} Allelúya. \emph{Ps.} Laudáte púeri.

\emph{Deinde ad gradum fontis orientalem puer ferens librum: deinde ad gradum fontis occidentalem executor officii. Post illum vero rectores principales. Ad fontes thurificandos thuribularius ad sacerdotem accedat quo facto redeat ⟨thurificandus⟩\footnote{1528:84v.} ad stationem suam, similiter ad versum et ad orationem dicendum, accedant ceroferarii ad sacerdotem, et dicta oratione resumant locum suum. Eodem vero ordine fiat consequens statio ante crucem exceptis rectoribus secundarii, qui stabunt proximi post sacerdotem: et exceptis tribus pueris qui cantaverunt} Allelúya. \emph{Sacerdos vero in fine psalmi} In éxitu Israel de Egípto. \emph{accedat ante cruciferarium ad thurificandum crucifixum: quo facto redeat ad locum suum ubi\footnote{`\emph{et ibi}', 1882:96.} dicat versiculum et orationem de cruce: et hoc modo faciat sacerdos per totam hebdomadam: ut patet in pictura vel in statione ⟨sequente⟩\footnote{1882:96.} hoc modo.}

\emph{¶ Sequitur statio.}

\begin{figure}
\centering
\includegraphics{images/92r-statio-fontes.jpg}
\caption{⟨¶ Hoc modo fiat statio ad fontes in die pasche, ut patet in pictura, et est procendum tali modo ut sequitur.⟩}
\end{figure}

\emph{¶ Deinde tres clerici\footnote{`\emph{pueri}', 1882:96.} in ipsa statione ante fontes conversi ad altare in superpelliceis simul cantent post psalmum} Laudate púeri Dóminum. \emph{hoc modo ut sequitur.}

\gregorioscore{g02078-alleluya-laudate-pueri-1}

\emph{Chorus repetatur.\footnote{`\emph{Chorus finiat}', 1882:96.}}

\begin{noinitial}

\gregorioscore{g02078-alleluya-laudate-pueri-2}

\emph{Pueri ⟨dicant hunc versum sequentem⟩.\footnote{1528:85v.}}

\gregorioscore{g02078-alleluya-laudate-pueri-3}

\emph{Chorus sic respondeat ⟨ut sequitur⟩.\footnote{1528:85v.}}

\gregorioscore{g02078-alleluya-laudate-pueri-4}

\emph{⟨Item⟩\footnote{1882:97.} pueri dicant sic ⟨ut sequitur⟩.}\footnote{1528:85v. At this point AS:239. includes a second verse: \gabcsnippet{(c4) Sit(hv) no(hv)men(gf~) Dó(hv)mi(jvhjgfh)ni(ev) be(hv)ne(hghfghg)díc(e)tum,(egfgffe) (,)
ex(d) hoc(fvfvfdgvFEdfhfgffvdfgef) (,)
nunc(fv) et(fv) us(fd)que(fv) in(fvfg) sé(evhv)cu(gfgEGfhhgfgevhdhgfgevhdGH)lum.(hj) (::)} In 1519:92v `nomen' is set EFG.GDG.}

\gregorioscore{g02078-alleluya-laudate-pueri-5}

\end{noinitial}

\emph{Non ulterius dicatur post repetitionem} Allelúya. \emph{sine neuma.}

\emph{Incensatis prius fontibus dicat sacerdos}

\emph{℣.} Surréxit Dóminus de sepúlchro.

\emph{℟.} Qui pro nobis pepéndit in ligno allelúya.

\emph{⟨℣.⟩} ⟨Orémus.⟩\footnote{1882:97.}

\begin{lesson}
\subsubsection{Oratio.}

\lettrine{P}{r}esta quésumus omnípotens Deus: ut qui resurrectiónis domínice sollénnia cólimus: ereptiónis nostre letíciam suscípere mereámur. Per eúndem Christum Dóminum nostrum. \emph{⟨Chorus respondent} Amen.⟩\footnote{1882:97.}
\end{lesson}

\emph{Nec procedat nec subsequatur} Dóminus vobíscum.

\emph{Deinde in eundo ad crucem ab omnibus rectoribus chori incipiatur antiphona sic. ⟨hoc modo.\footnote{`\emph{sequens sic dicendo}', 1882:97.⟩}}

\gregorioscore{920113-alleluya-in-exitu-1}

\emph{Chorus prosequatur ⟨sic⟩.\footnote{1882:97.}}

\begin{noinitial}

\gregorioscore{920113-alleluya-in-exitu-2}

\emph{Que licet brevis sic terminetur a choro. Statim rectores ex parte decani ad chorum conversi simul incipiant ⟨hunc⟩\footnote{1882:97.} psalmum ⟨sequentem hoc modo⟩.\footnote{1882:97.}}

\gregorioscore{920113-alleluya-in-exitu-3}

\emph{Chorus ex parte predicta ℣. prosequatur ⟨ut sequitur⟩.\footnote{`\emph{hoc modo}', 1882:97.}}

\gregorioscore{920113-alleluya-in-exitu-4}

\emph{Hinc procedat processio: alius ℣. ex alia parte chori dicens sic.}

\gregorioscore{920113-alleluya-in-exitu-5}

\emph{Et hoc modo dicatur totus psalmus cum} Glória Patri. \emph{et} Sicut erat. \emph{⟨cum⟩\footnote{1882:97.} uno} Allelúya. \emph{tantum\footnote{`\emph{tamen}', 1882:97.} post unumquemque versum \&c.}

\gregorioscore{920113-alleluya-in-exitu-6}

\emph{Hic thurificet sacerdos crucifixum.}

\gregorioscore{920113-alleluya-in-exitu-7}

\end{noinitial}

\label{008013-dicite-in-nationibus}
\cantusnote{008013}
\emph{Quo finito dicat sacerdos ℣.} Dícite in natiónibus.

\emph{℟.} Quia Dóminus regnávit a ligno, allelúya.

\emph{⟨℣.⟩} ⟨Orémus.⟩\footnote{1882:99.}

\begin{lesson}
\subsubsection{Oratio.}

\lettrine{D}{e}us qui pro nobis Fílium tuum crucis patíbulum subíre voluísti: ut inimíci a nobis expélleres potestátem concéde nobis fámulis tuis ut in resurrectiónis ejus gáudiis semper vivámus. Per eúndem Christum Dóminum nostrum. \emph{⟨℟.⟩} Amen.
\end{lesson}

\emph{Nec precedat nec subsequatur} Dóminus vobíscum.

\emph{In introitu chori dicatur antiphona de sancta Maria. Ant.} Alma redemptóris. \emph{terminata cum} allelúya. 293.

\label{800385-sancta-dei}
\cantusnote{800385}
\emph{℣.} Sancta Dei génitrix ⟨virgo semper María.

\emph{℟.} Intercéde pro nobis. \emph{et cetera.⟩}\footnote{1882:99.} 300.

\emph{⟨℣.⟩} ⟨Orémus.⟩\footnote{1882:99.}

\begin{lesson}
\subsubsection{Oratio.}

\label{gratiam-tuam}
\lettrine{G}{r}átiam tuam quésumus Dómine méntibus nostris infúnde: ut qui ángelo nunciánte Christi Fílii tui incarnatiónem cognóvimus: per passiónem ejus et crucem ad resurrectiónis glóriam perducámur. Per eúndem ⟨Dóminum nostrum Jesum Christum Fílium. \emph{et cetera⟩.\footnote{G1882:99.}}
\end{lesson}


\chapter{¶ ⟨Feria secunda.⟩}

\label{feria-ii-processio}
\emph{Feria ij. et quotidie per hanc hebdomadam fiat processio ad matutinas per medium chori ante crucifixum ad ostium occidentale: cum accolito deferente crucem in superpelliceo, ceroferariis et thuribulario ⟨albis indutis⟩,\footnote{1882:99.} cantando antiphonam} Christus resúrgens. \pageref{001796-christus-resurgens-1}. \emph{Hac die dicatur ℣.} Dicant nunc. \emph{a duobus de superiori gradu in superpelliceis, ante introitum chori ad clerum conversi. Sequentibus vero duobus diebus a duobus clericis de ij. forma loco et habitu supradictis. Tamen in v. et vj. feria et sabbato dicatur sine ℣. Finito ℣. incensato prius crucifixo dicat sacerdos ℣.} Dícite in natiónibus. \pageref{008013-dicite-in-nationibus}. \emph{Oratio.} Deus qui pro nobis. \pageref{deus-qui-pro-nobis}.

\emph{In introitu ⟨chori⟩\footnote{1528:88v.} de sancta Maria antiphona.} Ave regína celórum. 292. \emph{℣.} Post partum. Brev:{[}192{]}. \emph{Oratio.} Grátiam tuam. \pageref{gratiam-tuam}.

\emph{Ad vesperas processio fiat ad fontes eodem modo et ordine quo in die pasche ad vesperas cum oleo et chrismate cum cruce et ceroferariis et thuribulario cantando antiphonam} Sedit Angelus. \pageref{004858-sedit-angelus}. \emph{sine ℣. rectoribus incipientibus: excepto quod hac die pueri non cantent} Allelúya. \emph{in statione ante fontes nec in eundo post psalmum} Laudáte púeri ⟨Dóminum⟩.\footnote{1882:100.} \emph{nec ante crucem psalmum} In éxitu ⟨Israel⟩.\footnote{1882:100.} \emph{sed finito antiphona ⟨predicta⟩\footnote{1882:100.} incensatis prius fontibus, dicat sacerdos}

\emph{℣.} Surréxit Dóminus de sepúlchro.

\emph{℟.} Qui pro nobis ⟨pepéndit in ligno⟩.

\emph{⟨℣.⟩} Orémus.

\begin{lesson}
\subsubsection{Oratio.}

\lettrine{C}{o}ncéde quésumus omnípotens Deus ut festa paschália que venerándo cólimus: étiam vivéndo teneámus. Per Christum.
\end{lesson}

\emph{Deinde usque ad crucem antiphona} Christus resúrgens. \pageref{001796-christus-resurgens-1}. \emph{sine ℣. qua dicta incensato prius crucifixo dicat sacerdos versum} Dícite in natiónibus. \pageref{008013-dicite-in-nationibus}. \emph{Oratio.} Deus qui ⟨pro⟩\footnote{1882:100.} nobis. \pageref{deus-qui-pro-nobis}.

\emph{In introitu chori de sancta Maria antiphona.} Anima mea liquefácta est. 299.

\emph{℣.} ⟨Sancta Dei génitrix.⟩\footnote{1882:100.} \pageref{800385-sancta-dei}. \emph{et oratio} ⟨Grátiam tuam.⟩\footnote{1882:100.} \emph{ut supra ad vesperas in die pasche.} \pageref{gratiam-tuam}.

\emph{Hic ordo servetur de processionibus in eundo et ⟨in⟩\footnote{1882:100.} introitu chori quotidie ad vesperas usque ad sabbatum cum propriis orationibus ad fontes et diversis antiphonis in introitu chori.}


\chapter{¶ Feria iij.}

\emph{Fiat processio ad matutinas ut supra.} \pageref{feria-ii-processio}.

\emph{In redeundo de sancta Maria antiphona} Beáta Dei génitrix. 291. \emph{℣. et oratio ut supra.} \pageref{gratiam-tuam}.

\emph{¶ Feria iij.\footnote{`\emph{Feria ij.}' 1519:96r.} ad vesperas ad fontes.}

\begin{lesson}
\subsubsection{Oratio.}

\lettrine{P}{r}esta quésumus omnípotens Deus: ut per hec festa paschália que cólimus: devóti semper in tua laude vivámus. Per Christum Dóminum nostrum.
\end{lesson}

\emph{In redeundo antiphona} Descéndi ⟨in hortum⟩. 300. \emph{⟨Versus et oratio ut supra.⟩}\footnote{1882:100.} \pageref{800385-sancta-dei}.


\chapter{¶ Feria iiij.}

\emph{Ad matutinas fiat processio ut supra.} \pageref{feria-ii-processio}.

\emph{In redeundo de sancta Maria antiphona} Speciósa. 294. \emph{Versus et oratio ut supra.} \pageref{800385-sancta-dei}.

\emph{Ad vespere ad fontes.}

\begin{lesson}
\subsubsection{Oratio.}

\lettrine{C}{o}ncéde quésumus omnípotens Deus: ut qui festa paschília ágimus: celéstibus desidériis accénsi, fontem vite sitiámus\footnote{`sentiámus', 1882:101.} Dóminum nostrum Jesum Christum Fílium tuum.\footnote{1519:96v adds `Qui tecum', which seems contradictory to the rubric that follows.} \emph{Non dicatur ulterius.}
\end{lesson}

\emph{In redeundo\footnote{`\emph{eundo}', 1519:96v.} de sancta Maria antiphona} Alma redemptóris mater.\footnote{1882:101.} 293. \emph{℣. et oratio ut supra.} \pageref{800385-sancta-dei}.


\chapter{¶ Feria v.}

\emph{Ad matutinas fiat processio ut supra.} \pageref{feria-ii-processio}.

\emph{In redeundo antiphona} Ave regína. 292. \emph{℣. et oratio ut supra.} \pageref{800385-sancta-dei}.

\emph{⟨Ad vesperas⟩\footnote{1882:101.} ad fontes.}

\begin{lesson}
\subsubsection{Oratio.}

\lettrine{D}{a} quésumus omnípotens Deus: ut ecclésia tua et suórum firmitáte membrórum, et nova semper fecunditáte letétur. Per Christum Dóminum nostrum. \emph{⟨℟.⟩} Amen.
\end{lesson}

\emph{In revertendo de sancta Maria antiphona} Anima mea. 299. \emph{℣. et oratio ut supra.} \pageref{800385-sancta-dei}.


\chapter{¶ Feria vj.}

\emph{Ad matutinas\footnote{`\emph{missam}', 1519:96v.} processio.} \pageref{feria-ii-processio}.

\emph{In revertendo antiphona} Beáta Dei génitrix. 291. \emph{℣. et oratio ut supra.} \pageref{800385-sancta-dei}.

\emph{Ad vesperas ad fontes.}

\begin{lesson}
\subsubsection{Oratio.}

\lettrine{A}{d}ésto quésumus Dómine famílie tue et dignánter impénde: ut quibus grátiam fidéi contulísti et corónam largiáris etérnam. Per Christum Dóminum nostrum. \emph{⟨℟.⟩} Amen.
\end{lesson}

\emph{In introitu chori ⟨dicatur⟩\footnote{1528:89r.} de sancta Maria antiphona} Descéndi ⟨in ortum meum⟩. 300. \emph{℣. et oratio ut supra.} \pageref{800385-sancta-dei}.


\chapter{Sabbato.}

\emph{Ad matutinas fiat processio ut supra.} \pageref{feria-ii-processio}.

\emph{In revertendo antiphona} Speciósa. 294. \emph{℣. et oratio ut supra.} \pageref{800385-sancta-dei}.

\emph{Ad vesperas non fiat processio ad fontes cum oleo et crismate sicut in precedentibus diebus: sed tantum eat processio ante crucem, et exeat processio per medium chori cum ceroferariis et thuribulario ⟨albis indutis⟩\footnote{1882:101.} sine cruce, et puer librum deferens ante sacerdotem in superpelliceo: sacerdos autem in simili habitu cum cappa serica choro sequente in superpellicio cantando antiphonam} Christus resúrgens. \pageref{001796-christus-resurgens-1}. \emph{Rectoribus chori in medio processionis in cappis sericis incipientes. Finita antiphona duo clerici de superiori gradu in superpelliciis ante ostium chori ad populum conversi, dicant versum} Dicant nunc ⟨Judéi⟩.\footnote{1882:102.} \emph{Deinde dicat sacerdos ⟨post⟩\footnote{1882:102.} thurificationem crucifixi ℣.} Dícite in natiónibus. \pageref{008013-dicite-in-nationibus}. \emph{Oratio.} Deus qui pro nobis. \pageref{deus-qui-pro-nobis}.

\emph{In introitu chori dicatur una de predictis antiphonis per ordinem de sancta Maria ut supra.} ⟨Descéndi in ortum⟩. 300. \emph{℣.} Sancta Dei génitrix. \pageref{800385-sancta-dei}. \emph{Oratio.} Grátiam tuam. \pageref{gratiam-tuam}.

\emph{Et hoc in omnibus sabbatis omnium ebdomadarum observetur ad vesperas in eundo et in introitu chori usque ad ascensionem Domini: sive de dominica fit servitium sive de festo sanctorum: nisi in festo sancte crucis ad utrasque vesperas quod si in sabbato contigerit nulla fiat processio tunc ad secundas vesperas: et nisi quod in ceteris sabbatis dicatur versus} Dicant nunc ⟨Judéi⟩.\footnote{1882:102.} \emph{a duobus de secunda forma, loco et habitu observato: ad clerum conversis. Proxima tamen dominica ante ascensionem Domini dicatur ℣.} Dicant nunc ⟨Judéi⟩.\footnote{1882:102.} \emph{a duobus de superiori gradu, loco et habitu predicto.}

\emph{Similiter fiat omnibus duplicibus festis in dominicis tempore predicto contingentibus. Preterea rectores chori precedant in capis nigris\footnote{`\emph{habitu non mutato}', 1882:102.} in omnibus sabbatis sequentibus nisi duplex festum fuerit.}


\chapter{¶ Dominica in octavis pasche.}

\emph{Eat processio in cappis sericis sicut in omnibus festis duplicibus: que in dominicis contingunt per medium chori, circueundo chorum et claustrum sicut in die nativitatis Domini: cantando antiphonam} Sedit ángelus. 207. \emph{Tres clerici de superiori gradu dicant ℣.} Crucifíxum in carne. \emph{in pulpito sicut in die pasche.}

\emph{In introitu chori antiphona.} Christus resúrgens. \pageref{001796-christus-resurgens-1}. \emph{cum suo versu a toto choro.}

\emph{℣.} Surréxit Dóminus. 209.

\emph{Oratio.} Deus qui per Unigénitum. 209.


\chapter{¶ ⟨Dominica prima post octavas pasche.⟩}

\emph{Dominica prima post octavas pasche, et omnibus dominicis usque ad proximam dominicam ante ascensionem Domini quando de dominica agitur ad processionem in eundo cantor incipiat sequentem antiphonam hoc modo.}

\gregorioscore{a01129-ego-sum-alpha}

\emph{Duo clerici de secunda forma in superpelliciis ante introitum chori conversi ad populum dicant versum.}

\begin{noinitial}

\gregorioscore{a01129a-ego-sum}

\end{noinitial}

\emph{In introitu chori antiphona} Christus resúrgens. \emph{sine versu.} \pageref{001796-christus-resurgens-1}.

\emph{℣.} Surréxit Dóminus de sepúlchro. 209.

\emph{Oratio.} Deus qui per Unigénitum. 209.

\emph{Eodem modo fiat processio omnibus diebus dominicis usque ad dominicam ante ascesionem Domini quando de dominica agitur. Quando ultimo fit servitium de dominica ante ascensionem Domini ad processionem antiphona} Sedit ángelus. 207. \emph{Tres clerici in pulpito de superiori gradu ad populum conversi in superpelliciis dicant ℣.} Crucifíxum.

\emph{In introitu chori antiphona.} Christus resúrgens. \pageref{001796-christus-resurgens-1}. \emph{cum suo versu a toto choro.}

\emph{℣.} Surréxit Dóminus de sepúlchro. 209.

\emph{Oratio.} Deus qui per Unigénitum. 209.


\chapter{¶ Feria ij. in rogationibus.}

\emph{Si vacaverit: post sextam dicatur missa ⟨dominicalis scilicet⟩\footnote{1882:103.}} Vocem jocunditátis. \emph{ut supra in dominica.\footnote{`\emph{ut infra.}' 1882:103.}} 248. \emph{: nona cantata et peracta: omnibus que ad processionem pertinent ad gradum chori ordinetur processio cum aque bajulo cum cappa sua nigra cum cruce: ceroferariis in albis et thuribularius: deinde capsule reliquiarum deferantur a duobus dyaconis de secunda forma habitu non mutato: post hec dyaconus et subdyaconus cum sacerdote: omnibus albis indutis procedant, et processione per medium chori et ecclesie ⟨incedant⟩\footnote{1882:104.} et exeat processio per ostium ecclesie occidentale et per portam claustri\footnote{'Should probably run \emph{per portam} clausi \emph{borealem. Clausum} and \emph{claustrum} are confused in the Consuetudinary. ⟨1882:xv.⟩} borealem: ad aliquam ecclesiam in civitate, cantando antiphonas sequentes. Preterea in principio processionis deferatur dracho tribus vexillis rubeis precedentibus, ij. loco leo, iij. loco cetera vexilla: deinde sequatur processio suo ordine eodem modo et habitu sicut dicto: preter capsulam reliquiarum: ita tamen quod ⟨sacerdos⟩\footnote{1882:104.} absque cappa serica incedat ⟨ut patet in statio sequens⟩.\footnote{1528:90v.}}

\begin{figure}
\centering
\includegraphics{images/98v-ordo-rogationibus.jpg}
\caption{¶ Ordo processionis in secunda feria in rogationibus.}
\end{figure}

\emph{¶ Hec antiphona sequens dicatur a toto choro in stallis ⟨antequam exeat processio, cantore incipiente⟩.\footnote{1882:105.}}

\gregorioscore{002822-exurge-domine}

\emph{Non dicatur nisi primus versus sed statim sequatur} Glória Patri. 426. \emph{Deinde repetatur antiphona} Exúrge Dómine. \emph{et sic post ceteras antiphonas fiat que sequuntur. His dictis fiat processio cantore incipiente ⟨antiphonam sequentem⟩\footnote{1882:105.} hoc modo ⟨ut sequitur⟩.\footnote{1528:91v.}}

\gregorioscore{204825-surgite-sancti}

Glória Patri. 429. \emph{Repetatur antiphona.}

\emph{Alia antiphona.}

\gregorioscore{002109-de-hierusalem}

\emph{⟨Et repetatur antiphona.⟩\footnote{1882:105.}}

\emph{⟨Alia antiphona.⟩}

\gregorioscore{202443-in-nomine}

\emph{⟨cum} Glória Patri. \emph{et} Sicut erat.⟩\footnote{1882:105. This could be sung as in a psalm, i.e.~in two verses, or as in an officium, i.e.~a single verse, as follows:

  \gregorioscore{202443-gloria-patri}}

\emph{Si necesse fuerit ad pluviam postulandam cantentur hec sequens antiphonam, sinautem non dicitur.}

\gregorioscore{002376-domine-rex}

⟨Glória Patri. Sicut erat. 426. \emph{Et repetatur antiphona.⟩\footnote{1882:105.}}

\emph{⟨Alia antiphona.⟩}

\gregorioscore{003971-nunquid-est}

⟨Glória Patri. Sicut erat. 427. \emph{Et repetatur antiphona.⟩\footnote{1882:106.}}

\emph{⟨Alia antiphona.⟩}

\gregorioscore{002768-exaudi-domine}

⟨Glória Patri. Sicut erat. 427.

\emph{Alia antiphona.⟩\footnote{1882:106.}}

\gregorioscore{a01183.1-respice-domine}

\emph{Ad ⟨hanc⟩\footnote{1882:106.} antiphonam non sequatur psalmus.}

\emph{¶ Pro serenitate aeris ⟨dicuntur he sequentes antiphone⟩.\footnote{1882:106.}}

\gregorioscore{003393-inundaverunt-aque}

⟨Glória Patri. Sicut erat. 426. \emph{Et repetatur antiphona.}

\emph{Alia antiphona.⟩\footnote{1882:106.}}

\gregorioscore{003925-non-nos-demergat}

⟨Glória Patri. Sicut erat. 428.⟩\footnote{1882:106.}

\emph{¶ Contra mortalitatem hominum tempore belli dicatur hec antiphona.}

\gregorioscore{003615-libera-domine}

\emph{⟨cum} Gloria Patri. 427.⟩\footnote{1882:107.}

\emph{⟨Contra hostum impugnacionem.}
% TODO: chk

\gregorioscore{003400-invocantes-dominum}

\emph{Pro quacumque tribulacione.}

\gregorioscore{004219-parce-domine}⟩\footnote{Rylands-24:230.}

\emph{His dictis dicant septem psalmos penitentiales quantum\footnote{`\emph{si tantum}', 1882:107.} restat iter: cum} Glória Patri. \emph{et} Sicut erat. \emph{et antiphona} Ne reminiscáris. \emph{Require hos psalmos cum antiphona in quarta feria in capite jejunii:\footnote{`\emph{in fine libri}', 1882:107.}} \pageref{chap:capite-jejunii}. \emph{et dicatur cum hac letania et ℣. et oratione sequente: et hec omnia sine nota dicantur: ita tamen quod quicquid dicat sacerdos de letania, chorus idem repetat plene et integre usque ad prolationem} Peccatóres te rogámus audi nos. \emph{tunc enim post} Ut pacem nobis dones. \emph{Chorus respondent,} Te rogámus audi nos. \emph{Et ⟨sic⟩\footnote{1882:107.} post singulos ℣. usque ad} Fili Dei. Te rogámus ⟨audi nos⟩.\footnote{1882:107.}

\section[⟨Sequitur⟩ Letania.]{⟨Sequitur⟩\footnote{1528:95v.} Letania.}

\begin{lesson}

\lettrine{K}{y}rieléyson. Christeléyson. Christe audi nos.

Pater de celis Deus. Miserére nobis.

Fili Redémptor mundi Deus. Miserére nobis.

Spíritus Sancte Deus. Miserére nobis.

Sancta Trínitas unus Deus. Miserére nobis.

Sancta María. \hfill ora.

Sancta Dei génitrix. \hfill ora.

Sancta virgo vírginum. \hfill ora.

Sancte Míchael. \hfill ora.

Sancte Gábriel. \hfill ora.

Sancte Ráphael. \hfill ora.

Omnes sancti ángeli et archángeli Dei.\footnote{1882:107 omits `Dei'.} \hfill Oráte pra nobis.

Omnes sancti beatórum spirítuum órdines. \hfill Oráte pro.

Sancte Johánnes baptísta. \hfill ora.

Omnes sancti patriárche et prophéte. \hfill Oráte pro nobis.

Sancte Petre. \hfill ora.

Sancte Paule. \hfill ora.

Sancte Andréa. \hfill ora.

Sancte Johánnes. \hfill ora.

Sancte Jacóbe. \hfill ora.

Sancte Thoma. \hfill ora.

Sancte Philíppe. \hfill ora.

Sancte Jacóbe. \hfill ora.

Sancte Mathée. \hfill ora.

Sancte Bartholomée. \hfill ora.

Sancte Simon. \hfill ora.

Sancte Thadée. \hfill ora.

Sancte Mathía. \hfill ora.

Sancte Bárnaba. \hfill ora.

Sancte Marce. \hfill ora.

Sancte Luca. \hfill ora.

Omnes sancti apóstoli et evangelíste. \hfill Oráte ⟨pro nobis⟩.\footnote{1882:108.}

Omnes sancti discípuli Dómini et innocéntes. \hfill Oráte pro nobis.

Sancte Stéphane. \hfill ora.

Sancto Line. \hfill ora.

Saucte Clete. \hfill ora.

Sancte Clemens. \hfill ora.

Sancte Fabiáne. \hfill ora.

Sancte Sebastiáne. \hfill ora.

Sancte Cosma. \hfill ora.

Sancto Damiáne. \hfill ora.

Sancte Prime. \hfill ora.

Sancte Feliciáne. \hfill ora.

Sancto Dionísi cum sóciis tuis. \hfill Oráte\footnote{`Ora' 1882:108.} ⟨pro nobis⟩.\footnote{1528:95v.}

Sancte Victor cum sóciis tuis. \hfill Oráte.\footnote{`Ora' 1882:108.}

Omnes sancti mártyres. \hfill Oráte ⟨pro nobis⟩.

Sancte Silvéster. \hfill ora.

Sancte Leo. \hfill ora.

Sancte Hierónime. \hfill ora.

Sancte Augustíne. \hfill ora.

Sancte Ysidóre. \hfill ora.

Sancte Juliáne. \hfill ora.

Sancte Medárde. \hfill ora.

Sancte Gildárde. \hfill ora.

Sancte Albíne. \hfill ora.

Sancte Eusébi. \hfill ora.

Sancte Svvythún. \hfill ora.

Sancte Biríne. \hfill ora.

Omnes sancti confessóres. \hfill oráte.

Omnes sancti monáchi et heremíte. \hfill Oráte pro nobis.

Sancta María Magdaléna. \hfill ora.

Sancta María Egýptia. \hfill ora.

Sancta Margaréta. \hfill ora.

Sancta Scolástica. \hfill ora.

Sancta Petronílla. \hfill ora.

Sancta Genovéfa. \hfill ora.

Sancta Praxédis. \hfill ora.

Sancta Sothéris. \hfill ora.

Sancta Prisca. \hfill ora.

Sancta Tecla. \hfill ora.

Sancta Affra. \hfill ora.

Sancta Edítha. \hfill ora.

Omnes sancte vírgines. \hfill oráte.

Omnes sancti. \hfill Oráte pro.

Propítius esto. \hfill Parce nobis Dómine.

Ab omni malo. \hfill Líbera nos Dómine.

Ab insídiis diáboli. \hfill Líbera nos.

A damnatióne perpétua. \hfill Líbera.

Ab imminéntibus peccatórum nostrórum perículis. \hfill Líbera.

Ab infestatiónibus demónum. \hfill Líbera.

A spíritu fornicatiónis. \hfill Líbera.

Ab appetítu inánis glórie. \hfill Líbera.

Ab ⟨omni⟩\footnote{1882:109; 1531:49r.} immundítia mentis et córporis. \hfill Líbera.

Ab ira et ódio et omni mala voluntáte. \hfill Líbera.

Ab immúndis cogitatiónibus. \hfill Líbera.

A cecitáte cordis. \hfill Líbera.

A fúlgure et tempestáte. \hfill Líbera.

A subitánea et improvísa morte. \hfill Líbera.

Per mystérium sancte incarnatiónis tue. \hfill Líbera.

Per nativitátem tuam. \hfill Líbera.

Per circumcisiónem tuam. \hfill Líbera.

Per baptísmum tuum. \hfill Líbera.

Per jejúnium tuum. \hfill Líbera.

Per crucem et passiónem tuam. \hfill Líbera.

Per pretiósam mortem tuam. \hfill Líbera.

Per gloriósam resurrectiónem tuam. \hfill Líbera.

Per admirábilem ascensiónem tuam. \hfill Líbera.

Per grátiam Sancti Spíritus Paráclyti. \hfill Líbera.

In hora mortis. \hfill Succúrre nobis Dómine.\footnote{`In hora mortis \ldots{} Domine' usually does not appear in the Roman Breviaries (however it is to be found in the Pontificale Romanum Distributum Clementis 8 (Rome:1738) Vol 1:166.). The sources are not uniform in either punctuation or capitalization, ranging from `In hora mortis succúrre nobis Dómine' through to `In hora mortis. Succúrre nobis Dómine.' No indication of the musical setting is provided in the sources.}

In die judícii. \hfill Líbera.

Peccatóres. \hfill Te rogámus audi nos.

Ut pacem nobis dones. \hfill Te rogámus audi nos.

Ut misericórdia et píetas tua nos custódiat. \hfill Te rogámus.

Ut ecclésiam tuam cathólicam régere et defensáre dignéris. \hfill Te rogámus.

Ut domnum apostólicum et omnes gradus ecclésie in sancta\footnote{`sanctam', 1519:104r.} religióne conserváre dignéris. \hfill Te rogámus.

Ut regibus et princípibus nostris pacem et veram concórdiam atque victóriam donáre dignéris. \hfill Te rogámus.

Ut epíscopos et abbátes nostros in sancta religióne conserváre dignéris. \hfill Te rogámus.

Ut congregatiónes ómnium sanctórum morum in tuo sancto servítio conserváre dignéris. \hfill Te rogámus.

Ut cunctum pópulum Christiánum precióso sánguine tuo redémptum conserváre dignéris. \hfill Te rogámus.

Ut ómnibus benefactóribus nostris sempitérna bona retríbuas. \hfill Te rogámus.

Ut ánimas nostras et paréntum nostrórum ab etérna damnatióne erípias. \hfill Te rogámus.

Ut fructus terre dare et conserváre dignéris. \hfill Te rogámus.

Ut óculos misericórdie tue super nos redúcere dignéris. \hfill Te rogámus.

Ut obséquium servitútis nostre rationábile fácias. \hfill Te rogámus.

Ut mentes nostras ad celéstia desidéria érigas. \hfill Te rogámus.

Ut misérias páuperum et captivórum intúeri et releváre dignéris. \hfill Te rogámus audi nos.

Ut ómnibus fidélibus defúnctis réquiem etérnam dones. \hfill Te rogámus.

Ut nos exaudíre dignéris. \hfill Te rogámus.

Fili Dei. \hfill Te rogámus.

Agnus Dei qui tollis peccáta mundi. \hfill Exáudi nos Dómine.

Agnus Dei qui tollis peccáta mundi. \hfill Parce nobis Dómine.

Agnus Dei qui tollis peccáta mundi. \hfill Miserére nobis.

\end{lesson}

Kyrieléyson. Christeléyson. Kyrieléyson.

Pater noster.

Et ne nos indúcas ⟨in tentatiónem⟩.\footnote{1882:111.} Sed líbera nos ⟨a malo⟩.\footnote{1882:111.}

Osténde nobis Dómine misericórdiam tuam. Et\footnote{1882:111 omits `Et'.} salutáre tuum da nobis.

Peccávimus cum pátribus nostris. Injúste égimus ⟨iniquitátem fécimus⟩.\footnote{1882:111.}

Dómine non secúndum peccáta ⟨nostra fácias nobis⟩.\footnote{1882:111.} Neque secúndum ⟨iniquitátes nostras retríbuas nobis⟩.\footnote{1882:111.}

Orémus pro omni gradu ⟨ecclésie⟩.\footnote{1882:111.} Sacerdótes tui ⟨induántur justítiam et sancti tui exúltent⟩.\footnote{1882:111.}

Pro frátribus ⟨et soróribus nostris⟩.\footnote{1882:111.} Salvos fac servos tuos ⟨et ancíllas tuas Deus meus sperántes in te.⟩\footnote{1882:111.}

Pro cuncto pópulo Christiáno. Salvum fac pópulum ⟨tuum Dómine et bénedic hereditáti tue: et rege eos et extólle illos usque in etérnum⟩.\footnote{1882:111.}

Dómine fiat pax ⟨in virtúte tua⟩.\footnote{1882:111.} Et habundántia ⟨in túrribus tuis⟩.\footnote{1882:111.}

Anime famulórum famularúmque tuárum requiéscant in pace. Amen.

Dómine exáudi oratiónem meam. Et clamor meus ad te veniat.

Dóminus vobíscum. Et cum spíritu tuo.

Orémus.

\begin{lesson}

\subsubsection{Oratio.}

\lettrine{D}{e}us cui próprium est miseréri semper et párcere súscipe deprecatiónem nostram, et\footnote{`ut', 1882:111.} quos delictórum cáthena constríngit: miserátio tue piétatis absólvat. Per Christum Dóminum nostrum.

\subsubsection[⟨Oratio.⟩]{⟨Oratio.⟩\footnote{1882:111.}}

\lettrine{O}{m}nípotens sempitérne Deus: qui facis mirabília magna solus, preténde super fámulos tuos pontífices: et super cunctas congregatiónes illis commíssas spíritum grátie salutáris: et ut in veritáte tibi compláceant, perpétuum eis rorem tue benedictiónis infúnde.

\subsubsection[⟨Oratio.⟩]{⟨Oratio.⟩\footnote{1882:111.}}

\lettrine{D}{e}us qui charitátis dona per grátiam Sancti Spíritus tuórum córdibus fidélium infúndis: da fámulis et famulábus tuis frátribus et soróribus nostris pro quibus tuam deprecámur cleméntiam salútem mentis et córporis: ut te tota virtúte díligant: et que tibi plácita sunt tota dilectióne perfíciant.

\subsubsection[⟨Alia oratio.⟩]{⟨Alia oratio.⟩\footnote{1882:112.}}

\lettrine{D}{e}us a quo sancta desidéria recta consília ⟨et justa sunt ópera, da servis tuis illam quam mundas dare non potest pacem: ut et corda nostra mandátis tuis dédita, et hóstium subláta formídire: témpora sint tua protectióne tranquílla⟩.\footnote{1882:112. 1519:105r has `\emph{\&c.}'}

\lettrine{I}{n}effábilem misericórdiam tuam nobis quésumus Dómine cleménter osténde: ut simul nos ⟨et⟩\footnote{I1882:112.} a peccátis ⟨ómnibus⟩\footnote{1882:112.} éxuas et a penis quas pro his merémur benígnus erípias.

\lettrine{F}{i}délium Deus ómnium Cónditor et Redémptor animábus ⟨famulórum famularúmque tuárum remissiónem cunctórum tribus peccatórum: ut indulgéntiam quam semper optavérunt piis supplicatiónibus consequántur⟩.\footnote{1882:112. 1519:105r has `\emph{\&c.}'}

\lettrine{P}{i}etáte tua quésumus Dómine nostrórum solve víncula ómnium delictórum: et intercedénte beáta et gloriósa sempérque vírgine Dei genitríce María: cum ómnibus sanctis tuis: dómnum papam,\footnote{In Gough Missals 75 `papam' has been deleted.} reges et príncipes, epíscopos et abbátes nostros, et omnem plebem illis commíssam, nosque fámulos tuos, atque loca nostra in omni sanctitáte custódi omnésque consanguinitáte ac familiaritáte, vel confessióne et oratióne nobis junctos: seu omnem pópulum cathólicum a vítiis purga virtútibus illústra, pacem et salútem nobis tríbue, hostes visíbiles et invisíbles rémove: famem et pestem repélle: amícis et inimícis nostris veram charitátem: atque infírmis sanitátem largíre: et iter famulórum tuórum in salútis tue prosperitáte dispóne: et ómnibus fidélibus vivis ac defúnctis in terra vivéntium vitam et réquiem etérnam concéde. Per eúndem Christum Dóminum nostrum. \emph{⟨℟.⟩} ⟨Amen.⟩\footnote{1882:112.}

\end{lesson}

\emph{Et si transierit processio per ecclesiam dicatur responsorium cum suo ℣. antiphona cum versiculum et oratione de sancto ⟨de⟩\footnote{1882:112.} quo est ecclesia illa: ita quod ad januas cimiterii vel citius inchoetur non ibi processio circa cimiteriam sed directe in ecclesia. Cum autem pervenerit in ecclesia ubi fieri debet statio: et finita antiphona vel responsorio, dicat sacerdos ℣. et orationem de sancto de quo est ecclesia illa, deinde sequantur preces in prostratione cum dicitur.} Kyrieléyson. Christeléyson. Kyrieléyson. Pater noster. \emph{\&c.~ut prenotatum est in processione feriali quadragesime.} 82.

\emph{¶ His finitis incipiatur missa de jejunio, ⟨cantore incipiente⟩\footnote{1528:97v.} hoc modo ⟨ut sequitur⟩.\footnote{1528:97v.}}

\gregorioscore{g01075-exaudivit-de-templo}

\begin{lesson}
\subsubsection{Oratio.}

\lettrine{P}{r}esta quésumus omnípotens Deus ut ⟨qui⟩\footnote{1882:113.} in afflictióne nostra de tua pietáte confídimus: contra ómnia advérsa\footnote{`advérsa ómina', 1882:113.} tua semper protectióne muniámur. Per Dóminum ⟨nostrum Jesum Christum Fílium tuum. \emph{et cetera.⟩\footnote{1882:113.}}
\end{lesson}

\emph{Secunda oratio de sancto de quo est ecclesia illa, iij. de omnibus sanctis ⟨scilicet⟩\footnote{1882:113.} Oratio.} Concéde quésumus omnípotens Deus. \emph{⟨\&c.~ut supra⟩.\footnote{1882:113.}} 222.

\begin{lesson}
\subsubsection[Lectio epistole beati Jacobi apostoli. ⟨ultimo capitulo⟩. (v. 16--20.)]{Lectio epistole beati Jacobi apostoli. ⟨ultimo capitulo⟩.\footnote{1882:113.} (v. 16--20.)}

\lettrine{C}{h}aríssimi. Confitémini alterútrum peccáta vestra: et oráte pro ínvicem ut salvémini. Multum enim valet: deprecátio justi assídua. Helýas homo erat símilis nobis passíbilis: et oratióne orávit ut non plúeret super terram et non pluit annos tres et menses sex. Et rursum orávit, et celum dedit plúviam: et terra dedit fructum suum. Si quis autem ex vobis erráverit a veritáte et convérterit quis eum, scire debet quóniam qui convérti fécerit peccatórem ab erróre vite sue salvábit ánimam ejus a morte. Et óperit multitúdinem peccatórum.
\end{lesson}

\gregorioscore{507011-alleluya-confitemini}

\emph{⟨Diaconus legat evangelium sic dicendo.} Sequéntia sancti evangélii secúndum Lucam. \emph{Chorus respondeat} Glória tibi Dómine.⟩\footnote{1882:113.}

\begin{lesson}
\subsubsection{¶ Evangelium secundum Lucam. (xj. 5--13.)}

\lettrine{I}{n} illo témpore. Dixit Jesus discípulis suis, Quis vestrum habébit amícum, et ibit ad illum média nocte: et dicet illi, Amíce: commóda michi tres panes quóniam amícus meus venit de via ad me: et non hábeo quod ponam ante illum. Et ille deíntus respóndens dicat, Noli michi moléstus esse: jam óstium clausum est: et púeri mei mecum sunt in cúbili: non possum súrgere et dare tibi. Et ille si perseveráverit pulsans: dico vobis et si non dabit illi surgens eo quod amícus ejus sit, propter improbitátem tamen ejus surget: et dabit illi quotquot habet necessários: et ego vobis dico, Petíte et dábitur vobis. Quérite et inveniétis. Pulsáte: et aperiétur vobis. Omnis enim qui petit áccipit: et qui querit invénit: et pulsánti aperiétur. Quis autem ex vobis patrem petit panem, nunquid lápidem dabit illi? Aut piscem: nunquid pro pisce serpéntem dabit illi? Aut si petíerit ovum: nunquid pórriget illi scorpiónem? Si ergo vos cum sitis mali: nostis bona data dare fíliis vestris: quanto magis Pater vester de celo: dabit Spíritum bonum peténtibus se?
\end{lesson}

\gregorioscore{g01077-confitebor-domino}

\emph{¶ Hic fiat sermo ad populum si placet.}

\emph{⟨Postea cantetur communio sequens.⟩\footnote{1882:114.}}

\gregorioscore{g01078-petite-et-accipietis}

\filbreak
\emph{¶ Feria iij. dicatur magna missa de sancta Maria: nisi\footnote{1882:114. `\emph{ubi}', 1519:107r.} aliquod festum cum regimine chori contigerit, et tunc missa} Salus pópuli. Missale:{[}194{]}. \emph{dicatur in processione ubi fit statio et in vigilia ascensionis Domini dicatur missa de pace in processione. Tamen si festum alicujus sancti cum regimine chori in ij. feria contigerit tunc dicatur missa dominicalis in iij. feria, vel in vigilia ascensionis Domini in processione ubi fit statio. Similiter quando festum cum regimine chori in vigilia ascensionis Domini contigerit tunc dicatur missa de vigilia ubi fit statio et semper in his predictis tribus diebus dicatur missa in choro post sextam et post illam missam dicatur ⟨hora⟩\footnote{1882:114.} nona: et tunc eat processio ⟨ad aliquam ecclesiam ut supra dictum est⟩.\footnote{1882:114.}}


\section{¶ Missa dominicalis. Officium.}

\gregorioscore{g02089-vocem-jocunditatis}

\begin{lesson}
\subsubsection{Oratio.}

\lettrine{D}{e}us, a quo cuncta bona procédunt, ⟨largíre supplícibus tuis, ut cogitémus, te inspiránte, que recta sunt, et te gubernánte eádem faciámus. Per Dóminum.⟩\footnote{1882:115.}
\end{lesson}

\emph{Secunda oratio de sancto de quo est ecclesia illa.}

\emph{⟨Tertia de omnibus sanctis scilicet,} Concéde, quésumus, omnípotens Deus.⟩\footnote{1882:115.} 222.

\begin{lesson}
\subsubsection[Lectio epistole beati Jacobi, apostoli ⟨primo capitulo⟩. (j. 22--27.)]{Lectio epistole beati Jacobi, apostoli ⟨primo capitulo⟩.\footnote{1882:115.} (j. 22--27.)}

\lettrine{C}{h}aríssimi, Estóte factóres verbi et non auditóres tantum, falléntes nosmetípsos: quia si quis audítor est verbi et non factor: hic comparátur viro consideránti vultum nativitátis sue in spéculo. Considerávit enim se et ábiit: et statim oblítus est qualis fúerit. Qui autem prospéxerit in lege perfécte libertátis et permánserit in ea: non audítor obliviósus factus sed factor óperis: hic beátus in facto suo erit. Si quis autem putat se religiósum esse non refrénans linguam suam sed sedúcens cor suum, hujus vana est relígio. Relígio munda et immaculáta apud Deum et Patrem hec est: visitáre pupíllos et víduas in tribulatióne eórum. Et immaculátum se custodíre ab hoc século.
\end{lesson}

\gregorioscore{g02177-alleluya-usque-modo}

\emph{⟨Evangelium.} Sequéntia sancti evangélii secúndum Johánem, décimo sexto capítulo. Glória tibi Dómine.⟩\footnote{1882:115.}

\begin{lesson}
\subsubsection{¶ Secundum Johannem. (xvj. 23--30.)}

\lettrine{I}{n} illo témpore. Dixit Jesus discípulis suis, Amen amen dico vobis: si quid petiéritis Patrem in nómine meo: dabit vobis. Usque modo non petístis quicquam in nómine meo. Pétite et accipiétis: ut gáudium vestrum sit plenum. Hec in provérbiis locútus sum vobis. Venit hora cum jam non in provérbiis loquar vobis: sed palam de Patre meo annunciábo vobis. Illo die in nómine meo petétis. Et nunc\footnote{`non', \emph{Vulgate,} 1555.} dico vobis, quia ego rogábo Patrem de vobis. Ille enim\footnote{`Ipse enim', \emph{Vulgate,} 1555.} Pater amat vos quia vos me amástis et credidístis quia ego a Deo exívi. Exívi a Patre et veni in mundum: íterum relínquo mundum: et vado ad Patrem. Dicunt ei discípuli ejus, Ecce nunc palam lóqueris: et provérbium nullum dicis. Nunc scimus quia scis ómnia: et non opus est tibi ut quis te intérroget. In hoc crédimus: quia a Deo exísti.
\end{lesson}

\gregorioscore{g01070-benedicite-gentes}

\emph{⟨Postea cantetur communio sequens.⟩}

\gregorioscore{g01071-cantate-domino}


\chapter{Missa de vigilia ascensionis Domini.}

\gregorioscore{g01161-omnes-gentes}

\begin{lesson}
\subsubsection{Oratio.}

\lettrine{P}{r}esta quésumus omnípotens ⟨Pater, ut nostras mentis inténtio, quo Unigénitus Fílius tuus Dóminus noster ventúre solennitátis gloriósus auctor ingréssus est, semper inténdat; et quo fide pergit conversatióne pervéniat. Per.
\end{lesson}

\emph{Secunda oratio de sancto, de quo est ecclcsia illa. Tertia de omnibus sanctis.} Concéde, quésumus, omnípotens Deus. \emph{ut supra.⟩\footnote{1882:116.}}

\begin{lesson}
\subsubsection{Lectio Actuum Apostolorum iv. ⟨32--35.⟩}

\lettrine{I}{n} diébus illis. Multitúdinis autem credéntium erat cor unum et ánima una. Nec quisquam ⟨eórum⟩\footnote{1882:116; 1513:97v.} que possidebat áliquid suum esse dicébat: sed erant illis ómnia commúnia. Et virtúte magna reddébant apóstoli testimónium resurrectiónis Jesu Christi Dómini nostri: et grátia magna erat in ómnibus illis. Neque enim quisquam egens erat inter illos. Quotquot enim possessóres agrórum aut domórum erant: vendéntes afferébant précia eórum que vendébant: et ponébant ante pedes apostolórum. Dividebátur autem síngulis: prout cuíque opus erat.
\end{lesson}

\gregorioscore{g01164-alleluya-omnes-gentes}

\begin{lesson}
\subsubsection{Evangelium secundum Johannem. ⟨xvij. 1.-11.⟩}

\lettrine{I}{n} illo témpore. Sublevátis Jesus óculis in celum dixit, Pater venit hora: clarífica Fílium tuum, ut et Fílius tuus claríficet te. Sicut dedísti ei potestátem omnis carnis: ut omne quod dedísti ei det eis vitam etérnam. Hec est autem vita etérna: ut cognóscant te solum verum Deum: et quem misísti Jesum Christum. Ego te clarificávi super terram: opus consummávi quod dedísti michi ut fáciam. Et nunc clarífica me tu Pater apud temetípsum: claritáte quam hábui priúsquam mundus esset apud te. Magnifestávi\footnote{`Magnificávi', 1519:110r; 1523:110r. The later editions have `Manifestávi'.} nomen tuum homínibus quos dedísti michi de mundo. Tui erant, et michi eos dedísti et sermónem tuum servavérunt. Nunc cognovérunt quia ómnia que dedísti michi abs te sunt: quia verba que dedísti michi dedi eis. Et ipsi accepérunt, et cognovérunt vere quia misísti: credidérunt quia a te exívi.\footnote{1554:115v and 1555 have the \textit{Vulg.}: `et cognovérunt vere quia a te exívi, et credidérunt quia tu me misísti.'} Ego pro eis rogo. Non pro mundo rogo: sed pro his quos dedísti michi: quia tui sunt. Et ómnia mea tua sunt, et tua mea sunt: et clarificátus sum in eis. Et jam non sum in mundo: et hii in mundo sunt et ego ad te vénio.
\end{lesson}

\gregorioscore{g02180-viri-galilei}

\filbreak
\gregorioscore{g01087-pater-cum-essem}

\emph{¶ Cantata missa statim si duplex festum illa die contigerit tres clerici de superiori gradu in medio processionis cantent letaniam in revertendo habitu non mututo. Si duplex festum non fuerit dicatur a duobus clericis de secunda forma, et sic eat processio ad portam claustri orientalem et intrent chorum per ostium occidentale: et dicatur letania usque ud prolationem} Sancta María ⟨quésumus almum⟩.\footnote{1882:117.} \emph{antequam exeat processio ad gradum altaris.}

\section{Prima letania.}

\gregorioscore{000000-kyrie-prima-letania-1}

\emph{Chorus idem.}

\emph{Clerici.}

\emph{⟨Chorus idem repetat post unumquemque versum.}

\emph{Hic procedat processio et clerici predicti prosequantur.⟩\footnote{1882:117.}}

\begin{noinitial}

\gregorioscore{000000-kyrie-prima-letania-2}

\emph{Chorus} Kýrie.

\emph{Clerici ⟨cantent versum⟩.}\footnote{1882:118.}

\gregorioscore{000000-kyrie-prima-letania-3}

\emph{Chorus} Kýrie.

\emph{Clerici ⟨dicant versum⟩.\footnote{1882:118.}}

\gregorioscore{000000-kyrie-prima-letania-4}

\end{noinitial}

\emph{Chorus ut supra.} ⟨Kýrie.⟩\footnote{1882:118.}

\emph{Et clerici predicti prosequantur de ceteris ordinibus quantum sufficit iter usque ad gradum chori in ecclesia propria: ita tamen quod in hac letania et iiij. non dicatur versus sic,} Omnes sancti ángeli, Omnes sancti apóstoli. \emph{sed ita pronuntietur} Omnis chorus angelórum et archangelórum \emph{vel} apostolórum \emph{vel} evangelistárum. \emph{Similiter de ceteris ordinibus.}

\emph{Ad gradum chori finiatur sic:} Omnis chorus sanctórum.

\emph{Deinde sequatur versus et oratio de omnibus sanctis.}

\emph{Quando vero dicatur} Kýrie qui pretióso.

\emph{¶ In die sancti Marce evangeliste ⟨tunc⟩\footnote{1882:118.} in secunda feria in rogationibus dicatur ista secunda letania sic.}

\section[⟨Secunda letania.⟩]{⟨Secunda letania.⟩\footnote{1882:118.}}

\gregorioscore{909041-kyrie-secunda-letania-1}

\emph{Chorus idem repetat post unumquemque ℣.}

\emph{Clerici dicant versum.}

\begin{noinitial}

\gregorioscore{909041-kyrie-secunda-letania-2}

\emph{⟨Item⟩\footnote{1882:118.} chorus ⟨repetat⟩\footnote{1882:118.}} Kýrie. \emph{⟨et cetera⟩.\footnote{1882:118.}}

\emph{Clerici ⟨cantent versum⟩.\footnote{1882:118.}}

\gregorioscore{909041-kyrie-secunda-letania-3}

\emph{⟨Iterum⟩\footnote{1882:118.} chorus ⟨repetat⟩\footnote{1882:118.}} Kýrie.

\emph{Clerici ⟨cantent versum⟩.\footnote{1882:118.}}

\gregorioscore{909041-kyrie-secunda-letania-4}

\emph{Chorus} Kýrie ⟨eléison. \emph{et cetera.⟩\footnote{1882:118.}}

\emph{Clerici ⟨dicant versum⟩.\footnote{1882:118.}}

\gregorioscore{909041-kyrie-secunda-letania-5}

\emph{Chorus ⟨repetat⟩\footnote{1882:118.}} Kýrie.

\emph{⟨Postea⟩\footnote{1882:118.} clerici ⟨dicant versum⟩.\footnote{1882:118.}}

\gregorioscore{909041-kyrie-secunda-letania-6}

\emph{Chorus} Kýrie.

\emph{Hic procedat processio et clerici predicti procedant de ceteris ordinibus, quantum sufficit iter.}

\emph{Item clerici.}

\gregorioscore{909041-kyrie-secunda-letania-7}

\emph{⟨Item⟩\footnote{1882:118.} chorus} Kýrie.

\emph{⟨Postea⟩\footnote{1882:119.} clerici ⟨dicant versum⟩.\footnote{1882:119.}}

\gregorioscore{909041-kyrie-secunda-letania-8}

\emph{⟨Iterum⟩\footnote{1882:119.} chorus} Kýrie.

\emph{⟨Postea⟩\footnote{1882:119.} clerici ⟨dicant versum⟩.\footnote{1882:119.}}

\gregorioscore{909041-kyrie-secunda-letania-9}

\emph{⟨Postea⟩\footnote{1882:119.} chorus ⟨repetat⟩\footnote{1882:119.}} Kýrie.

\emph{⟨Item⟩\footnote{1882:119.} clerici ⟨versum⟩.\footnote{1882:119.}}

\gregorioscore{909041-kyrie-secunda-letania-10}

\emph{⟨Item⟩\footnote{1882:119.} chorus ⟨repetat⟩\footnote{1882:119.}} Kýrie.

\emph{⟨Postea⟩\footnote{1882:119.} clerici ⟨dicant versum⟩.\footnote{1882:119.}}

\gregorioscore{909041-kyrie-secunda-letania-11}

\emph{⟨Item⟩\footnote{1882:119.} chorus ⟨repetat⟩\footnote{1882:119.}} Kýrie.

\emph{⟨Deinde⟩\footnote{1882:119.} clerici ⟨dicant versum⟩.\footnote{1882:119.}}

\gregorioscore{909041-kyrie-secunda-letania-12}

\emph{⟨Deinde⟩\footnote{1882:119.} chorus ⟨repetat⟩\footnote{1882:119.}} Kýrie.

\emph{⟨Postea⟩\footnote{1882:119.} clerici ⟨dicant versum⟩.\footnote{1882:119.}}

\gregorioscore{909041-kyrie-secunda-letania-13}

\emph{⟨Postea⟩\footnote{1882:119.} chorus ⟨repetat⟩\footnote{1882:119.}} Kýrie.

\emph{⟨Item⟩\footnote{1882:119.} clerici ⟨versum⟩.\footnote{1882:119.}}

\gregorioscore{909041-kyrie-secunda-letania-14}

\emph{⟨Deinde⟩\footnote{1882:119.} chorus ⟨repetat⟩\footnote{1882:119.}} Kýrie ⟨eleison. \emph{et cetera.}

\emph{Iterum⟩\footnote{1882:119.} clerici ⟨dicant versum⟩.}\footnote{1882:119.}

\gregorioscore{909041-kyrie-secunda-letania-15}

\end{noinitial}

\emph{⟨Chorus repetat} Kýrie.⟩\footnote{1882:119.}

\emph{Et sic de apostolis et de ceteris ordinibus usque ad} Omnes Sancti. \emph{Sequatur ℣. et oratio de omnibus sanctis ⟨ut supra⟩.}


\section{¶ Tertia letania.}

\gregorioscore{909041-kyrie-tertia-letania-1}

\emph{Chorus idem ⟨repetat post unumquemque versum⟩.\footnote{1882:119.}}

\emph{⟨Postea⟩\footnote{1882:119.} clerici ⟨cantant hunc versum⟩.\footnote{1882:119.}}

\begin{noinitial}

\gregorioscore{909041-kyrie-tertia-letania-2}

\emph{⟨Item⟩\footnote{1882:119.} chorus ⟨repetat⟩\footnote{1882:119.}} Kýrie.

\emph{⟨Deinde⟩\footnote{1882:119.} clerici ⟨versum⟩.\footnote{1882:119.}}

\gregorioscore{909041-kyrie-tertia-letania-3}

\emph{⟨Iterum⟩\footnote{1882:119.} chorus ⟨repetat⟩\footnote{1882:119.}} Kýrie.

\emph{⟨Postea⟩\footnote{1882:119.} clerici ⟨versum⟩.\footnote{1882:119.}}

\gregorioscore{909041-kyrie-tertia-letania-4}

\end{noinitial}

\emph{Chorus} Kýrie.

\emph{Et clerici predicti prosequantur de ceteris ordinibus quantum sufficit ⟨iter⟩\footnote{1882:119.} ⟨usque ad gradum chori⟩,\footnote{1517. ⟨1882:119.⟩} usque ad} Omnes Sancti. \emph{⟨et cetera.⟩\footnote{1882:119. Rylands-24:232 includes another litany here, Rex Kyrie. See the appendix.}}


\section[¶ Quarta letania.]{¶ Quarta letania.\footnote{In 1508. this litany has the page heading `Letania in tempore belli.'}}

\gregorioscore{000000-kyrie-quarta-letania-1}

\emph{Chorus idem ⟨repetat⟩.\footnote{1882:119.}}

\emph{⟨Postea⟩\footnote{1882:119.} clerici ⟨dicant versum⟩.\footnote{1882:119.}}

\begin{noinitial}

\gregorioscore{000000-kyrie-quarta-letania-2}

\emph{Chorus idem ⟨repetat⟩.\footnote{1882:120.}}

\emph{⟨Postea⟩\footnote{1882:120.} clerici ⟨dicant versum⟩.\footnote{1882:120.}}

\gregorioscore{000000-kyrie-quarta-letania-3}

\emph{Chorus idem ⟨repetat⟩.\footnote{1882:120.}}

\emph{⟨Postea⟩\footnote{1882:120.} clerici ⟨dicant versum⟩.\footnote{1882:120.}}

\gregorioscore{000000-kyrie-quarta-letania-4}

\emph{Chorus} Kýrie.

\emph{⟨Item⟩\footnote{1882:120.} clerici.}

\gregorioscore{000000-kyrie-quarta-letania-5}

\emph{Chorus ⟨repetat} Dómine⟩\footnote{1882:120.} miserére.

\emph{⟨Item⟩\footnote{1882:120.} clerici ⟨dicant versum⟩.\footnote{1882:120.}}

\gregorioscore{000000-kyrie-quarta-letania-6}

\emph{Chorus ⟨repetat⟩\footnote{1882:120.}} Miserére.

\emph{⟨Postea⟩\footnote{1882:120.} clerici ⟨dicant versum⟩.\footnote{1882:120.}}

\gregorioscore{000000-kyrie-quarta-letania-7}

\emph{Chorus ⟨repetat⟩\footnote{1882:120.}} Kýrie.

\emph{Clerici ⟨dicant versum⟩.\footnote{1882:120.}}

\gregorioscore{000000-kyrie-quarta-letania-8}

\emph{⟨Item⟩\footnote{1882:120.} chorus} Dómine ⟨miserére⟩.\footnote{1882:120.}

\emph{⟨Deinde⟩\footnote{1882:120.} clerici ⟨versum⟩.\footnote{1882:120.}}

\gregorioscore{000000-kyrie-quarta-letania-9}

\emph{⟨Iterum⟩\footnote{1882:120.} chorus} Miserére.

\emph{Clerici ⟨dicant versum⟩.\footnote{1882:120.}}

\gregorioscore{000000-kyrie-quarta-letania-10}

\emph{⟨Item⟩\footnote{1882:120.} chorus ⟨repetat⟩\footnote{1882:120.}} Kýrie.

\emph{⟨Deinde⟩\footnote{1882:120.} clerici ⟨dicant versum⟩.\footnote{1882:120.}}

\gregorioscore{000000-kyrie-quarta-letania-11}

\emph{Hic modus servetur per omnes ordines usque ad} Omnis chorus sanctórum oret. \emph{Si necesse fuerit ℣. sequentes dicantur a predictis clericis tempore belli.}

\emph{Versus.}

\gregorioscore{000000-kyrie-quarta-letania-12}

\emph{Chorus idem ⟨repetat⟩.\footnote{1882:120.}}

\emph{⟨Deinde⟩\footnote{1882:120.} clerici ⟨dicant versum⟩.\footnote{1882:120.}}

\gregorioscore{000000-kyrie-quarta-letania-13}

\emph{Chorus idem ⟨repetat⟩.\footnote{1882:120.}}

\emph{Clerici ⟨versum⟩.\footnote{1882:120.}}

\gregorioscore{000000-kyrie-quarta-letania-14}

\emph{Chorus idem ⟨repetat⟩.\footnote{1882:120.}}

\emph{⟨Deinde⟩\footnote{1882:120.} clerici ⟨dicant versum⟩.\footnote{1882:120.}}

\gregorioscore{000000-kyrie-quarta-letania-15}

\emph{Chorus idem ⟨repetat⟩.\footnote{1882:120.}}

\emph{⟨Iterim⟩\footnote{1882:120.} clerici ⟨dicant versum⟩.\footnote{1882:120.}}

\gregorioscore{000000-kyrie-quarta-letania-16}

\emph{Chorus idem ⟨repetat⟩.\footnote{1882:120.}}

\emph{⟨Postea⟩\footnote{1882:120.} clerici ⟨dicant versum⟩.\footnote{1882:120.}}

\gregorioscore{000000-kyrie-quarta-letania-17}

\emph{Chorus idem ⟨repetat⟩.}\footnote{1882:120.}

\emph{⟨Deinde⟩\footnote{1882:120.} clerici ⟨dicant versum⟩.}\footnote{1882:120.}

\gregorioscore{000000-kyrie-quarta-letania-18}

\emph{Chorus idem ⟨repetat⟩.}\footnote{1882:120.}

\emph{⟨Iterum⟩\footnote{1882:121.} clerici ⟨dicant versum⟩.}\footnote{1882:121.}

\gregorioscore{000000-kyrie-quarta-letania-19}

\end{noinitial}

\emph{¶ Chorus idem repetat.}

\emph{Finita aliqua letania in his tribus diebus precedentibus dicat sacerdos versum} Vox letície et exultatiónis. \emph{Resp.} In tabernáculis justórum allelúya.

\begin{lesson}
\subsubsection{Oratio.}

\lettrine{P}{r}esta quésumus omnípotens Deus ut in resurrectióne Dómini nostri Jesu Christi Fílii tui cum ómnibus sanctis percipiámus veráciter portiónem. Qui tecum vivit et regnat.
\end{lesson}

\emph{Tamen in vigilia ascensionis finita aliqua letania dicat sacerdos loco predicto ℣.} Letámini in Dómino ⟨et exultáte justi. \emph{Resp.} Et gloriámini omnes recti corde.⟩

\begin{lesson}
\subsubsection{Oratio.}

\lettrine{I}{n}firmitátem nostram quésumus Dómine propícius réspice et mala ómnia que juste mereámur ómnium sanctórum tuórum intercessióne avérte. Per Christum Dómimim nostrum. \emph{⟨℟.⟩} ⟨Amen.⟩\footnote{1882:121.}
\end{lesson}

\emph{¶ Feria iij. et iv. fiat processio in eundo et redeundo eodem modo et ordine quo prediximus: excepto quod in vigilia ascensionis retrocedat dracho: videlicet proximo loco ante crucem tam in eundo quam in redeundo.}


\chapter{¶ In die ascensionis Domini.}

\emph{Ordinetur processio sicut in die pasche: excepto quod hac die vexilla processionis procedant primo videlicet leo:\footnote{`\emph{loco}', 1519:114r; 1523:114r; 1528:105v; 1554:120v. 1508, unpaged, 1530:135r and 1555, unpaged, have `\emph{leo}'.} deinde minora vexilla per ordinem: ultimo loco draconis, deinde inter subdyaconum et thuribularium duo de secunda forma capsulam reliquiarum in cappis sericis deferant: ipse quoque dyaconus ⟨in⟩\footnote{1882:121.} eundo reliquias deferat, pro dispositione sacriste. Preterea hac die procedat processio per ostium chori et ecclesie ⟨et⟩\footnote{1882:121.} exiet per ostium occidentale circumeundo extrinsecus totam ecclesiam et atrium intrando per portam juxta cimiterium canonicorum circumeundo claustrum et rediet in ecclesiam per idem ostium quo egressa est. Hac itaque processio ad gradum chori prius ordinata ut patet in ⟨ista⟩\footnote{1882:121.} statione sequenti.}

\begin{figure}
\centering
\includegraphics{images/114v-ordo-ascensionis.jpg}
\caption[¶ ⟨Statio et⟩ ordo processionis in die ascensionis ⟨Domini⟩ ante missam.]{¶ ⟨Statio et⟩\footnote{1882:122.} ordo processionis in die ascensionis ⟨Domini⟩\footnote{1882:122.} ante missam.}
\end{figure}

\emph{Tres clerici de superiori gradu in medio processionis in cappis sericis dicant hanc prosam sequentem.}

\gregorioscore{a00177-salve-qua-deus-1}

\emph{Chorus idem repetat post unumquemque versum.}

\emph{Clerici ⟨dicant⟩\footnote{1882:122.} ℣.}

\begin{noinitial}

\gregorioscore{a00177-salve-qua-deus-2}

\end{noinitial}

\emph{Per idem ostium quo egressa est ⟨processio⟩\footnote{1882:123.} regredietur usque ad crucem in ecclesia cantando ⟨hoc sequens⟩\footnote{1882:123.} ℟. In revertendo cantore incipiat.}

\begin{highnotes}

\label{007904-viri-galilei}
\gregorioscore{007904-viri-galilei}

\end{highnotes}

\emph{¶ In introitu chori dicatur aliud responsorium hoc modo ⟨quo sequitur⟩.\footnote{1882:123.}}

\label{007225-non-conturbetur}
\gregorioscore{007225-non-conturbetur}

\cantusnote{007952}
\emph{℣.} Ascéndit Deus in jubilatióne.

\emph{Resp.} Dóminus in voce tube ⟨allelúya⟩.\footnote{1882:123.}

\begin{lesson}
\subsubsection{Oratio.}

\lettrine{C}{o}ncéde quésumus omnípotens Deus: ut qui hodiérna die Unigénitum tuum redemptórem nostrum ad celos ascendísse crédimus: ipsi quoque mente in celéstibus habitémus. Per eúndem Christum Dóminum nostrum. ⟨Amen.⟩\footnote{1882:124.}
\end{lesson}


\chapter{¶ Dominica infra octavas ascensionis.}

\emph{Ad processionem. ℟.} Viri Galiléi. \pageref{007904-viri-galilei}.

\emph{In introitu chori. ℟.} Non conturbétur. \pageref{007225-non-conturbetur}.

\emph{℣. et oratio ut supra in die ascensionis.}


\chapter{¶ In vigilia penthecostes.}

\emph{Cantentur letanie et fiat benedictio fontium sicut in vigilia pasche eodem modo et ordine, tam in eundo quam in redeundo et in statione.}

\emph{In redeundo dicatur} Rex sanctórum. \pageref{a00176-rex-sanctorum-1}.


\chapter{¶ In die penthecostes.}

\emph{Ante iij. aspergatur aqua benedicta: et post aspersionem aque ordinetur processio sine vexillis et sine reliquiis: circumeundo ecclesiam et claustrum sicut in die pasche: et cantetur prosa sequens in medio processionis a tribus clericis de superiori gradu in cappis sericis ⟨qui⟩\footnote{1882:124.} dicant hoc modo.}

\gregorioscore{a00177-salve-qua-nova-1}

\emph{Chorus idem repetat post unumquemque versum.}

\emph{Clerici.}

\begin{noindent}

\gregorioscore{a00177-salve-qua-nova-2}

\end{noindent}

\emph{In redeundo per idem ostium quo egressa est regrediatur ⟨processio⟩\footnote{1882:125.} usque ad crucem cantando hoc ℟. sequens hoc modo.}

\gregorioscore{007693-spiritus-sanctus}

\emph{In introitu chori dicatur hec ⟨sequens⟩\footnote{1882:125.} antiphona ⟨hoc modo quo sequitur⟩.\footnote{1882:125.}}

\gregorioscore{003096-hodie-completi}

\emph{℣.} Loquebántur váriis linguis apóstoli.

\emph{℟.} Magnália Dei allelúya.

\begin{lesson}
\subsubsection{Oratio.}

\lettrine{D}{e}us, qui hodiérna die corda fidélium Sancti Spíritus illustratióne docuísti: da nobis in eódem Spíritu recta sápere: et de ejus semper consolatióne gaudére. Per Christum Dóminum ⟨nostrum. Amen.⟩\footnote{1882:125.}
\end{lesson}

\emph{¶ Et nota quod hac die cantetur hora iij. et vj. a toto choro in cappis sericis et non alias per totum annum.}
