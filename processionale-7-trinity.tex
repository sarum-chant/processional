\chapter{¶ In die sancte Trinitatis.}

\emph{Ordinetur ⟨processio⟩\footnote{1882:125.} sicut in die natalis Domini. Sexta\footnote{Note: `1517---Tertia', 1882:125.} cantata eat processio per medium chori circumeundo claustrum cantore incipiente ⟨hoc responsorium sequens⟩\footnote{1882:125.} hoc modo.}

\gregorioscore{007718-summe-trinitati}

\emph{In introitu chori dicatur istud responsorium ⟨sequens hoc modo quo sequitur⟩.\footnote{1882:126.}}

\gregorioscore{006870-honor-virtus}

\emph{In tempore paschali hoc modo dicatur.}

\begin{noinitial}

\gregorioscore{006870-alleluya}

\gregorioscore{006870a-trinitati-lux-perennis}

\end{noinitial}

\emph{℣.} Sit nomen Dómini benedíctum.

\emph{℟.} Ex hoc nunc ⟨et usque in séculum⟩.\footnote{1882:126.}

\emph{⟨℣.} Orémus.⟩

\begin{lesson}
\subsubsection{Oratio.}

\lettrine{O}{m}nípotens sempitérne Deus qui dedísti fámulis tuis in confessióne vere fídei etérne Trinitátis glóriam agnóscere et in poténtia majestátis adoráre Unitátem: quésumus ut ejúsdem fídei firmitáte ab ómnibus semper muniámur advérsis. Qui vivis et regnas in unitáte.\footnote{1554. has `In qua vivis. \emph{\&c.}' 1882:126 has `In qua vivis.' The edition follows the Breviary and Missal here.}
\end{lesson}


\chapter{¶ In festo Corporis Christi.}

\emph{Ante missam procedat processio per medium chori et ecclesie, exiens per ostium occidentale circumeundo ecclesiam et atrium ut in die ascensionis Domini ordinata prius processione ad gradum chori cum sacerdote alba et cappa serica induto: Corpus Christi in tabernaculo deferente sub quodam pallio de serico ⟨quod⟩\footnote{1882:126.} super iiij. hastas deferatur, cum cereis illuminatis a clericis in superpelliceis. Tres clerici de superiori gradu in medio processionis in cappis sericis cantent prosam et cantetur ℣. antequam exeat processio: chorus idem repetat post unumquemque versiculum, clerici vero sequentur.}

\gregorioscore{a00177-salve-qua-caro-messie}

\emph{In revertendo usque ad crucem dicatur ⟨hoc responsorium modo quo sequitur⟩.\footnote{1882:127.}}

\gregorioscore{601997-respexit-helyas}

\begin{noinitial}

\gregorioscore{601997a-si-quis}

\emph{Si necesse fuerit dicatur istum versiculum.}

\gregorioscore{601997-gloria-patri}

\end{noinitial}

\emph{In introitu chori dicatur ista antiphona.}

\gregorioscore{203576-o-sacrum-convivium}

\emph{℣.} Panem de celo prestitísti eis.

\emph{℟.} Omne delectaméntum ⟨in se habéntem⟩.\footnote{1882:128.}

\emph{⟨℣.} Orémus.⟩

\begin{lesson}
\subsubsection{Oratio.}

\lettrine{D}{e}us qui nobis sub sacraménto mirábili passiónis tue memóriam reliquísti tríbue quésumus: ita nos córporis et sánguinis tui sacra mistéria venerári: ut redemptiónis tue fructum in nobis júgiter sentiámus. Qui vivis et regnas.
\end{lesson}


\chapter{¶ Dominica infra octavas.}

\emph{Ubi habentur cum regimine chori\footnote{`1517, \&c.---\emph{ubi octave habentur cum regimine chori}', 1882:128.} ad processionem. ℟.} Respéxit Helýas. 285.

\emph{In introitu chori de sancta Maria.\footnote{`1517---In introitu chori, ut supra in prima die, nisi historia inchoata fuerit, sicut in processione ante missam. Coram ecclesia dicitur antiphona Adoremus. Et in introitu chori antiphona Ave regina celorum de sancta Maria', 1882:128.}} 291.

\lettrine{I}{}\emph{N sabbatis per estatem scilicet a Trinitate usque ad adventum Domini: ad vesperas post omnes memorias eat\footnote{\emph{I}'\emph{et}', 1519:122v and most other editions. 1530:144v and 1555. have `\emph{eat}'.} processiones: ante crucem de quocunque fit servitium per medium chori ubi\footnote{`\emph{nisi}', 1555.} duplex festum fuerit, ordinata prius processione ad gradum chori cum duobus ceroferariis, albis tantum indutis, et thuribulario in simili habitu sine cruce, deinde puer librum ferens ante sacerdotem in superpelliceo deinde executor officii in simili habitu cum cappa serica, post eum vero duo rectores\footnote{1882:128 has `\emph{cantores}', with the note: `1517 and later edd.---rectores.'} in medio processioni in simili habitu antiphonam in eundo et in introitu incipientes, choro sequente habitu non mutato.}


\chapter{⟨Ante crucem in sabbatis post Trinitatem.⟩}

\begin{figure}
\centering
% \includegraphics{images/1v-statio-aqua.png}
\caption{¶ Statio ad vesperas ante crucem in sabbatis per estatem.}
\end{figure}

\emph{¶ Tunc cantent unam istarum antiphonarum per ordinem.}

1523:123v; AS:424; Ant-1519-S:70v; Brev-1531-S:47v.\footnote{1523:123v has a B♭ signature throughout. In 1523:123v `et' is set CB♭ADB♭B♭A.}

\gregorioscore{missing}

\emph{Et finiatur cum} Allelúya. \emph{qandocunque dicatur} Allelúya.

1519:123v; AS:428; Ant-1519-S:73v; Brev-1531-S:48v.

\gregorioscore{missing}

\emph{⟨Sequitur alia antiphona.⟩\footnote{1882:130.}}

\gregorioscore{001962-crux-fidelis}

\emph{Statio post thurificationem crucifixi.}

\emph{℣.} Adorámus te Christe et benedícimus tibi.

\emph{℟.} Quia per sanctam crucem tuam ⟨redemísti mundum⟩.\footnote{1882:130.}

\emph{⟨℣.} Orémus.⟩

\begin{lesson}
\subsubsection{Oratio.}

\lettrine{D}{e}us qui unigéniti Fílii tui Dómini nostri Jesu Christi precióso sánguine vivífice crucis vexíllum sanctificáre voluísti: concéde quésumus eos qui ejúsdem sancte crucis gáudent honóre, tua quoque ubíque protectióne gaudére. Per eúndem Christum.
\end{lesson}

\emph{In introitu chori dicatur una istarum antiphonarum sequentium per ordinem tam in processione in sabbatis quam ad processionem ante missam in dominicis pro dispositione cantoris.\footnote{GS:141 and Rylands-24:251 list two further antiphons: O gloriósa. (see Brev:{[}524{]}) and Sancta María. (see Brev.:{[}512{]}.}}

1519:124v; AS:519; Ant-1520-S:105r; Brev-1531-S:130r.\footnote{In the Antiphonal the `Allelúya' is different.}

\gregorioscore{missing}

\emph{In tempore paschali.}

\gregorioscore{missing}

1519:124v; AS:529; Ant-1519-C:49r; Ant-1520-C:49r; Brev-1531:171v.

\gregorioscore{missing}

\emph{In tempore paschali ⟨ita finiatur⟩.\footnote{1882:130.}}

\gregorioscore{missing}

1519:125r; AS:529; Ant-1519-C:48v; Ant-1520-C:48v; Brev-1531:171v.

\gregorioscore{missing}

\emph{In tempore paschali ⟨finiatur cum⟩.}

\gregorioscore{missing}

1519:125v; Ant-1519-C:50r; Ant-1520-C:50r; Brev-1531:171v.

\gregorioscore{missing}

\emph{In tempore paschali ⟨finiatur cum⟩.}

\gregorioscore{missing}

1519:125v; AS:528; Ant-1520-S:111v; Brev-1531-S:132v.

\gregorioscore{missing}

\emph{⟨Postea sequitur alia antiphona⟩.\footnote{1555. unpaged.}}

1519:126r; AS:528; Ant-1520-S:112r; Brev-1531-S:132v.

\gregorioscore{missing}

\emph{Infra assumptionem et nativitatem beate Marie tam ad processionem ad vesperas in sabbatis quam ante missam in dominicis dicatur una istarum antiphonarum.}

1519:126v; AS:490; Ant-1520-S:89v; Brev-1531-S:115v.

\gregorioscore{missing}

1519:127r; AS:492; Ant-1520-S:91r; Brev-1531-S:105v.

\gregorioscore{missing}

1519:127v; AS:491; Ant-1520-S:90r; Brev-1531-S:115v. \gregorioscore{missing}

\emph{In tempore paschali.}

\gregorioscore{missing}

1519:128r; AS:492; Ant-1520-S:90v; Brev-1531-S:115v.

\gregorioscore{missing}

\emph{In tempore paschali.}

\gregorioscore{missing}

\emph{℣.} Sancta Dei génitrix ⟨virgo semper Maria⟩.\footnote{1882:132.}

\emph{℟.} Intercéde ⟨pro nobis ad Dóminum Deum nostrum⟩.\footnote{1882:132.}

\emph{Ad processionem ante missam. ℣.} Post partum ⟨virgo⟩.\footnote{1882:132.} \emph{℟.} Dei génitrix. 44. \emph{et semper cum hac oratione.}

\emph{⟨℣.} Orémus.⟩

\begin{lesson}
\subsubsection{Oratio.}

\lettrine{C}{o}ncéde quésumus omnípotens et miséricors Deus fragilitáti nostre presídium: ut qui sancte Dei genitrícis et vírginis Maríe commemoratiónem ágimus: intercessiónis ejus auxílio a nostris iniquitátibus resurgámus. Per eúndem Dóminum. \emph{vel} ⟨Per⟩\footnote{1882:132.} eúndem Christum ⟨Dóminum nostrum⟩.\footnote{1882:132.}
\end{lesson}

\emph{¶ Sabbato vero infra octavas assumptionis et nativitatis beate Marie dicatur in introitu chori de omnibus sanctis ant.} Salvátor mundi. 315. \emph{cum versu} Letámini in Dómino. 316. \emph{et cum oratione} Infirmitátem. 316.


\chapter[¶ Omnibus dominicis ab octavis sancte Trinitatis.]{¶ Omnibus dominicis ab octavis sancte Trinitatis usque ad adventum Domini.}

\emph{Quando fit plenum servitium de dominica ad processionem ante missam dicatur unum istorum responsoriorum sequentium per ordinem de Trinitate quando ad matutinas exeat processio et redeat ut in dominica j. adventus ut supra dictum est.}

1519:128v; AS:288; Ant-1520:3v; Brev-1531:157v.\footnote{1519:128v has no flats. In 1519:128v `Patri' is set A.AG.}

\gregorioscore{missing}

\emph{⟨Item sequitur aliud responsorium.⟩\footnote{1882:132.}}

1519:128v; AS:288; Ant-1520:3v; Brev-1531:157v.\footnote{In 1519:128v `majestátis' is set FE.FG.GFFEDC.F.}

\gregorioscore{missing}

6249a. \gregorioscore{missing}

\emph{⟨Postea sequitur aliud responsorium.⟩\footnote{1882:132.}}

1519:129r; AS:289; Ant-1520:4r; Brev-1531:158r.

\gregorioscore{missing} 7498.

\gregorioscore{missing} 7498a.

\gregorioscore{missing}

\emph{⟨Item sequitur aliud responsorium.⟩\footnote{1882:133.}}

1519:129v; AS:289; Ant-1520:4v; Brev-1531:158r.\footnote{1519:129v has no flats. In 1519:129v `Magnus' is set EDCDE.E; `magna' is set GA.BGGAG; the final four notes of `est' are AFDF; in the ℣. `Glória Patri', `et' is set G.}

\gregorioscore{missing} 7117.

\gregorioscore{missing}

\emph{⟨Item sequitur aliud responsorium.⟩\footnote{1882:133.}}

1519:130r; AS:290; Ant-1520:4v; Brev-1531:158r.\footnote{1519:130r has a flat only at `proli'. In 1519:130r the second `Da' is set AB.}

6777. \gregorioscore{missing}

\emph{⟨Postea sequitur aliud responsorium.⟩\footnote{1882:133.}}

1519:130v.

\gregorioscore{missing}

\emph{⟨Postea sequitur aliud responsorium.⟩\footnote{1882:133.}}

\emph{¶ In festo Sancte Trinitatis.}

1519:130v; AS:291; Ant-1520:5v; Brev-1531:158v.\footnote{1519:130v has no flats.}

7764. \gregorioscore{missing}

7764a. \gregorioscore{missing}

\emph{⟨Deinde sequitur aliud responsorium.⟩\footnote{1882:133.}}

1519:131r; AS:292; Ant-1520:6r; Brev-1531:158v.

6239.

\gregorioscore{missing} 6239a.

\emph{⟨Item aliud responsorium} Summe Trinitáti. \emph{et cetera.⟩\footnote{1882:133.} Require ⟨ut supra⟩\footnote{1882:133.} in festo Sancte Trinitatis.} 277.

\emph{Et semper tunc post ℟. ante crucem in ecclesia dicatur una istarum antiphonarum de cruce et si de aliquo festo fit processio, licet ad vesperas in sabbato processio ante crucem facta non fuerit nisi tantum aliquod festum duplex in ipsa dominica contigerit.}

1519:131v; AS:533; Ant-1520-S:117r; Brev-1531-S:137v.

\gregorioscore{missing}

\emph{⟨Item sequitur alia antiphona.⟩\footnote{1882:134.}}

1519:131v; AS:532; Ant-1520-S:117r; Brev-1531-S:137v.\footnote{In 1528:122r, 1530:154v, and 1554:141r the music appears a fifth lower. In AS:532. the music is a fourth lower, in Mode VII.}

\gregorioscore{missing}

\emph{℣.} Hoc signum crucis erit in celo.

\emph{℟.} Cum Dóminus ad judicándum vénerit.

\emph{⟨℣.⟩} Orémus.

\begin{lesson}
\subsubsection{Oratio.}

\lettrine{A}{d}ésto nobis Dómine Deus noster et quos sancte crucis letári facis honóre: ejus quoque perpétuis defénde subsídiis. Per Christum Dóminum nostrum. ⟨Amen.⟩\footnote{1882:134.}
\end{lesson}

\emph{Deinde dicantur preces consuete ut supra in dominica prima adventus Domini. Quibus dictis, dicatur in introitu chori una ex istis per ordinem de sancta Maria antiphona.} Beáta Dei génitrix. 291. Ave regína celórum. 292. Alma redemptóris mater. 293. Speciósa facta est. 294. Ibo michi ad. 295. Quam pulchra. 295. \emph{Infra ⟨octavis⟩ assumptionem beate Marie ⟨usque ad nativitatem ejusdem⟩\footnote{1882:134.} dicatur una istarum antiphonarum.} Tota pulchra es. 296. Ascéndit Christus. 298. Anima mea. 299. Descéndi in hortum. 300. \emph{℣.} Post partum. 44. \emph{Et semper cum hac oratione ⟨scilicet⟩\footnote{1882:134.}} Concéde quésumus miséricors Deus. \emph{ut supra} 50. \emph{: que terminetur sic,} Per eúndem ⟨Dóminum. \emph{vel} eúndem Christum⟩.\footnote{1882:134.}

