\chapter{¶ In dedicatione ecclesie.}

\emph{Eat processio sicut in die penthecostes circumeundo ecclesiam et claustrum, tres clerici de superiori gradu in medio processionis in cappis sericis dicant prosam sequentem hoc modo.}

\gregorioscore{a00177-salve-qua-sponso}

\emph{Chorus idem repetat post unumquemque versum.}

\gregorioscore{missing}

\emph{¶ In introitu chori dicatur hoc responsorium ⟨sequens⟩.\footnote{1882:135.}}

1519:132v; AS:pl. p; Ant-1520:58v; Brev-1531:215v.\footnote{In 1519:132v. the first `est' is set CDEDF. In 1519:132v. and the other editions no repeat is indicated at `vere', but a repeat is indicated at `et'; yet the indicated repetition cue at the end is `†~Vere.' In 1519:132v. `in' is set CD. In 1519:133r. `somno' is set thus:

  \gregorioscore{missing}}

\gregorioscore{missing}

\emph{In tempore paschali ⟨dicitur⟩.\footnote{1882:135.}}

\gregorioscore{missing}

\emph{¶ Hac die non dicatur} Glória Patri.

\emph{℣.} Beáti qui hábitant in domo tua Dómine.

\emph{℟.} In sécula seculórum laudábunt te.

\emph{⟨℣.} Orémus.⟩

\begin{lesson}
\subsubsection{Oratio.}

\lettrine{D}{e}us qui nobis per síngulos annos hujus sancti templi tui consecratiónis réparas diem: et sacris semper mystériis represéntas incólumes, exáudi preces pópuli tui et presta: ut quisquis hoc templum benefícia petitúrus ingréditur: cuncta se impetrásse letétur. Per Christum Dóminum.
\end{lesson}

\emph{¶ Dominica vero infra octavas\footnote{`\emph{et in octava}, MS.' 1882:135.} si dominica fuerit ad processionem Resp.} Terríbilis. 312. \emph{cum versu} Cumque evigilásset. \emph{ut supra et cum} Glória Patri. \emph{et ⟨cum⟩\footnote{1882:135.}} ‡~Et ego.\footnote{1882:135.

  \gregorioscore{missing}} \emph{In paschali tempore cum} Allelúya.

\emph{In introitu chori de sancta Maria.} 309.

\chapter{¶ In vigilia sancti Andree apostoli.}

\emph{Si hoc festum infra adventum Domini vel extra contigerit: fiat processio ad j. vesperas post omnes memorias que habentur ad altare sancti Andree si habeatur, cum ceroferariis et thuribulario a puero librum deferente ante sacerdotem sine cruce, choro sequente habitu non mutato cantando ℟. rectoribus incipientibus cum suo versu, itaque rectores chori incipiant ℟. hoc modo.}

1519:133v; AS:350; Ant-1519-S:6r; Brev-1531-S:3v.\footnote{In 1519:133v. `iste est' is set G.GBAGA AFFE. 1519:133v. has no flat.}

7899. \gregorioscore{missing}

\emph{Non dicatur} Glória Patri.

\emph{Dum ℣. canitur thurificet sacerdos altare illud deinde ymaginem sancti Andree et postea dicat ℣.} In omnem ⟨terram exívit sonus eórum⟩.\footnote{1882:136.}

\emph{℟.} Et in ⟨fines orbis terre verba eórum⟩.\footnote{1882:136.}

\emph{⟨℣.} Orémus.⟩

\begin{lesson}
\subsubsection{Oratio.}

\lettrine{M}{a}jestátem tuam quésumus Dómine supplíciter exorámus: ut sicut ecclésie tue beátus Andréas éxtitit predicátor ⟨et rector⟩\footnote{1882:136.}: ita apud te sit pro nobis perpétuus intercéssor. Per Christum Dóminum nostrum.
\end{lesson}

\emph{Nec precedat nec subsequatur} Dóminus vobíscum. \emph{sed tantum cum} Orémus. \emph{ante orationem.}

\emph{Cum vero oratio} Majestátem. \emph{dicatur ad j. veperas tunc ad processionem dicatur ista oratio.}

\begin{lesson}
\lettrine{Q}{u}ésumus omnípotens Deus: ut beátus Andréas apóstolus tuum pro nobis implóret auxílium: ut ⟨a nostris reátibus absolúti⟩\footnote{1555, unpaged.} a cunctis étiam perículis eruámur. Per Christum.
\end{lesson}

\emph{In redeundo extra adventum de sancta Maria. Infra adventum vero dicatur ista antiphona de omnibus sanctis.}

1519:134r; AS:543, 579; Ant-1520-S:155r; Brev-1531-S:79r.\footnote{1519:134r. has no flats. In 1519:134r. `sanctórum' is set C.AB.A; `semper' is set DF.DC.}

\gregorioscore{missing}

\emph{℣.} Letámini in Dómino et exultáte justi.

\emph{℟.} Et glorámini ⟨omnes recti corde⟩.\footnote{1882:136.}

\emph{⟨℣.} Orémus.⟩

\begin{lesson}
\subsubsection{Oratio.}

\lettrine{I}{n}firmitátem nostram quésumus Dómine propícius réspice, et mala ómnia que juste merémur ómnium sanctórum tuórum intercessióne avérte. Per Dóminum ⟨nostrum⟩.\footnote{1882:136.}
\end{lesson}

\emph{¶ Et eat processio in omni festo sanctorum ad j. vesperas post memorias que habentur semel per annum quorum altaria sunt in ecclesia cum propriis responsoriis vel de communi cum ℣. et oratione de sanctis: ita tamen quod quando oratio de vigilia dicitur ad primas vesperas ut supra diximus: tunc ad processionem dicatur oratio de die. Similiter quando oratio de festo dicatur ad j. vesperas tunc ad proccessionem dicatur oratio de communi: nisi quando de festo propria habeatur: ut in festo sancti Nicolai et consimilibus. Et semper in revertendo dicatur de sancta Maria: nisi quando de ea ad vesperas prius fuerit memoria tunc enim dicatur de omnibus sanctis. In revertendo ut supra notatum est.}

\chapter{¶ In die sancti Nicolai.}

\emph{Ad j. vesperas post memorias dictas eat processio ad altare ejusdem cum ceroferariis et thuribulario, ⟨et⟩\footnote{1882:137.} puero librum deferente ante sacerdotem, sine cruce, choro sequente habitu non mutato, cantando ⟨responsorium sequens cum suo versu sine prosa⟩.\footnote{1882:137.}}

1519:134v; AS:359; Ant-1519-S:14r; Brev-1531-S:7r.\footnote{1519:134v. has no flats.}

\gregorioscore{missing}

\emph{Cum suo ℣. sed sine prosa in hac processione.}

\gregorioscore{missing}

\emph{Non dicatur} Glória Patri.

\emph{⟨Dum⟩\footnote{1882:137.} ℣. canitur thurificet sacerdos altare, deinde ymaginem sancti Nicolai et postea dicat ℣. sic.} Ora pro nobis, beáte Nícholáe.

\emph{℟.} Ut digni efficiámur ⟨promissiónibus Christi⟩.\footnote{1882:137.}

\emph{⟨℣.} Orémus.⟩

\begin{lesson}
\subsubsection{Oratio.}

\lettrine{D}{e}us bonitátis auctor et bonórum dispensátor: concéde propítius: ut qui beáti Nicolái confessóris tui atque pontíficis sollémnia venerámur: ejus patrocínio atque suffrágio, majestátis tue propitiatiónem consequámur. Per Christum Dóminum.
\end{lesson}

\emph{In redeundo ut supra. ℣.} Letámini in Dómino. 316. \emph{Oratio.} Infirmitátem nostram. 316.

\chapter{¶ In conceptione beate Marie virginis.}

\emph{Omnia fiant sicut in nativitate} 369. \emph{verbis tamen mutatis in} Conceptióne.

\emph{Si festum alicujus sancti ix. lectionum ab octavis epiphanie usque ad lxx. vel ab octavis pasche usque ad penthecostes\footnote{`\emph{ascensionem}', 1555, unpaged. 1882:138. has `\emph{Ascensionem}' with the note `1517, \&c.---Pentecosten.'} et a} Deus ómnium. \emph{usque ad adventum Domini in dominica contigerit, fiat servitium de ipso festo cum ix. responsorio communis historie vel cum aliquo responsorio de historia ipsius festi si propria habeantur. In revertendo scilicet in introitu chori dicatur aliqua antiphona vel ℟. de sancta Maria cum versu et oratione que tempori conveniant. Tamen si fuerit festum majus duplex alicujus sancti, tam in eundo et in introitu chori totum fiat de sancto. Dominica vero infra octavas assumptionis et nativitatis beate Marie si dominica fuerit, in eundo dicatur de sancta Maria, et in introitu chori de omnibus sanctis antiphona} Salvátor mundi. 315. \emph{℣.} Letámini in Dómino. 316. \emph{Oratio.} Infirmitátem nostram. 316. \emph{Quando vero antiphona} Salvátor mundi. \emph{dicatur ad vesperas tunc ad processionem ante missam dicatur hec antiphona sequens hoc modo.}

1519:135v; AS:543; Ant-1520-S:40v; Brev-1531-S:79r.\footnote{In 1519:135v. `beáta' is set GBCD.D.GA.}

\gregorioscore{missing}

\emph{cum versiculo et oratione predictis de omnibus sanctis ⟨ut supra⟩.\footnote{1882:138.}} 316.

\chapter{¶ Sanctorum Fabiani et Sebastiani martyrum.}

\emph{Si in dominica fuerit ad processionem ⟨dicatur responsorium sequens⟩.\footnote{1882:138.}}

1519:135v; AS:370; Ant-1519-S:23r; Brev-1531-S-S:15r.\footnote{`pax in homínibus' \emph{Chevallon.} ⟨SB:83.⟩ SB:83 has `pax homínibus.' AS:368 has no indication of the second repeat `‡~Et in terra.'}

\gregorioscore{missing} 6647.

\gregorioscore{missing}

6647a. \gregorioscore{missing}

\emph{⟨℣.⟩} Glória Patri ⟨et Fílio et Spirítui Sancto⟩.\footnote{1882:138.} 425. ‡~Et in terra.

\emph{⟨Post versum, si necesse fuerit, quotienscunque dicitur responsorium ad processionem, dicitur} Glória Patri. \emph{per totum annum extra passionem Domini, nisi cum dicitur responsorium in introitu chori.⟩\footnote{1882:138.}}

\emph{In introitu chori de sancta Maria.} 50.

\chapter{¶ Sancti Agnetis virginis et martyris.}

\emph{Si dominica fuerit ad processionem ⟨dicatur responsorium sequens⟩.\footnote{1882:139.}}

1519:136r; AS:373; Ant-1519-S:26r; Brev-1531-S:15v.\footnote{1519:136r. has no flats. 1519:136v. has `†~Et intra.' 1528:127r. has `†~Et tanquam.'}

\gregorioscore{missing} 6955.

\gregorioscore{missing} 6955b.

\gregorioscore{missing}

\emph{⟨℣.⟩} Glória. 425. †~Et tanquam.

\emph{In introitu chori de sancta Maria ⟨cum versu et oratione que tempori conveniunt⟩.\footnote{1882:139.}} 50.

\chapter[¶ ⟨In die⟩ sancti Vincentii.]{¶ ⟨In die⟩\footnote{1882:139.} sancti Vincentii.}

\emph{Si dominica fuerit ante lxx. fiat processio sicut in j. dominica adventus. Ad processionem dicatur ⟨hoc⟩\footnote{1882:139.} ℟. sequens.}

1519:136v; AS:384; Ant-1519-S:33v; Brev-1531-S:19r.\footnote{1519:136v. has no flats. In 1519:136v. `-ciósus martyr Vincén-' is a third lower, as if the clef is wrong. In 1519:136v. `resolútus' is set ED.CD.DEFD.D; at the ℣. `de' is set GAF.}

sar0658. \gregorioscore{missing}

\emph{In introitu chori de sancta Maria.} 50.

\chapter{¶ In conversione sancti Pauli.}

\emph{Si dominica fuerit ad processionem dicatur ℟. sequens.}

1519:137v; AS:393; Ant-1519-S:40v; Brev-1531-S21r.\footnote{1519:137v. has no flats. In 1519:137v. `hódie' is set FG.DEFEDCD.CDC; in the V. the second `et' is set G.}

\gregorioscore{missing} 6272.

\gregorioscore{missing}

\emph{⟨℣.⟩} Glória Patri ⟨et Fílio et Spirítui⟩.\footnote{1882:139.} 424. †~Quia.

\emph{In introitu chori de sancta Maria.} 50.

\chapter{In purificatione beate Marie.}

\emph{Cantata hora sexta fiat benedictio luminis solemniter a pontifice vel a sacerdote cappa serica induta cum aliis indumentis sacerdotalibus: et super supremum gradum altaris converso ad australem\footnote{1882:139 has `\emph{orientem}' with the note `1517, \&c.---ad australe.'} sic incipiente.}

\subsubsection{¶ Sequitur statio.}

\begin{figure}
\centering
\includegraphics{images/138r-statio-candele.jpg}
\caption[¶ Statio dum benedicuntur candele ⟨luminaria⟩ in die purificationis ⟨beate Marie⟩.]{¶ Statio dum benedicuntur candele ⟨luminaria⟩\footnote{1882:140.} in die purificationis ⟨beate Marie⟩.\footnote{1882:140.}}
\end{figure}

\grecommentary{1519:138r; Grad-1508-S:378; Missal-1513-S:10v}

\gregorioscore{dominus-vobiscum-hh}

\begin{lesson}
\lettrine{B}{e}ne✠dic Dómine Jesu Christe hanc ✠ creatúram cere supplicántibus nobis et infúnde ei per virtútem sancte crucis bene⟨✠⟩\footnote{B1555, unpaged.}dictiónem celéstem: ut qui eam ad repellándas ténebras humáno úsui tribuísti, talem signáculo sancte crucis tue fortitúdinem et benedictiónem accípiat: ut quibuscúnque locis accénsa sive appósita fúerit, discédat diábolus et contremíscat et fúgiat pállidus cum ómnibus minístris suis de habitatiónibus illis nec presúmat ámplius inquietáre \emph{et finiatur ⟨benedictio⟩\footnote{1882:140.} sic}

\end{lesson}

\gregorioscore{servientes-tibi}

\emph{Et dicantur omnes orationes cum} Orémus. \emph{sub eodem tono. Non dicatur} Dóminus vobíscum. \emph{nisi ante primum orationem tantum.}

Orémus.

\begin{lesson}
\subsubsection{Oratio.}

\lettrine{D}{o}mine sancte Pater omnípotens etérne Deus qui ómnia ex níhilo creásti et jussu tuo per ópera apum hunc liquórem ad perfectiónem cereórum veníre fecísti: ut\footnote{`et', 1555, unpaged.} qui hodiérna die petitiónem justi Symeónis implésti: te humíliter deprecámur, ut has cándelas ad usum hóminum ⟨et⟩\footnote{1555, unpaged.} sanitátem córporum et animárum preparátas: sive in terra sive in aquis per invocatiónem sanctíssimi nóminis tui et per intercessiónem sancte Marie ⟨semper⟩\footnote{1882:141.} vírginis: cujus hódie festa devóte celebrántur et per preces ómnium sanctórum tuórum bene✠dícere et sanctifi⟨✠⟩\footnote{1555, unpaged.}cáre dignéris: ut et hujus plebis tue que illas honorífice in mánibus portáre desíderat, teque laudándo exsultáno exáudias voces de celo sancto tuo et de sede majestátis tue, et propítius sis ómnibus clamántibus ad te quos redemísti pretióso sánguine Fílii tui. Qui tecum et cum Spíritu Sancto vivit et gloriátur Deus. Per ómnia sécula seculórum. \emph{⟨℣.⟩} Amen.

Orémus.

\subsubsection{Oratio.}

\lettrine{O}{m}nípotens sempitérne Deus qui hodiérna die Unigénitum tuum ulnis sancti Symeónis in templo sancto tuo suscipiéndum presentári voluísti: tuam súpplices deprecámur cleméntiam ut hos céreos, quos nos fámuli tui in tui nóminis magnificéntia suscipiéntes gestáre cúpimus luce accénsos bene✠dícere et sancti✠ficáre atque lúmine supérne benedictiónis accéndere dignéris quátinus eos tibi Dómino Deo nostro offeréndo digni et sancto igne dulcíssime tue claritátis succéndi in templo sancto glórie
\end{lesson}

\begin{noinitial}

\gregorioscore{tue-representati-mereamur}

\end{noinitial}

\gregorioscore{per-omnia-secula-vere-dignum}

\emph{Hic mutet vocem suam quasi legendo sic.} Qui tecum vivit et regnat in unitáte Spíritus Sancti Deus. Per ómnia sécula seculórum amen.

\emph{Hic aspergantur candele aqua benedicta, et thurificet. Deinde sequantur ⟨he due orationes super aliquo cereo accenso cum⟩\footnote{1882:142.}}

\grecommentary{1519:140v}

\gregorioscore{dominus-vobiscum}

\begin{lesson}
\subsubsection{Oratio.}

\lettrine{D}{o}mine sancte Pater omnípotens lumen indefíciens qui ea cónditor ómnium lúminum: béne✠dic hoc lumen tuis fidélibus in honórem nóminis tui portándum: quátenus a te sanctificáti atque benedícti lúminis tue claritátis accendámur et illuminémur: et propítius concédere dignéris, ut velúti eódem igne quondam illuminásti Móysen fámulum tuum: ita illúmines corda nostra et sensus nostros, quátinus ad visiónem etérne claritátis perveníre mereámur. Per Christum ⟨Dóminum⟩.\footnote{1882:142.}

\emph{⟨Alia oratio cum} Orémus.⟩\footnote{1882:143.}

\lettrine{O}{m}nípotens sempitérne Deus: qui Unigénitum tuum ante témpora de te génitum, sed temporáliter de María vírgine incarnátum lumen verum et indefíciens ad repelléndas humáni géneris ténebras et ad incendéndum lumen fidéi et veritátis misísti in mundum: concéde propítius: ut sicut extérius corporáli ita et\footnote{`étiam', 1882:143.} intérius luce spirituáli irradiári mereámur. Per eúndem Christum ⟨Dóminum nostrum⟩.\footnote{1882:143.}
\end{lesson}

\emph{¶ His dictis sacerdos cum suis ministris descendat ad gradum ⟨chori⟩\footnote{1882:143.} et si episcopus presens fuerit officium exsequatur et in sedem suam se recipiat. Deinde accendantur candele et distribuantur cantore incipiente hanc antiphonam sequentem hoc modo.}

\gregorioscore{003645-lumen-ad-revelationem}

\emph{Repetatur antiphona post unumquemque versum.}

\begin{lesson}

Quia vidérunt óculi mei: salutáre tuum.

Quod parásti: ante fáciem ómnium populórum.

Lumen ad revelatiónem géntium: et glóriam plebis tue Israel.

Glória Patri.

Sicut erat.

\end{lesson}

\emph{Si necesse fuerit, reincipiatur psalmus cum antiphona. Deinde eat processio sicut in die nativitatis Domini singuli clerici cum cereis ardentibus in manibus suis. Adjecto quod unus sacristarum\footnote{`1517---altaristarum.' 1882:143.} in superpellicio post thuribularium ante subdyaconum deferat cereum cum aliis benedictum usui benedictionis fontium in vigiliis pasche et penthecostes specialiter reservatur, cantore incipiente ⟨antiphonam⟩.\footnote{1882:143.}}

\gregorioscore{200456-ave-gratia-plena}

\emph{⟨Postea sequitur alia antiphona.⟩\footnote{1882:143.}}

\gregorioscore{001293-adorna-thalamum-tuum}

\emph{Postea sequitur Responsorium.\footnote{`\emph{Antiphonam}', 1519:142r.}}

\gregorioscore{007537-responsum-accepit}

\emph{Tres clerici seniores in pulpito conversi ad populum in cappis sericis simul cantant hunc versum.}

\begin{noinitial}

\gregorioscore{007537a-hodie-beata}

\end{noinitial}

\emph{In introitu chori dicatur hoc ℟.} ⟨s\emph{equens, sine versu et cum versu pro dispositione precentoris⟩.\footnote{1882:144.}}

\gregorioscore{007869-videte-miraculum}

\emph{℣.} Suscépimus Deus misericórdiam tuam.

\emph{℟.} In médio templi ⟨tui⟩.

\emph{⟨℣.} Orémus.⟩

\begin{lesson}
\subsubsection{Oratio.}

\lettrine{E}{x}áudi Dómine plebem tuam et quem intrínsecus ánnua tríbuis devotióne venerári: interveniénte beáta Dei genitríce sempérque vírgine María intérius asséqui grátie tue luce concéde. Per eúndem Christum ⟨Dóminum nostrum. Amen⟩.
\end{lesson}

\chapter{¶ Sancte Agathe virginis.}

\emph{Si dominica fuerit ad processionem ℟.} Regnum mundi. 399.

\emph{In introitu chori de sancta Maria.} 50.

\chapter{¶ In annuciatione beate Marie.}

\emph{In quacumque feria contigerit eat processio sicut in festis majoribus duplicibus cantando ℟.}

1519:143v; AS:419; Ant-1519-S:64v; Brev-1531-S:42v.\footnote{1519:143v. omits the T.P. idication. It is found in 1555, unpaged.}

\gregorioscore{missing}

\emph{⟨In introitu chori cantetur hec antiphona sequens:}

1555:unpaged; AS:415; Ant-1519-S:61v; Brev-1531-S:40v.

3340. \gregorioscore{missing}

\emph{Tempore paschali.}

\gregorioscore{missing}

\emph{℣.} Roráte celi désuper.

\emph{℟.} Et nubes pluant justum: aperiátur terra, et gérminet salvatórem.

\emph{⟨℣.} Orémus.⟩

\begin{lesson}
\subsubsection{Oratio.}

\lettrine{D}{e}us, qui de beáte Maríe vírginis útero Verbum tuum ángelo nuntiánte carnem suscípere voluísti; presta supplícibus tuis, ut qui vere eam genitrícem Dei crédimus, ejus apud te intercessiónibus adjuvémur. Per eúndem Christum.⟩\footnote{1555, unpaged.}
\end{lesson}

\emph{In revertendo de omnibus sanctis ant.} Salvátor mundi. 315. \emph{℣.} Letámini in Dómino. 316. \emph{Oratio.} Infirmitátem nostram. 316.

\emph{In tempore paschali} Christus resúrgens. \emph{cum suo ℣. a toto choro.} 201. \emph{℣.} Surréxit Dóminus de sepúlchro. 209. \emph{Oratio.} Deus qui per Unigénitum. 209.

\chapter{¶ Sancti Georgii martyris.}

\emph{Si in dominica fuerit ad processionem responsorium.} Fílie Hierúsalem. 392.

\emph{In introitu chori de sancta Maria.} 291. \emph{Quare predictum ℟. in communi unius martyris tempore paschali.}

\chapter{¶ Sancti Marci evangeliste.}

\emph{Si dominica fuerit ad processionem responsorium.} Cándidi facti sunt. \emph{require in communi unius apostoli paschali tempore. In introitu chori de sancta Maria. Et tunc nihil fiat de jejunio nec de processione que solet fieri ipso die propter dominicam illo anno. Si hoc festum in aliqua feria post octavas pasche contigerit tunc fiat processio post magnam missam de sancto Marco. Hoc modo ordinetur processio ad gradum chori sicut in simplicibus dominicis dum hora ix. canitur: qua cantata incipiat cantor antiphonam} Exúrge Dómine. 228. \emph{et cantetur a toto choro loco nec habitu mutato, in stallis antequam exeat processio. Deinde sequantur omnes antiphone per ordinem sicut in feria secunda rogationum et exeat processio eodem modo et ordine quo in ⟨predicta⟩\footnote{1882:146.} secunda feria rogationum sed absque leone et drachone ad aliquam ecclesiam in urbe vel in suburbio, et dicatur missa dominicalis vel missa de pace} Missale:{[}258{]}. \emph{vel} Salus pópuli. Missale:{[}194{]}. \emph{pro dispositione cantoris vel pro quacunque necessitate cum memoria de festo loci et de omnibus sanctis et cum uno} Allelúya. \emph{tantum.}

\emph{Cantata missa tres clerici de superiori gradu habitu non mutato dicant hanc letaniam} Kyrie⟨léyson⟩ qui precióso. 256. \emph{in redeundo ad ecclesiam propriam et finiatur ad gradum chori deinde dicat sacerdos ℣.} Vox letície et exultatiónis.\footnote{`1517---Letámini in Dómino.' 1882:146.} 266. \emph{Oratio.} Presta quésumus omnípotens Deus ut in.\footnote{`1517---Infirmitatem.' 1882:146.} 266.

\chapter{¶ Sancti Vitalis martyris.}

\emph{Si dominica fuerit ad processionem resp.} Fílie Hierúsalem. \emph{⟨in communi unius martyris tempore paschali.⟩\footnote{1882:146.}} 392.

\emph{In introitu chori de sancta Maria.} 291.

\chapter{¶ Apostolorum Philippi et Jacobi.}

\emph{Si dominica fuerit ad processionem} Cándidi facti. \emph{⟨quere in communi unius apostoli tempore paschali.⟩\footnote{1882:146.}} 391.

\emph{In introitu chori de sancta Maria.} 291.

\chapter{¶ In inventione sancte crucis.}

\emph{Ad j. vesperas post memorias eat processio ordinata cum ceroferariis et thuribulario ante crucem per medium chori sine cruce cantando antiphonam sequens rectoribus incipientibus.}

1519:144v; AS:424; Ant-1519-S:70v; Brev-1531-S:47v.\footnote{1519:144v. has no flats.}

\gregorioscore{missing}

\emph{Et finiatur cum} Allelúya. \emph{quacumque dicatur cum} Allelúya.

\emph{Dum antiphona canitur thurificet sacerdos crucem: et postea dicat ℣.} Dícite in natiónibus. 220. \emph{Oratio.} Deus qui pro nobis Fílium tuum. \emph{ut supra in die pasche ad vesperas.} 220.

\emph{In introitu chori de sancta Maria. ℣.} Sancta Dei génitrix. 220. \emph{Oratio.} Grátiam tuam. \emph{⟨ut supra in die pasche ad vesperas.⟩\footnote{1882:146.}} 220.

\emph{Si hoc festum post diem ascensionis Domini celebretur tunc oratio} Concéde quésumus miséricors Deus. \emph{loco} Grátiam tuam. \emph{dicetur.} 220.

\emph{Si hoc festum in sabbato contigerit nulla fiat processio\footnote{`MS.---ad secudas vesperas.' 1882:147.} nec ulla memoria nisi de sequente dominica tantum.}

\emph{Si vero in die ascensionis evenerit differatur in crastinum ⟨et⟩\footnote{1882:147.} ad secundas vesperas de ascensione fiat sollemnis memoria de cruce procedendo ante crucem ⟨cum cruce⟩\footnote{1882:147.} cum hac antiphona:} O crux gloriósa. 341. \emph{et non dicatur} O crux benedícta. \emph{propter nimiam suam brevitatem.}

\emph{℣.} Hoc signum crucis erit in celo.

\emph{⟨Resp.} Cum Dóminus.⟩\footnote{1882:147.} 308.

\emph{⟨℣.} Orémus.⟩

\begin{lesson}
\subsubsection{Oratio.}

\lettrine{D}{e}us qui in preclára salutífere crucis inventióne passiónis tue mirácula suscitásti: concéde ut vitális ligni précio etérne vite suffrágia consequámur. Qui vivis et regnas cum Deo Patre.
\end{lesson}

\emph{In introitu chori de sancta Maria.} 291.

\emph{¶ In die ejusdem si dominica fuerit ad processionem dicitur hoc responsorium.}

1519:145r; AS:425; ant-1519-S:71v; Brev-153-S:48r.\footnote{In 1519-S:71v. the first clef is an F-clef. The guide at the end of the line suggests that the clef should be a C-clef.}

\gregorioscore{missing}

\emph{In introitu chori de sancta Maria.} 291.

\chapter{¶ Sancti Johannis ante portam Latinam.}

\emph{Si dominica fuerit ad processionem ℟.} Cándidi facti sunt. 391.

\emph{In introitu chori de sancta Maria.} 291.

\chapter[¶ ⟨In die⟩ sancti Dunstani episcopi et confessoris.]{¶ ⟨In die⟩\footnote{1882:147.} sancti Dunstani episcopi et confessoris.}

\emph{Si dominica fuerit ad processionem ℟.} Fílie Hierúsalem. 392.

\emph{In introitu chori de sancta Maria.} 291. \emph{vel} 309.

\chapter{¶ In vigilia sancti Johannis baptiste.}

\emph{Post vesperas fiat processio ad altare sancti Johannis cum ceroferariis, thuribulario et puero librum deferente ante sacerdotem sine cruce cantando sequentem ℟. rectore incipiente.}

1519:145v; AS:431; Ant-1520-S:5r; Brev-1531-S:58v.\footnote{1519:145v. has no accidentals. `1517--- Rectores simul cantent versum.' 1882:148.}

\gregorioscore{missing}

\emph{⟨Sine} Glória Patri.⟩\footnote{1555, unpaged.}

\emph{Dum versus canitur thurificet sacerdos altare, deinde imaginem sancti Johannis: et postea dicat ℣.} Glória et honóre coronásti eum Dómine. 363.

\begin{lesson}
\subsubsection{Oratio.}

\lettrine{S}{a}ncti Johánnis baptíste mártyris tui Dómine quésumus veneránda festívitas salutáris auxílii nobis prestet efféctum. Per Christum Dóminum nostrum. Amen.
\end{lesson}

\emph{In redeundo de sancta Maria.} 309.

\chapter[⟨In die sancti Johannis baptiste.⟩]{⟨In die sancti Johannis baptiste.⟩\footnote{1882:148.}}

\emph{In die ejusdem si dominica fuerit ⟨dicitur hoc responsorium⟩.\footnote{1882:148.}}

1519:146r; AS:434; Ant-1520-S:7r; Brev-1531-S:59v.\footnote{1519:146r.has no flats.}

\gregorioscore{missing}

\emph{⟨℣.⟩} Glória Patri. 424. †~Qui viam.

\emph{In introitu chori de sancta Maria.} 309.

\chapter{¶ In festo sanctorum Johannis et Pauli.}

\emph{⟨Si dominica fuerit.⟩\footnote{1882:148.}}

\emph{Ad primas vesperas in sabbato fiat processio ante crucem, nisi aliqua dominica omnino oporteat differri: tunc memoria de dominica et de Trinitate non fiat, nec processio ⟨In die eorum ante missam si dominica fuerit ad processionem⟩ ℟.} In circúitu tuo. \emph{quere in communi plurimorum martyrum.} 396.

\emph{In introitu chori de sancta Maria.} 309.

\chapter{¶ In die apostolorum Petri et Pauli.}

\emph{Si dominica fuerit ad processionem ante missam dicatur istud ℟. ⟨sequens hoc modo.⟩\footnote{1882:149.}}

1519:146v; AS:446; Ant-1520-S:16v; Brev-1531-S:64r.\footnote{1519:146v. has no flats.}

\gregorioscore{missing}

\emph{In introitu chori de sancta Maria.} 309.

\chapter{¶ Commemoratio sancti Pauli apostoli.}

\emph{Si dominica fuerit ad processionem dicatur hoc responsorium.}

1519:147r; AS:393; Ant-1519-S:40v; Ant-1520-S:21r; Brev-1531-S:65v.\footnote{1519:147r. has no flats.}

\gregorioscore{missing} \gregorioscore{missing}

\emph{℣.} Glória Patri. 420. ‡~Qui te ⟨elégit⟩.\footnote{SB-S:386.}

\emph{In introitu chori de sancta Maria.} 309.

\chapter{¶ In festo visitationis Marie.}

\emph{Ad processionem}.

\gregorioscore{a00177-salve-qua-christi-mater}

\emph{In redeundo de omnibus sanctis} Salvátor mundi. \emph{tantum.} 315. \emph{℣.} Letámini in Dómino. 316. \emph{Oratio.} Infirmitátem. 316.

\chapter{¶ In octava apostolorum Petri et Pauli.}

\emph{Si dominica fuerit ad processionem resp.} Cives apostolórum. 393.

\emph{In introitu chori de sancta Maria.} 309.

\chapter[¶ Translatio sancti Thome martyris.]{¶ Translatio sancti Thome martyris.\footnote{In 1519:148v. this procession is crossed out. In 1523. the folio is missing.}}

\emph{Si dominica fuerit ad processionem ℟.} Jacet granum. \emph{require in festo ejusdem ebdomade natalis Domini} 38. \emph{: et dicetur sine prosa.}

\emph{In introitu chori de sancta Maria.} 309.

\chapter{¶ In festo reliquiarum.}

\emph{Ad processionem ℟.} Concéde nobis. \emph{℣.} Adjúvent. \emph{require in festo omnium sanctorum.} 381. \emph{Finito responsorio cum suo ℣. et} Glória Patri. \emph{si tantum restat iter fiat statio in ecclesia, ibique legantur nomina reliquiarum in lingua materna: et interim abluantur ibi reliquie, choro sequente. Quibus finitis processio more solito in chorum redeat cantore incipiente antiphonam} Salvátor mundi. 315. \emph{cum versu} Letámini in Dómino. 316. \emph{Oratio.} Infirmitátem nostram. 316.

\chapter{¶ In festo sancte Margarete virginis et martyris.}

\emph{Ad j. vesperas fiat processio ad altare ejusdem cantando ℟. sequentem.}

1519:149r; AS:665; Ant-1519-C:43r; Brev-1531-P:81r.\footnote{1519:149r. has no flats. In 1519:149r. `tui' is set CDE.C.}

\gregorioscore{missing} 6142a.

\gregorioscore{missing}

\emph{℣.} Ora pro nobis beáta Margaréta.

\emph{℟.} Ut digni ⟨efficiámur promissiónibus Christi⟩.\footnote{1882:151.}

\begin{lesson}
\subsubsection{Oratio.}

\lettrine{E}{x}áudi nos Deus salutáris noster ut sicut de beáte Margaréte vírginis tue et mártyris festivitáte gaudémus ita pie devotiónis erudiámur afféctu. Per christum Dóminum. ⟨Qui tecum.⟩\footnote{1882:151.}
\end{lesson}

\emph{In introitu chori de sancta Maria.}

\emph{In die ejusdem ⟨si dominica fuerit⟩\footnote{1882:151.} ad processionem ⟨ante missam dicitur⟩\footnote{1882:151.} ℟.} Regnum mundi. 399.

\emph{In introitu chori de sancta Maria.} 309.

\chapter{¶ In festo sancte Marie Magdalene.}

\emph{Ad j. vesperas eat processio ad altare ejusdem cantando ℟.} Regnum mundi. 399.

\emph{℣.} Dimíssa sunt ei peccáta multa.

\emph{⟨℟.} Quóniam diléxit multum.⟩\footnote{1882:151.}

\emph{⟨℣.} Orémus.⟩

\begin{lesson}
\subsubsection{Oratio.}

\lettrine{S}{a}cratíssimam Dómine beáte Maríe Magdaléne qua celos súbiit sollennitátem recenséntes supplíciter rogámus cleméntiam tuam: ut qui devotiónis ejus recólimus insígnia ipsíus mereámur compótes fíeri glórie. Per Christum.
\end{lesson}

\emph{In redeundo de sancta Maria.} 309.

\emph{¶ In die ejusdem si dominica fuerit, ad processionem ante missam ⟨dicitur responsorium⟩}.\footnote{1882:151.}

1519:149v; AS:462; Ant-1520-S:46v; Brev-1531-S:89v.\footnote{1519:149v. has a flat only at `Glória Patri.' In 1519:149v. `éthere' is set FAAFGFD.CD.D.}

\gregorioscore{missing}

\emph{¶ In introitu chori de sancta Maria.} 309.

\chapter{¶ In festo sancte Anne.}

\emph{Si dominica fuerit ad processionem ℟.}

1519:150r; Ant-1520-S:52v; Brev-1531-S:94v.\footnote{1519:150r. has a flats only at `letos', `cunctis', and `et.' In 1519:150r. `remissionem' is set A.AGADAA.BCA.GA.A. In 1519:150r. `fac' is set AC; `post' is set BAG.}

\gregorioscore{missing}

\emph{In redeundo de sancta Maria.} 309.

\chapter{¶ Ad vincula Petri.}

\emph{Si fuerit dominica fuerit ad processionem.}

1519:150v; AS:464, 469; Ant-1520-S:57v; Brev-1531-S:97v.

\gregorioscore{missing}

\emph{In introitu chori dicatur aliqua antiphona de sancta Maria.} 309.

\chapter{¶ In festo transfigurationis Domini.}

\emph{Ad processionem.}

\gregorioscore{007859-videns-petrus}

\emph{In redeundo ⟨dicatur⟩\footnote{1882:152.} de sancta Maria.} 309.

\chapter{¶ In festo nominis Jesu.}

\emph{Ad processionem.}

\gregorioscore{a00177-salve-qua-jesus}

\emph{⟨Chorus idem repetat post unumquemque versum. Clerici versus.⟩\footnote{1555, unpaged.}}

\gregorioscore{missing}

\emph{In redeundo de sancta Maria.} 309.

\chapter{¶ In festo sancti Laurentii.}

\emph{Ad primas vesperas eat processio ad altare ejusdem cantando.}

1519:153r; AS:485; Ant-1520-S:85v; Brev-1531-S:110v.

\gregorioscore{missing}

\emph{⟨℣.⟩} Glória Patri. 420. †~Et ideo.

\emph{℣.} Glória et honóre coronásti eum Dómine.

\emph{℟.} Et constituísti ⟨eum super ópera mánuum tuárum⟩.\footnote{1882:153.}

\emph{⟨℣.} Orémus.⟩

\begin{lesson}
\subsubsection{Oratio.}

\lettrine{D}{a} nobis quésumus omnípotens Deus: viciórum nostrórum flammas extínguere, qui beáto Lauréntio tribuísti tormentórum suórum incéndia superáre. Per Christum Dóminum nostrum. \emph{⟨℟.⟩} Amen.
\end{lesson}

\emph{In introitu chori de sancta Maria.} 309.

\emph{In die ejusdem si dominica fuerit ad processionem ℟.} Méruit esse. \emph{ut supra ad vesperas.}

\emph{In introitu chori de sancta Maria.} 309.

\chapter[¶ ⟨In die⟩ sancti Ypolyti cum sociis suis martyribus.]{¶ ⟨In die⟩\footnote{1882:154.} sancti Ypolyti cum sociis suis martyribus.}

\emph{Si hoc festum in dominica contigerit totum servitium fiat de festo et memoria de dominica ei de Trinitate et processio ante crucem ad j. vesperas.}

\emph{In die eorundem si dominica fuerit ad processionem.} In circúitu tuo. \emph{quere in communi plurimorum martyrum.} 396.

\emph{In introitu chori de sancta Maria.} 309.

\chapter{¶ In assumptione beate Marie.}

\emph{Quacunque die contigerit ad processionem.}

1519:153v; AS:498, 572; Ant-1519:100v; Ant-1520-S:94v; Brev-1531-S:117v.\footnote{1519:153v. has no flats. At the repeat after ℣. Glória. `Quia.' \emph{Chevallon.} ⟨SB-S:698.⟩ 1520-S:94v. omits `\emph{V.} Glória. †~Quia.' SB-S:98 has at the repeat after ℣. `Glória. Christus.' The settings in AS, for Assumption and for All Saints, have neither the `Allelúya' nor the final word of the ℣. `commemoratiónem.' In AS:498. the last syllable of `justície' is set AGDEFGEDEDDC.}

\gregorioscore{missing}

\emph{⟨℣.⟩} Glória Patri. 424. †~Quia ext te.

\emph{⟨Sequitur⟩\footnote{1882:154.} aliud responsorium.}

1519:154r; AS:496; Ant-1520-S:93v; Brev-1531:117r.\footnote{In 1519:154e. `concíves' is set G.GEG.EDG.}

\gregorioscore{missing}

\emph{⟨℣.⟩} Glória Patri. 426. ‡~Consórtes et concíves.

\emph{In introitu chori dicatur ista antiphona.}

1519:154r; AS:492; Ant-1520-S:91r; Brev-1531-S:105v.\footnote{In 1519:154r. the final note of `immortalitátis' is the first note of `locum.'}

\gregorioscore{missing}

\emph{℣.} Exaltáta es sancta Dei génitrix.

\emph{℟.} Super choros angelórum ad celéstia regna.

\emph{⟨℣.} Orémus.⟩

\begin{lesson}
\subsubsection{Oratio.}

\lettrine{V}{e}neránda nobis Dómine hujus diéi festívitas opem cónferat sempitérnam: in qua sancta Dei génitrix mortem súbiit temporálem, nec tamen mortis néxibus déprimi pótuit que Fílium tuum Dóminum nostrum de se génuit incarnátum. Qui tecum ⟨vivit⟩.\footnote{V1882:154.}
\end{lesson}

\emph{Si dies dominica infra octavos assumptionis contigerit processio fiat ante crucem nisi aliqua dominica omnino differatur propter prolixitatem temporis. Quando dicitur de omnibus sanctis in revertendo ad vesperas semper dicitur ista antiphona.} Salvátor mundi. 315. \emph{℣.} Letámini. 316. \emph{Oratio.} Infirmitátem. 316.

\emph{Eodem die fiat processio ante missam.}

1519:155r; AS:500; Ant-1520-S:96r; Brev-1531-S:118r.\footnote{1519:155r. has no flats.}

\gregorioscore{missing}

\emph{In introitu chori de omnibus sanctis antiphona.} Te gloriósus. 319. \emph{quando vero antiphona} Salvátor mundi. \emph{dicitur ad vesperas tunc in crastino ante missam in introitu chori dicitur antiphona} Te gloriósus. \emph{quando de omnibus sanctis dicitur in revertendo et semper cum hoc ℣.} Letámini. 316. \emph{et cum oratione} Infirmitátem. 316.

\emph{¶ In octava assumptionis si dominica fuerit ad processionem omnia fiant sicut in dominica infra octavas ejusdem ⟨ut⟩\footnote{1882:155.} supra notatum est.} 367.

\chapter{¶ In decollatione sancti Johannis baptiste.}

\emph{Si in dominica fuerit ad processionem ℟.} Perceptúrus. \emph{quere in communi unius martyris.} 390.

\emph{In introitu chori de sancta Maria.} 309.

\chapter{¶ In die nativitatis beate Marie.}

\emph{Quacunque die contigerit ⟨ad processionem⟩.\footnote{1882:155.}}

1519:155v; AS:524; Ant-1520-S:109v; Brev-1531-S:132r.\footnote{1519:155v. has no flats. In 1519:155v. `maris' is set GFDFFG.DF.}

\gregorioscore{missing}

\emph{⟨Sequitur⟩\footnote{1882:155.} aliud ⟨resp.⟩\footnote{1882:155.}}

1519:155v; AS:523; Ant-1520-S:108v; Brev-1531-S:131v.\footnote{In 1519.155v. `génuit' is set GAEDF.AGC.AGC.}

\gregorioscore{missing}

\emph{In introitu chori ⟨cantetur hec sequens antiphona⟩.\footnote{1882:156. `\emph{In redeundo ℟}. Styrps Jesse. \emph{vel} Natívitas tua.' Rylands-24:370.}}

1519:156r; AS:519; Ant-1520-S:106r; Brev-1531:130r.\footnote{1519:156r. has no flats.}

\gregorioscore{missing}

\emph{℣.} Sancta Dei génitix virgo semper María. 300.

\begin{lesson}
\subsubsection{Oratio.}

\lettrine{S}{u}pplicatiónem servórum tuórum Deus miserátor exáudi: ut qui in nativitáte Dei genitrícis Maríe congregámur, ejus intercessiónibus a te de instántibus perículis eruámur. Per eúndem Christum.
\end{lesson}

\emph{¶ Si dies dominica infra octavas ante festum exaltationis sancte crucis contigerit: fiat processio ante crucem. In introitu chori de omnibus sanctis ut supra in alio festo.}

\emph{¶ Dominica infra octavas beate nativitatis Marie cum extra festum exaltationis sancte crucis evenerit. Ad processionem ante missam ℟.} Ad nutum. 366. \emph{In introitu chori de omnibus sanctis ut in octava beate Marie.}

\chapter{¶ In exaltatione sancte crucis.}

\emph{Ad primas vesperas eat processio ante crucem cantando antiphonam.} O crux gloriósa. \emph{ut supra in alio festo.} 341. \emph{℣.} Dícite in natiónibus. 220. \emph{Oratio.} Deus qui Unigéniti. \emph{ut supra in historie} Deus ómnium. \emph{ad primas vesperas.} 291.

\emph{In introitu chori dicatur ista antiphona} Beáta Dei génitrix. 291. \emph{℣.} Elégit eam. Brev:36. \emph{Oratio.} Supplicatiónem servórum. \emph{ut supra in nativitate beate Marie.} 372.

\emph{In die ante missam si dominica fuerit ad processionem.}

1519:156v; AS:427; Ant-1520-S:118r; Brev-1531-S:139r.\footnote{1519:156v. has no flats. In 1519:156v. `cleménter' is set D.Dc.F; `nobis' is set AFAGAFEDEFD.D.}

\gregorioscore{missing}

\emph{¶ In introitu chori de sancta Maria.} 309.

\emph{In octava beate Marie si dominica fuerit ad processionem omnia fiant sicut in dominica infra octavas.} 372.

\chapter{¶ In festo sancti Matthei apostoli et evangeliste.}

\emph{Si dominica fuerit ad processionem ⟨cantetur hoc responsorium sequens⟩.\footnote{1882:157.}}

1519:157r; AS:549; Ant-1520-S:125r; Brev-1531-S:143v.\footnote{1519:157r. has no flats. In 1519:157r. juxta is set AAGEFED.EFE.}

\gregorioscore{missing}

\emph{⟨℣.⟩} Glória Patri. 424. ‡~Illuc.

\emph{In introitu chori de sancta Maria.} 309.

\chapter{¶ In festo sancti Michaelis archangeli.}

\emph{Ad primas vesperas fiat processio ad altare ⟨ejusdem⟩\footnote{1882:157.} cantando responsorium.}

1519:157v; AS:555; Ant-1520-S:129v; Brev-1531-S:147v.\footnote{1519:157v. has no flats. The editions maintain the cue `‡~Et' even though there is no ℣. Glória Patri' or final repetition, as there is at matins.}

\gregorioscore{missing}

\emph{Non dicatur ⟨℣.⟩} Glória Patri.

\emph{℣.} Stetit ángelus ⟨juxta aram templi.

\emph{℟.} Habens thuríbulum áureum in manu sua⟩.\footnote{1531-S:146r.}

\begin{lesson}
\subsubsection{Oratio.}

\lettrine{B}{e}áti archángeli tui Michaélis\footnote{BIn some Processionals this prayer begins `Beátus Michaélis archángeli.'} interventióne suffúlti te Dómine súpplices deprecámur: ut quos honóre proséquimur, contingámus et mente. Per Christum.
\end{lesson}

\emph{In redeundo de sancta Maria.} 309.

\emph{In die ante missam si dominica fuerit ad processionem ⟨dicitur hoc⟩.\footnote{1882:157.}}

1519:158r; AS:556; Ant-1520-S:130v; Brev-1531-S:146r.\footnote{1519:158r. has a flat only at `omnis.'}

\gregorioscore{missing}

\emph{In introitu chori de sancta Maria.} 309.

\chapter[¶ ⟨In die⟩ sancti Dionysii sociorumque ejus.]{¶ ⟨In die⟩\footnote{1882:157.} sancti Dionysii sociorumque ejus.}

\emph{Si dominica fuerit ad processionem responsorium.}

1519:158v; AS:564; Ant-1520-S:139r; Brev-1531:152v.\footnote{1519:158v. has no flats. In 1519:158v. `Beátus' is set C.D.DEFEDDC; `intulérunt' is set FEGA.FE.DCEFGED.}

\gregorioscore{missing}

\emph{⟨℣.⟩} Glória Patri. 420. ‡~Beátas.

\emph{In introitu chori de sancta Maria.} 309.

\chapter{¶ In translatione sancti Edwardi regis et confessoris.}

\emph{Eat processio ad primas vesperas ad altare ejusdem cantando responsorium.}

1519:159r; AS:653; Ant-1519-C:29v; Brev-1531-P:76v.\footnote{In 1519:159r. `rogántes' is set CD.D.BE.}

7580. \gregorioscore{missing}

\emph{¶ Dum versus canitur thurificet sacerdos altare, deinde imaginem sancti Edvvardi regis: deinde ⟨dicat⟩\footnote{1882:158.}}

\emph{℣.} Ora pro nobis beáte Edvvárde.

\emph{℟.} Ut digni efficiámur promissiónibus Christi.

\emph{⟨℣.} Orémus.⟩

\begin{lesson}
\subsubsection{Oratio.}

\lettrine{D}{e}us qui unigénitum tuum Dóminum nostrum Jesum Christum gloriosíssimo regi Edvvárdo in forma visíbili demonstrásti: tríbue quésumus ut ejus méritis et précibus ad etérnam ipsíus Dómini nostri Jesu Christi visiónem pertíngere mereámur. Qui tecum vivit et regnat.
\end{lesson}

\emph{In redeundo ⟨de⟩ sancta Maria.} 309.

\chapter{¶ Sancti Michaelis in monte Tumba.}

\emph{Si dominica fuerit ad processionem responsorium.}

1519:159v; AS:567; Ant-1520-S:131r; Brev-1531-S:148r.\footnote{In 1519:159v. `Dómine' is set DFGA.G.G.}

\gregorioscore{missing}

\emph{⟨℣.⟩} Glória Patri. 426. ‡~Contingámus.

\emph{In introitu chori de sancta Maria.} 309.

\chapter{¶ Sancti Luce evangeliste.}

\emph{Si dominica fuerit ad processionem ℟.} Cum ambulárent animália. \emph{ut supra in festo sancti Matthei ⟨apostoli⟩.\footnote{1882:158.}} 373.

\chapter{¶ In die omnium sanctorum.}

\emph{Qacunque die contigerit ad processionem ⟨cantetur istud responsorium sequens⟩.\footnote{1882:159.}}

1519:160v; AS:542, 577; Ant-1520-S:153r; Brev-1531:165v.\footnote{1519:160v. has no flats.}

\gregorioscore{missing}

\emph{In introitu chori antiphona} Salvátor mundi. \emph{require in festo Sancti Andree apostoli ad j. vesperas.} 315.

\emph{℣.} Letámini in Dómino ⟨et exultáte justi.

\emph{Resp.} Et gloriámini.⟩\footnote{1882:159.} 316.

\emph{⟨℣.} Orémus.⟩

\begin{lesson}
\subsubsection{Oratio.}

\lettrine{O}{m}nípotens sempitérne Deus qui nos ómnium sanctórum tuórum mérita sub una tribuísti sollennitáte venerári: quésumus ut desiderátam nobis tue propitiatiónis abundántiam multiplicátis intercessiónibus largiáris. Per Christum Dóminum nostrum.
\end{lesson}

\chapter{¶ In festo sancti Martini.}

\emph{Ad primas vesperas eat processio ad altare ejusdem cantando ℟.}

1519:161r; AS:592; Ant-1520-S:161r; Brev-1531:173r.\footnote{1519:161r. has no flats.}

\gregorioscore{missing}

\emph{⟨℣.⟩} Glória. 424. ‡~Celum.

\emph{Dum versus canitur thurificet sacerdos altare deinde ymaginem sancti Martini, postea dicat sacerdos}

\emph{℣.} Ora pro nobis beáte Martíne.

\emph{℟.} Ut digni efficiámur ⟨promissiónibus Christi.⟩\footnote{1882:159.}

\emph{⟨℣.} Orémus.⟩

\begin{lesson}
\subsubsection{Oratio.}

\lettrine{D}{e}us qui bonitátis auctor es et bonórum dispensátor concéde propítius: ut beáti Martíni confessóris tui atque pontíficis suffrágio, majestátis tue propitiatiónem consequámur. Per Christum.
\end{lesson}

\emph{In introitu chori de sancta Maria.} 309.

\emph{In die ejusdem ante missam si dominica fuerit ad processionem responsorium.} Martínus Abrahe. \emph{ut supra.} 383.

\emph{In introitu chori de sancta Maria.} 309.

\chapter{¶ In festo sancti Bricii.}

\emph{Si dominica fuerit totum fiat de festo: et ad primas vesperas processio fiat ante crucem.}

\chapter{¶ In festo sancti Edmundi episcopi et confessoris.}

\emph{Fiat processio ad j. vesperas ad altare ejusdem cantando ℟.} Sancte Edmúnde. \emph{cum ℣.} O sancte. \emph{sicut in translatione sancti Edvvardi regis et confessoris.} 378. \emph{Dum ℣. canitur thurificet sacerdos altare deinde ymaginem sancti Edmundi: deinde dicat sacerdos versiculum.} Ora pro nobis sancte Edmúnde. \emph{℟.} Ut digni efficiámur ⟨promissiónibus Christi⟩.\footnote{1882:160.}

\emph{⟨℣.} Orémus.⟩

\begin{lesson}
\subsubsection{Oratio.}

\lettrine{P}{l}enam in nobis etérne Salvátor tue virtútis operáre médelam: ut qui preclára beáti Edmúndi confessóris tui atque pontíficis mérita venerémur ipsíus adjúti suffrágiis, a cunctis animárum nostrórum languóribus liberémur. Per Christum.
\end{lesson}

\emph{In introitu chori de sancta Maria.} 309.

\chapter{¶ De sancto Edmundo rege et martyre.}

\emph{Fiat processio ad altare ejusdem ad j. vesperas in eundo et redeundo sicut in festo sancti Edmundi episcopi et confessoris cantando ℟.} Beátus vir qui suffert. \emph{in communi unius martyris.} 395.

\emph{℣.} Ora pro nobis beáte Edmúnde.

\emph{℟.} Ut digni efficiámur promissiónibus Christi.

\emph{⟨℣.⟩} Orémus.

\begin{lesson}
\subsubsection{Oratio.}

\lettrine{P}{r}esta quésumus omnípotens Deus: ut qui beáti Edmúndi regis et mértyris tui natalítia cólimus: intercessióne ejus in tui nóminis amóre roborémur. Per Christum.
\end{lesson}

\emph{In introitu chori de sancta Maria.} 309.

\chapter{¶ Sancte Cecilie virginis et martyris.}

\emph{Si dominica fuerit ad processionem.}

1519:162r; AS:pl. S; Ant-1520-S:169v; Brev-1531:185r.

\gregorioscore{missing}

\emph{⟨℣.⟩} Glória Patri. 421. ‡~Abjícite.

\emph{In introitu chori de sancta Maria.} 309.

\chapter[¶ Sancti Clementis pape et martyris.]{¶ Sancti Clementis pape\footnote{In 1519:162v. `\emph{pape}' is erased.} et martyris.}

\emph{Si dominica fuerit ad processionem.}

1519:162v; AS:pl. S; Ant-1520-S:171r; Brev-1531-S:187r.\footnote{`abundantia' \emph{Chevallon.} ⟨SB:1094.⟩ `abundántia', 1519:162v. 1519:162v. has a flat only at `ad' and throughout the ℣. In 1519:162v. `confessóribus' is set F.AC.CD.CCB.CBA. In 1555, unpaged, the repeat is to `‡~Ut de benefíciis.'}

\gregorioscore{missing}

\emph{⟨℣.⟩} Glória Patri. 421. †~Ut ⟨confessóribus⟩.\footnote{1882:161.}

\emph{In introitu chori de sancta Maria.} 309.

\chapter[¶ ⟨In festo⟩ De sancte Katharine ⟨virginis⟩.]{¶ ⟨In festo⟩\footnote{1882:161.} De sancte Katharine ⟨virginis⟩.\footnote{1882:161.}}

\emph{Ad primas vesperas fiat processio ad altare ejusdem cum ceroferariis et thuribulario cantando.}

1519:163r; AS:pl. Y; Ant-1520-S:176r; Brev-1531-S:189r.

\gregorioscore{missing}

\emph{Non dicatur} Glória Patri.

\emph{Dum ℣. canitur thurificet sacerdos altare deinde imaginem beate Katharine: et postea dicat ⟨sacerdos⟩\footnote{1882:161.} ℣.} Ora pro nobis beáta Katharína.

\emph{℟.} Ut digni ⟨efficiámur promissiónibus Christi⟩.\footnote{1882:161.}

\emph{⟨℣.} Orémus.⟩

\begin{lesson}
\subsubsection{Oratio.}

\lettrine{D}{e}us qui dedísti legem Moysi in summitáte montis Sýnay et in eódem loco per sanctos ángelos tuos corpus beáte Katharíne vírginis et mártyris tue mirabíliter collocásti: tríbue quésumus, ut ejus mentis et intercessiónibus ad montem qui Christus est valeámus perveníre. Per Christum Dóminum nostrum.
\end{lesson}

\emph{In introitu chori de sancta Maria.} 309.

\emph{¶ In die ejusdem si dominica fuerit ad processionem ⟨dicitur hoc resp⟩.\footnote{1882:162.}}

1519:163v; ⟨AS:pl. W; Ant-1520-S:173v; Brev-1531-S:189r.\footnote{1519:163v. has no flats.}

\gregorioscore{missing}

\emph{In introitu chori de sancta Maria.} 309.

