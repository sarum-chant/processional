\chapter{¶ In natali unius apostoli ⟨sive⟩ evangeliste in paschali tempore.}

\emph{Si dominica fuerit ad processionem ⟨dicatur istud responsorium sequens⟩.\footnote{1882:162.}}

1519:163v; AS:plate H.; Ant-1519-C:1r; Brev-1531-P:66r.

\gregorioscore{} 6263.

\gregorioscore{}

\emph{⟨℣.⟩} Glória Patri. 424. ‡~Allelúya.

\emph{In introitu chori de sancta Maria.} 291.

\chapter{In natali unius martyris sive confessoris paschali tempore.}

\emph{Si dominica fuerit ad processionem.}

1519:164r; AS:pl. L; Ant-1519-C:3v; Brev-1531-P:66v.

\gregorioscore{}

\emph{⟨℣.⟩} Glória Patri. 424. ‡~Allelúya.

\emph{In introitu chori de sancta Maria.} 291.

\chapter{¶ In natali unius apostoli sive plurimorum apostolorum extra tempus paschale.}

\emph{Si dominica fuerit ad processionem.}

1519:164v; AS:plate Q; Ant-1519-C:9r; Brev-1531-P:67v.

\gregorioscore{} 6289.

\gregorioscore{} 6289a.

\gregorioscore{}

\emph{⟨℣.⟩} Glória Patri. 425. †~Et liberáre.

\emph{In introitu chori de sancta Maria.} 309.

\chapter{¶ In natali unius martyris decollati quando communis historia dicitur.}

\emph{Si dominica fuerit ad processionem.}

1519:165r; AS:639; Ant-1519-C:15v; Brev-1531-P:70v.\footnote{1519:165r. has no flats. In 1519:165r. `gaudens' is set D.CEE; `vívere' is set FGAGFG.FE.E.}

\gregorioscore{}

\emph{In introitu chori de sancta Maria.} 309.

\chapter{¶ De uno martyre non decollato quando communis historia dicitur.}

\emph{Si dominica fuerit ad processionem.}

1519:165v; AS:638; Ant-1519-C:16r; Brev-1531-P:70v.\footnote{AS:639. does not indicate `℣. Glória. †~Quam repromísit.'}

\gregorioscore{}

6232a. \gregorioscore{}

\emph{⟨℣.⟩} Glória Patri. 425. †~Quam.

\emph{In introitu chori de sancta Maria.} 309.

\chapter{¶ In nativitate plurimorum martyrum quando communis historia dicitur.}

\emph{Si dominica fuerit ad processionem ⟨dicatur istud responsorium sequens⟩.\footnote{1882:163.}}

1519:166r; AS:648; Ant-1519-C:24r; Brev-1531-P:74v.

6891. \gregorioscore{}

\emph{⟨℣.⟩} Glória Patri. 424. ‡~Ibi.

\emph{In introitu chori de sancta Maria.} 309.

\chapter[¶ In natali unius confessoris et pontificis ⟨sive doctoris⟩.]{¶ In natali unius confessoris et pontificis ⟨sive doctoris⟩.\footnote{1882:163.}}

\emph{Si dominica fuerit ad processionem.}

1519:166v; AS;655; Ant-1519-C:31r; Brev-1531-P:77r.\footnote{In 1519:166v. `\emph{N.}' is set GBG.A.A.}

\gregorioscore{}

\emph{In introitu chori de sancta Maria.} 309.

\chapter{¶ In nativitate unius confessoris et doctoris sive abbatis quando historia communis dicitur.}

\emph{Si dominica fuerit ad processionem ℟.} Miles Christi. \emph{ut supra.} 395.

\emph{In introitu chori de sancta Maria.} 305.

\chapter{¶ In natali plurimorum confessorum.}

\emph{Si dominica fuerit ad processionem ⟨dicitur⟩ responsorium.\footnote{1882:164.}}

1519:166v; AS:576, 661; Ant-1519-C:38v; Brev-1531-P:80r.⟩\footnote{In 1519:166v. `precíncti' ends CDE.D; `homínibus' is set CD.D.DEFD.C. In 1519:167r. a repeat is indicated by the capital at `†~Et.' as at matins of many confessors in the Breviary, but the cue for the repetition is `‡~Quando.'}

\gregorioscore{} 7675a.

\gregorioscore{}

\emph{⟨℣.} Glória Patri. 424. ‡~Quando.⟩\footnote{1555, unpaged.

  \gregorioscore{}}

\emph{In introitu chori de sancta Maria.} 309.

\chapter{¶ In nativitate unius virginis et martyris quando communis historia dicitur.}

\emph{Si dominica fuerit ad processionem.}

1519:167r; AS:666; Ant-1519-C:44r; Brev-1531-P:81v.

\gregorioscore{}

\emph{In tempore paschali in processione contra reginam suscpiendam finiatur sic.}

7524a. \gregorioscore{}

\emph{In tempore paschali post} Glória Patri. \emph{repetatur a choro} ‡~Allelúya.

\emph{In introitu chori ⟨dicatur aliqua antiphona⟩\footnote{1882:164.} de sancta Maria.} 309.

\chapter[⟨¶ De processionibus causa necessitatis factis.⟩]{⟨¶ De processionibus causa necessitatis factis.⟩\footnote{1882:164.}}

\lettrine{F}{}\emph{Iunt autem quedam processiones causu necessitatis vel tribulationis scilicet pro serenitate aeri vel ad pluviam postulandam, vel contra mortalitatem hominum vel in tempore belli, vel pro pace ecclesie vel pro aliqua tribulatione que eodem modo et ordine ordinantur per omnia quo in processionibus ferialibus xl.}

\emph{Procedant autem per medium chori ad unum altare ejusdem ecclesie vel ad aliquam ecclesiam in urbe vel in suburbio: eodem modo et ordine per omnia quo in ij. feria in rogationibus sed absque drachone et leone et vexillis que non deferantur, choro sequente habitu non mutato cantando antiphonam vel ℟. ut patet inferius: et cum vij. psalmis penitentialibus si tantum restat iter, cum letania et collectis.}

\chapter{¶ Pro serenitate aeris.}

\emph{Ant.} Inundavérunt. \emph{Ps.} Salvum me fac. \emph{require post primum v. psalmi statim dicatur} Glória Patri. \emph{et} Sicut erat. \emph{Deinde repetatur antiphona.} 234.

\emph{Alia antiphona.} Non nos demérgat. \emph{Ps.} Exáudi\footnote{`Exáudi nos', 1555, unpaged.} Dómine. Glória Patri. Sicut erat. \emph{Deinde repetatur antiphona ⟨ut supra feria secunda rogationum⟩}.\footnote{1882:165.} 235.

\chapter{¶ Ad pluviam postulandam.}

\emph{Ant.} Dómine Rex ⟨Deus⟩ Abraham. \emph{Ps.} Exúrgat Deus. Glória Patri. Sicut erat. \emph{⟨Deinde⟩\footnote{1882:165.} repetatur antiphona. Similiter fiat post omnes alias ⟨antiphonas⟩}.\footnote{1882:165.} 231.

\emph{Alia ant.} Numquid est in ydólis. \emph{Ps.} Exúrgat Deus. 231.

\emph{Ant.} Exáudi Dómine. \emph{Ps.} Plúviam voluntáriam.\footnote{In 1882:165. the order is confused: `\emph{Ant.} Plúviam voluatáriam. \emph{Ps.} Exáudi Dómine.'} 232.

\emph{⟨Ant.} Réspice, Dómine. \emph{Non sequitur psalmus. Require has antiphonas feria secunda rogationum.⟩\footnote{1882:165.}} 234.

\chapter{¶ Contra mortalitatem hominum tempore belli.}

\emph{Ant.} Líbera Dómine. \emph{Ps.} Dómine Dóminus noster. Glória Patri. Sicut. \emph{Repetatur antiphona ⟨ut supra⟩\footnote{1882:165.} require has antiphonas feria ij. rogationum.} 236.

\chapter{¶ Pro pace petenda.}

\emph{Dicitur unum vel plura de ℟. subscriptis per ordinem in eundo quantum sufficit iter.}

\emph{℟.} Afflícti pro peccátis. \emph{require in processionibus ferialibus xl.} 76.

1519:168r; AS:321; Ant-1520:40r; Brev-1531:194r.\footnote{In 1520:40r. `convérte' is set CD.D.D.}

\gregorioscore{}

\emph{In tempore paschali.}

\gregorioscore{}

\emph{⟨Sequitur aliud responsorium.⟩\footnote{1882:165.}}

1519:168v; AS:321; Ant-1520:40r; Brev-1531:194v.

\gregorioscore{}

\emph{⟨Postea sequitur aliud responsorium.⟩\footnote{1882:166.}}

1519:169r; AS:324; Ant-1520:43v; Brev-1531:196v.\footnote{1519:169r. has no flats.}

6687. \gregorioscore{}

6687a. \gregorioscore{}

1519:169r; AS:325; Ant-1520:44r; Brev-1531:196v.\footnote{`pátentia', 1519:169r.}

\gregorioscore{} 7793.

\gregorioscore{}

\emph{In tempore paschali.}

\gregorioscore{} 7793a.

\gregorioscore{}

1519:169v; AS:320; Ant-1520:39r; Brev-1531:192v.\footnote{1519:169v. has no flat.}

\gregorioscore{}

\emph{In tempore paschali.}

\gregorioscore{}

\emph{¶ Si transierit processio ad aliquam ecclesiam in urbe vel in suburbio ubi fieri debet statio dicitur antiphona vel ℟. de sancto de quo est ecclesia illa: ita quod ad januas cymiterii vel citius inchoetur. Finita antiphona vel ℟. statim dicat sacerdos ad gradum chori ℣. et orationem de sancto ejusdem ecclesie sine} Dóminus vobíscum. \emph{sed tantum cum} Orémus. \emph{qua dicta clerici eodem modo et ordine quo in processione ordinantur se prosternant ita quod sacerdos ad gradum altaris cum diacono a dextris et subdiacono a sinistris suam faciant prostrationem cum dicitur} Kyrieléyson. Christeléyson. Kyrieléyson. \emph{sine nota sequatur} Pater noster. \emph{cetera omnia sicut predictum est in processione feriali xl.} 80. \emph{His dictis incipiatur ⟨missa⟩\footnote{1882:166.} ex sua causa processione precedente a cantore.}

\emph{¶ Missa pro fratribus et sororibus. Officium.} Salus pópuli. \emph{Epistola} Hec dicit. \emph{⟨et cetera⟩.\footnote{1882:166.}} Missale:{[}194{]}.

\emph{¶ Post missam duo clerici de secunda forma habitu non mutato dicant hanc letaniam que terminetur ab eisdem ad gradum chori in ecclesia propria hoc modo sequente.}

1519:170r; 1519-P:172r; Brev-1531-P:48v.\footnote{The processionals do not indicate the repetition of the petitions. In 1519:170r. `de celis' is set A C.A. 1519:170r. omits `Spíritus sancte Deus: Miserére nobis.'; In 1519:170r. `unus' is set C.C.}

\gregorioscore{}

\emph{Cetera ut supra} 83. \emph{usque ad}

\gregorioscore{}

\emph{Deinde statim sequatur.}

\gregorioscore{}

\emph{Chorus idem repetat.}

\emph{Clerici prosequantur sic versum.\footnote{In 1519:170v. `Ab' is set D.}}

\gregorioscore{}

\emph{Chorus idem repetat et sic de singulis ℣. ut in secunda feria rogationum.}

\gregorioscore{}

\emph{Chorus idem.}

\emph{Clerici.}

\gregorioscore{}

\emph{Chorus respondent sic.\footnote{1519:170v. omits `audi nos.'}}

\gregorioscore{}

\emph{Clerici.}

\gregorioscore{}

\emph{Chorus respondeat.}

\gregorioscore{}

\emph{Et sic de singulis ℣. ut supra in ij. feria rogationum.}

\emph{Clerici versum.}

\gregorioscore{}

\emph{Chorus idem versus.}

\emph{Clerici.\footnote{1519:171r. omits the flats.}}

\gregorioscore{}

\emph{Chorus idem versus.}

\emph{Clerici.}

\gregorioscore{}

\emph{Chorus idem versus.}

\emph{Clerici.}

\gregorioscore{}

\emph{Chorus idem.}

\emph{Clerici.}

\gregorioscore{}

\emph{Chorus idem.}

\emph{Clerici.}

\gregorioscore{}

\emph{Chorus idem.}

\emph{Clerici.\footnote{1519:171r. omits the flat.}}

\gregorioscore{}

\emph{Deinde dicat sacerdos ℣.} Letámini in Dómino. 312.

\begin{lesson}
\subsubsection{Oratio.}

\lettrine{I}{n}firmitátem nostram quésumus propícius réspice: et mala ómnia que juste merémur: ómnium sanctórum tuórum intercessióne avérte. Per Dóminum.
\end{lesson}

\emph{⟨require in festo sancti Andree.⟩\footnote{1882:167.}} 316.

\chapter{¶ Cum corpus defuncti portatur ad ecclesiam.}

\emph{Dicitur hec antiphona.}

1519:171v.

\gregorioscore{}

\emph{Repetatur antiphona. Deinde dicatur iste psalmus} De profúndis ⟨clamávi⟩.\footnote{1882:168.} 62. \emph{et post unumquemque versum repetatur antiphona. Deinde si necesse fuerit dicatur iste psalmus} ⟨In éxitu.⟩\footnote{1882:168.} Brev:{[}318{]}. \emph{in ordine.}

\emph{In introitu vel citius ⟨sequens⟩\footnote{1882:168.} ℟. inchoetur.}

1519:171v; AS:583; Ant-1519-P:188v; Brev-1531-P:53r.

\gregorioscore{}

7091g. \gregorioscore{}

\emph{Et dicitur cum uno versu tantum.}

\emph{In introitu chori dicatur sequens antiphona, et aspergatur corpus aqua benedicta et nunquam debet corpus alicujus defuncti portari circa cimiterium sed directe in ecclesiam et sequatur ⟨alia antiphona hoc modo quo sequitur⟩.\footnote{1882:168.}}

1519:172r.

\gregorioscore{}

\emph{Repetatur antiphona et statim dicatur quod sequitur.}

1519:172r.

\gregorioscore{}

\emph{Tunc sacerdos aspergat corpus cum aqua benedicta et thurificet rogans} Oráte pro ánima \emph{N.} et pro animábus fidélium defunctórum.

\emph{⟨℣.⟩} Pater noster. Brev:{[}5{]}.

\emph{⟨℣.⟩} Et ne nos. \emph{⟨℟.⟩} Sed líbera.

\emph{⟨℣.⟩} A porta ínferi. \emph{⟨℟.⟩} Erue Dómine ánimas eórum.

\emph{⟨℣.⟩} Non intres in judícium servo tuo. \emph{⟨℟.⟩} Quia non justificábitur ⟨in conspéctu tuo omnis vivens.⟩\footnote{1882:168.}

\emph{⟨℣.⟩} Dóminus vobíscum. \emph{⟨℟.⟩} Et cum spíritu tuo.

\emph{⟨℣.⟩} Orémus.

\begin{lesson}
\subsubsection{Oratio.}

\lettrine{S}{u}scipe Dómine servum tuum \emph{vel} ancíllam tuam \emph{N.} in bonum habitáculum: et da ei réquiem in regno celésti Hierúsalem, ut in sinu Abrahe patriárche tui collocátus resurrectiónis diem prestolétur, et inter resurgéntes ad glóriam resúrgat: et cum benedíctis ad déxteram Dei veniéntibus véniat: et cum possidéntibus vitam etérnam possídeat. Per Christum.
\end{lesson}

\emph{Tunc sic.} Anima ejus et ánime ómnium fidélium defunctórum per misericórdiam Dei in pace requiéscant. Amen. Pater noster.

\chapter[⟨¶ Processio causa venerationis.⟩]{⟨¶ Processio causa venerationis.⟩\footnote{1882:169.}}

\lettrine{F}{}\emph{Iunt autem quedam processiones causa venerationis ad suscipiendum archiepiscopum, proprium episcopum, legatum vel cardinalem, regem vel reginam: que eodem modo ordinantur, et habitu quo in die nativitatis Domini: procedant autem per medium chori et ecclesie per ostium occidentale usque ad locum destinatum, ibique ad personam suscipiendum disponantur in processione. Duo excellentiores persone in cappis sericis: et ipsam personam reverenter in revertendo suscipiant. Post thurificationem et aque benedicte aspersionem cantore incipiat ⟨responsorium⟩\footnote{1882:169.} scilicet contra archiepiscopum: proprium episcopum, legatum vel cardinalem: dicatur hoc ℟.} Summe Trinitáti. \emph{ut in festo Sancte Trinitatis.} 277. \emph{Contra regem dicatur ℟.} Honor virtus. \emph{⟨ut supra in festo Sancte Trinitatis⟩}.\footnote{1882:169.} 279. \emph{Contra reginam dicatur hoc ℟.} Regnum mundi. \emph{Require in communi unius virginis} 395. \emph{: eadem quoque via qua accesserunt usque ad gradum altaris accedant.}

\emph{Finito responsorio cum suo ℣. a toto choro sequatur} Kyrieléyson. Christeléyson. Kyrieléyson. \emph{sine nota.} Pater noster. Brev:{[}5{]}. \emph{Deinde super archiepiscopum: episcopum proprium: legatum vel cardinalem cum prosternant se in oratione ad gradum altaris, dicat sacerdos in cappa serica ℣. et orationes}

\emph{⟨℣.⟩} Et ne nos ⟨indúcas in tentatiónem⟩. \emph{⟨℟.⟩} Sed líbera ⟨nos a malo⟩.

\emph{⟨℣.⟩} Salvum fac servum tuum Dómine. \emph{⟨℟.⟩} Deus meus sperántem in te.

\emph{⟨℣.⟩} Mitte ei Dómine auxílium de sancto. \emph{⟨℟.⟩} Et de Syon túere eum.

\emph{⟨℣.⟩} Nichil profíciat inimícus in eo. \emph{⟨℟.⟩} Et fílius iniquitátis non appónat nocére ei.

\emph{⟨℣.⟩} Esto ei Dómine turris fortitúdinis. \emph{⟨℟.⟩} A fácie inimíci.

\emph{⟨℣.⟩} Dómine exáudi ⟨oratiónem meam⟩. \emph{⟨℟.⟩} Et clamor ⟨meus at de véniat⟩.

\emph{⟨℣.⟩} Dóminus vobíscum. \emph{⟨℟.} Et cum spíritu tuo.⟩

\emph{⟨℣.⟩} Orémus.

\begin{lesson}
\subsubsection{Oratio.}

\lettrine{C}{o}ncéde quésumus Dómine, fámulo tuo \emph{N.} metropolitáno nostro \emph{vel} epíscopo \emph{vel} preláto ut predicándo et exercéndo que recta sunt exémplo bonórum óperum ánimas suórum ínstruat subditórum: et etérne remuneratiónis mercédem a te piíssimo pastóre percípiat. Per Christum.
\end{lesson}

\emph{¶ Super regem vel reginam in prostratione ad gradum altaris dicat sacerdos in cappa serica ℣.} Et ne ⟨nos indúcas in tentatiónem⟩ \emph{⟨℟.⟩} Sed líbera ⟨nos a malo⟩.

\emph{⟨℣.⟩} Osténde nobis ⟨Dómine misericórdiam tuam⟩.\footnote{1882:170.} \emph{⟨℟.⟩} Et salutáre tuum ⟨da nobis⟩.\footnote{1882:170.}

\emph{⟨℣.⟩} ⟨Dómine⟩\footnote{1882:170.} salvum fac regem \emph{vel} ancíllam tuam. \emph{⟨℟.⟩} Et exáudi nos in die qua invocavérimus te.

\emph{⟨℣.⟩} Mitte ei Dómine auxílium de sancto. \emph{⟨℟.⟩} Et de Syon tuére eum.

\emph{⟨℣.⟩} Nichil profíciat inimícus in eo. \emph{⟨℟.⟩} Et fílius iniquitátis non appónat nocére ei.

\emph{⟨℣.⟩} Dómine Deus virtútum convérte nos. \emph{⟨℟.⟩} Et osténde fáciem tuam et salvi érimus.

\emph{⟨℣.⟩} Dómine exáudi ⟨oratiónem meam. \emph{℟.} Et clamor meus ad te véniat.⟩

\emph{⟨℣.⟩} Dóminus vobíscum. \emph{⟨℟.} Et cum spíritu tuo.⟩

\emph{⟨℣.⟩} Orémus.

\begin{lesson}
\subsubsection{Oratio.}

\lettrine{D}{e}us in cujus manu corda sunt regum qui es humílium consolátor et fidélium fortitúdo: et protéctor ómnium in te sperántium, da regi nostro \emph{vel} regíne nostre populóque Christiáno triúmphum virtútis tue sciénter excólere ut per te semper reparéntur ad véniam. Per Christum.
\end{lesson}

\chapter[⟨¶ Antiphone de beata Maria.⟩]{⟨¶ Antiphone de beata Maria.⟩\footnote{1882:170.}}

1519:173r.

\gregorioscore{}

1519:174v.

\gregorioscore{}

1519:175r; AS: 55; Ant-1519:63r, 97v; Brev-1531:31v.⟩

\gregorioscore{}

1519:175r.

\gregorioscore{}

\emph{℣.} Ave María grátia plena Dóminus tecum.

\emph{℟.} Benedícta ⟨tu in muliéribus.⟩\footnote{1882:171.}

\emph{⟨℣.⟩} ⟨Orémus.⟩\footnote{1882:171.}

\begin{lesson}
\subsubsection{Oratio.}

\lettrine{O}{m}nípotens sempitérne Deus qui glorióse vírginis et matris Maríe corpus et ánimam: ut dignum Fílii tui habitáculum éffici mererétur Spíritu Sancto cooperánte mirabíliter preparásti: da ut cujus commemoratióne letámur ejus pia intercessióne ab instántibus malis et subitánea morte et improvísa liberémur. Per eúndem ⟨Christum Dóminum nostrum.⟩\footnote{1882:171.}
\end{lesson}

\emph{⟨Postea sequitur responsorium.⟩\footnote{1882:171. This is in fact an Antiphon with Verse.}}

1519:175r.

\gregorioscore{}

1519:175v.

\gregorioscore{}

1519:175v; AS:518; Ant-1520-S:105r; Brev-1531-S:7v.

\gregorioscore{}\footnote{1508. and 1555. include here five proses, which are not properly part of the Processional but of the Antiphonale (\emph{O morum doctor egregie}, \emph{Sospitati dedit}, \emph{Inviolata integra et casta}, \emph{Crux fidelis}, and \emph{Eterne virgo memorie},) followed by the seven penitential psalms.}

\begin{verse}
⟨Quis satis enumeret, quantam pressoria nobis\\*
Conferat ars cunctis utilitatis opem\\*
Tempore sub modico facili commissa papiro\\*
Et precio parvo grandia lector habes\\*
Innumeris mendis scriptorum ignavia quondam\\*
Tedia quam multis grandia sepe dabat\\*
Sed cum pressores patuerunt arte periti\\*
Palladis occulte tunc rediere faces\\*
Ergo tibi hoc munus Sarensis mater habeto\\*
Quod tibi divinis proderit officiis.\\*
Sane hoc pressorum digessit in arte magister\\*
Martinus Morin incola Rothomagi.⟩\footnote{1508:unpaged; 1555:unpaged.}
\end{verse}

\begin{centering}
⟨¶ Processionale cum bonis notulis et bonis li-\\
gaturis: atque cum stationibus picturatis infra ap-\\
positis Secundum usum insignis ad preclare\\
ecclesie Sarum: noviter correctum, ac rursus\\
emendatum, per Christophorum Endoviensis.\\
Antwerpie excusum. Impensis\\
honesti viri Francisci Byrk-\\
man. Anno ab incarna-\\
tione Domini. 1523.\\
Die vero 16.\\
Martii.\\
Laus soli Deo.

✠

¶ Venales reperiuntur Londoniis in cimiterio\\
sancti Pauli, a Francisco Byrckman,\\
vel as suis servitoribus.⟩\footnote{1523:76r, 76v.}
\end{centering}

\chapter{Versi responsoria.}

Primus modus.

ST:x.

\gregorioscore{}

Secundus modus.

ST:xvij.

\gregorioscore{}

Tertius modus.

ST:xxj.

\gregorioscore{}

Quartus modus.

ST:xxxj.

\gregorioscore{}

Quintus modus.

ST:xxxv.

\gregorioscore{}

Sextus modus.

ST:xliij.

\gregorioscore{}

%TODO: add flat?

\gregorioscore{}

Septimus modus.

ST:liij.

\gregorioscore{}

Octavus modus.

ST:lxv.

\gregorioscore{}

\chapter{Ad Officium.}

\gregorioscore{}