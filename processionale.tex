\documentclass[
  fontsize=14pt,
  paper=letter,
  DIV=12,
  headings=normal,
  parskip=never]
{scrbook}

\usepackage[autocompile]{gregoriotex}
\usepackage{unicode-math}
\usepackage{microtype}
\usepackage[open]{bookmark}
\usepackage{xurl}
\usepackage{marginnote}

\usepackage[canadian, latin]{babel}

\addto\captionslatin{
  \RenewDocumentCommand\contentsname{}
    {Tabula contentorum}
}

\defaultfontfeatures{Scale=MatchLowercase}
\defaultfontfeatures[\rmfamily]{Ligatures=TeX,Scale=1}
\setmainfont[Numbers={Lowercase,Proportional}]{JuniusX}

% Decorative fonts
\newfontfamily\gothic{DS Caslon Gotisch}[Scale=1]
\newfontfamily\lombardic{Luminari}
\newfontfamily\woodcut{TypographerWoodcutInitialsOne}

% heading formatting
\setcounter{secnumdepth}{-\maxdimen} % remove section numbers
\RenewDocumentCommand\raggedsection{}
  {\centering} % centre all headings
\setkomafont{disposition}{\normalfont\color{gregoriocolor}} % all headings in red (gregoriocolor = rubric colour)
\addtokomafont{caption}{\color{gregoriocolor}}
\addtokomafont{pageheadfoot}{\color{gregoriocolor}}
\addtokomafont{pagenumber}{\color{gregoriocolor}}
\RenewDocumentCommand\marginfont{}
  {\noindent\rule{0pt}{0.7\baselineskip}\color{black}\footnotesize}

% title page formatting
\addtokomafont{subject}{\normalfont\scshape\addfontfeatures{Letters=UppercaseSmallCaps}\Huge}
\addtokomafont{publishers}{\scshape\addfontfeatures{Letters=UppercaseSmallCaps}}
\addtokomafont{date}{\scshape\addfontfeatures{Letters=UppercaseSmallCaps}}

\RedeclareSectionCommand[afterskip=1sp]{subsubsection}

% Use \emph for rubrics
\RenewDocumentCommand\emph{m}
  {{\color{gregoriocolor}{#1}}}

\hypersetup{
  pdfusetitle,
  pdflang=la,
  unicode,
  hidelinks
}

% Line spacing
\usepackage{setspace}
\setstretch{1.2}

% Columns with vertical rule
\usepackage{multicol}
\setlength{\columnsep}{1.5pc}
\setlength{\columnseprule}{0.7pt}

% Drop caps
\usepackage{lettrine}
\LettrineRealHeighttrue
\RenewDocumentCommand\LettrineTextFont{}
  {\MakeUppercase} % make word following cap all caps
\RenewDocumentCommand\LettrineFontHook{}
  {\color{gregoriocolor}}
\setlength{\DefaultNindent}{0em}

\urlstyle{same} % disable monospaced font for URLs

\setlength{\emergencystretch}{3em} % prevent overfull lines

% Move footnote marker into left margin
\deffootnote{0em}{1.6em}{\thefootnotemark\enskip}
\setfootnoterule{0pt} % Remove footnote rule

% use all small caps for \textsc
\RenewDocumentCommand\textsc{m}
  {{\scshape\addfontfeatures{Letters=UppercaseSmallCaps}#1}}

% Font substitution
\usepackage{newunicodechar}
\newfontfamily{\coelacanth}{Coelacanth}
\newunicodechar{♭}{{\coelacanth ♭}}

\newunicodechar{℣}{{\scshape\addfontfeatures{Letters=UppercaseSmallCaps} ℣}}
\newunicodechar{℟}{{\scshape\addfontfeatures{Letters=UppercaseSmallCaps} ℟}}

% TODO: should these be coloured only in scores?
\newunicodechar{†}{{\color{gregoriocolor} †}}
\newunicodechar{‡}{{\color{gregoriocolor} ‡}}
\newunicodechar{⹋}{{\color{gregoriocolor} ⹋}}

\newfontfamily{\jenson}{Adobe Jenson Pro}
\newunicodechar{¶}{{\jenson ¶}}

% TODO: licence for commercial use of Brill requested; alternatively STIX Two Math renders the character this way, but it needs to be raised to sit on the baseline
\newfontfamily{\brill}{Brill}
\newunicodechar{✠}{{\brill ✠}}
\newunicodechar{𝔄}{{\brill 𝔄}}
\newunicodechar{𝔈}{{\brill 𝔈}}
\newunicodechar{𝔒}{{\brill 𝔒}}

% Endnotes
\usepackage{enotez}
\setenotez{
  list-name={Notae},
  backref=true
}
\AtEveryEndnotesList{
  \markboth{Notae}{Notae}
  \setlength{\columnsep}{2pc}
  \setlength{\columnseprule}{0pt}
  \begin{multicols}{2}
}
\AfterEveryEndnotesList{\end{multicols}}
\let\footnote=\endnote % turn footnotes into endnotes

% remove figure numbers
\RenewDocumentCommand\figureformat{}{}
\RenewDocumentCommand\captionformat{}{}

% Graphics
\usepackage{graphicx}
\makeatletter
\def\maxwidth{\ifdim\Gin@nat@width>\linewidth\linewidth\else\Gin@nat@width\fi}
\def\maxheight{\ifdim\Gin@nat@height>\textheight\textheight\else\Gin@nat@height\fi}
\makeatother
% Scale images if necessary, so that they will not overflow the page
% margins by default, and it is still possible to overwrite the defaults
% using explicit options in \includegraphics[width, height, ...]{}
\setkeys{Gin}{width=\maxwidth,height=\maxheight,keepaspectratio}


% Gregorio settings
\gresetgregpath{{./scores/}}

\grechangestaffsize{20} % larger staff
\grechangedim{spacebeneathtext}{1ex}{scalable}

\gresetlinecolor{gregoriocolor} % staff lines in red
\RenewDocumentCommand\GreItalic{m}
  {{\color{gregoriocolor}{#1}}} % use red for italics in scores
\grechangestyle{initial}{\woodcut\fontsize{90}{90}\selectfont}
\grechangestyle{commentary}{\footnotesize}
\grechangestyle{annotation}{\footnotesize}
\grechangestyle{abovelinestext}{\scriptsize}
\grechangestyle{firstsyllableinitial}{\MakeUppercase}
\grechangedim{annotationraise}{0mm}{scalable}
\grechangedim{commentaryraise}{5mm}{scalable}
\grechangedim{abovelinestextraise}{5mm}{scalable}

% set chant symbols
\RenewDocumentCommand\GreStar{}
  {{\color{gregoriocolor}*}}

% styles for chant lines
\NewDocumentEnvironment{noinitial}{}
  {\gresetinitiallines{0}\grechangestyle{firstsyllableinitial}{}}
\NewDocumentEnvironment{largeinitial}{}
  {\grechangestyle{initial}{\woodcut\fontsize{120}{120}\selectfont}}

% chant headers
\NewDocumentCommand\cantusmargin{m}
  {\marginnote{\href{http://cantusindex.org/id/#1}{#1}}}
\gresetheadercapture{cantus}{cantusmargin}{}
\gresetheadercapture{commentary}{grecommentary}{}
\gresetheadercapture{annotation}{greannotation}{}
\gresetheadercapture{annotation}{greannotation}{}

% two-column text layout
\NewDocumentEnvironment{lesson}{}
  {\begin{multicols}{2}
  \RenewDocumentCommand\LettrineFontHook{}
    {\lombardic\color{gregoriocolor}}
  \setcounter{DefaultLines}{2}}
  {\end{multicols}}

\NewDocumentCommand\newrubric{}
  {\hfill \break \noindent}

\input{ushyphex} % standard English exceptions

\recalctypearea

\begin{document}

\frontmatter

\extratitle{
  \begin{figure}
  \centering
  \includegraphics[height=0.7\textheight]{images/1r-title.png}
  % \caption{Oxford, Bodleian Library, Gough Missals 75, fol. 1r}
  \end{figure}
}

\frontispiece{%
    {\gothic\Large\par

    \begin{minipage}[t]{\textwidth}
      ¶ Processionale ad vsum insignis ac preclare ecclesie Sarum nouiter ac rursus castigatum per excellentissimum ac vigilantissimum et reuerendissimum In Christo patrem Dominum nostrum dominum episcopum de Wyntonii feliciter incipit. Parisiensis impressum per Wolfgangum Hopylium impensis honesti viri Francisci Bryckman ciuis Coloniensis. Anno Domini M.CCCCC.xix. Die vero xxviij. Octobris.
    \end{minipage}
    \vfill
    \begin{minipage}[b]{\textwidth}
      Venundantur Londonii apud Franciscum Bryckman, in cimiterio sancti Pauli.
    \end{minipage}}
}

\title{Processionale ad usum Sarum}

\author{edited by\\William Renwick\\Brandon Wild\\Andrew Dunning}

\date{MMXXI}

\publishers{Gregorian Institute of Canada}

\lowertitleback{
\KOMAoptions{parskip=half}

\begin{otherlanguage}{canadian}
© 2021 Gregorian Institute of Canada

School of the Arts, McMaster University, 1280 Main Street West, Hamilton, Ontario, \textsc{L8S 4L8}, Canada

\url{http://sarum-chant.ca}

Images courtesy of the Bodleian Library, University of Oxford.
\end{otherlanguage}
}

\maketitle

\tableofcontents

\listoffigures

\chapter{Introduction}

\section{Sources}

\textit{Processionale ad vsum insignis ac preclare ecclesie Sarum nouiter castigatum per dominum episcopum de vuynton̄. feliciter incipit.} Par. venundātur Lond. apud F. byrckman (1519). STC 16235. The only surviving copy is Oxford, Bodleian Library, Gough Missals 75. Missing fol.~123.

\section{Editorial Method}

\clearpage

\begin{figure}
\thispagestyle{empty}  
\centering
\includegraphics{images/1v-statio-aqua.png}
\caption{¶ Statio dum benedicitur aqua benedicta in omnibus dominicis.}
\end{figure}

\mainmatter

\chapter{Benedictio salis et aque.}

\marginnote{2r}\lettrine[lines=7,image=true,findent=.5em]{images/2r-initial.png}{}\emph{Mnibus dominicis diebus per annum post primam et capitulum~: nisi in duplicibus festis et in dominica ramis palmarum, a sacerdote ebdomadario alba et cum cappa serica induto cum diacono et subdiacono textum deferentibus cum thuribulario et duobus ceroferariis~: et acolito crucem deferente omnibus albis cum amictibus indutis et in medio presbyterii ad altare conversis~: et ⟨cum⟩\footnote{Barlow 7:1r; 1882:1.} duobus pueris quorum unus\footnote{`\emph{alter}', 1882:1.} scilicet puer\footnote{`\emph{quorum alter scilicet}', Barlow 7:1r.} qui ad aquam scribitur in tabula in sale\footnote{`\emph{salem}', 1882:1.} tenendo et aquam benedictam gestando~: alter puer scilicet\footnote{`\emph{scilicet puer}', 1882:2.} ebdomadarius lector ad manus in libro tenendo eidem sacerdoti in superpellicio ministrent ad gradum chori fiat benedictio salis et aqua sic incipiente.\footnote{`\emph{hoc modo}', Barlow 7:1r; `\emph{incipiendo}', 1882:2.} ✠}

\begin{largeinitial}

\grecommentary{1519:2r.\footnote{In Barlow 7:1r ff. the versicles appear a fifth higher, in the C clef.}}

\gabcsnippet{(f3) Ex(hv)or(h)cí(h)zo(h) te(hv) cre(hv)a(h)tú(h)ra(h) sa(hv)lis(h) per(hv) De(hv)um(h) vi(hv)✠vum(h)}

\end{largeinitial}

\begin{lesson}
\noindent per Deum ve✠rum\footnote{In Balrow 7:1r. the three crosses are a later addition.} per Deum sanc✠tum~: per Deum qui te per Helíseum prophétam in aquam mitti jussit~: ut sanarétur sterílitas aque ut efficiáris sal exorcizátus\footnote{`exorcizátum', Barlow 7:1r; 1882:2.} in salútem credéntium, et sis\footnote{`ut sit', Barlow 7:1r.} ómnibus te suméntibus sánitas ánime et córporis~: et effúgiat atque discédat ab eo loco quo aspérsus\footnote{`aspérsum', Barlow 7:1r; 1882:2.} fúeris~: omnis fantásia et nequítia vel versútia diabólice fraudis~: omnísque spíritus immúndus adjurátus. Per eum qui ventúrus est judicáre vivos et mórtuos et

\begin{noinitial}

\gabcsnippet{(f3) sé(hv)cu(h)lum(h) per(hv) ig(hv)nem.<v>\footnote{SP:2r. has no flat. 1519:2r. has no flat.}</v>(dxd) (::)}

\end{noinitial}

\end{lesson}

\emph{Chorus respondeat sic.}

\begin{noinitial}

\gabcsnippet{(f3) A(gh)men.(hv) (::)}

\emph{Et sic finiatur omnes exorcismi per totum annum.}

\emph{Sequitur oratio sine} Dóminus vobíscum. \emph{sed tantum cum}

\gabcsnippet{(f3) O(hv)ré(h)mus.(f) (::)}

\end{noinitial}

\begin{lesson}
\lettrine{I}{m}ménsam cleméntiam tuam omnípotens etérne Deus humíliter implorámus~: ut hanc creatúram salis\footnote{1517: \emph{hic respiciat salem.} ⟨1882:2.⟩} quam in usum géneris humáni\footnote{`humáni géneris', Balrow 7:1r.} tribúisti, bene✠dícere et sanctificáre\footnote{1517---sancti✠ficáre. ⟨1882:2.⟩ In Barlow 7:1r. a second cross has been added here in a later hand.} tua pietáte dignéris~: ut sit\footnote{`sis', Barlow 7:1r.} ómnibus suméntibus salus men\marginnote{2v}tis et córporis~: ut\footnote{`et', Barlow 7:1r; 1882:3.} quicquid ex eo tactum vel respérsum fúerit cáreat omni immundícia~: omníque impugnatióne spirituális \emph{⟨qui finietur hoc modo⟩\footnote{Barlow 7:1r.}}
\end{lesson}

\begin{noinitial}

\grecommentary{1519:2v.\footnote{SP:2v. has no flats. 1519--2v. has no flats.}}

\gabcsnippet{(f3) ne(hv)quí(h)ti(dxd)e.(d) (:)
Per(hv) Dó(h)mi(h)num(hv) nos(hv)trum(h) Je(h)sum(hv) Chris(h)tum(h) Fí(hv)li(h)um(h) tu(h)um(f) (,)
qui(h) te(h)cum(h) vi(h)vit(h) et(h) reg(h)nat(h) in(hv) u(h)ni(h)tá(h)te(h) Spí(h)ri(h)tus(h) Sanc(hv)ti(h) De(h)us(h) (,)
per(h) óm(h)ni(h)a(h) sé(h)cu(h)la(h)
se(h)cu(h)ló(h)rum.(dxd) (::)}

\emph{Chorus respondeat sic}

\gabcsnippet{(f3) A(gh)men.(hv) (::)}

\end{noinitial}

\emph{¶ Sub eodem tono finiantur omnes orationes sequentes ⟨et etiam oratio post aspersionem aque⟩.\footnote{‡ 1517, \&c.~add---Sub eodem tono finiantur omnes orationes sequentes et etiam oratio post aspersionem aque. ⟨1882:2.⟩; also Barlow 7:1r.}}

\subsubsection[¶ Sequatur exorcismus aque ita incipiens.]{¶ Sequatur exorcismus aque ita incipiens.\footnote{`\emph{incipiendo}', 1882:2.}}

\begin{lesson}
\lettrine{E}{x}orcízo te creatúra aque in nómine Dei Patris omnipo✠téntis~: et in nómine Jesu Christi ✠ fílii ejus Dómini nostri et in virtúte Spíritus ✠ Sancti\footnote{Barlow 7:1r-1v. omits the crosses.}~: ut fias aqua exorcizáta ad effugándam omnem potestátem inimíci~: et ipsum inimícum eradicáre et explantáre váleas cum ángelis suis apostáticis, per virtútem ejúsdem Dómini nostri Jesu Christi qui ventúrus est judicáre vivos et mórtuos et séculum per ignem. \emph{Chorus.} Amen.
\end{lesson}

\emph{Oratio sine} Dóminus vobíscum. \emph{sed tantum cum} Orémus. \emph{⟨supradicto modo.⟩\footnote{Barlow 7:1v.}}

\begin{lesson}
\subsubsection{Oratio.}

\lettrine{D}{e}us qui ad salútem humáni géneris máxima queque sacraménta in aquárum substántia condidísti~: adésto propítius invocatiónibus nostris~: et eleménto huic~: \emph{Hic respiciat sacerdos aquam.} multímodis purificatiónibus preparáto virtútem tue bene✠dictiónis infúnde~: ut creatúra tua mystériis tuis sérviens ad abjiciéndos demónes~: morbósque pelléndos divíne grátie sumat efféctum~: ut quicquid in dómi\marginnote{3r}bus vel in locis fidélium hec unda respérserit~: cáreat omni immundícia liberétur a noxa~: non illic resídeat spíritus péstilens~: non aura corrúmpens~: discédant omnes insídie laténtis inimíci et siquid est quod aut incolumitáti habitántium ínvidet aut quiéti~: aspersióne hujus aque effúgiat~: ut salúbritas per invocatiónem sancti tui nóminis expetíta~: ab ómnibus sit impugnatiónibus\footnote{`impugnatiónibus sit', Barlow 7:1v.} defénsa. Per Dóminum nostrum.\footnote{`\emph{et cetera}', 1882:3. `Jsum Christum Fílium.' Barlow 7:1v.}
\end{lesson}

\emph{¶ Hic mittat sacerdos sal in aquam in modum crucis ita per se privatim dicens.\footnote{`\emph{dicens per se privatim}', 1882:3.}} Commíxtio salis et aque páriter fiat in nómine Patris et Fílii et Spíritus Sancti. Amen.

\subsubsection[¶ Sequitur benedictio salis et aque pariter hoc modo.]{¶ Sequitur benedictio salis et aque pariter hoc modo.\footnote{`\emph{hoc modo incipiens}', 1882:3.}}

\begin{noinitial}

\grecommentary{1519:3r.\footnote{Barlow 7:1v. omits `Dóminus vobíscum \ldots{} Orémus.'}}

\gabcsnippet{(f3) Dó(hv)mi(h)nus(h) vo(hv)bís(h)cum.(f) (::)
<i>℟.</i> Et(hv) cum(h) spí(hv)ri(h)tu(h) tu(hv)o.(f) (::Z)
<i>℣.</i> O(hv)ré(h)mus.(f) (::)}

\end{noinitial}

\begin{lesson}
\subsubsection[Oratio.]{Oratio.\footnote{1882:3.}}

\lettrine{D}{e}us invícte virtútis auctor et insuperábilis impérii rex ac semper magníficus triumphátor~: qui advérse dominatiónis vires reprímis~: qui inimíci rugiéntis sevítiam súperas~: qui hóstiles nequítia poténs expúgnas, te Dómine treméntes et súpplices deprecámur ac pétimus, ut hanc creatúram \emph{Hic respiciat aquam sale mixtam.} salis et aque dignánter accípias~: benígnus illústres pietátis tue more\footnote{`rore~: \emph{Missale Herf.}~: amore~: \emph{Miss. Sar.} 1498, 1513 ; more~: \emph{Miss. Leofr.}, \emph{Rom.} 1474, \emph{Ebor.}, \emph{Sarum} 1492, 1527, \&c.' ⟨1901:19.⟩} sanctí✠fices ut ubicúnque fúerit aspérsa per invocatiónem tui sancti nóminis omnis infestátio immúndi spíritus abjiciátur~: terrórque venenósi serpéntis procul pellátur~: et preséntia Sancti Spíritus nobis misericórdiam tuam poscéntibus ubíque adésse dignétur. Per Dóminum nostrum Jesum Christum Fílium tuum. \emph{et cetera.\footnote{`qui tecum vivit et regnat', 1882:3. `qui tecum \emph{\&c.}' Barlow 7:1v.}} in unitáte ejúsdem Spíritus Sancti Deus. Per ómnia sécula seculórum. \emph{⟨℟.⟩} Amen.
\end{lesson}

\emph{¶ Si fuerit duplex festum in dominica extra chorum fiat benedictio salis et aque privatim ante aliquod altare~: et hora sexta\footnote{1517---hora tertia vel sexta. 1528, \&c.\emph{---}sexta. ⟨1882:3.⟩ `\emph{tertia}', 1882:3.} cantata aspergatur. In aliis vero dominicis in choro benedicatur~: et ante tertiam aspergatur~: nisi in dominica ra\marginnote{3v}mis palmarum tunc vero sicut in duplicibus diebus alibi benedicetur et sexta aspergatur licet duplex festum fuerit\footnote{Manual--- tunc enim extra chorum benedicatur et post sextam aspergatur, more duplicis festi, licet duplex festum non fuerit. ⟨1882:4.⟩ ⟨Barlow 7:1v.⟩ \emph{sicut in duplicibus festis observetur.} 1882:4.}~: et sic\footnote{1528:3r; 1519 has `non si hic.'} peracta benedictione salis et aque accedat ipse sacerdos ad principale altare et ipsum stando circumquaque aspergat. In redeundo in primis\footnote{`\emph{imprimis}', 1882:4.} aspergat ministros ordinatim incipiendo ab accolito qui crucem defert. Deinde ad gradum chori rediens ibidem singulos ab se accedentes clericos aspergat incipiens a majoribus. Tamen si episcopus presens fuerit ad eum pertinet aspersio clericorum tantum in sede sua incipiens a majorbus. Post aspersionem clericorum laicos in presbyterio hinc\footnote{Manual---hinc inde. ⟨1882:4.⟩} stantes aspergat. Et dum aspergatur aqua benedicta, cantetur sequens antiphona a cantor ita incipiente.}

\gregorioscore{001494-asperges-me-1}

\end{document}

\emph{Deinde repetatur antiphona} Aspérges me. \emph{Versus.}

\begin{noinitial}

\gregorioscore{001494-asperges-me-2}

\emph{Iterum repetatur ant.} Aspérges. \emph{Versus.}

\gregorioscore{001494-asperges-me-3}

\marginnote{4r}\emph{Repetatur in hunc modum.}

\gregorioscore{001494-asperges-me-4}

\end{noinitial}

\emph{Hec antiphona ⟨predicta⟩\footnote{1882:4.} dicatur in aspersione aque supradicto modo omnibus dominicis per annum de quocunque fit servitium~: et in dominica passionis ⟨Domini⟩\footnote{1882:4.} et dominica in ramis palmarum cum} Glória Patri \emph{et} Sicut erat \emph{preter quod\footnote{`\emph{preterquam}', 1882:4; Barlow 7:2r.} a pascha usque ad festum Sancte Trinitatis.}

\emph{⟨Et in hoc tempore dicatur sequens antiphona.}

\gregorioscore{005403-vidi-aquam-1}

\emph{Deinde repetatur antiphona.}

\begin{noinitial}

\gregorioscore{005403-vidi-aquam-2}

\emph{Deinde repetatur sic.}

\gregorioscore{005403-vidi-aquam-3}

\end{noinitial}

\emph{Hec antiphona predicta dicatur in aspersione aque in die pasche et omnibus dominicis a pascha usque ad festum Trinitatis.⟩\footnote{1530:4r.}}

\emph{Post aspersionem aque\footnote{`\emph{aque aspersionem}', 1882:5.} dicat sacerdos ⟨hunc versum⟩\footnote{1882:5.} ad gradum chori.}

\begin{noinitial}

\gregorioscore{008166-ostende-nobis}

\emph{sine\footnote{`\emph{Non dicatur}', 1882:5.}} Dóminus vobíscum. \emph{sed tantum cum}

\gabcsnippet{(c3) O(h)ré(hv)mus.(f) (::)}

\end{noinitial}

\begin{lesson}
\lettrine{E}{x}áudi nos Dómine sancte Pater omnípotens etérne Deus~: et míttere dignáre sanctum ángelum tuum de celis~: qui custódiat fóveat prótegat vísitet et deféndat omnes habitántes in hoc
\end{lesson}

\begin{noinitial}

\gabcsnippet{(c3) ha(hv)bi(h)tá(h)cu(d)lo.(d) (:)
Per(h) Chris(h)tum(h) Dó(h)mi(h)num(h) nos(hv)trum.(f) (::)}

\end{noinitial}


\emph{Chorus respondeat sic.}

\begin{noinitial}

\gabcsnippet{(c3) A(gh)men.(hv) (::)}

\end{noinitial}

\emph{Deinde eat processio hoc ordine. In primis\footnote{`\emph{Imprimis}', 1882:5.} procedat ministri virga\footnote{`\emph{virgam}', 1882:5.} manu gestans locum faciens processio.\footnote{`\emph{processioni}', 1882:5. Barlow 7:2r. omits the verger.} Deinde pueri in superpelliceis cum aqua benedicta gestante,\footnote{`\emph{puer in super pelliceo aquam benedictam gestans}', 1882:5.} deinde accolitus crucem ferens~: et post ipsum duo ceroferarii pariter incedentes deinde thuribularius~: \marginnote{4v}post eum subdiaconus, deinde diaconus omnes albis cum amictibus induti, et absque tunicis vel casulis et post diaconum~: eat sacerdos in simili habitu cum cappa serica. Deinde sequantur pueri et clerici de secunda forma habitu non mutato non bini et bini~: sed ex duabus partibus juxta ordinem quo disponuntur in choro. Et clerici\footnote{`\emph{reliqui clerici}', note, 1901:21.} de superiori gradu eodem ordine quo disponuntur in capitulo\footnote{`\emph{capite}', 1528:4r; 1544:5r; 1554:4v.} habitu non mutato~: sed excellentioribus sequentur per ordinem quod in omnibus dominicis non duplicibus observetur per totum annum.}

\emph{Et exeat processio per hostium presbyterii septentrionale, circuiens presbyterium. Episcopus si presens fuerit mitram gerat et baculum in fine processionis. Sacerdos vero sive presens fuerit episcopus sive non in anteriori parte post suos ministros procedat et in eundo singula altaria aspergat.}

\newrubric
\emph{¶ In duplicibus tamen festis que in dominica contingunt in procedendo altaria non aspergat~: sed eat post diaconum in habitu antedicto. Deinde ab australi parte ipsius ecclesie per fontes venientes procedant ad crucem, sacerdote cum suis ministris predictis ⟨in⟩\footnote{1882:6.} medio suo ordine stante. Ita quod puer deferens aquam et accolitus stent juxta\footnote{`\emph{ad}', 1882:6.} gradum ante crucem. Omnes autem clerici interesse possunt processioni totius anni~: licet nulle hore diei precedentis interfuerint.\footnote{`\emph{Quilibet autem processionem totius anni intrare possunt, licet nulli hore diei precedenti interfuerint.}' Consuetudinary (Cap. xiv. ad fin.) ⟨1882:xv.⟩ Henderson declares that the text of the Consuetudinary is correct here.}}

\chapter{⟨¶ Dominica prima adventus.⟩}

\emph{¶ Omnibus dominicis diebus per adventum ad processionem in eundo dicatur ista antiphonam hoc modo.}

\label{003792-missus-est}
\gregorioscore{003792-missus-est}

% TODO: Missing text from 5v
\marginnote{5r}\emph{¶ Quando vero pervenerit processio ante magnam crucem in ecclesia nisi fieri debeat statio non dicatur antiphona de cruce usque ad inceptionem\footnote{Barlow 7:2r-2v. breaks of the rubric at this point.} historie} Deus ómnium. \emph{Sed statim post antiphonam vertat se sacerdos ad populum et dicat in lingua materna sic.\footnote{Here Barlow 7:2v. includes a later marginal notation that appears to say `X. Ecclesia Salisburiensis orationes? et officium.' Then follows a form of the bidding of the bedes in English. See Appendix.}}

Orémus pro ecclésia Romána et pro papa et archiepíscopis, et epíscopis, et speciáliter pro epíscopo nostro \emph{N.} et pro decáno nostro \emph{vel} pro rectóre hujus ecclésie \emph{scilicet in ecclesiis parochialibus}~: et pro terra sancta, pro pace ecclésie et terre et ⟨pro⟩\footnote{1882:6.} rege\footnote{1528:5r. omits `\emph{et rege}.'} et regína et eórum\footnote{`suis', Dickinson:37**.} líberis.\footnote{`Let us pray for the church of Rome, and ⟨for⟩ the pope and ⟨for⟩ the archbishops, and ⟨for⟩ the bishops, and especially for our bishop \emph{N.} and for the dean \emph{or} for the rector of this church, \emph{of course in parish churches}~: and for the holy land, and for the peace of the church and ⟨these⟩ lands and ⟨for⟩ the king and queen and their children.' (trans. ed.)

  The form found in 1544:6r. is `Orémus pro ecclésia Anglicána et pro rege nostro et archiepíscopis, epíscopis et speciáliter pro epíscopo nostro \emph{N.} et pro decáno \emph{vel} rectóre hujus ecclésie, \emph{scilicet in ecclesiis parochialibus} et pro terra sancta pro pace ecclésie et terre et regína et suis líberis.' `Let us pray for the church of England, and ⟨for⟩ our kinge and ⟨for⟩ the archbishops, ⟨and for⟩ the bishops, and especially for our bishop \emph{N.} and for the dean \emph{or} rector of this church, \emph{of course in parish churches}~: and for the holy land, ⟨and⟩ for the peace of the church and ⟨these⟩ lands and ⟨for⟩ the queen and their children.' trans. ed.

  The rubric `\emph{et cetera more solito}' suggests that particular prayers would be added here according to local circumstances, as shown in the extract from Salisbury Cathedral MS 152:12r. given below. A simple order derived from MS 152 would be: `and for \emph{N.} and \emph{N.} (benefactors), and for our friends, for our brethren and sistren, and for all our parishioners, with all those that doth any good to this church, and for all true Christian people.'} \emph{et cetera more solito.\footnote{MS 152:12r. indicates here an additional `Pater noster' before the psalm.}}

\emph{Deinde vertat se sacerdos et dicatur\footnote{`\emph{dicatur et}', 1519:5v.} iste psalmus} Deus misereátur nostri. Brev:⟨51⟩. \emph{ex utraque parte chori more solito sine nota~: ex parte chori principali incipiatur. Finito psalmo cum} Glória Patri. \emph{et} Sicut ⟨erat⟩.\footnote{1882:6.} \emph{sequatur} Kyrieléyson. Christeléyson. Kyrieléyson. Pater noster. Brev:⟨5⟩. \emph{Deinde dicat sacerdos in audientia~: sed sine nota}

\emph{⟨℣.⟩} Et ne nos ⟨indúcas in tentatiónem.

\emph{℟.⟩} Sed libera ⟨nos a malo.

\emph{℣.⟩} Osténde nobis Dómine ⟨misericórdiam tuam⟩.\footnote{Barlow 7:3r; 1882:7.}

\emph{⟨℟.⟩} Et salutáre ⟨tuum da nobis⟩.\footnote{1882:7.}

\emph{⟨℣.⟩} Sacerdótes tui induántur justítiam.

\emph{⟨℟.⟩} Et sancti tui exúltent.

\emph{⟨℣.⟩} Dómine salvum fac regem.

\emph{⟨℟.⟩} Et exáudi nos ⟨in die qua invocavérimus te⟩.\footnote{1882:7.}

\emph{⟨℣.⟩} Salvum fac servum ⟨tuum⟩.\footnote{1882:7. `Salvum fac servos tuos et ancíllas tuas.' Barlow 7:3r.}

\emph{⟨℟.⟩} Deus meus sperántem in te.

\emph{⟨℣.⟩} Salvum fac pópulum tuum ⟨Dómine⟩.\footnote{⟨Dómine et bénedic hereditáti tue.⟩, Barlow 7:3r.}

\emph{⟨℟.⟩} Et rege eos ⟨et extólle illos usque in etérnum⟩.\footnote{1882:7.}

\emph{⟨℣.⟩} Dómine fiat pax ⟨in virtúte tua⟩.\footnote{1882:7; Barlow 7:3r.}

\emph{⟨℟.⟩} Et abundántia ⟨in túrribus tuis⟩.\footnote{1882:7.}

\emph{⟨℣.⟩} Dómine éxáudi oratiónem ⟨meam⟩.\footnote{1882:7; Barlow 7:3r.}

\emph{⟨℟.⟩} Et clamor ⟨meus ad te véniet⟩.\footnote{1882:7.}

\emph{⟨℣.⟩} Dóminus vobíscum. \emph{⟨℟.⟩} Et cum spíritu tuo.

\emph{⟨℣.⟩} Orémus.

\begin{lesson}
\lettrine{D}{e}us qui charitátis dona per grátiam Sancti Spíritus tuórum córdibus fidélium infúndis~: da fámulis et famulábus tuis~: pro quibus tuam deprecámur cleméntiam salútem mentis et córporis~: ut te tota virtúte díligant~: et que tibi plácita sunt tota dilectióne perfíciant~: et pacem tuam nostris concéde tempóribus. Per Christum Dominum nostrum. \emph{⟨℟.⟩} Amen.
\end{lesson}

\newrubric
\emph{¶ Item\footnote{`\emph{Deinde}', 1882:7.} conversus ad populum dicat sacerdos in lingua materna,} Orémus pro animábus \emph{N.} et \emph{N.\footnote{`Let us pray for the souls of \emph{N.} and \emph{N.}' trans. Ed.

  The rubric `\emph{more solito}' again suggests that particular prayers would be added here according to local circumstances, as shown in the extract from Salisbury Cathedral MS 152:12r. given below. A simple order derived from MS 152 would be: `For the souls of all bishops whose bodies rest in this holy place, especially \emph{N.} and \emph{N.} which bishops have in their time honoured this church with precious vestements and many other jewels. And for the souls of all the patrons of this church, and all other lords that have honoured it with their bodies, rents, or any other jewels, especially for \emph{N.} and \emph{N.} And for the souls of all other servants of this church which have truly serve it or done any good therero in their days, especially for \emph{N.} and \emph{N.} And for all the souls whose bones rest in this church and churchyard. And for the souls of all those that have given to this church rents, vestments, or any other goods, whereby God is more worshipped in this church, and the ministers thereof better sustained. And for the souls of all our friends, our brethren and our sistren, and for the souls of all our parishioners. And for the souls of all that have done any good to this church. And for all Christian souls.'} more solito\footnote{Ms 152:15r. again inserts and additional `Pater noster' here.}~: et postea vertat se sacerdos et dicat psalmum\footnote{`\emph{dicatur iste psalmus}', 1882:7.}} De profúndis clamávi. Brev:⟨340⟩. \emph{supradicto modo sine} Glória Patri. \emph{⟨et⟩\footnote{1882:7.} cum} Kyrieléyson. Christeléyson. Kyrieléyson. Pater noster. Brev:⟨5⟩.

\emph{⟨℣.⟩} Et ne nos indúcas in temptatiónem. \emph{⟨℟.⟩} Sed líbera nos a malo.

\emph{⟨℣.⟩} Réquiem etérnam dona eis Dómine. \emph{⟨℟.⟩} Et lux perpétua lucéat eis.

\emph{⟨℣.⟩} A porta ínferi. \emph{⟨℟.⟩} Erue Dómine ⟨ánimas eórum⟩.\footnote{1882:7.}

\emph{⟨℣.⟩} Credo vidére ⟨bona Dómini⟩.\footnote{1882:7.} \emph{⟨℟.⟩} In terra vivéntium.

\emph{⟨℣.⟩} Dóminus vobíscum. \emph{⟨℟.⟩} Et cum spíritu tuo.

\emph{⟨℣.⟩} Orémus.

\begin{lesson}
\lettrine{A}{b}sólve quésumus Dómine ánimas famulórum tuórum pontíficum et sacerdótum~: et ánimas famulórum familarúmque tuárum paréntum parrochianórum amicórum et\footnote{Barlow 7:3v. omits `et ánimas \ldots{} amicorum et.'} benefactórum nostrórum et ánimas\footnote{Barlow 7:3v. omits `ánimas.'} ómnium fidélium defunctórum ab omni vínculo delictórum~: ut in resurrectiónis glória inter sanctos eléctos\footnote{`el eléctos', 1519:6r.} resuscitári respírent. Per Christum ⟨Dóminum nostrum⟩.\footnote{Barlow 7:3v.}

\end{lesson}

\emph{⟨℣.⟩} Requiéscant in pace. \emph{⟨℟.⟩} Amen.

\newrubric
% TODO: he/hee?
\emph{He preces predicta dicuntur supradicto modo omnibus dominicis per annum~: sive de dominica~: sive ⟨de⟩\footnote{1882:8.} aliquo festo fit servitium~: nisi duplex festum fuerit~: et nisi in sexta die a nativitate Domini et in die sancti Silvestri si in dominica evenerit et nisi in dominica ⟨in⟩\footnote{1882:8.} ramispalmarum.}

\emph{¶ Ita tamen quod in ecclesiis parochialibus non ad processionem sed post evangelium et offertorium supradicto modo dicuntur ante aliquod altare in ecclesia vel in pulpito ad hoc constituto~: tunc psalmo} De profúndis. Brev:⟨340⟩. \emph{cum versiculi et oratione} Absólve quésumus Dómine. 11. \emph{semper dicatur in statione ante crucem in ecclesie supradicto modo nisi in duplicibus festis et in vj. die a nativitate Domini et in die sancti Silvestri quando in dominica evenerit, et nisi in dominica palmarum ut supra diximus.}

\emph{Finitis precibus intrent chorum cantore incipiente.\footnote{`\emph{In introitu chori vel citius cantor incipiat responsorium.}' 1882:8. In Cam-Queens-MS-28 (Gradual):10.: `I\emph{n stacione ante crucem nulla dicatur antiphona usque ad incepcionem} Deus ómnium. \emph{Finiatur antiphona statim incipiatur ℟. in introitu chori hoc modo.}' Beginning at \emph{Deus ómnium} the chant for the entry into the chancel is an \emph{antiphon} to the Virgin.}}

\gregorioscore{007068-letentur-celi-1}

\emph{Et ⟨dicatur⟩\footnote{1882:8.} cum versu vel sine versu per dispositione cantoris.}

\begin{noinitial}

\gregorioscore{007068-letentur-celi-2}

\end{noinitial}

\marginnote{6v}\emph{Sed sine} Glória Patri. \emph{Quod per totum annum observetur quando dicitur responsorium in introitu chori, et quando dicatur ℣. cantetur a toto choro.}

\emph{Deinde dicat sacerdos ad gradum chori versiculum sequentem.}

\begin{noinitial}

\gregorioscore{008246-vox-clamantis}

\emph{Non dicatur} Dóminus vobíscum \emph{post processionem, similiter fiat per totum annum sed tantum cum.}

\gabcsnippet{(c3) O(hv)ré(h)mus.(f) (::)}

\begin{lesson}
\lettrine{E}{x}cita quésumus\footnote{1882:8. omits `quésumus.'} Dómine poténtiam tuam et veni~: ut ab imminéntibus peccatórum nostrórum perículis~: te mereámur protegénte éripi~: te liberánte salvári. Qui vivis et regnas cum Deo Patre in unitáte Spíritus Sancti Deus. Per ómnia
\end{lesson}

\gabcsnippet{(c3) sé(hv)cu(h)la(h) sé(h)cu(h)ló(h)rum.(d) (::)
<i>℟.</i> A(gh)men.(hv) (::)}

\end{noinitial}

\emph{Deinde eat sacerdos cum suis ministris in cimiterium canonicorum, aspergendo et orando pro defunctis cum isto psalmo} De profúndis. Brev:⟨340⟩. \emph{et cum oratione.}

\begin{lesson}
\lettrine{D}{e}us cujus miseratióne ánime fidélium ⟨defunctórum⟩\footnote{1882:8.} requiéscunt animábus famulórum famularúmque tuárum~: hic et ubíque in Christo quiescéntium~: da propícius suórum véniam peccatórum~: ut a cunctis reátibus absolúte~: tecum sine fine leténtur. Per eúndem Christum Dóminum nostrum. \emph{⟨℟.⟩} Amen.

\end{lesson}

\emph{⟨℣.⟩} Requiéscant in pace. \emph{⟨℟.⟩} Amen.

\newrubric
\emph{¶ Iste modus et ordo processionis servetur generaliter omnibus dominicis non duplicibus per ⟨totum⟩\footnote{1882:9.} annum cum suis antiphonis vel responsoriis, cum oratione et ℣.\footnote{`\emph{et versibus, et orationibus ipsis dominicis diversimode intitulatis.}' 1882:9.}}

\emph{¶ In dominicis ⟨tamen⟩\footnote{1882:9.} a lxx. usque ad lx. dicatur ℣. post antiphonam in ipsa statione ad gradum ante crucem a duobus clericis de ij. forma ad populum conversis habitu non mutato.}

\emph{¶ Similiter a dominica ab octavis pasche usque ad proximam dominicam ante ascensionem Domini dicatur versus a ij. clericis de ij. forma ad populum conversus in superpelliceis coram cruce.\footnote{`\emph{ad populum conversis, ut supra.}' 1882:9.}}

\emph{¶ In proxima dominica ante ascensionem Domini dicatur ℣. a iij. de superiori gradu in pulpito conversis ad populum. Et si episcopus in hac dominica vel in quacunque alia officium \marginnote{7r}exequatur~: ipse indutus in cappa serica cum mitra et baculo ac omnibus supradictis ministris ad benedictionem aque chorum solet intrare~: qui dum fit benedictio salis et aque~: sacerdos ut predicitur ad hoc induto in sedem recipiat episcopalem~: ibique post aspersionem principali altaris a sacerdote factam tam canonicos quam ceteros clericos~: ad sedem episcopalem accedente~: modo et ordine prenotato aspergat~: et tam versiculum quam orationem post antiphonam} Aspérges me. \emph{et post reversionem processionis in chorum dicat ubi debet~: ipse similiter in dominicis per estatem si officium secutus sit misse, vel cum oratione dicatur in statione ante crucem. Sacerdos cum predictis ad processionem vestitus~: preces cum orationibus sequentibus easdem dicet more solito~: et post reversionem in chorum dum hore tertia et sexta cantate fuerint~: pontifex induet se pro missa. Si vero episcopus executor officii non fuerit tunc in habitu clericali cum cirotecis tamen et baculo, clericos ut supra aspergat~: sed in processione cum cappa serica cum mitra et baculo semper in fine procedat~: et predictus sacerdos ebdomadarius super omnia exequatur que ad processionem pertinent loco et habitu consueto.}

\chapter{¶ Dominica secunda adventus.}

\emph{Ad processionem ant.} Missus est ⟨ángelus⟩.\footnote{1882:9.} \emph{ut supra ⟨in prima dominica adventus Domini⟩.\footnote{1882:9.}} \pageref{003792-missus-est}.

\emph{In introitu chori dicatur istud responsorium.}

\gregorioscore{007547-rex-noster}

\emph{℣.} Vox clamántis in desérto.

\emph{℟.} Paráte viam Dómini rectas fácite sémitas Dei nostri.

\emph{⟨℣.⟩} Orémus.

\begin{lesson}
\subsubsection{Oratio.}

\lettrine{E}{x}cíta quésumus Dómine corda nostra ad preparándas Unigéniti tui vias~: ut per ejus advéntum purificátis tibi méntibus servíre mereámur. Qui vivis et regnas.\footnote{`Qui tecum.' 1882:10.}
\end{lesson}

\chapter[¶ Dominica tertia ⟨adventus⟩.]{¶ Dominica tertia ⟨adventus⟩.\footnote{1882:10.}}

\emph{Ad processionem ant.} Missus est ⟨ángelus⟩.\footnote{1882:10.} \emph{ut supra.} \pageref{003792-missus-est}.

\emph{In introitu chori\footnote{`\emph{In redeundo}', 1882:10.} dicatur istud responsorium.}

\gregorioscore{006606-ecce-radix}

\emph{℣.} Vox clámantis in desérto.

\emph{℟.} Paráte viam Dómini rectas fácite sémitas Dei nostri.

\emph{⟨℣.⟩} Orémus.

\begin{lesson}
\subsubsection{Oratio.}

\lettrine{A}{u}rem tuam quésumus Dómini précibus nostris accómmoda~: et mentis nostre ténebras grátia tue visitatiónis illústra. Qui vivis. \emph{\&c.}
\end{lesson}

\chapter[¶ Dominica quarta ⟨adventus⟩.]{¶ Dominica quarta ⟨adventus⟩.\footnote{1882:10.}}

\emph{Ad processionem ant.} Missus est. \emph{ut supra.} \pageref{003792-missus-est}.

\emph{In introitu chori\footnote{`\emph{In redeundo}', 1882:10.} responsorium.}

\label{007177-montes-israel}
\gregorioscore{007177-montes-israel}

\emph{℣.} Vox clamántis in desérto.

\emph{℟.} Paráte viam Dómini rectas fácite sémitas Dei nostri.

\emph{⟨℣.⟩} Orémus.

\begin{lesson}
\subsubsection{Oratio.}

\lettrine{E}{x}cita quésumus Dómine poténtiam tuam et veni~: et magna nobis virtúte succúrre~: ut per\footnote{`per' is missing in 1519:8r. It appears in 1530:8v. and 1555:unpaged.} auxílium grátie tue quod nostra peccáta prepédiunt\footnote{`impédiunt', 1882:10. 1519:8r. has `prepediuut.'}~: indulgéntia tue propitiatiónis accéleret. Qui vivis et regnas.
\end{lesson}

\chapter{In vigilia nativitatis Domini.}

\emph{Si dominica fuerit ad processionem ant.} Missus est. \emph{⟨et cetera⟩\footnote{1882:11.} ut supra ⟨in dominica precedente⟩.\footnote{1882:11.}} \pageref{003792-missus-est}.

\emph{In introitu chori ℟.} Montes Israel. \pageref{007177-montes-israel}.

\emph{℣.} Vox clamántis. 17.

\emph{Oratio.} Excita quésumus Dómine poténtiam tuam et veni~: et magna. 17.

\chapter{¶ In die nativitatis.}

\emph{Quacunque die\footnote{`feria', 1882:11.} contigerit~: dum hora prima ante missam canitur~: sex pueri ad ministrandum vestiti cappis sericis, in chorum deferant quibus ceteri clerici ad processionem et ad missam donec cantatur} Agnus Dei. \emph{et} Pax Dómini. \emph{per totum chorum data fuerit induantur preter sacerdotes et ministri. Quod totiens fiat~: quotiens in festis duplicibus dominicis videlicet~: vel aliis festis quando fit processio causa festivitatis.}

\emph{¶ Dicta hora sexta\footnote{`\emph{tertia hora}', 1882:11.} processio per medium chori exeat per hostium occidentale circumeundo chorum ut in omnibus aliis festis duplicibus per annum quo ingredietur ecclesiam\footnote{`\emph{quando non egredietur ecclesiam}', 1882:11.}~: et sic eat processio circa claustrum hoc ordine. Precedat minister virgam manu gestans\footnote{`\emph{Imprimis sacriste virgas in manibus gestantes}', 1882:11.} locum faciens processioni~: deinde ⟨puer cum⟩\footnote{1882:11.} aqua benedicta~: deinde tres cruces a tribus accolitis deferentibus~: albis et tunicis\footnote{`\emph{albis cum amictibus indutis}', 1882:11.} : deinde ceroferarius ij. albis cum amictibus induti tantum~: deinde duo thuribularii in simili habitu~: deinde subdiaconus~: tunc diaconus dalmatica et tunica indutus textus singulos deferat.\footnote{`\emph{textum uterque deferens}', 1882:11.} Post diaconum eat sacerdos in alba ⟨cum amictu⟩\footnote{1882:11.} et cum cappa serica~: chorus itaque sequatur in cappis sericis. In primis\footnote{`\emph{Imprimis}', 1882:11.} pueri~: deinde clerici de secunda forma~: et clerici de superiori gradu juxta predictam ordinem ⟨videlicet in prima dominica adventus Domini prenotatum⟩\footnote{1882:11.} videlicet excellentioribus ⟨personis⟩\footnote{1882:11.} subsequentibus. Quod in omnibus duplicibus festis observetur in quibus fit processio. Ita tamen quod in festis minoribus duplicibus non habentur nisi due cruces tantum. Preterea in die ascensionis Domini~: et in festo Corporis Christi precedentibus vexillis per ordinem et capsula reliquiarum qui a duobus clericis de secunda forma in cappis sericis deferantur inter subdiaconum et t⟨h⟩uribularium. ⟨ut patet in statione sequenti⟩.\footnote{1528:8r.}}

\begin{figure}
\centering
\includegraphics{images/9r-ordo-processionis-nativitas.jpg}
\caption{¶ Ordo processionis in die nativitas Domini ante missam.}
\end{figure}

\emph{¶ In eundo cantor incipiat.}

\gregorioscore{006411-descendit-de-celis}

\emph{Tres\footnote{`MS. Bodl. --- \emph{Tres clerici de superiori gradu in capis sericis dicant prosam ; ita tamen dum dicti clerici santant versum, stent gradibus fixis unsa cum choro, et dum chorus prosequitur primum versum, procedant cum toto choro, quod etiam observetur per totum annum, quando versus in processione habentur.}' ⟨1882:13.⟩} clerici de superiori gradu in medio processioni in cappis sericis simul cantent in eundo prosam sequentem in hunc modum.}
% TODO: processionis?

\gregorioscore{006411Pzf-felix-maria}

\emph{Chorus respondeat.}

\begin{noinitial}

\gregorioscore{006411a-tanquam-sponsus}

\end{noinitial}

\emph{Clerici cantent hanc prosam sequentem in hunc modum.}

\gregorioscore{006411Pb-familiam-custodi}

\emph{Chorus ⟨cantat versum ut sequitur⟩}.\footnote{1528:9v. 1519:10r. has `iui', not `tui.'}

\begin{noinitial}

\gregorioscore{006411-gloria-patri}

\end{noinitial}

\emph{Clerici aliam prosam dicatur.}

\gregorioscore{006411Pzi-te-laudant-alme-1}

\emph{Que scilicet prosa in ipsa statione ante crucem ab ipsis finiabitur~: et post unamquamque ℣. respondeat chorus cantum prose super iij. vocales,} 𝔄. 𝔒. 𝔈. \emph{quod in omnibus prosis observetur.\footnote{`\emph{respondeat cantum prose more solito,} 𝔄.' 1882:13.}}

\begin{noinitial}

\gregorioscore{006411Pzi-te-laudant-alme-2}

\emph{Chorus respondeat sic ⟨dicens⟩\footnote{1882:13.}}

\gregorioscore{006411Pzi-te-laudant-alme-3}

\end{noinitial}

\emph{In introitu chori dicatur hec ⟨sequens⟩\footnote{1882:13.} antiphona cantore incipiente.}

\gregorioscore{003093-hodie-christus}

\emph{Si hec antiphona non sufficiat ad introitum chori~: tunc repetitur in predicta antiphona} † Hódie in terra canunt ángeli.

\emph{℣.} Benedíctus qui venit ⟨in nómine Dómini⟩.\footnote{1882:14.}

\emph{℟.} Deus Dóminus illúxit nobis.

\begin{lesson}
\subsubsection{Oratio.}

\lettrine{C}{o}ncéde quésumus omnípotens Deus~: ut nos Unigéniti tui nova per carnem natívitas líberet~: quos sub peccáti jugo vetústa sérvitus tenet. Per eúndem Christum Dóminum nostrum.
\end{lesson}

\emph{Modus et ordo processionis hujus diei locum habent in omni duplici festo per totum annum quod ex sollennitate sua processionem habet excepto quod in aliis festis non dicatur prosa excepta purificatione si episcopus presens fuerit et exequatur officium in processione omnes diaconi et subdiaconi processionem in simili habitu incedant. Sciendum est quod in omnibus majoribus festis duplicibus, tres accoliti in processione ante crucem ad tres cruces deferendas tunicis induantur in quibus ad missam subsequentem ministrent. Principalis accolitus est ille videlicet in tabula dominicalis notatus vel per ebdomadam suum exequatur officium mediam crucem defert~: secundus ex altera parte chori principalis~: tertius ex ea parte iiij.\footnote{`\emph{Principalis accolitus, ille videlicet in tabula dominicali notatus per ebdomadam suum exsequitur officium, mediam crucem defert ; secundus ex altera parte chori principalis ; tertius ex ea parte qua. primus crucem bajulat ex altera parte chori.}' ⟨1882:xv.⟩ This is Henderson's proposed suggestion.} primus crucem bajulat ex altera parte chori.}

\emph{¶ In die nativitatis Domini post vesperas finito primo} Benedicámus. \emph{a duobus de ij. forma in superpelliceis conveniant omnes diaconi in cappis sericis portantes cereos ardentes in manibus~: et sic eat processio per medium chori ad altare sancti Stephani cantando hoc ℟. cantore\footnote{`\emph{diacono}', 1882:14.} incipiente hoc modo.}

1519:11r; AS:60; Ant-1519:61v; Brev-1531:31v.\footnote{1519:11r. has no flats. In 1519:11r. `Dóminum' is set DFFG.DDCD.D; inimíco is set FGA.GF.GA.CAA; `funde' is set ABC.GA; `nos purgátos' is set GA G.B♭.AB♭C.AGF.}

\gregorioscore{}

\emph{Tres diaconi dicant simul hunc ℣.}

\gregorioscore{}

\emph{¶ Omnes\footnote{1508---tres. ⟨1882:14.⟩} diaconi dicant simul hanc prosam.}

1519:11v; AS:60; Ant-1519:62r; Brev-1531:31v.\footnote{In 1519:11v. a flat appears only in first phrase. 1519:12r. has `sequens' for sequéris'. In 1519:12r. there appears to be a missing C-clef at'crímina', making the music a third higher. In 1519:12r. `vénia' is set G.E.C.}

\gregorioscore{}

\emph{Chorus vel organa respondeat cantum prose super litteram} 𝔄. \emph{post unumquemque versum.}

\gregorioscore{}

\emph{Ad hanc processionem ⟨non⟩\footnote{Most editions omit `\emph{non}', but compare with the \emph{Noted breviary}, p.~334. The 1555 \emph{Processional} (unpaged) includes `\emph{non}.' `\emph{non dicitur}', 1882:15.} dicatur} Glória Patri. \emph{sed dum prosa canitur~: thurificet sacerdos altare~: deinde imaginem sancti Stephani~: et postea dicat modesta voce ⟨versiculum⟩.\footnote{1882:15.}}

1519:12r.\footnote{1519:12r. omits the response.}

\gregorioscore{}

\begin{lesson}
\subsubsection{Oratio.}

\lettrine{D}{a} nobis quésumus Dómine imitári quod cólimus ut discámus et inimícos dilígere~: quia ejus natalícia celebrámus~: qui novit étiam pro persecutóribus exoráre Dóminum nostrum Jesum Christum Fílium. \emph{⟨et cetera.⟩\footnote{1882:15.} Chorus.} Amen.
\end{lesson}

\emph{In redeundo dicatur aliqua antiphona de sancta Maria, vel responsorium.}

\gregorioscore{007709-stirps-jesse}

\emph{Non dicitur} Glória Patri. \emph{Sed sacerdos ad gradum chori dicat}

\emph{Versiculum.} Speciósus forma pre fíliis hóminum.

\emph{℟.} Diffúsa est grátia in lábiis tuis. \emph{Non dicatur ulterius.}

\emph{⟨℣.⟩} Orémus.

\begin{lesson}
\subsubsection{Oratio.}

\lettrine{D}{e}us qui salútis etérne beáte Maríe virginitáte fecúnda humáno géneri prémia prestitísti~: tríbue quésumus ut ipsam pro nobis intercédere sentiámus per quam merúimus auctórem vite suscípere, Dóminum nostrum Jesum Christum Fílium tuum. Qui tecum vivit.
\end{lesson}

\chapter{¶ In die sancti Stephani.}

\emph{Si dominica fuerit eodem modo fiat processio que in ceteris diebus dominicis preter habitum et excepto quod hac die tres diaconi prosam in eundo cantent in medio procedentes que in ipsa statione ante crucem ab eisdem terminetur~: cetera ut supra.}

\emph{℟.} Sancte Dei Precióse. 24. \emph{℣.} Ut tuo. \emph{et percantetur a toto choro~: tres diaconi dicant prosam} Te mundi. 25. \emph{Chorus} 𝔄. \emph{et sic deinceps a toto choro.} Glória Patri. \emph{et dicatur hoc modo.}

1519:13r.\footnote{1519:13r. has no flat.}

\gregorioscore{}

\emph{In redeundo ant.} Hódie Christus natus. 23.

\emph{℣.} Benedíctus qui venit. 23.

\emph{Oratio.} Concéde quésumus omnípotens Deus. 23.

\emph{Simili modo de sancto Johanne et de Innocentibus et de sancto Thoma si in dominica festum illorum evenerit cum ix. ℟. et ℣. et prosa, et} Glória Patri. \emph{et semper in redeundo usque ad circumcisionem dicatur antiphona} Hódie Christus. \emph{cum ℣. et oratione ut supra.} 23.

\emph{¶ In die sancti Stephani ad ij. vesperas post memoriam de nativitate conveniant omnes sacerdotes in cappis sericis cum cereis ardentibus in manibus~: et sic eat processio ad altare apostolorum per medium chori cantando responsorium.}

\gregorioscore{006913-in-medio-ecclesie}

\emph{⟨Tres diaconi dicant versum,} Misit. \emph{\&c.⟩\footnote{MS. Harl. 2945---Tres diaconi dicant versum, Misit, \&c.~⟨1882:16.⟩}}

\gregorioscore{006913a-misit-dominus}

\emph{Omnes\footnote{1508---Tres. MSS. and other Edd.---Omnes. ⟨1882:16.⟩} sacerdotes simul dicant prosam.}

\gregorioscore{006913Pzd-nascitur-ex-patre}

\emph{Ad hanc processionem non dicatur} Glória Patri. \emph{sed dum prosa canitur thurificet sacerdos ad altare deinde imaginem sancti Johannis~: et postea dicat modesta voce versiculum.} Valde honorándus est beátus Johánnes.

\emph{℟.} Qui supra pectus Dómini in cena recúbuit.

\begin{lesson}
\subsubsection{Oratio.}

\lettrine{E}{c}clésiam tuam quésumus Dómine benígnus illústra~: ut beáti Johánnis apóstoli tui et evangelíste illumináta doctrinis~: ad dona pervéniat sempitérna. Per Christum Dóminum nostrum. \emph{⟨℟.⟩} Amen.
\end{lesson}

\emph{In redeundo dicitur aliqua antiphona de sancta Maria vel ℟.} Solem justície. \emph{quere in nativitate ejusdem} 369. \emph{: versus ⟨et⟩\footnote{1882:17.} oratio ut supra.} 23.

\chapter{¶ In die sancti Johannis apostoli.}

\emph{Si dominica fuerit ad processionem eodem modo fiat ut in die sancti Stephani~: excepto quod hac ⟨die⟩\footnote{1882:17.} iij. sacerdotes in eundo dicant prosam in medio chori~: que in ipsa statione ante crucem terminetur.}

\emph{In eundo ℟.} In médio. 29. \emph{℣.} Misit Dóminus. \emph{⟨Tres sacerdotes dicant prosam} Náscitur. \emph{ut supra et dicitur cum versu.⟩\footnote{1517. ⟨1882:17.⟩}}

\emph{In introitu chori antiphona.} Hódie Christus. 23.

\emph{Versus.} Benedíctus qui venit. 23.

\emph{Oratio ut supra.} 23.

1519:14r.\footnote{1519:14r. has no flat. In 1519:14r. this ℣. appears \emph{after the} rubric `\emph{Oratio ut supra,}'}

\begin{noinitial}

\gregorioscore{006913Pzd-gloria-patri}

\end{noinitial}

\emph{¶ In die sancti Johannes ad vesperas post memoriam de sancto Stephano eat processio puerorum ad altare sancte Trinitatis~: et omnium sanctorum quod dicatur} Salve.\footnote{1517 omits `\emph{quod dictur} Salve.' ⟨1882:17.⟩} \emph{in cappis cereis ardentibus in manibus cantando~: episcopo puerorum incipiente hoc modo.}

\gregorioscore{006273-centum-quadraginta}

\emph{Tres pueri dicant hunc versum sequentem.}

\begin{noinitial}

\gregorioscore{006273b-hi-empti-sunt}

\end{noinitial}

\emph{Omnes pueri dicant hanc prosam. Chorus post unumquemque versum respondeat cantum prose super ultimum litteram.}

\gregorioscore{ah10068-sedentem-in-superne}

\emph{Ad hanc prosam non dicatur} Glória Patri. \emph{sed dum prosa canitur tunc episcopus puerorum thurificet altare deinde imaginem Sancte Trinitatis~: et postea dicat modesta voce ℣.} Letámini in Dómino et exultáte justi.

\emph{℟.} Et gloriámini omnes recti corde.

\emph{⟨℣.⟩} Orémus.

\begin{lesson}
\subsubsection[⟨Oratio.⟩]{⟨Oratio.⟩\footnote{1882:18.}}

\lettrine{D}{e}us cujus hodiérna die precónium innocéntes mártyres non loquéndo sed moriéndo conféssi sunt, ómnia in nobis vitiórum mala mortífica~: ut fidem tuum quam lingua nostra lóquitur~: étiam móribus vita fateátur. Qui cum Deo Patre et Spíritu Sancto vivis et regnas Deus. Per ómnia sécula seculórum. \emph{⟨℟.⟩} Amen.
\end{lesson}

\emph{In revertendo\footnote{`\emph{redeundo}', 1882:18.} precentor\footnote{`\emph{preceptor}', 1519:15v; 1523:15v. `The editions vary between \emph{precentor puerorum} and \emph{preceptor}, as the \emph{preceptor} or \emph{terminator processionis.} The two words are elsewhere interchanged in the same kind of reference. See Ducange under \emph{Preceptor}.' ⟨1882:xv.⟩} puerorum incipiat de sancta Maria ℟.} Felix namque. \emph{Require in ⟨die⟩\footnote{1882:18.} assumptionis beata Marie} 364. \emph{: et si necesse fuerit dicatur ℣.} Ora pro pópulo. \emph{et loco} Assumptiónem. \emph{dicatur} Commemorationem.\footnote{1882:18. has `Sollennitátem.' with the following note: 1528, \&c.---Commemorationem.} \emph{et sic processio chorum intret per ostium occidentale ut supra~: et omnes pueri ex utraque parte chori in superiori gradu se recipiant~: et ab hac hora usque post processionem diei proximi succedentis nullus clericorum solet gradum superiorem ascendere cujuscunque conditionis fuerit.}

\emph{¶ Ad istam processionem pro dispositione puerorum scribuntur canonici ad ministrandum eisdem majores ad thuribulandum et ad librum deferendum minores ad candelabra deferenda.}

\emph{Episcopus in sede sua dicat versiculum.} Speciósus forma pre fíliis hóminum. \emph{℟.} Diffúsa est. 27. \emph{Oratio.} Deus qui salútis etérne. \emph{que terminetur sic} Qui tecum vivit et regnat. \emph{⟨et cetera.⟩\footnote{1882:18.}} 28. Pax vobis. Et cum spiritu tuo. \emph{Sequatur} Benedicámus Dómino. \emph{a duobus vicariis vel a tribus extra regulam.}

\emph{Post hec episcopus puerorum in sede sua benedicat populum in hunc modum. Cruciferarius accipiat baculum episcopi in manu conversus ad episcopum~: et incipiat hanc antiphonam sequentem in hunc modum.}

\gregorioscore{000000-princeps-ecclesie-1}

\emph{Hic convertat se ad populum sic dicendo.}

\begin{noinitial}

\gregorioscore{000000-princeps-ecclesie-2}

\emph{Chorus respondent sic.}

\gregorioscore{000000-princeps-ecclesie-3}

\end{noinitial}

\emph{Deinde tradat baculum episcopo et tunc episcopus puerorum primo signando se in fronte sic dicendo.}

\gregorioscore{}

\emph{Chorus respondeat sic.}

\gregorioscore{}

\emph{Item episcopus signando se in pectore dicat sic.}

\gregorioscore{}

\emph{Deinde episcopus puerorum conversus ad clerum~: elevet brachium suum~: et dicat hanc benedictionem hoc modo.}

\gregorioscore{}

\emph{Hic convertat se ad populum dicendo sic.\footnote{`Nostra', 1882:19.}}

\gregorioscore{}

\emph{Deinde convertat se ad altare sic dicendo.\footnote{`Nos', 1882:19. and 1519. `vos' appears in the Breviary-1531.}}

\gregorioscore{}

\emph{Postea ad seipsum reversus ponat manum super pectus suum dicendo.}

\gregorioscore{}

\emph{Chorus respondeat sic.}

\gregorioscore{}

\emph{His itaque peractis incipiat episcopus puerorum completorium more solito et post completorium dicat episcopus puerorum ad chorum conversus sub tono supradicto modo\footnote{cf.~Breviarium, 408.}} Adjutórium nostrum. \emph{et} Sit nomen Dómini. 35. \emph{Deinde dicat episcopus.} Benedícat vos omnípotens Deus Pater et Fílius et

1519:16v.\footnote{1519:16v. has no flat.}

\gregorioscore{}

\emph{⟨Chorus respondeat sic.⟩\footnote{1519:78r.}}

\gregorioscore{}⟨\gregorioscore{}⟩\footnote{1519:77v; 1531:37v.}

\chapter{¶ In die sanctorum innocentium.}

\emph{Si dominica fuerit eodem modo processio ⟨fiat⟩\footnote{1882:19.} ut in die sancti Stephani excepta quod hac die iij. pueri precedent in eundo dicant in medio processione\footnote{`\emph{prosam in eundo dicant in medio procedentes}', 1882:19.} que in ipsa statione ante crucem ab eisdem terminetur.}

\emph{In eundo ℟.} Centum quadragínta. 31. \emph{℣.} Hi empti. \emph{Prosa.} Sedéntem in supérne. \emph{Sequatur}

\begin{noinitial}

\gregorioscore{006273-gloria-patri}

\end{noinitial}

\emph{In introitu chori de nativitate ut supra.} 23.

\emph{¶ In die sanctorum innocentium ad vesperas post memoriam de sancto Johanne accipiat cruciferarius baculum episcopi puerorum et cantet antiphonam} Princeps ecclésie. \emph{sicut ad j. vesperas.} 35. \emph{Similiter episcopus puerorum benedicat populum supradicto modo, et sic compleatur servitium hujus diei.}

\emph{Deinde\footnote{This and the Procession for S. Thomes' day are omitted in 1544 and later edd. ⟨1882:20.⟩} eat processio ad altare sancti Thome martyris habitu non mutato absque cereis in manibus cantando ℟. cantore incipiente hoc modo.}

\gregorioscore{601260-jacet-granum}

\emph{Deinde dicatur prosa in superpelliciis ab omnibus qui voluerint si placet et chorus respondent cantum prose super literam} 𝔄.

\gregorioscore{601260Pa-clangat-pastor}

\emph{¶ In hac processione non dicatur} Glória Patri. \emph{sed dum canitur prosa thurificet sacerdos altare, deinde imaginem sancti Thome et postea dicat ℣.} Ora pro nobis beáte Thome.

\emph{℟.} Ut ⟨digni efficiámur promissiónibus Christi⟩.\footnote{1882:20.}

\begin{lesson}
\subsubsection[Oratio.]{Oratio.\footnote{In 1519:17v. `gloriósus póntifex Thomas' is crossed out.}}

\lettrine{D}{e}us pro cujus ecclésia gloriósus póntifex Thomas gládiis impiórum occúbuit~: presta quésumus~: ut omnes qui ejus implórant auxílium petitiónis sue salutárem consequántur efféctum. Per Christum.
\end{lesson}

\emph{In revevertendo dicatur antiphona vel ℟. de sancta Maria.}

\chapter[¶ In die sancti Thome martyris.]{¶ In die sancti Thome martyris.\footnote{In 1519:17v. `In die sancti Thomas' is crossed out.}}

\emph{Si dominica fuerit, ad processionem ante missam fiat eodem modo et ordine quo in die sancti Stephani~: excepto quod in hac die iij. clerici de superiori gradu dicant prosam in medio processionis que in ipsa statione ante crucem finietur.}

\emph{In eundo dicatur ℟.} Jacet granum. 38. \emph{℣.} Cadit custos. \emph{Prosa.} Clangat pastor. \emph{et dicatur cum} Glória Patri.

1519:17v.\footnote{In 1519:17v. `tu' is set B♭. In 1555 (unpaged), `tu' is set B♭C.}

\begin{noinitial}

\gregorioscore{601260-gloria-patri}

\end{noinitial}

\emph{In introitu chori de nativitate ut supra.} 23.

\chapter{¶ Sexta die a nativitate Domini.}

\emph{Si dominica fuerit ad processionem ℟.} Descéndit. \emph{℣.} Tanquam sponsus. 19. \emph{et dicitur sic.}

1519:18r.

\gregorioscore{}

\emph{\&c.~sine prosa cum} Glória Patri.\footnote{`\emph{vel ℟}. Verbum caro. \emph{℣}. In princípio. \emph{sine prosa}.' GS:18.} 21.

\emph{¶ In redeundo de nativitate ut supra.\footnote{`\emph{In redeundo} Te laudant. \emph{vel} Hódie Christus. \emph{Sequatur ℣. et or. de nat.}' GS:18.}} 23.

\chapter{¶ In die sancti Silvestri.}

\emph{Si dominica fuerit Resp.} Miles Christi.\footnote{`Christi miles', in the Processionals; however, surely `Miles Christi' is intended. `Christi miles' is proper to St.~Vincent.} 397.

\emph{In introitu chori de nativitate ut supra.} 23.

\chapter[¶ In die circuncisionis ⟨Domini⟩.]{¶ In die circuncisionis ⟨Domini⟩.\footnote{1882:21.}}

\emph{Si dominica fuerit modus et ordo processionis fiat hac die ut in die sancti Thome martyris. Dicatur responsorium sequens.}

\gregorioscore{007840-verbum-caro}

\emph{Tres clerici de superiori gradu in cappis sericis in medio processionis dicant ⟨hanc⟩\footnote{1882:22.} prosam ⟨sequentem⟩.\footnote{1882:22.}}

\gregorioscore{007840Pzc-quem-ethera}

\begin{noinitial}

\gregorioscore{007840-gloria-patri}

\end{noinitial}

\emph{In introitu chori dicatur ℟. ⟨sequens⟩\footnote{1882:22.} de sancta Maria.}

1519:18v; AS:49; Ant-1519:54v; Brev-1531:29r.\footnote{In 1519:18v. `ángeli' is set F.EEGFEFED.ED; `Dei' is set GAE.FGAGFABA. 1519:18v. has a flat only at `cognovísti'; `cognovísti is set A.AG.AB♭AGA.AGGEFE; `aurem' is set AB♭AGAAGG.EFE; `Dóminum nostrum' is set EGAG.FEDC.DEDC DEF.EDFFE; `genuísti' is set DG.A.AB♭.A.}

\gregorioscore{}

\emph{¶ Cum versu vel sine versu pro dispositione cantoris.}

\gregorioscore{}

\emph{℣.} Post partum ⟨virgo invioláta permansísti.

\emph{℟.} Dei génitrix intercéde pro nobis⟩.

\emph{Oratio.} Deus qui salútis. \emph{ut supra in processione sancti Stephani.} 28.

\chapter[¶ In octavis sancti Stephani et Johannis.]{¶ In octavis sancti Stephani et Johannis et sanctorum innocentium.}

\emph{Si dominica fuerit fiat processio per omnia sicut in die preter habitum et sine prosis.}

\emph{In introitu chori a circumcisione usque ad purificationem et si lxx. infra purificationem evenerit semper de sancta Maria dicatur ⟨hoc⟩\footnote{1882:22.} ℟.} Te laudant ángeli. \emph{ut supra in ⟨die⟩\footnote{1882:22.} circumcisionis} 43. \emph{excipiuntur cum\footnote{`\emph{tamen}', 1882:22.} dies epiphanie et dominica infra octavas ejusdem et in octava si ⟨in⟩\footnote{1882:22.} dominica evenerit.}

\chapter{¶ In vigilia epiphanie.}

\emph{Si ⟨in⟩\footnote{1882:22.} dominica evenerit ad processionem in eundo dicatur ℟.} Verbum caro. 41. \emph{cum ℣. et cum} Glória Patri. \emph{sed sine prosa.}

\emph{In introitu chori de sancta Maria ut supra.}

\chapter{¶ In die epyphanie.}

\emph{Quacunque feria contigerit modus et ordo fiat processionis per omnia sicut in die natvitatis Domini preter prosam.} 18.

\emph{In eundo dicatur.}

1519:19v; AS:86; Ant-1519:106v; Brev-1531:52v.\footnote{In 1519:19v. `Tria sunt' is set D.F GA; `consídera' is set A.GA.FED.DFDDC. 1519:19v. has no flats.}

\gregorioscore{}

\emph{⟨Sequitur⟩\footnote{1882:23.} aliud ℟.}

\gregorioscore{007523-reges-tharsis}

\emph{⟨℣.⟩} Glória Patri. 420. ‡Dómino Deo.

\emph{In introitu chori dicatur ℟. sequentem.}

\gregorioscore{006892-in-columbe-specie}

\emph{℣.} Vox Dómini super aquas.\footnote{1519:20v; 1523:20v; 1528:19v; 1544:23v; and 1554:21r. have `aquas multas.'}

\emph{℟.} Dóminus ⟨super aquas multas⟩.\footnote{1882:23.}

\emph{⟨℣.⟩} Orémus.

\begin{lesson}
\subsubsection{Oratio.}

\lettrine{D}{e}us qui hodiérna die Unigénitum tuum géntibus stella duce revelásti concéde propítius~: ut qui te jam\footnote{`jam te', 1882:23.} ex fide cognóvimus~: usque ad contemplándam spéciem tue celsitúdinis perducámur. Per eúndem Christum ⟨Dóminum nostrum. Amen⟩.\footnote{1882:23.}
\end{lesson}

\chapter{¶ Dominica infra octavas epyphanie et in octava.}

\emph{Si dominica fuerit ad processionem ℟.} Tria sunt ⟨múnera⟩.\footnote{1882:23.} 45.

\emph{In introitu chori ℟.} In colúmbe spécie. 47.

\emph{℣.} Vox Dómini. 48.

\emph{Oratio.} Deus qui hodiérna. 48.

\chapter[¶ Dominica j. post octavas epiphanie.]{¶ Dominica j. post octavas epiphanie et in omnibus dominicis abhinc usque ad lxx. quando de dominica agitur.}

\emph{Modus et ordo processionis fiant ut in dominica prima adventus Domini ⟨nostri Jesu Christi⟩.\footnote{1882:23.}} 9.

\emph{Ad processionem ⟨cantetur⟩\footnote{1882:24.} ℟. hoc modo ⟨quo sequitur⟩.\footnote{1882:24.}}

\gregorioscore{006011-abscondi-tanquam-aurum}

\emph{℣.} Glória. 421. † Miserére.

\emph{In introitu chori usque ad purificationem dicatur de sancta Maria ℟.} Te laudant. \emph{ut supra.} 43. \emph{Post purificationem vero usque ad septuagesimam dicatur in introitu chori antiphona} Ave regína ⟨celórum⟩.\footnote{1882:24. `\emph{vel} Alma redemptóris', GS:20.} 292. \emph{vel alia antiphona de sancta Maria cum hoc versu} Post partum ⟨virgo invioláta permansísti⟩.\footnote{1882:24.} 44.

Orémus.

\begin{lesson}
\subsubsection{Oratio.}

\lettrine{C}{o}ncéde quésumus miséricors Deus fragilitáti nostre presídium~: ut qui sancte Dei genitrícis et vírginis Maríe commemoratiónem ágimus, intercessiónis ejus auxílio, a nostris iniquitátibus resurgámus. Per eúndem Christum Dóminum \emph{\&c.} ⟨nostrum. \emph{⟨℟.⟩} Amen⟩.\footnote{1882:24. 1519:21v. has simply `\emph{\&c.}' `\emph{vel} Concéde nos fa. \emph{vel alia or. de sancta Maria.}' GS:20.}
\end{lesson}

\chapter{¶ Dominica septuagesime.}

\emph{Ad processionem ⟨dicatur antiphona que sequitur⟩.\footnote{1882:24.}}

\gregorioscore{002497-ecce-charissimi}

\emph{Duo clerici de secunda forma in statione ante crucem ad populum conversi habitu non mutato cantant ⟨versum⟩.\footnote{1882:24.}}

\begin{noinitial}

\gregorioscore{002497a-ecce-mater}

\end{noinitial}

\emph{In introitu chori ante purificationem dicatur ⟨responsorium⟩\footnote{1882:24.}} Te laudant. 22. \emph{Si post dicatur ⟨hoc responsorium sequens⟩.\footnote{1882:24.}}

\gregorioscore{007804-ubi-est-abel}

\emph{Versus.} Dómine refúgium factus es nobis.

\emph{Resp.} A generatióne et progénie.

\emph{⟨℣.⟩} Orémus.

\begin{lesson}
\subsubsection{Oratio.}

\lettrine{P}{r}eces pópuli tui quésumus Dómine cleménter exáudi ut qui juste pro peccátis nostris afflígimur~: pro nóminis tui glória misericórditer liberémur. Per Christum.
\end{lesson}

\chapter{¶ Dominica in lx.}

\emph{Ad processionem ant.} Ecce charíssimi. \emph{ut supra.} 50.

\emph{In introitu chori de sancta Maria~: si lx. ante purificationem evenerit ⟨ut supra⟩\footnote{Compare Dominica in l. below.}} 50. \emph{: si vero post purificationem evenerit.}

\gregorioscore{600283-benedicens}

\emph{℣.} Dómine refúgium factus es nobis.

\emph{℟.} A generatióne et progénie.

\emph{⟨℣.⟩} Orémus.

\begin{lesson}
\subsubsection{Oratio.}

\lettrine{D}{e}us qui cónspicis quia ex nulla nostra actióne confídimus~: concéde propítius~: ut contra ómnia advérsa doctóris géntium protectióne muniámur. Per Christum Dóminum nostrum. \emph{⟨℟.⟩} Amen.
\end{lesson}

\chapter{¶ Dominica in quinquagesima.}

\emph{Ad processionem ant.} Ecce caríssimi. 50.

\emph{In introitu chori de sancta Maria~: si l. ante purificationem evenerit ut supra, si vero post purificationem evenerit dicatur hoc.}

\gregorioscore{007762-tentavit-deus}

\emph{Versus.} Dómine refúgium factus es nobis.

\emph{Resp.} A generatióne et progénie.

\emph{⟨℣.⟩} Orémus.

\begin{lesson}
\subsubsection{Oratio.}

\lettrine{P}{r}eces nostras quésumus Dómine cleménter exáudi~: atque a peccatórum nostrórum vínculis absolútos ab omni nos adversitáte custódi. Per Christum Dóminum nostrum. \emph{⟨℟.⟩} Amen.
\end{lesson}

\chapter{¶ Feria quarta in capite jejunii.}

\emph{Post sextam in primis\footnote{`\emph{imprimis}', 1882:26.} fiat sermo ad populum si placuerit~: deinde prosternant se clerici in choro, et dicant septem psalmos penitentiales cum} Glória Patri. \emph{et} Sicut erat. \emph{et antiphona} Ne reminiscáris.

\emph{Excellentior vero procedat sacerdos\footnote{`\emph{sacerdos procedat}', 1882:26.} indutus cappa serica rubea, cum aliis vestibus sacerdotalibus, cum diacono a dextris et subdiacono a sinistris et ceteris ministris altaris qui omnes sint albis cum amictibus induti a vestibulo ad gradum altaris procedant~: et ibi dicant septem psalmos penitentiales in prostratione videlicet.}

\subsection[¶ Sequuntur septem psalmi penitentiales.]{¶ Sequuntur septem psalmi penitentiales.\footnote{`Dómine ne in furóre. \emph{et cetera. Require predictos psalmos in fine libri.}' 1882:26.}}

\emph{Antiphona.} Ne reminiscáris. \emph{Psalmus.\footnote{US-II:56. The Penitential Psalms were recited beginning on Ash Wednesday. The full rite appears in TUS:III-17. and in SP:22v.}}

⟨1523:24v.\footnote{1519:24v. has one of the standard images of David and Bathsheba:⟩}

\begin{lesson}
\subsubsection{⟨Psalmus. vj.⟩}

\lettrine{D}{o}mine ne in furóre tuo árguas me~: neque in ira tua corrípias me.

Miserére mei Dómine quóniam infírmus sum~: sana me Dómine quóniam conturbáta sunt ossa mea.

Et ánima mea turbáta est valde~: et tu Dómine úsquequo~?

Convértere Dómine et eripe ánimam meam~: salvum me fac propter misericórdiam tuam.

Quóniam non est in morte qui memor sit tui~: in inférno autem quis confitébitur tibi~?

Laborávi in gémitu meo~: lavábo per síngulas noctes lectum meum~: láchrymis meis stratum meum rigábo.

Turbátus est a furóre óculus meus~: inveterávi inter omnes inimícos meos.

Discédite a me omnes qui operámini iniquitátem~: quóniam exaudívit Dóminus vocem fletus mei.

Exaudívit Dóminus deprecatiónem meam~: Dóminus oratiónem meam suscépit.

Erubéscant et conturbéntur veheménter omnes inimíci mei~: convertántur et erubéscant valde velóciter.

Glória Patri.

Sicut erat.

\subsubsection{Psalmus. ⟨xxxj.⟩}

\lettrine{B}{e}áti quorum remísse sunt iniquitátes~: et quorum tecta sunt peccáta.

Beatus vir cui non imputávit Dóminus peccátum~: nec est in spíritu ejus dolus.

Quóniam tácui inveteravérunt ossa mea~: dum clamárem tota die.

Quóniam die ac nocte graváta est super me manus tua~: convérsus sum in erúmna mea dum confígitur spina.

Delíctum meum cógnitum tibi feci~: et injustítiam meam non abscóndi.

Dixi Confitébor advérsum me injustítiam meam Dómino~: et tu remisísti impietátem peccáti mei.

Pro hac orábit ad te ómnis sanctus~: in témpore opportúno.

Verúntamen in dilúvio aquárum multárum~: ad eum non approximábunt.

Tu es refúgium meum a tribulatióne que circundédit me~: exultátio mea erue me a circundántibus me.

Intelléctum tibi dabo et ínstruam te in via hac qua gradiéris~: firmábo super te óculos meos.

Nolíte fíeri sicut equus et mulus~: quibus non est intelléctus.

In chamo et freno maxíllas eórum constrínge~: qui non appróximant ad te.

Multa flagélla peccatóris~: sperántem autem in Dómino misericórdia circundábit.

Letámini in Dómino et exultáte justi~: et gloriámini omnes recti corde.

Glória.

\subsubsection{Psalmus. ⟨xxxvij.⟩}

\lettrine{D}{o}mine ne in furóre tuo árguas me~: neque in ira tua corrípias me.

Quóniam sagítte tue infíxe sunt michi~: et confirmásti super me manum tuam.

Non est sánitas in carne mea\footnote{In 1519:25r. the punctuation differs from the Breviary. For consistency, this edition follows the Breviary. The following are the variants in 1519. 

  3. Non est sánitas in carne mea~: * a fácie ire tue non est pax óssibus meis a fácie peccatórum meórum.

  10. Cor meum conturbátum est derelíquit me virtus mea~: * et lumen oculórum meórum, et ipsum non est mecum.} a fácie ire tue~: non est pax óssibus meis a fácie peccatórum meórum.

Quóniam iniquitátes mee supergrésse sunt caput meum~: et sicut onus grave graváte sunt super me.

Putruérunt et corrúpte sunt cicatríces mee~: a facie insipiéntie mee.

Miser factus sum et curvátus sum usque in finem~: tota die contristátus ingrediébar.

Quóniam lumbi mei impléti sunt illusiónibus~: et non est sánitas in carno mea.

Afflíctus sum et humiliátus sum nimis~: rugiébam a gémitu cordis mei.

Dómine ante te omne desidérium meum~: et gémitus meus a te non est abscónditus.

Cor meum conturbátum est~: derelíquit me virtus mea et lumen oculórum meórum, et ipsum non est mecum.

Amíci mei et próximi mei~: advérsum me appropinquavérunt et stetérunt.

Et qui juxta me erant de longe stetérunt~: et vim faciébant qui querébant ánimam meam.

Et qui inquirébant mala michi locúti sunt vanitátes~: et dolos tota die meditabántur.

Ego autem tanquam surdus non audiébam~: et sicut mutus non apériens os suum.

Et factus sum sicut homo non áudiens~: et non habens in ore suo redargutiónes.

Quóniam in te Dómine sperávi~: tu exáudies me\footnote{``me'' is missing in 1531-P:47v, but is present in 1531-P:16v.} Dómine Deus meus.

Quia dixi Nequándo supergáudeant michi inimíci mei~: et dum commovéntur pedes mei super me magna locúti sunt.

Quóniam ego in flagélla parátus sum~: et dolor meus in conspéctu meo semper.

Quóniam iniquitátem meam annunciábo~: et cogitábo pro peccáto meo.

Inimíci autem mei vivunt, et confirmáti sunt super me~: et multiplicáti sunt qui odérunt me iníque.

Qui retríbuunt mala pro bonis detrahébant michi~: quóniam sequébar bonitátem.

Ne derelínquas me Dómine~: Deus meus ne discésseris a me.

Inténde in adjutórium meum Dómine~: Deus salútis mee.

Glória Patri et Fílio~: et Spirítui Sancto.

Sicut erat in princípio.

\subsubsection{Psalmus. ⟨l.⟩}

\lettrine{M}{i}serére mei Deus~: secúndum magnam misericórdiam tuam.

Et secúndum multitúdinem miseratiónum tuárum~: dele iniquitátem meam.

Amplius lava me ab iniquitáte mea~: et a peccáto meo munda me.

Quóniam iniquitátem meam ego cognósco~: et peccátum meum contra me est semper.

Tibi soli peccávi et malum coram te feci~: ut justificéris in sermónibus tuis, et vincas cum judicáris.

Ecce enim in iniquitátibus concéptus sum~: et in peccátis concépit me mater mea.

Ecce enim veritátem dilexísti~: incérta et occúlta sapiéntie tue manifestásti michi.

Aspérges me ysópo et mundábor~: lavábis me et super nivem dealbábor.

Audítui meo dabis gáudium et letíciam~: et exultábunt ossa humiliáta.

Avérte fáciem tuam a peccátis meis~: et omnes iniquitátes meas dele.

Cor mundum crea in me Deus~: et spíritum rectum ínnova in viscéribus meis.

Ne projícias me a fácie tua~: et spíritum sanctum tuum ne áuferas a me.

Redde michi letíciam salutáris tui~: et spíritu principáli confírma me.

Docébo iníquos vias tuas~: et ímpii ad te converténtur.

Líbera me de sanguínibus Deus Deus salútis mee~: et exultábit lingua mea justíciam tuam.

Dómine lábia mea apéries~: et os meum annunciábit laudem tuam.

Quóniam si voluísses sacrifícium dedíssem~: útique holocáustis non delectáberis.

Sacrifícium Deo spíritus contribulátus~: cor contrítum et humiliátum Deus non despícies.

Benígne fac Dómine in bona voluntáte tua Syon~: et edificéntur muri Hierúsalem.

Tunc acceptábis sacrifícium justície\footnote{In 1519:25v. the punctuation differs from the Breviary. For consistency, this edition follows the Breviary. The following is a variant in 1519.

  20. Tunc acceptábis sacrifícium justície~: * oblatiónes et holocáusta tunc impónent super altáre tuum vítulos.} oblatiónes et holocáusta~: tunc impónent super altáre tuum vítulos.

Glória.

Sicut.

\subsubsection{Psalmus. ⟨cj.⟩}

\lettrine{D}{o}mine exáudi oratiónem meam~: et clamor meus ad te véniat.

Non avértas fáciem tuam a me~: in quacúnque die tríbulor inclína ad me aurem tuam.

In quacúnque die invocávero te~: velóciter exáudi me.

Quia defecérunt sicut fumus dies mei~: et ossa mea sicut crémium aruérunt.

Percússus sum ut fenum et áruit cor meum~: quia oblítus sum comédere panem meum.

A voce gémitus mei~: adhésit os meum carni mee.

Símilis factus sum pellicáno solitúdinis~: factus sum sicut nictícorax in domicílio.

Vigilávi~: et factus sum\footnote{In 1519:25r. the punctuation differs from the Breviary. For consistency, this edition follows the Breviary. The following are the variants in 1519.

  8. Vigilávi et factus sum~: * sicut passer solitárius in tecto.} sicut passer solitárius in tecto.

Tota die exprobrábant michi inimíci mei~: et qui laudábant me advérsum me jurábant.

Quia cínerem tanquam panem manducábam~: et potum meum cum fletu miscébam.

A fácie ire indignatiónis tue~: quia élevans allisísti me.

Dies mei sicut umbra declinavérunt~: et ego sicut fenum árui.

Tu autem Dómine in etérnum pérmanes~: et memoriále tuum in generatióne et generatiónem.

Tu exúrgens miseréberis Syon~: quia tempus miseréndi ejus quia venit tempus.

Quóniam placuérunt servis tuis lápides ejus~: et terra ejus miserebúntur.

Et timébunt gentes nomen tuum Dómine~: et omnes reges terre glóriam tuam.

Quia edificávit Dóminus Syon~: et vidébitur in glória sua.

Respéxit in oratiónem humílium~: et non sprevit precem eórum.

Scribántur hec in generatióne áltera~: et pópulus qui creábitur laudábit Dóminum.

Quia prospéxit de excélso sancto suo~: Dóminus de celo in terram aspéxit.

Ut audíret gémitus compeditórum~: ut sólveret fílios interemptórum.

Ut annúntient in Syon nomen Dómini~: et laudem ejus in Hierúsalem.

In conveniéndo pópulos in unum~: et reges ut sérviant Dómino.

Respóndit ei in via virtútis sue~: Paucitátem diérum meórum núncia michi.

Ne révoces me in dimídio diérum meórum~: in generatióne et generatiónem anni tui.

Inítio\footnote{``In inítio.'' 1531-P:34r, 48r.} tu Dómine terram fundásti~: et ópera mánuum tuárum sunt celi.

Ipsi períbunt tu autem pérmanes~: et omnes sicut vestiméntum veteráscent.

Et sicut opertórium mutábis eos et mutabúntur~: tu autem idem ipse es, et anni tui non defícient.

Fílii servórum tuórum habitábunt~: et semen eórum in séculum dirigétur.

Glória Patri.

\subsubsection{Psalmus. ⟨cxxix.⟩}

\lettrine{D}{e} profúndis clamávi ad te Dómine~: Dómine exáudi vocem meam.

Fiant aures tue intendéntes~: in vocem deprecatiónis mee.

Si iniquitátes observáveris Dómine~: Dómine quis sustinébit~?

Quia apud te propiciátio est~: et propter legem tuam sustínui te Dómine.

Sustínuit ánima mea in verbo ejus~: sperávit ánima mea in Dómino.

A custódia matutína usque ad noctem~: speret Israel in Dómino.

Quia apud Dóminum misericórdia~: et copiósa apud eum redémptio.

Et ipse rédimet Israel~: ex ómnibus iniquitátibus ejus.

Glória.

\subsubsection{Psalmus. ⟨cxlij.⟩}

\lettrine{D}{o}mine exáudi oratiónem meam~: áuribus pércipe obsecratiónem meam in veritáte tua\footnote{In 1519:25v. the punctuation differs from the Breviary. For consistency, this edition follows the Breviary. The following is a variant in 1519.

  1. Dómine exáudi oratiónem meam áuribus pércipe obsecratiónem meam in veritáte tua~: * exáudi me in tua justícia.} exáudi me in tua justícia.

Et non intres in judícium cum servo tuo\footnote{1519:26v. has `tuo Dómine.'}~: quia non justificábitur in conspéctu tuo omnis vivens.

Quia persecútus est inimícus ánimam meam~: humiliávit in terra vitam meam.

Collocávit me in obscúris sicut mórtuos séculi~: et anxiátus est super me spíritus meus in me turbátum est cor meum.

Memor fui diérum antiquórum meditátus sum in ómnibus opéribus tuis~: in factis mánuum tuárum meditábar.

Expándi manus meas ad te~: ánima mea sicut terra sine aqua tibi.

Velóciter exáudi me Dómine~: defécit spíritus meus.

Non avértas fáciem tuam a me~: et símilis ero descendéntibus in lacum.

Audítam fac michi mane misericórdiam tuam~: quia in te sperávi.

Notam fac michi viam in qua ámbulem~: quia ad te levávi ánimam meam.

Eripe me de inimícis meis Dómine ad te confúgi~: doce me fácere voluntátem tuam, quia Deus meus es tu.

Spíritus tuus bonus dedúcet me in terram rectam~: propter nomen tuum Dómine, vivificábis me in equitáte tua.

Edúces de tribulatióne ánimam meam~: et in misericórdia tua dispérdes inimícos meos.

Et perdes omnes qui tríbulant ánimam meam~: quóniam ego servus tuus sum.

Glória Patri.
\end{lesson}

\emph{Antiphona}. Ne reminiscáris Dómine delícta nostra vel paréntum nostrórum, neque vindíctam sumas de peccátis nostris.

Kýrie eléyson. Christe eléyson. Kýrie eléyson.

Pater noster. Brev:⟨5⟩.

\emph{Et hec omnia sine nota dicantur tam a sacerdote quam a toto choro.}

\emph{Deinde erigat se sacerdos cum dyacono et subdiacono et cum puero librum sibi administrante et stando conversus ad orientem dicat super populum cum nota hoc modo.}

1519:27r.\footnote{SP:27r. has one less note than required for the next.}

\gregorioscore{}

\emph{Clerici respondeant.\footnote{`\emph{Chorus respondeat}', 1882:26.}} Sed líbera nos a malo.

\emph{⟨℣.⟩} Salvos fac servos tuos et ancíllas tuas. \emph{⟨℟.⟩} Deus meus sperántes in te.

\emph{⟨℣.⟩} Mitte eis Dómine auxílium de sancto. \emph{⟨℟.⟩} Et de Syon tuére eos.

\emph{⟨℣.⟩} Convértere Domine úsquequo. \emph{⟨℟.⟩} Et deprecábilis ⟨esto super servos tuos⟩.\footnote{1882:26.}

\emph{⟨℣.⟩} Adjuva nos Deus salutáris noster. \emph{⟨℟.⟩} Et propter glóriam nóminis ⟨tui Dómine líbera nos et propítius esto peccátis nostris propter nomen tuum⟩.\footnote{1882:26.}

\emph{⟨℣.⟩} Dómine exáudi ⟨oratiónem meam⟩.\footnote{1882:26.} \emph{⟨℟.⟩} Et clamor meus ⟨ad te véniat⟩.\footnote{1882:26.}

\emph{⟨℣.⟩} Dóminus vobíscum. \emph{⟨℟.⟩} Et cum spíritu tuo.

\emph{⟨℣.⟩} Orémus.

\begin{lesson}
\subsubsection{Oratio.}

\lettrine{E}{x}áudi Dómine preces nostras et confiténtium tibi parce peccátis ut quos consciéntie reátus accúsat, indulgéntia tue miseratiónis absólvat. Per Christum Dóminum ⟨nostrum. \emph{℟.} Amen.⟩\footnote{1882:27.}
\end{lesson}

\emph{Et omnes orationes dicantur cum} Orémus. \emph{sub tono supradicto. Non dicatur} Dóminus vobíscum. \emph{nisi ante primam orationem tantum~: tamen alie orationes cum} Orémus.

\begin{lesson}
\subsubsection{Oratio.}

\lettrine{A}{d}sit quésumus Dómine fámulis tuis inspirátio grátie salutáris que corda eórum flétuum ubertáte resólvat sicque macerándo confíciat~: ut iracúndie tue motus~: idónea satisfactióne compéscat. Per Christum Dóminum nostrum. ⟨Amen.⟩\footnote{1882:27.}
\end{lesson}

\emph{⟨℣.⟩} Orémus.

\begin{lesson}
\subsubsection{Oratio.}

\lettrine{D}{a} quésumus Dómine Deus noster his fámulis tuis contínuam purgatiónis sue observántiam peniténdo ágere~: et ut hoc efficátius implére váleant grátia eos tue visitatiónis prevéniat et subsequátur. Per Christum.
\end{lesson}

\emph{⟨℣.⟩} Orémus.

\begin{lesson}
\subsubsection{Oratio.}

\lettrine{P}{r}evéniat hos fámulos ⟨tuos⟩\footnote{1882:27.} quésumus Dómine misericórdia tua~: ut omnes iniquitátes eórum céleri indulgéntia deleántur. Per Christum Dóminum nostrum.
\end{lesson}

\emph{⟨℣.⟩} Orémus.

\begin{lesson}
\subsubsection{Oratio.}

\lettrine{A}{d}ésto Dómine supplicatiónibus nostris nec sit ab his fámulis tuis cleméntie tue longínqua miserátio sana vúlnera, eorúmque peccáta remítte~: ut nullis a te iniquitátibus separáti tibi Dómine semper váleant adherére. Per Christum Dóminum nostrum.
\end{lesson}

\emph{⟨℣.⟩} Orémus.

\begin{lesson}
\subsubsection{Oratio.}

\lettrine{D}{o}mine Deus noster qui offensióne nostra non vínceris sed satisfactióne placáris~: réspice quésumus super hos fámulus tuos qui se tibi gráviter peccásse confiténtur, tuum est enim absolutiónem críminum dare, et véniam prestáre peccántibus qui dixísti peniténtiam te male peccatórum quam mortem~: concéde ergo Dómine his fámulis tuis ut tibi penténtie excúbias célebrent, et corréctis áctibus suis~: conférri sibi a te sempitérna gáudia gratuléntur. Per Christum.
\end{lesson}

\emph{⟨℣.⟩} Orémus.

\begin{lesson}
\subsubsection{Oratio.}

\lettrine{D}{e}us cujus indulgéntia omnis homo índiget meménto famulórum famularúmque tuárum et quia lúbrica terrenáque córporis fragilitáte nudáti virtúte~: in multis deliquérunt~: quésumus ut des véniam confiténtibus parcas supplicántibus\footnote{`supplícibus', 1882:28.}~: ut qui suis méritis accusántur~: tua miseratióne salvéntur. Per Christum Dóminum nostrum. \emph{⟨℟.⟩} Amen.
\end{lesson}

\emph{Hic non dicatur} Dóminus vobíscum. \emph{neque} Orémus. \emph{Sed vertat se sacerdos ad populum, et dicat super eos.}

\begin{lesson}
\subsubsection{Absolutio.}

\lettrine{A}{b}sólvimus vos vice beáti Petri apostolórum príncipis cui colláta est a Dómino potéstas ligándi atque solvéndi et quantum ad vos pértinet accusátio, et ad nos remíssio sit vobis omnípotens Deus vita et salus et ómnium peccatórum vestrórum pius indúltor. Qui vivit et regnat ⟨cum Deo Patre⟩. \emph{\&c.}
\end{lesson}

\emph{Deinde surgant omnes a prostratione osculantes formulas vel terram sacerdote sic dicente} Qui vivit et regnat.

\emph{Deinde accedat sacerdos ad altare cum suis ministris et ibi super altare in dextera parte altaris ad orientem conversus fiat benedictio cinerum prius cineribus in pelvibus argenteis super altare positis.}

\emph{¶ Hic fiat benedictio cinerum sine} Dóminus vobíscum. \emph{et sine} Orémus. \emph{Hoc modo incipiat.}

\begin{lesson}
\subsubsection{Oratio.}

\lettrine{O}{m}nípotens sempitérne Deus qui miseréris ómnium et nichil odísti eórum que fecísti dissímulans peccáta hóminum propter peniténtiam, qui étiam súbvenis in necessitáte laborántibus bene✠dícere et sanctifi✠cáre hos cíneres dignáre quos causa humilitátis et sancte religionis ad emundándam delícta nostra super cápita nostra more Ninevitárum ferre constituísti~: et da per invocatiónem sancti tui nóminis ut omnes qui eos ad deprecándam misericórdiam tuam super cápita sua túlerint a te mereántur ómnium delictórum suórum véniam accípere et hódie sic eórum inchoáre sancta jejúnia~: ut in die resurrectiónis purificátis méntibus ad sanctum mereántur accédere pascha~: et in futúro perpétuam accípere glóriam. Per Dóminum nostrum Jesum Christum Fílium tuum. Qui tecum.
\end{lesson}

\emph{Hic aspergatur aqua benedicta super cineres, deinde dicatur} Dóminus vobíscum. \emph{et} Orémus.

\begin{lesson}
\subsubsection{Oratio.}

\lettrine{D}{e}us qui non mortem sed peniténtiam desíderas peccatórum fragilitátem conditiónis humáne benigníssime réspice, et hos cíneres quos causa preferénde humilitátis atque promerénde vénie capítibus nostris impóni decrévimus~: bene✠dícere pro tua pietáte dignéris~: ut qui nos cíneres esse monuísti et ob pravitátis nostre méritum in púlverem reversúros cognóscimus~: peccatórum ómnium véniam et prémia peniténtibus repromíssa misericórditer cónsequi mereámur. Per Dóminum nostrum Jesum Christum Fílium tuum. Qui tecum vivit et regnat Deus.
\end{lesson}

\emph{Postea distribuantur cineres.}

\emph{¶ Executor officii in sede episcopali ac duo clerici excellentiores persone, stolis amicti ex utraque parte chori in stallis suis capita quorumcunque venientium cineribus aspergent ut sic dicendo,} Meménto homo quod cinis es et in cínerem revertáris. In nómine Patris et Fílii et Spíritus Sancti. Amen.

\emph{Et interim cantentur iste sequens antiphone sequente.}

\gregorioscore{002770-exaudi-nos}

\emph{Non dicatur nisi primus ℣. psalmi sed statim sequatur} Glória Patri. 428. \emph{Deinde repetatur antiphona} Exáudi nos.\footnote{Rylands-24:82. indicates `\emph{repetatur ant.} Glória Patri.'}

\emph{Alia antiphona.}

\gregorioscore{003554-juxta-vestibulum}

\emph{⟨Alia⟩\footnote{1882:29.} antiphona.}

\gregorioscore{003193-immutemur-habitu}

\emph{¶ Peracto officio dicat sacerdos ad gradum chori sic} Dóminus vobíscum.

\emph{Chorus respondeat.} Et cum spíritu tuo.

\emph{⟨Sacerdos.⟩\footnote{1882:29.}} Orémus.

\begin{lesson}
\subsubsection{Oratio.}

\lettrine{D}{e}us qui juste irásceris et cleménter ignóscis afflícti pópuli tui lácrymas súscipe~: et iram tue indignatiónis quam juste merétur ⟨propitiátus⟩\footnote{1882:29.} avérte. Per Christum Dóminum nostrum. \emph{⟨℟.⟩} Amen.

\emph{⟨℣.⟩} Orémus.

\subsubsection{Oratio.}

\lettrine{C}{o}ncéde nobis quésumus Dómine presídia milítie Christiáne sanctis\footnote{`sic sancte', 1882:29.} inchoáre jejúniis~: ut contra spiritáles nequítias pugnatúri continéntie muniámur auxílio.\footnote{`auxíliis', 1882:29.} Per Dóminum nostrum Jesum Christum Fílium tuum. Qui tecum vivit. \emph{⟨et cetera.⟩\footnote{1882:29.}}
\end{lesson}

\emph{His finitis eat processio per medium chori sine cruce cum ceroferariis et thuribulariis ad ostium occidentale excellentioribus precedentibus~: precedente vexillo cilicino. Deinde executor officii penitentes singillatim per manus ejiciat per ministrationem alicujus sacerdotis de choro tradentes eos per manus dextras eidem~: ipsi vero penitentes osculantes manum executoris exeant, tunc si episcopis presens fuerit archidiaconus subministret ei dicto modo, et interim cantentur hec duo responsoria cum suis versiculus~: sine} Glória Patri. \emph{cantore incipiente ut patet in pictura vel in statione sequente hoc modo.}

\begin{figure}
\centering
\includegraphics{images/30r-statio-die-cinerum.jpg}
\caption{\emph{¶ Statio in die cinerum dum episcopus ejiciat penitentes ⟨tunc incipiatur istud responsorium hoc modo ut sequitur⟩.}}
\end{figure}

\gregorioscore{006571-ecce-adam}

\gregorioscore{006937-in-sudore}

\emph{Ejectis penitentibus claudatur ostium ecclesie et redeundo processio more solito cantore incipiente hoc modo.}

\gregorioscore{006653-emendus-in-melius}

\emph{Sine} Glória Patri.

\emph{Non dicatur ℣. neque oratio sed statim incipiatur missa a cantore.}

\chapter{¶ Dominica j. xl.}

\emph{Ad processionem.}

\gregorioscore{002042-cum-venerimus}

\emph{In redeundo cantatur ℟. ⟨ut sequitur⟩.}

\gregorioscore{006529-ductus-est}

\emph{℣.} Scuto circúndabit te véritas ejus.

\emph{℟.} Non timébit ⟨a timóre noctúrno⟩.\footnote{1882:31.}

\emph{⟨℣.⟩} Orémus.

\begin{lesson}
\subsubsection{Oratio.}

\lettrine{D}{e}us qui ecclésiam tuam ánnua quadragesimáli observatióne puríficas, presta famílie tue~: ut quod a te obtinére abstinéndo nítitur, hoc bonis opéribus exequátur. Per Christum Dóminum nostrum. \emph{⟨℟.⟩} Amen.
\end{lesson}

\chapter[⟨Feria quarta et sexta per totam quadragesimam.⟩]{⟨Feria quarta et sexta per totam quadragesimam.⟩\footnote{1882:32.}}

\emph{¶ Sciendum est quod per totam xl. in omni feria iiij. et vj. usque ad cenam Domini fiat processio ad unum altare ecclesie per ordinem post ix. dictam ante inchoationem misse~: nisi festum ix. lectionum ibidem contigerit. Exeat itaque processio per ostium presbyterii borealis ad unum altare ex ejus latere~: sacerdos vero cum suis ministris albis cum amictibus induti sine cruce choro sequente habitu non mutato cantando unum istorum responsoriorum per ordinem cantore incipiente.}

1519:33r;l AS:106; Brev-1531:65r.

\gregorioscore{}

\emph{⟨Resp.} Eméndemus in mélius. \emph{et cetera ut supra in statione cinerum.} 72.

\emph{Aliud resp.⟩\footnote{1882:32.}}

\gregorioscore{007348-paradisi-portas}

\emph{Aliud ℟.}

\gregorioscore{007626-scindite-corda}

\emph{Aliud ℟.}

\gregorioscore{006012-abscondite-elemosinam}

\emph{Finito Responsorio cum ℣. ⟨et⟩\footnote{1882:32.} absque} Glória Patri. \emph{clerici eodem modo quo in processione ordinantur prostrationem faciant. Ita quod sacerdos ad gradum altaris cum diacono a dextris et subdyacono a sinistris et ceroferariis cereis interim super altare dimissis suam faciant prostrationem cum dicitur sine\footnote{`\emph{sine}', 1530:39v; 1555; 1882:32; 1901:65. `\emph{cum}', 1519:34v.} nota.}

Kyrieléyson. Christeléyson. Kyrieléyson.

Pater noster. Brev:⟨5⟩.

\emph{Deinde dicat sacerdos cum nota ⟨sic⟩.\footnote{1882:33.}}

1519:34v.

\gregorioscore{}

\emph{⟨℣.⟩} Osténde nobis Dómine misericórdiam tuam.

\emph{⟨℟.⟩} Et salutáre tuum da nobis.

\emph{⟨℣.⟩} Peccávimus cum pátribus nostris.

\emph{⟨℟.⟩} Injúste égimus iniquitátem fécimus.

\emph{⟨℣.⟩} Dómine non secúndum péccata nostra fácias nobis.

\emph{⟨℟.⟩} Neque secúndum iniquitátes nostras retríbuas nobis.

\emph{⟨℣.⟩} Ne memíneris iniquitátum nostrárum antiquárum.

\emph{⟨℟.⟩} Cito antícipent nos misericórdie tue, quia páuperes facti sumus nimis.

\emph{⟨℣.⟩} Adjuva nos Deus salutáris noster.

\emph{⟨℟.⟩} Et propter glóriam nóminis tui Dómine líbera nos.

\emph{⟨℣.⟩} Exáudi Dómine vocem meam qua clamávi ad te.

\emph{⟨℟.⟩} Miserére mei et exáudi me.

\emph{Ps}. Miserére mei Deus. 59. \emph{Totus psalmus dicatur sine nota sed cum} Glória Patri. \emph{et} Sicut erat. \emph{alternando ex utraque parte chori. Quo finito solus sacerdos se erigat dicendo cum nota sic.}

\emph{⟨Versus.⟩\footnote{1882:33.}}

1519:35r.

\gregorioscore{}

\begin{lesson}
\lettrine{P}{r}eces nostras quésumus Dómine cleménter exáudi~: ut qui juste pro peccátis nostris afflígimur, pro tui nóminis glória misericórditer liberémur. Per Christum Dóminum.
\end{lesson}

\emph{Et sic surgant omnes a prostratione osculantes terram vel formulas sacerdote sic dicente} Per Christum Dóminum nostrum.

\emph{Finita oratione duo clerici de ij. forma habitu non mutato dicant letaniam in revertendo, et dicatur usque ad prolationem} Sancta María. \emph{coram altare antequam processio procedat.}

1519:35v; Ant-1519-P:172r; SP:164r.; Brev-1531-P:48v.

\gregorioscore{}

Pater de celis ⟨audi nos⟩. \emph{⟨ij.⟩}

Fili redémptor mundi Deus⟨audi nos⟩. \emph{⟨ij.⟩}

Spíritus Sancte Deus⟨audi nos⟩. \emph{⟨ij.⟩}

Sancta Trínitas Unus Deus⟨audi nos⟩. \emph{⟨ij.⟩}

1519:35v.\footnote{In 1555. `pro' is set B. This is the melodic form that was taken up in the English Litany of 1544.}

\gregorioscore{}

\emph{Et processio presbyterorum circumeundo chorum intret~: et clerici predicti prosequantur de ceteris ordinibus quantum sufficit iter, scilicet tres vel quattuor~: et si tantum restat iter sub tono predicto ita quod ad gradum chori finiatur sic.}

1519:35v.\footnote{In 1555. `pro' is set B.}

\gregorioscore{}

\emph{Hic ordo letaniarum servetur in processionibus faciendis per totam quadragesimam scilicet in quartam feriam et sextam.}

\chapter[¶ Feria quarta prime ebdomade ⟨quadragesime⟩.]{¶ Feria quarta prime ebdomade ⟨quadragesime⟩.\footnote{1882.34.}}

\emph{⟨Sequitur letania.⟩\footnote{1882.34.}}

Kyrieléyson. Christeléyson. Christe audi nos.

Pater de celis Deus. Miserére nobis.

Fili Redémptor mundi Deus. Miserére nobis.

Spíritus Sancte Deus. Miserére nobis.

Sancta Trínitas Unus Deus. Miserére nobis.

Sancta María. Ora pro nobis.

Sancta Dei génitrix. ora.

Sancta virgo vírginum. ora.

Sancte Míchael. ora.

Sancte Gábriel. ora.

Sancte Ráphael. ora.

Omnes sancti angeli et archángeli.

Oráte pro nobis.

Omnes sancti beatórum spirítuum órdines. oráte.

Sancte Johánnes baptísta. ora.

Omnes sancti patriárche et prophéte. Oráte pro nobis.

Sancte Petre. ora.

Sancte Paule. ora.

Sancte Andréa. ora.

Omnes sancti apóstoli et evangelíste. oráte.

Omnes sancti discípuli Dómini et innocéntes. oráte.

Sancte Stéphane. ora.

Sancte Line. ora.

Sancte Clete. ora.

Omnes sancti mártyres. oráte.

Sancte Silvéster. ora.

Sancte Leo. ora.

Sancte Hierónyme. ora.

Omnes sancti confessóres. oráte.\\
Omnes sancti monáchi et heremíte. oráte.

Sancta María Magdaléna. ora.

Sancta María Egyptíaca. ora.

Sancta Margaréta. ora.

Omnes sancte vírgines. oráte.

Omnes sancti. oráte.

\chapter{¶ Feria vj. ejusdem ebdomada.}

Sancte Johánnes. ora.

Sancte Jacóbe. ora.

Sancte Thoma. ora.

Omnes sancti apóstoli et evangelíste. oráte.

Omnes sancti discípuli Dómini et innocéntes. orate.

Sancte Clemens. ora.

Sancte Fabiáne. ora.

Sancte Sebastiáne. ora.

Omnes sancti mártyres. orate.

Sancte Augustíne. ora.

Sancte Ysidóre. ora.

Sancte Juliáne. ora.

Omnes sancti confessóres. oráte.

Omnes sancti monácbi et heremíte.

Oráte\footnote{`Ora', 1519:36r.} pro nobis.

Sancta Scolástica. ora.

Sancta Petronílla. ora.

Sancta Genovéfa. ora.

Omnes sancte vírgines. oráte.

Omnes sancti. oráte pro nobis.

\chapter{¶ Feria iiij. ij. ebdomade.}

Sancte Philíppe. ora.

Sancte Jacóbe. ora.

Sancte Mathée. ora.

Omnes sancti apóstoli et evangelíste. oráte.

Omnes sancti discípuli Dómini et innocéntes. oráte.

Sancte Cosma. ora.

Sancte Damiáne. ora.

Sancte Prime. ora.

Omnes sancti mártyres. oráte.

Sancte Gildárde. ora.

Sancte Medárde. ora.

Sancte Albíne. ora.

Omnes sancti confessóres. oráte.\\
Omnes sancti monáchi et heremíte. Oráte pro nobis.

Sancta Praxédis. ora.

Sancta Sothéris. ora.

Sancta Prisca. ora.

Omnes sancti vírgines. oráte.

Omnes sancti. oráte.

\chapter{¶ Feria vj. ejusdem ebdomade.}

Sancte Bartholomée. ora.

Sancte Symon. ora.

Sancte Thadée. ora.

Omnes sancti apóstoli et

evangelíste. oráte.

Omnes sancti discípuli Dómini et innocéntes. oráte.

Sancte Feliciáne. ora.

Sancte Dionísi cum sóciis tuis. oráte.

Sancte Victor cum sóciis tuis. oráte.

Omnes sancti mártyres. oráte.

Sancte Eusébi. ora.

Sancte Swythúne. ora.

Sancte Biríne ora.\\
Omnes sancti confessóres. oráte.\\
Omnes sancti monáchi et heremíte.

Oráte pro nobis.

Sancta Tecla. ora.

Sancta Affra. ora.\\
Sancta Edítha. ora.

Omnes sancti vírgines. oráte.

Omnes sancti. oráte.

\chapter{¶ Feria iiij. iij. ebdomade.}

Sancte Mathía. ora.

Sancte Bárnaba. ora.

Sancte Marce. ora.

Omnes sancti apóstoli et

evangelíste. oráte.

Omnes sancti discípuli Dómini et

innocéntes. oráte.

Sancte Thoma. ora.

Sancte Cornéli. ora.\\
Sancte Cypriáne. ora.

Omnes sancti mártyres. oráte.

Sancte Gregóri. ora.

Sancte Augustíne. ora.

Sancte Ambrósi. ora.

Omnes sancti confessóres. oráte.

Omnes sancti monáchi et heremíte.

Oráte pro nobis.

Sancta Felícitas. ora.

Sancta Perpétua. ora.

Sancta Colúmba. ora.

Omnes sancte vírgines. Oráte.

Omnes sancti. Oráte.

\chapter{¶ Feria vj. ejusdem ebdomade.}

Sancte Petre. ora.

Sancte Paule. ora.

Sancte Andréa. ora.

Omnes sancti apóstoli et evangelíste. oráte.

Omnes sancti discípuli Dómini et innocéntes. oráte.

Sancte Johánnes. ora.

Sancte Kenélme. ora.

Sancte Osvuálde. ora.

Omnes sancti mártyres. oráte.

Sancte Remígi. ora.

Sancte Donatiáne. ora.

Sancte Elígi. ora.

Omnes sancti confessóres. oráte.

Omnes sancti monáchi et heremíte. oráte pro nobis.

Sancta Cristiána.\footnote{`Christína', 1519:37r.} ora.

Sancta Eulália. ora.

Sancta Eufémia. ora.

Omnes sancte vírgines. oráte.

Omnes sancti. oráte pro nobis.

\chapter{¶ Feria iiij. iiij. ebdomade.}

Sancte Johánnes. ora.

Sancte Jacóbe. ora.

Sancte Thoma. ora.

Omnes sancti apóstoli et

evangelíste. oráte.

Omnes sancti discípuli Dómini et innocéntes. Oráte.

Sancte Laurénti. ora.

Sancte Tybúrci. ora.

Sancte Valeriáne. ora.

Omnes sancti mártyres. oráte.

Sancte Dúnstane. ora.

Sancte Edmúnde. ora.

Sancte Leonárde. ora.

Omnes sancti confessóres. oráte.

Omnes sancti monáchi et heremíte. oráte pro nobis.

Sancta Katherína. ora.

Sancta Cecília. ora.

Sancta Anastásia. ora.

Omnes sancte vírgines. oráte.

Omnes sancti. oráte.

\chapter{¶ Feria vj. ejusdem ebdomade.}

Sancte Philíppe. ora.

Sancte Jacóbe. ora.

Sancte Mathée. ora.

Omnes sancti apóstoli et

evangelíste. oráte.

Omnes sancti discípuli Dómini et innocéntes. oráte.

Sancte Vincénti. ora.

Sancte Géreon cum sóciis tuis. oráte.

Sancte Mauríci cum sóciis tuis. oráte pro nobis.

Omnes sancti mártyres. oráte.

Sancte Nicoláe. ora.

Sancte Richárde. ora.

Sancte Machúte. ora.

Omnes sancti confessóres. oráte.

Omnes sancti monáchi et heremíte. oráte pro nobis.

Sancta Agnes. ora.

Sancta Juliána. ora.

Sancta Cuthbúrga. ora.

Omnes sancte vírgines. oráte.

Omnes sancti. oráte.

\chapter{¶ Feria iiij. v. ebdomade.}

Sancte Bartholomée. ora.

Sancte Symon. ora.

Sancte Thadée. ora.

Omnes sancti apóstoli et evangelíste. Oráte pro nobis.

Omnes sancti discípuli Dómini et innocéntes. oráte.

Sancte Albáne. ora.

Sancte Ypólite cum sóciis tuis. oráte.

Sancte Luciáne cum sóciis tuis. oráte.\footnote{1519:37v. has `ora.'}

Omnes sancti mártyres. oráte.

Sancte Sanson. ora.\footnote{1519:37v. has `oráte.'}

Sancte Plácide. ora.

Sancte Columbáne. ora.

Omnes sancti confessóres. oráte.\footnote{1519:37v. has `ora.'}

Omnes sancti monáchi et heremíte.

Oráte pro nobis.

Sancta Felícula. ora.

Sancta Susánna. ora.

Sancta Brigída. ora.

Omnes sancte vírgines. oráte.\footnote{1519:37v. has `ora.'}

Omnes sancti. oráte.

\chapter{¶ Feria vj. ejusdem ebdomade.}

Sancte Mathía. ora.

Sancte Bárnaba. ora.

Sancte Marce. ora.

Omnes sancti apóstoli et evangelíste. oráte.\footnote{1519:37v. has `ora.'}

Omnes sancti discípuli Dómini et innocéntes. oráte.

Sancte Gervási ora.

Sancte Prothási ora.

Sancte Timothée ora.

Omnes sancti mártyres. oráte.

Sancte Benedícte. ora.

Sancte Maure. ora.

Sancte Egídi. ora.

Omnes sancti confessóres. oráte.

Omnes sancti monáchi et heremíte.

Oráte pro nobis.

Sancta Scholástica. ora.

Sancta Sabína. ora.

Sancta Justína. ora.

Omnes sancte vírgines. oráte.

Omnes sancti. oráte.

\chapter{¶ Feria iiij. vj. ebdomade.}

Sancte Petre. ora.

Sancte Paule. ora.

Sancte Andréa. ora.

Omnes sancti apóstoli et

evangelíste. oráte.\footnote{1519:37v. has `ora.'}

Omnes sancti discípuli Dómini et innocéntes. oráte.

Sancte Quintíne. ora.

Sancte Bonifáci cum sóciis tuis. oráte.\footnote{1519:37v. has `ora.'}

Sancte Kyliáne cum sóciis tuis.

Ora pro nobis.

Omnes sancti mártyres. oráte.

Sancte Brici. ora.

Sancte Amánde. ora.

Sancte Cuthbérte. ora.\\
Omnes sancti confessóres. oráte.

Omnes sancti monáchi et heremíte.

Oráte pro nobis.

Sancta Margaréta. ora.

Sancta Walbúrgis. ora.

Sancta Radegúndis. ora.

Omnes sancte vírgines. oráte.

Omnes sancti. Oráte pro nobis.

\emph{Non sequatur versiculus neque oratio~: sed sacerdos cum suis ministris accedat, et statim a cantore incipiatur officium misse ⟨et precedat crux lignea absque imagine crucifixi.⟩\footnote{1882:41.}}

\emph{⟨Dominica secunda quadragesime et in omnibus dominicis per totam xl. excepta prima dominica, deferatur crux lignea ante processionem sine imagine crucifixi, quando de dominica agitur. In omnibus aliis processionibus in xl. contingentibus ordinetur ut in alio tempore.⟩\footnote{Added in 1517 edition. ⟨1882:41⟩.}}

\chapter{¶ Dominica secunda xl.}

\emph{Ad processionem ant.} Cum venérimus. \emph{ut supra.} 73.

\emph{In introitu chori dicatur hoc ℟. sequens.}

\gregorioscore{007156-minor-sum}

\emph{℣.} Scuto circúndabit te véritas ejus.

\emph{℟.} Non timébis ⟨a timóre noctúrno⟩.\footnote{1882:41.}

\emph{⟨℣.⟩} Orémus.

\begin{lesson}
\subsubsection{Oratio.}

\lettrine{D}{e}us qui cónspicis omni nos virtúte destítui intérius exteriúsque custódi~: ut ab ómnibus adversitátibus muniámur in córpore, et a pravis cogitatiónibus mundémur in mente. Per ⟨Dóminum nostrum. \emph{⟨℟.⟩} Amen.⟩\footnote{1882:41.}
\end{lesson}

\chapter[¶ Dominica iij. ⟨quadragesime⟩.]{¶ Dominica iij. ⟨quadragesime⟩.\footnote{1882:41.}}

\emph{Ad processionem.}

\gregorioscore{a00236-in-die}

\emph{In introitu chori dicatur ⟨hoc responsorium sequens⟩.\footnote{1882:42.}}

\gregorioscore{007102-loquens-joseph}

\emph{℣.} Scuto circúndabit te véritas ejus.

\emph{℟.} Non timébis ⟨a timóre noctúrno⟩.\footnote{1882:42.}

\emph{⟨℣.⟩} Orémus.

\begin{lesson}
\subsubsection{Oratio.}

\lettrine{Q}{u}ésumus omnípotens Deus vota humílium réspice atque ad defensiónem nostram déxteram tue majestátis exténde. Per ⟨Christum Dóminum nostrum. \emph{⟨℟.⟩} Amen.⟩\footnote{1882:42.}
\end{lesson}

\chapter{¶ Dominica medie xl.}

\emph{Ad processionem ant.} In die quando. \emph{ut supra.} 90.

\emph{In redeundo ⟨cantatur⟩\footnote{1882:42.}}

\gregorioscore{006143-audi-israel}

\emph{℣.} Scuto circúndabit te véritas ejus.

\emph{℟.} Non timébis ⟨a timóre noctúrno⟩.\footnote{1882:42.}

\emph{⟨℣.⟩} Orémus.

\begin{lesson}
\subsubsection{Oratio.}

\lettrine{C}{o}ncéde quésumus omnípotens Deus~: ut qui ex mérito nostre actiónis afflígimur~: tue grátie consolatióne respirémus. Per Christum ⟨Dóminum nostrum. \emph{⟨℟.⟩} Amen⟩.\footnote{1882:42.}
\end{lesson}

\chapter{¶ Dominica in passione Domini.}

\emph{Ad processionem.}

\gregorioscore{006287-circundederunt-me}

\emph{Non dicatur} Glória Patri. \emph{Si necesse fuerit repetatur ℟. cum versu.}

\emph{In introitu chori.}

\gregorioscore{006931-in-proximo-est}

\emph{℣.} Eripe me de inimícis meis Deus meus.

\emph{℟.} Et ab insurgéntibus in me líbera me.

\emph{⟨℣.⟩} Orémus.

\begin{lesson}
\subsubsection{Oratio.}

\lettrine{Q}{u}ésumus\footnote{The breviary omits `Quésumus.'} omnípotens Deus famíliam tuam propícius réspice~: ut te largiénte regátur in córpore~: et te servánte custodiátur in mente. Per Christum.
\end{lesson}

\chapter{¶ Dominica in ramis palmarum.}

\emph{Post aque benedicte aspersionem legatur hec lectio ante gradum altaris ab accolito alba induto et hoc ex parte australi super flores et frondes cum titulo suo hoc modo.}

\begin{lesson}
\subsubsection{Lectio libri Exodi. (xv. 27.)}

\lettrine{I}{n} diébus illis. Venérunt fílii Israel in Helym~: ubi erant duódecim fontes aquárum\footnote{`duódecim fontes aquárum', 1882:43.} et septuagínta palme et castrametáti\footnote{`castrameáti', 1523:41v.} sunt juxta aquas. Profectíque sunt de Helym~: et venit omnis multitúdo filiórum Israel in desértum Syn~: quod est inter Helym et Synai quintadécima die mensis secúndi postquam egréssi sunt de terra Egýpti et murmurávit omnis congregátio filiórum Israel contra Móysen et Aaron in solitúdine. Dixerúntque ad eos fílii Israel, Utinam mórtui essémus per manum Dómini in terra Egýpti~: quando sedebámus super ollas cárnium~: et comedebámus panem in saturitáte. Cur induxístis nos in desértum istud~: ut occiderétis omnem multitúdinem fame~? Dixit autem Dóminus ad Móysen, Ecce ego pluam vobis panes de celo. Egrediátur pópulus et cólligat que suffíciunt per síngulos dies~: ut temptem ⟨eum⟩\footnote{1882:43; \emph{Vulgate.}} utrum ámbulet in lege mea, an non. Die autem sexta parent quod ínferant~: et sit duplum quam collígere solébant per síngulos dies. Dixerúntque Móyses et Aaron ad omnes fílios Israel, Véspere sciétis quod Dóminus edúxerit vos de terra Egýpti~: et mane vidébitis glóriam Dómini. Audívi enim murmur vestrum contra Dóminum. Nos vero quid sumus~: quia musitástis contra nos~? Et ait Móyses, Dabit vobis Dóminus véspere carnes édere, et mane panes in saturitáte~: eo quod audívit\footnote{`audíverit', 1882:43; \emph{Vulgate.}} murmuratiónes vestras quibus murmuráti estis contra eum, Nos enim quid sumus~? Nec contra nos est murmur vestrum~: sed contra Dóminum. Dixítque Móyses ad Aaron, Dic univérse congregatióni filiórum Israel, Accédite coram Dómino~: audívit enim murmur vestrum. Cumque enim\footnote{\emph{Vulgate} omits `enim.'} loquerétur Aaron ad omnem cetum filiórum Israel~: respexérunt ad solitúdinem, et ecce glória Dómini appáruit in nube.
\end{lesson}

\emph{⟨Sintque ministri processionis albis cum amictibus induti tantum sine tunicis vel casulis~: executor tamen officii in capa rubea serica, choro itaque sequente habitu non mutato.⟩\footnote{1517. 1882:44.}}

\emph{Deinde accepto ⟨statim⟩\footnote{1882:44.} benedictione legatur evangelium super lectrinum ⟨ubi leguntur evangelia a diacono⟩\footnote{1882:44; `\emph{a diacono}' in 1517. 1882:44.} ad boreale\footnote{1517---ad oriéntem. 1882:44.} converso more simplicis festo ⟨post acceptam benedictionem⟩\footnote{1882:44.} cum} Dóminus vobíscum.

\begin{lesson}
\subsubsection[⟨Evangelium⟩ secundum Johannem. (xij. 12.)]{⟨Evangelium⟩\footnote{1882:44.} secundum Johannem. (xij. 12.)}

\lettrine{I}{n} illo témpore. Turba multa que convénerat\footnote{`vénerat', \emph{Vulgate.}} ad diem festum cum audíssent quia venit Jesus Hierosólimam~: accepérunt ramos palmárum et processérunt óbviam ei~: et clamábant, Osánna~: benedíctus qui venit in nómine Dómini Rex Israel. Et invénit Jesus aséllum et sedit super eum~: sicut scriptum est, Noli timére fília Syon~: ecce rex tuus venit tibi mansuétus~: sedens super pullum ásine. Hec non cognovérunt discípuli ejus primum~: sed quando glorificátus est Jesus~: tunc recordáti sunt quia hec erant scripta de eo et hec fecérunt ei. Testimónium\footnote{`Testiónium', 1519:52r.} ergo perhibébat turba que erat cum eo quando Lázarum vocávit de monuménto~: et suscitávit eum a mórtuis. Proptérea et óbviam venit ei turba~: quia audiérant eum fecísse hoc signum. Phariséi ergo dixérunt ad semetípsos, Vidétis quia nichil profícimus~? Ecce mundus totus post eum ábiit.
\end{lesson}

\emph{Finito evangelio statim executor officii in gradu iij. ad altare positis prius palmis\footnote{`\emph{ramis}', 1882:45.} cum frondibus coram altare pro clericis pro aliis vero super gradum predictum in parte australi.}

\begin{figure}
\centering
\includegraphics{images/42r-statio-rami.jpg}
\caption[⟨Statio dum benedicuntur rami in dominica ramis palmarum.⟩]{⟨Statio dum benedicuntur rami in dominica ramis palmarum.⟩\footnote{1882:45.}}
\end{figure}

\emph{¶ Sequatur benedictio florum et frondium a sacerdote induto cappa serica rubea super gradum altaris ad austrum\footnote{1517---orientem. 1882:45.} converso ita dicens.}

\begin{lesson}
\lettrine{E}{x}orcízo te, creatúra florum vel fróndium in nómine Dei ✠ Patris omnipoténtis et in nómine Jesu Christi ✠ Fílii ejus Dómini nostri~: et in virtúte ✠ Spíritus Sancti~: próinde omnis virtus adversérii~: omnis exércitus diáboli~: omnis potéstas inimíci~: omnis incúrsio démonum eradicáre et explantáre ab hac creatúra florum vel fróndium ut ad Dei grátiam festinántium vestígia \emph{Que terminetur sic.}
\end{lesson}

1519:42v.

\gregorioscore{}

\emph{¶ Deinde dicantur omnes orationes sine} Dóminus vobíscum. \emph{sed tantum cum} Orémus.

\begin{lesson}
\subsubsection{Oratio.}

\lettrine{O}{m}nípotens sempitérne Deus qui in dilúvii effusióne Noe fámulo tuo per os colúmbe gestántis ramum ólive pacem terris rédditam nunciásti, te súpplices deprecámur~: ut hanc creatúram florum spatulásque palmárum seu frondes árborum quas ante conspéctum glórie tue offérimus verítas tua sanctí✠ficet, ut devótus pópulus in mánibus eas suscípiens\footnote{`suscípiens eas', 1882:46.} bene✠dictiónis tue grátiam cónsequi mereátur.
\end{lesson}

\emph{Et dicantur et terminentur omnes orationes sub tono isto legendo hoc modo.}

1519:42v.

\gregorioscore{}

\emph{Et omnes orationes dicantur cum} Orémus. \emph{sub tono supradicto.}

\begin{lesson}
\subsubsection{Oratio.}

\lettrine{D}{e}us cujus Fílius pro salúte géneris humáni de celo descéndit ad terras~: et appropinquánte hora passiónis sue Hierosólimam in ásino sedens veníre et a turbis rex appellári et laudári vóluit~: auge fidem in te sperántium~: et súpplicum preces cleménter exáudi~: véniat super nos quésumus Dómine misericórdia\footnote{`benedíctio', 1882:46.} tua~: et hos palmárum ceterarúmque árborum ramos bene✠dícere dignáre~: ut omnes qui eos latúri sunt, benedictiónis tue dono repleántur, concéde ergo ut sicut Hebreórum púeri osánna in excélsis clamántes eídem Fílio tuo Dómino nostro cum ramis palmárum occurrérunt nos ita árborum ramos gestántes cum bonis opéribus occurrámus óbviam Christo et perveniámus ad gáudium sempitérnum. Per eúndem Christum Dóminum nostrum.

⟨Orémus.

\subsubsection[Oratio.⟩]{Oratio.⟩\footnote{1882:46.}}

\lettrine{D}{e}us qui dispérsa cóngregas, et congregáta consérvas qui pópulis óbviam Christo Jesu ramos palmárum gestántibus\footnote{`portántibus', 1882:46.} benedixísti~: béne✠dic\footnote{The ✠ appears in 1882:46, but not in 1519:43r.} ⟨étiam⟩\footnote{1882:46.} et hos ramos palmárum ceterarúmque árborum quos tui fámuli ad nóminis tui benedictiónem fidéliter suscípiunt~: ut in quemcúnque locum introdúcti fúerint, tuam benedictiónem habitatóres illíus loci omnes consequántur~: ita ut omni advérsa valitúdine effugáta~: déxtera tua prótegat quos rédemit. Per eúndem Christum\footnote{1882:47. omits `Christum.'} Dóminum nostrum.
\end{lesson}

\emph{Hic asrpergatur aqua benedicta super flores et frondes et thurificentur. Deinde dicatur} Dóminus vobíscum. \emph{⟨cum⟩\footnote{1882:47.}} Orémus.

\begin{lesson}
\subsubsection{Oratio.}

\lettrine{D}{o}mine Jesu Christe mundi Cónditor et Redémptor~: qui nostre liberatiónis et sanatiónis grátia ex summa celi arce descéndere, et carnem súmere, et passiónem subíre dignátus es~: quique sponte própria loco ejúsdem propínquans passiónis turbis\footnote{`passiónis appropínquans a turbis', 1882:47.} cum ramis palmárum obviántibus benedíci, laudári, et rex benedíctus in nómine Dómini véniens clara voce appellári voluísti~: tu nunc nostre confessiónis laudatiónem acceptáre, et hos palmárum ceterarúmque árborum ac florum ramos bene✠dícere et sancti✠ficáre dignéris~: ut quicúnque ⟨in tue⟩\footnote{1882:47.} virtútis obséquio exínde áliquid túlerit celésti benedictióne sanctificátus peccatórum remissiónem et vite etérne prémia percípere mereátur. Per te Jesu Christe salvátor mundi.\footnote{1882:47. omits `mundi.'} Qui cum Patre et Spíritu Sancto vivis et regnas Deus. Per ómnia sécula seculórum. \emph{⟨℟.⟩\footnote{1882:47.}} Amen.
\end{lesson}

\emph{¶ His itaque peractis statim distribuantur palme, et interim cantentur hee sequentes antiphone cantore incipiente antiphonam hoc modo.}

\gregorioscore{004415-pueri-hebreorum-tollentes}

\emph{⟨Sequitur⟩\footnote{1882:47.} alius cantus.}

\gregorioscore{004416-pueri-hebreorum-vestimenta}

\emph{¶ Dum distribuuntur rami benedicti preparetur feretrum cum reliquiis in quo corpus Christi dependeat~: et ad locum prime stationis a duobus clericis de secunda forma non tamen processionem sequendo sed ad locum prime stationis obviam veniendo habitu non mutato deferatur lumen in lanterna precedente, cum cruce et duobus vexillis precedentibus~: deinde exeat processio ad locum prime stationis sintque ministri processionis albis cum amictibus ahsque tunicis vel casulis. Ita tamen quod sacerdos eat in cappa serica rubea ut supra, choro itaque sequente habitu non mutato. Si tamen assit episcopus mitram et baculum in fine proccessionis.\footnote{`\emph{Deinde per medium chori et ecclesie exeat processio per ostium chori occidentale usque ad locum prime stationis, videlicet ante crucem septentrionali cimiterio, precedente cruce sine imagine, ut in aliis dominicis quadragesime, episcopo vel excellentiore sacerdote exsequente officium processionis, sintque ministri albis cum amictibus induti absque tunicis vel casulis~: ita tamen quod exsecutor officii processionis eat in capa serica rubea, ckoro itaque sequente habitu non mutato. Si tamen assit episcopus, mitram gerat et baculum in fine proccessionis.}' 1882:47.}}

\emph{In eundo ⟨cantor⟩\footnote{1882:48.} dicitur hec antiphonam ⟨sequentem⟩.\footnote{1882:48.}}

\gregorioscore{203956-prima-autem-azimorum}

\emph{Unde\footnote{1517---Unde cum hec, \&c.~1544, \&c.---Cum autem. ⟨1882:48.⟩} hec antiphona canitur cum sequentibus, exeat processio per ostium occidentale~: et sic circa claustrum, et ita per portam canonicorum usque ad locum prime stationis que fit ex parte ecclesie borealis, in extrema parte orientalis tunc laicorum\footnote{`should run--- \emph{in extrema parte orientali cimiterii laicorum.}' ⟨1882:xv.⟩} legetur evangelium.\footnote{\emph{⟨cimiterii laicorum, ubi legetur evangelium modo quo sequitur.⟩} So 1517 ; but 1508, \&c.---in extrema parte orientalis. Tunc laicorum legetur evangelium. 1528---leaf wanting. ⟨1882:48.⟩}}

\gregorioscore{001976-cum-appropinquaret}

\emph{Si ⟨autem⟩\footnote{1882:48.} non sufficiant hee due antiphone usque ad locum prime stationis~: tunc dicantur hee sequentes antiphone.}

\gregorioscore{001983-cum-audisset}

\emph{⟨Sequitur⟩ alia antiphona.}

\gregorioscore{001437-ante-sollennitatis}

\emph{⟨Item⟩\footnote{1882:49.} alia antiphona.}

\gregorioscore{001437-ante-passionis}

\emph{¶ Hic fiat prima statio ex parte ecclesie boreali in extrema parte orientalis~: et legatur evangelium} Cum appropinquásset Jesus. \emph{ab ipso diacono induto ad processionem non juxta crucem sed ante sacerdotem loco parumper secus stationem illius diaconi mutato, ab ipso diacono ad boreale\footnote{1517---oriéntem. ⟨1882:49.⟩} converso. Et legatur more simplicis festi ix. lectionum post acceptam benedictionem cum} Dóminus vobíscum. \emph{et} Glória tibi Dómine.

\emph{Et tunc feretrum cum reliquiis preparatum in quo Christus\footnote{1511---Christus. ⟨1882:49.⟩} in pixide dependeat, obviam veniendo ad locum prime stationis, a duobus clericis de secunda forma deferatur habitu non mutato, ut supra~: et in fine evangelii ad hec verba} Benedíctus qui venit in nómine Dómini. \emph{primo se ostendat.}

\begin{lesson}
\subsubsection{Evangelium secundum Mattheum. xxj. (1--9.)}

\lettrine{I}{n} illo témpore. Cum appropinquásset Jesus Hierosólymam, et venísset Bethpháge ad montem Olivéti~: tunc misit duos discípulos dicens eis, Ite in castéllam quod contra vos est~: et statim inveniétis ásinam alligátam, et pullum cum ea. Sólvite~: et addúcite michi. Et si quis vobis áliquid díxerit, dícite quia Dóminus his opus habet~: et conféstim dimíttet eos. Hoc autem totum factum est~: ut adimplerétur quod dictum est per prophétam dicéntem, Dícite fílie Syon, Ecce rex tuus venit tibi mansuétus~: sedens super ásinam, et pullum fílium subjugális. Eúntes autem discípuli~: fecérunt sicut precépit illis Jesus. Et adduxérunt ásinam et pullum, et imposuérunt super eos vestiménta sua~: et eum désuper sédere fecérunt. Plúrima autem turba~: stravérunt vestiménta sua in via. Alii autem cedébant ramos de arbóribus~: et sternébant in via. Turbe autem que precedébant et subsequebántur~: clamábant dicéntes, Osánna fílio David~: Benedíctus qui venit in nómine Dómini.
\end{lesson}

\emph{¶ Finito evangelio ⟨unus puer ad modum prophete indutus stans in aliquo eminenti loco cantet lectionem propheticam modo quo sequitur.}

\gregorioscore{003481-hierusalem-respice}

\emph{Tres clerici de secunda forma habitu non mutato exeuntes ab eadem processione conversi ad populum stantes ante magnam crucem ex parte occidentali simul cantent antiphonam\footnote{`\emph{hunc versum}', 1882:50.} hoc modo.}

\gregorioscore{a01485-en-rex-venit}

\emph{Post singulos\footnote{'\emph{singulas'}, 1519:47v.} ℣. executor officii incipiat antiphonam} Salve. \emph{conversus ad reliquias quam prosequatur chorus cum genuflexione osculando terram ab ipso quoque executore officii primo cum flectione. Senior genuflectendo dicat sic.\footnote{`\emph{primo cum choro genufiectendo, dicatur sic}', 1882:50.}}

\gregorioscore{a01485-salve-quem-jesum}

\emph{Chorus in prostratione deosculando terram prosequatur resurgendo.}

\gregorioscore{a01485-testatur-plebs}

\emph{⟨Item propheta cantet hoc quod sequitur.}

1508, unpaged; 1517, unpaged; 1555, unpaged.\footnote{The prophet's part appears to be found only in the Edd. of 1508 and 1517 (and 1555). ⟨1882:50.⟩ The music is taken from the capitulum at First Vespers of Advent I.}

\gregorioscore{}\footnote{1555, unpaged.}

\emph{Iterum clerici stantes ante reliquias ⟨simul⟩\footnote{1882:51.} loco nec habitu mutato dicant ⟨hunc⟩\footnote{1882:51.} ℣. ⟨sequentem⟩.\footnote{1882:51.}}

\gregorioscore{a01485-hic-est-qui}

\emph{¶ Item senior\footnote{`\emph{executor officii}', 1882:51.} loco nec habitu mutato incipiat ⟨antiphonam⟩.\footnote{1882:51.}}

\gregorioscore{a01485-salve-lux-mundi}

\emph{Chorus resurgendo prosequatur.}

\begin{noinitial}

\gregorioscore{a01485-rex-regum}

\end{noinitial}

\emph{⟨Iterum propheta.}

1508, unpaged; 1517, unpaged; 1555, unpaged.\footnote{The prophet's part appears to be found only in the Edd. of 1508 and 1517 (and 1555). ⟨1882:50.⟩ The music is taken from the capitulum at First Vespers of Advent I.}

\gregorioscore{}

\emph{Postea evanescat.⟩\footnote{1882:51.}}

\emph{Item clerici ante reliquias loco nec habitu mutato ⟨dicant hunc⟩\footnote{1882:51.} ℣.\footnote{In 1882:51 the ℣. concludes `cecinerunt prophetice.'}}

\gregorioscore{a01485-hic-est-ille}

\emph{Item senior\footnote{`\emph{executor officii}', 1882:51.} loco nec habitu mutato ⟨dicat antiphonam sequentem⟩.\footnote{1882:51.}}

\gregorioscore{a01485-salve-nostra}

\emph{Chorus resurgendo prosequatur.}

\gregorioscore{a01485-pax-vera}

\footnote{1517---Tunc ministri in prima processione, videlicet accolitus cum cruce et ceroferarii procedentes aliis ministris cum feretro se associent ; et statim reportata prima cruce illa, videlicet sine imagine, predictus accolitus crucem argenteam accipiat deferendam. Et tunc procedat processio usque ad locum secunde stationis. ⟨1882:51.⟩}

\emph{Deinde eat processio ad locum secunde stationis et feretrum cum capsula reliquiarum pariter cum lumine in lanternam inter subdiaconum et thuribularium deferatur cum vexillis ex utraque parte, cantore incipiente antiphonam ⟨sequentem⟩.\footnote{1882:51.}}

\gregorioscore{201244-dignus-est}

\emph{Alia antiphona.}

\gregorioscore{004107-occurrunt-turbe}

\emph{Si non sufficiunt hec due antiphone usque ad locum ij. stationis, tunc cantentur hec duo ℟. vel unum eorum cantore incipiente.}

\gregorioscore{600630-dominus-jesus}

\emph{⟨Item sequitur⟩\footnote{1882:52.} aliud ℟.}

\gregorioscore{600374-cogitaverunt}

\emph{Hic fiat secunda statio ex parte ecclesica australi~: ubi septem pueri ⟨in⟩\footnote{1882:52.} eminentiores ⟨loco⟩\footnote{1882:52.} simul cantent ⟨hanc antiphonam⟩\footnote{1882:52.} ⟨hoc modo⟩.}

\gregorioscore{008310-gloria-laus}

\emph{Chorus idem repetat post vnumquemque versum.}

\emph{Pueri vero dicant versum sequens.}

\gregorioscore{008310a-israel-es-tu}

\emph{Chorus idem ⟨repetat⟩\footnote{1882:52.}} Glória laus.

\emph{Septem pueri versus.}

\gregorioscore{008310b-cetus-in-excelsis}

\emph{Chorus idem ⟨repetat⟩\footnote{1882:52.}} Glória laus.

\emph{Septem pueri versus.}

\gregorioscore{008310c-plebs-hebrea}

\emph{¶ Item chorus repetatur} Glória laus et honór.

\emph{Peracta hac\footnote{`\emph{autem}', 1882:52.} statione eat processio per medium claustri a dextera manu usque ad hostium ecclesie occidentale, cantando ⟨hanc⟩ antiphonam ⟨sequentem⟩ cantore incipiente hoc modo.}

\gregorioscore{001852-collegerunt}

\emph{Hic fiat tertia statio ante predictum ostium ecclesie occidentale, ubi tres clerici de superiori gradu in ipso habitu non mutato\footnote{`For \emph{in ipso habitu non mutato}, read \emph{in ipso} ostio\emph{, habitu non mutato}.' 1882:xv.} conversi ad populum simul ⟨incipiant et⟩\footnote{1882:53.} cantent ⟨hunc⟩ ⟨sequentem⟩\footnote{1882:53.} ℣. ⟨hoc modo⟩ ⟨quo sequitur⟩.\footnote{1882:53.}}

\gregorioscore{001852a-unus-autem}

\emph{His finitis intrent ecclesiam per idem ostium sub feretro et capsula\footnote{`\emph{casula}', 1519:51v.} reliquiarum ex transverso ostii elevatis cantando cantore incipiente.}

\gregorioscore{006961-ingrediente}

\emph{Hic fiat quarta statio ante crucem in ecclesia et in ipsa statione executor officii incipiet antiphonam ⟨sequentem⟩.\footnote{1882:53.}}

\gregorioscore{001543-ave-rex-noster-1}

\emph{Cruce jam discooperta, et respondeat chorus cum genuflectione osculando terram.}

\gregorioscore{001543-ave-rex-noster-2}

\emph{Et sic incipiat sacerdos antiphonam ter~: singulis vicibus vocem exaltando et una cum choro fiat genuflexio et post tertiam reinceptionem chorus eadem antiphonam in statione stando totam prosequatur.}

\emph{Item sacerdos vocem exaltando hoc modo incipiat ⟨antiphonam⟩.\footnote{1882:53.}}

\gregorioscore{001543-ave-rex-noster-3}

\emph{Chorus ⟨cum genuflexione osculando terram respondeat⟩\footnote{1882:53.}} Rex noster.

\emph{⟨Item sacerdos⟩\footnote{1882:53.} se modo altissimum.\footnote{1508---semen altisisimum. 1517, 1523, \&c.---modo altissim̄. 1528, 1532---et modo altissimo. 1544, \&c.---modo altissimo, probably from MS, sē. m̄. altissim̄., \emph{i.e.}, senior modo altissimo, \emph{i.e.,} exsecutor officii, as above, p.~51, notes. ⟨1882:53.⟩ `se', 1517:52r.}}

\gregorioscore{001543-ave-rex-noster-3}

\emph{Chorus in eadem statione totam prosequatur antiphonam sic.}

\gregorioscore{001543-ave-rex-noster-4}

\emph{Qua finita intrent chorum et omnes cruces per ecclesiam sint discooperte usque post vesperas.}

\emph{In introitu chori ℟.} Circumdedérunt me. \emph{ut supra in dominica in passione ⟨Domini⟩.\footnote{1882:54.}} 94.

\emph{℣.} Eripe me de inimícis meis Deus meus.

\emph{℟.} Et ab insurgéntibus in me líbera me.

\emph{⟨℣.⟩} Orémus.

\begin{lesson}
\subsubsection[Oratio.⟩]{Oratio.⟩\footnote{1882:54.}}

\lettrine{O}{m}nípotens sempitérne Deus qui humáno géneri ad imitándum humilitátis exémplum~: Salvatórem nostrum carnem súmere, et crucem subíre fecísti~: concéde propítius, ut et patiéntie ipsíus habére documénta, et resurrectiónis consórtia mereámur. Per Christum Dóminum nostrum. \emph{⟨℟.⟩} Amen.
\end{lesson}

\chapter{¶ Feria v. in cena Domini.}

\emph{In primis\footnote{`\emph{Imprimis}', 1882:54.} fiat reconsiliatio penitentium hoc modo. ix. cantata pergat episcopus vel ejus vicarius excellentior sacerdos ad ostium ecclesie occidentale, indutus vestibus sacerdotalibus in cappa serica rubea cum duobus dyaconis cum amictibus indutis absque subdyacono et sine cruce per medium chori precedente vexillo cilicino. Sint\footnote{`\emph{Sintque}', 1882:54.} presentes in atrio ecclesie qui reconciliandi sunt, et si episcopus adest principalis archidyaconus ex parte penitentium scilicet extra ostium ecclesie in cappa serica legat hanc lectionem.} Adest tempus o veneránde. \emph{que non legatur episcopo absente.}

\begin{lesson}
\subsubsection{Lectio.}

\lettrine{A}{d}est tempus o veneránde póntifex, votívum afflíctis~: cóngruum peniténtibus, optábile tribulátis. Adsunt fílii tui pater, quos Deo per Spíritum Sanctum vera mater ecclésia cum letícia péperit. Sed íterum suadénte diábolo a sua integritáte corrúptos aut míseros factos aut éxules~: novis quotídie dolóribus ingemíscit. Pro hiis supplíciter orant quicúnque felíces in sinu suo remansérunt, quique divína protegénte se cleméntia fide stábiles prestitérunt. Parce hódie pater et tota bonitátis tue ímpetu fluat nobis fons ille David ⟨patens⟩\footnote{1882:54.} in ablutióne menstruáte~: ⟨muliéris⟩\footnote{1544, \&c.---menstruáti panni. ⟨1882:54.⟩ `muliéris' not in 1519:53r.} nulli impróperans~: nullum rejíciens nullum exclúdens. Quamvis enim a divínis supérne pietátis nichil témporis vacet~: nunc tamen lárgior est per indulgéntiam remíssio peccatórum et copiósior per grátiam assúmptio renascéntium. Accepísti a Dómino ánnulum discretiónis et honóris~: ut que signánda sunt signes, et que aperiénda sunt prodas ⟨: et que ligánda sunt liges~: et que solvénda sunt solvas⟩\footnote{1882:55.}~: atque credéntibus fidem baptismátis~: lapsis autem et peniténtibus per mystérium reconciliatiónis jánuas regni celéstis apérias. Lavant aque lavant láchryme. María flevit et suscépta est. Negávit apóstolus~: et quia flevit amaríssime restitútus est. Publicánus de thelóneo apostolátum méruit, latro in\footnote{`de', 1882:55.} cruce~: paradísum intrávit, suscéptus a patre clementíssimo fílius ejus\footnote{`illi', 1882:55.} júnior~: qui támdiu porcos pavit, qui patérne hereditátis preciósa stipéndia infelíciter dissipávit. Inde est étiam quod hódie tui súpplices póstea quam in várias formas críminum negléctum mandatórum celéstium et morum probabílium transgressióne cecidérunt humiliáti atque\footnote{`et', 1882:55.} prostráti prophética voce clamant dicéntes, Peccávimus cum pátribus nostris~: injuste\footnote{`iníque', 1882:55.} égimus iniquitátem fécimus. Manducavérunt sicut scriptum est~: panem dolóris láchrymis stratum rigavérunt, cor suum luctu, corpus suum afflixérunt jejúniis, ut animárum recíperent quam perdíderant sanitátem, plorat cum istis immo pro istis ploráre et oráre non desístit mater sancta ecclésia, et láchryme in maxíllis ejus quia venit tempus miseréndi ejus. Móveat pietátem tuam Pater vox fidélis et flébilis móveat gémitus et hábitus ipse miserórum.
\end{lesson}

\emph{¶ Finita lectione incipiat episcopus vel executor officii antiphonam hoc modo.}

\gregorioscore{205144-venite}

\emph{Scilicet infra predictum ostium conversus ad borealem signum faciendo cum manu dextra ad penitentes quasi vocando. Deinde ex parte penitentium scilicet extra ostium\footnote{`\emph{hostium}', 1519:53v.} dicat dyaconus hoc modo, quo sequitur ex parte episcopi.}

\gregorioscore{}

\emph{Alius diaconus ex parte episcopi dicat\footnote{`\emph{respondeat}', 1882:55.} sic.}

\gregorioscore{}

\emph{Et ita respondeat dyaconus. ⟨et⟩\footnote{1882:55.} Fiat tribus vicibus ita tamen quod post tertiam repetitionem antiphone non dicatur} Flectámus génua. \emph{sed chorus prosequatur totam antiphonam cantore incipiente hoc modo sequenter.\footnote{1882:55. begins `Veníte fílii audíte.'}}

\gregorioscore{}

\emph{¶ Totus psalmus dicatur sine} Glória Patri. \emph{et post unumquemque versum repetatur antiphona} Veníte. \emph{Dum psalmus canitur a toto choro cum antiphona~: semper manuatim penitentes a presbyteris\footnote{`\emph{plebesanis}', Munich, Bayerische Staatsbibliothek, Clm 3909, fo. 220r. cited in Sarah Hamilton, \emph{The Practice of Penitence, 900--1050} (Woodbridge, Suffolk~: Boydell and Brewer, 2001), p.~120. i.e.~by the people.} archidyacono et ab archidyacono reddantur episcopo~: et ab episcopo restituantur ecclesie gremio. Tamen si episcopus presens non fuerit tunc penitentes manuatim ab aliquo presbytero de choro habitu non mutato reddantur executori officii~: et ab ipso restituantur ecclesie gremio.}

\emph{Quibus expletis processio more solito in chorum redeat. Deinde prosternant se omnes clerici in choro~: et dicant septem psalmos penitentiales. Sacerdos vero cum suis ministris per se ante altare dicat predictos septem psalmos cum} Glória Patri. \emph{et} Sicut erat. \emph{et antiphonam} Ne reminiscáris. \emph{psalmi et antiphone ut supra in quarta feria in capite jejunii.} 56. \emph{⟨Qua⟩\footnote{1882:56.} finita antiphona dicitur} Kyrieléyson. Christeléyson. Kyrieléyson. Pater noster. \emph{Et hec omnia sine nota dicantur tam a sacerdote cum suis ministris quam a toto choro.}

\emph{Deinde erigat se sacerdos cum suis ministris et dicat super populum conversus ad australe\footnote{1517---ad orientem. ⟨1882:56.⟩} coram dextro cornu altaris.} Et ne nos. \emph{⟨℟.⟩} ⟨Sed líbera.⟩\footnote{1882:56.}

\emph{℣.} Salvos fac servos tuos et ancíllas tuas. \emph{℟.} Deus meus sperántes in te.

\emph{℣.} Convértere Dómine úsquequo. \emph{℟.} Et deprecábilis esto super servos tuos.

\emph{℣.} Mitte eis Dómine auxílium de sancto. \emph{℟.} Et de Syon tuére eos.

\emph{℣.} Adjuva nos Deus salutáris noster. \emph{℟.} Et propter glóriam nóminis tui Dómine líbera nos~: et propítius esto peccátis nostris propter nomen tuum.

1519:54r.

\gregorioscore{}

\begin{lesson}
\lettrine{A}{d}ésto Dómine supplicatiónibus nostris et me qui étiam misericórdia tua primus indígeo cleménter exáudi~: et quem non electióne mériti, sed dono grátie tue constituísti hujus óperis\footnote{`óperis hujus', 1882:56.} minístrum~: da fidúciam tui múneris exequéndi~: et ⟨tu⟩\footnote{1882:56.} ipse in nostro ministério quod tue pietátis\footnote{`pietátis tue', 1882:56.} est
\end{lesson}

\gregorioscore{}

\emph{Et omnes\footnote{`\emph{omnino}', 1882:57.} orationes dicantur cum} Orémus.\footnote{In Gough Missals 75 a blank space appears where `Orémus' is omitted.} \emph{et finiantur suh tono supradicto. Non dicatur} Dóminus vobscum. \emph{nisi ante primam orationem.}

\begin{lesson}
\subsubsection{Oratio.}

\lettrine{D}{e}us humáni géneris benigníssime Cónditor et misericordíssime Reformátor qui hóminem invídia diáboli ab eternitáte dejéctum~: únici Fílii tui sánguine redemísti, vivífica hos fámulos ⟨tuos⟩\footnote{1882:57.} quos tibi nullátenus mori desíderas et qui non derelínquis dévios, assúme corréctos. Móveant pietátem tuam quésumus Dómine horum famulórum tuórum lachrymósa suspíria tuórum\footnote{`Tu eórum', 1882:57.} médere vulnéribus tu jacéntibus manum pórrige salutárem ne ecclésia tua áliqua sui córporis portióne vastétur~: ne grex tuus detriméntum sustíneat~: ne de famílie tue damno inimícus exúltet~: ne renátos laváchro salutári mors secúnda possídeat. Tibi ergo Dómine súpplices preces offérimus~: tibi cordis fletum effúndimus tu parce confiténtibus~: ut sic in hac mortalitáte peccáta sua te adjuvánte défleant~: quátenus in treméndi judícii die senténtiam damnatiónis evádant~: et nésciant ⟨piíssime Pater⟩\footnote{1882:57.} quod terret in ténebris, quod stridet in flammis, atque ab erróris sui via, ad iter revérsi justície nequáquam ultra vulnéribus sauciéntur~: sed intégrum sit eis atque perpétuum, et quod grátia tua cóntulit~: et quod misericórdia tua reformávit. Per eúndem Christum Dómnum nostrum. \emph{⟨℟.⟩} Amen.

\emph{Dicatur} Orémus.

\subsubsection{Alia oratio.}

\lettrine{D}{o}mine sancte Pater omnípotens etérne Deus qui vúlnera nostra curáre dignátus es te súpplices exorámus et pétimus nos húmiles tui sacerdótes ut précibus nostris aures\footnote{`aurem', 1882:57.} tue pietátis inclináre dignéris~: atque ad peniténtiam confessióne moveáris, remittéque ómnia crímina et peccáta univérsa condónes~: desque his fámulis tuis Dómine pro supplíciis véniam, pro meróre letítiam, pro morte vitam~: ut qui ad tantam spem celéstis ápicis devolúti sunt de tua misericórdia confidéntes, ad bona pacíferi\footnote{`pacífici', 1882.57.} prémii tui atque ad celéstia dona perveníre mereántur. Per Christum Dóminum nostrum.
\end{lesson}

\emph{Hic non dicatur} ⟨Dóminus vobíscum. \emph{necque⟩\footnote{1882:57.}} Orémus. \emph{⟨sed vertat se sacerdos ad populum, et elevata manu sua dextera sine nota prosequatur in audientia hanc absolutionem sequentem.⟩\footnote{1882:57.}}

\begin{lesson}
\subsubsection{Absolutio.}

\lettrine{A}{b}sólvimus vos vice beáti Petri apostolórum príncipis cui Dóminus dedit potestátem\footnote{`cui colláta est a Dómino potéstas', 1882:58.} ligándi atque solvéndi~: et quantum ad vos pértinet accusátio~: et ad nos remíssio, sit vobis omnípotens Deus vita et salus et ómnium peccatórum vestrórum pius indúltor. Qui vivit et regnat cum Deo Patre in unitáte Spíritus Sancti Deus. Per ómnia sécula seculórum. \emph{⟨℟.⟩} Amen.
\end{lesson}

\emph{Et sic surgant omnes a prostratione osculantes terram vel formulas sacerdote dicente} Qui vivit et regnat ⟨cum Deo⟩.\footnote{1882:58.}

\emph{⟨Si episcopus non celebraverit, sacerdos celebraturus post absolutionem penitentium casulam suam ad altare autenticum induat.⟩\footnote{1882:58.} Tamen si episcopus presens fuerit fiat benedictio super populum adhuc prostratione hoc modo.}

\begin{lesson}
\lettrine{B}{e}n✠díctio Dei Patris Omnipoténtis et Fílii et Spíritus Sancti descéndat super vos et máneat semper. \emph{Chorus.} Amen.
\end{lesson}

\emph{¶ Deinde incipiatur missa sollenniter a cantore incipiente sine regimine chori.}

\emph{Officium.} Nos autem ⟨gloriári⟩.\footnote{1882:58.} Missal:641. \emph{Et cum sollenne} Kýrieléyson. \emph{videlicet} Cónditor Kýrie.\footnote{Cónditor, Kýrie, ómnium ymas creaturárum eléison.

  Tu nostra delens crímina, nobis inceesánter eléison.

  Ne sinas períre factúram tuam, sed clemens et eléison.

  Christe, Patris únice, natus de vírgine, nobis eléison.

  Mundum pérditum qui tuo sánguine salvásti de morte, eléison.

  Ad te nunc clamántum preces exáudiens, pius eléison.

  Spíritus alme, tua nos reple grátia, eléison.

  A Patre et Natu qui mansa júgiter, nobis eléison.

  Trínitas una, trina Unitas, simul adoránda,

  Nostrórum scélerum vincla resólve rédimens a morte.

  Onmes proclamémus nunc voce dulcíflua, Deus, eléison.

  (See reprint of the Hereford Missal, p.~xxxviii.--xlii. for this and other Kyries\emph{.}) ⟨1882:58.⟩} \emph{absque ℣.} Missal:18*. \emph{sive episcopus celebraverit sive non.}

\emph{Tres pueri ante olei consecrationem in superpelliceis ad gradum chori simul cantent hunc hymnum.}

\gregorioscore{830247-o-redemptor-1}

\emph{Chorus idem repetat post unumquemque versum. Clerici prosequantur.}

\gregorioscore{830247-o-redemptor-2}

\emph{⟨Hic pergat episcopus ad altare. Chorus interim repetat} O redémptor.⟩\footnote{1882:59.}

\gregorioscore{830247-o-redemptor-3}

\emph{⟨Hac die post} Per ómnia sécula seculórum. \emph{et post} Pax Dómini. \emph{dicta} Et cum spíritu tuo. \emph{more solito, non dicatur} Agnus Dei. \emph{nec} Pax Dei. \emph{nisi episcopus cebraverit~: tunc enim dicitur} Agnus Dei. \emph{sollenniter, sicut in festis minoribus duplicibus~: et sacerdos chrismatis ampullam pro pace deosculetur, et diaconus et subdiaconus dalmatica et tunica induantur propter sollennitatem cene, licet episcopus celebraverit.⟩\footnote{1882:59.}}

¶ \emph{⟨Item⟩ post prandium ⟨omnes sacerdotes atque clerici⟩ conveniant ad ecclesiam clerici ad altaria abluenda, et ad mandatum faciendum et ad completorium dicendum.}

\emph{⟨Item⟩ in primis\footnote{`\emph{Imprimis}', 1882:59.} benedicatur aqua more dominicali~: extra chorum privatim scilicet in vestibulo ante altare. Deinde preparantur duo sacerdotes de excellentioribus cum dyaconis et subdyaconis\footnote{1528, \&c.---cum duobus diaconis et subdiaconis. ⟨1882:59.⟩} de secunda forma et ceroferariis de prima forma, qui omnes sint albis cum amictibus indutis ⟨absque paruris⟩,\footnote{1882:59.} et duo clerici\footnote{`\emph{et cum duobus pueris in simili habitu}', 1882:59.} vinum et aquam deferentes ⟨majoribus precedentibus et a majore altari⟩\footnote{1882:59.} incipientes abluant illud infundentes vinum et aquam ⟨super cruces in medio et in cornibus ejusdem. Postea ministri infundant aquam⟩\footnote{1882:60.} et interim cantetur responsorium sequens hoc modo.}

\emph{⟨Si episcopus presens fuerit mitram gerat et baculum in lotione altarium et similiter ad mandatum, lotisque pedibus canonicorum, postquam pedes laverint, executor ipse episcopus ultimo lavetur.⟩\footnote{1882:60.}}

\gregorioscore{006916-in-monte-oliveti}

\emph{Et dicatur a toto choro coram altari cum suo ℣. a cantore incipiente hoc modo.}

\gregorioscore{006916a-veruntamen}

\emph{Et dicitur sine\footnote{In Gough Missals 75, fol.~57r. `sine' is added by hand in the margin.}} Glória Patri. \emph{quo finito dicatur ℣. et oratio de sancto in cujus honore consecratum est altare~: ab excellentiori sacerdote modesta voce id est sine nota que terminetur sic,} Per Christum Dóminum ⟨nostrum⟩.\footnote{1882:60.} \emph{nec precedat, nec sequatur\footnote{`\emph{subsequatur}', 1882:60.}} Dóminus vobíscum. \emph{sed tantum ⟨cum⟩\footnote{1882:60.}} Orémus. \emph{ante orationem ⟨et postea deosculentur altaria executores officii et alii clerici consequantur priusquam recedant.⟩\footnote{1882:60.} Eodem modo omnia altaria ecclesie abluantur cum responsoriis ⟨et versibus suis⟩\footnote{1882:60.} sicut ordinati sunt in ⟨predicta⟩\footnote{1882:60.} hystoria scilicet} In monte Olivéti. \emph{et cum ℣. et orationibus de sanctis ut supradictum est~: ita ⟨tamen⟩\footnote{1882:60.} quod nullum ℟. incipiatur nisi tantum coram altari ubi totum cantentur ⟨ut supra diximus et semper⟩\footnote{1882:60.}~: et si necesse fuerit reincipiatur hystoria. Ita in extrema ablutione cantetur ℟.} Circundedérunt me. \emph{cum suis ℣. sine} Glória Patri. 94. \emph{Finita ablutione altaria nuda usque ad sabbatum pasche~: altaria conserventur\footnote{`\emph{conversentur}', 1519:57r.} preter principale altare. Si vero plura fuerint altaria in ecclesia quam ℟. et predicta hystoria tunc reincipiatur hystoria ordine predicto servatur itaque\footnote{`\emph{scilicet quod}', 1882:60.} ℟.} Circundedérunt me. \emph{semper in ultimo loco\footnote{`\emph{super ultimo loco}', 1519:57r.} cantetur.} 94.

\emph{⟨Ad altare sanctissime Trinitatis cantetur hoc responsorium sequens.⟩\footnote{1882:60.}}

\gregorioscore{007780-tristis-est}

\emph{⟨℣.} Sit nomen Dómini benedíctum.

\emph{℟.} Ex hoc nunc et usque in séculum.

\emph{⟨℣.⟩} Orémus.

\begin{lesson}
\subsubsection{Oratio.}

\lettrine{O}{m}nípotens sempitérne Deus, qui dedísti fámulis tuis in confessióne vere fidéi etérne Trinitátis glóriam agnóscere, et in poténtia majestátis tue adoráre unitátem~: quésumus ut ejúsdem fidéi firmitátc ab ómnibus semper muniámur advérsis, In qua vivis et regnas Deus. Per ómnia sécula seculórum. \emph{Chorus respondeat} Amen.
\end{lesson}

\emph{Ad altare sancti Michaelis archangeli ceterorumque angelorum cantetur hoc responsorium sequens.⟩\footnote{1882:61.}}

\emph{Aliud responsorium.}

\gregorioscore{006618-ecce-vidimus}

\emph{⟨℣.} In conspéctu angelórum psallam tibi Dens meus.

\emph{℟.} Adorábo ad templum sanctum tuum et confitébor nómini tuo.

\emph{⟨℣.⟩} Orémus.

\begin{lesson}
\subsubsection{Oratio.}

\lettrine{D}{e}us, qui miro órdine angelórum mystéria hominúmque dispénsas, concéde propítius, ut quibus tibi ministrántibus in celo semper assístitur, ab his in terra vita nostra muniátur. Per.
\end{lesson}

\emph{Ad altare sanctorum apostolorum cantetur sequens responsorium.⟩\footnote{1882:61.}}

\gregorioscore{007809-unus-ex}

\emph{⟨℣.} In omnem terram exívit sonus eórum.

\emph{℟.} Et in fines. 315.

\emph{⟨℣.⟩} Orémus.

\begin{lesson}
\subsubsection{Oratio.}

\lettrine{Q}{u}ésumus omnípotens Deus ut beáti apóstoli tui Petrus et Paulus cum sóciis suis pro nobis implórent auxílium~: ut a nostris reátibus absolúti a cunctis étiam perículis eruámur. Per Christum Dóminum nostrum. Amen.
\end{lesson}

\emph{Ad altare sanctorum martyrum cantatur hoc responsorium sequens modo, quo sequitur.⟩\footnote{1882:61.}}

\gregorioscore{007041-judas-mercator}

\emph{⟨℣.} Letámini in Dómino\footnote{1508:unpaged. 1882:62.has `Deo.'} et exultáte justi.

\emph{℟.} Et gloriámini. 136.

\emph{⟨℣.⟩} Orémus.

\begin{lesson}
\subsubsection{Oratio.}

\lettrine{C}{o}ncéde quésumus omnípotens Deus ut qui sanctórum mártyrum tuórum Johánnis baptíste et Thome mártyris sociorúmque éorum recólimus victórias~: participémur et prémiis. Per Christum Dóminum nostrum. \emph{⟨℟.⟩} Amen.
\end{lesson}

\emph{Ad altare sanctorum confessorum cantetur istud responsorium.⟩\footnote{1882:62.}}

\gregorioscore{007807-una-hora}

\emph{⟨℣.} Letámini in Dómino et exultáte justi.

\emph{℟.} Et gloriámini omnes recti corde.

\emph{⟨℣.⟩} Orémus.

\begin{lesson}
\subsubsection{Oratio.}

\lettrine{D}{e}us qui nos sanctórum confessórum tuórum Nícolai et Edmúndi sociorúmque eórum confessiónibus gloriósis circúmdas et protégis, da nobis eórum imitatióne profícere et intercessióne gaudére. Per Christum Dóminum nostrum. Amen.
\end{lesson}

\emph{Ad altare sanctarum virginum cantetur sequens responsorium.⟩\footnote{1882:62.}}

\gregorioscore{007636-seniores-populi}

\emph{⟨℣.} Adducéntur regi vírgines.

\emph{℟.} Próxime ejus afferéntur tibi.

\emph{⟨℣.⟩} Orémus.

\begin{lesson}
\subsubsection{Oratio.}

\lettrine{I}{n}dulgéntiam nobis, Dómine, quésumus, beáte vírginas tue Katharína et Margaréta sociéque eárum semper implórent, que tibi grátie extitérunt et mérito caatitátis et tue confessióne virtútis. Per Christum Dóminum nostrum.
\end{lesson}

\emph{Ad superaltare in vestibulo resp}. Circundedérunt. \emph{Require ut supra in dominica in passione Domini.} 94.

\emph{℣.} Letámini in Dómino.

\emph{℟.} Et gloriámini. 136.

\emph{⟨℣.} Orémus.⟩

\begin{lesson}
\subsubsection{Oratio.}

\lettrine{C}{o}ncéde, quésumus, omnípotens Deus, ut intercéssio Dei genitrícis Maríe sanctarúmque ómnium celéstium virtútum et beatórum patriarchárum, prophetárum, apostolórum, evangelistárum, mártyrum, confessórum atque vírginum, et ómnium electórum tuórum nos ubíque letíficet, ut dum eórum mérita recólimus, patrocínia sentiámus. Per eúndem Dóminum nostrum. \emph{⟨℟.⟩} Amen.
\end{lesson}

\emph{Finita ablutione altarium, nuda usque ad sabbatum pasche reserventur.⟩\footnote{1882:63.}}

\emph{⟨Aliud responsorium.⟩}

\gregorioscore{007272-o-juda-qui}

\emph{⟨Sequitur⟩\footnote{1882:63.} aliud responsorium.}

\gregorioscore{007543-revelabunt-celi}

\emph{Post ablutionem altarium intrent capitulum ⟨et⟩\footnote{1882:63.} ibidem dyaconus legat evangelium sicut ad missam lectum est ⟨videlicet⟩.\footnote{1882:63.}}

\begin{lesson}
\subsubsection{Secundum Johannem. xiij. ⟨1--15.⟩}

\lettrine{A}{n}te diem festum pascbe, sciens Jesus quia venit hora ejus ut tránseat ex hoc mundo ad Patrem~: cum dilexísset suos qui erant in mundo~: in finem diléxit eos. Et cena facta, cum dyábolus jam se\footnote{`se' is not in \emph{Vulgate.}} misísset in cor ut tráderet eum Judas Symónis Scarióthis~: sciens quia dedit ei ómnia\footnote{`ómnia dedit ei', \emph{Vulgate.}} Pater in manus et quia a Deo exívit, et ad Deum vadit~: surgit a cena, et ponit vestiménta sua. Et cum accepísset líntheum~: precínxit se. Deinde misit aquam in pelvim~: et cepit laváre pedes discipulórum\footnote{`discipulórum suórum', 1882:64.}~: et extérgere líntheo quo erat precínctus. Venit ergo ad Symónem Petrum~: et dixit ei Petrus, Dómine~: tu michi lavas pedes~? Respóndit Jesus et dixit ei, Quod ego fácio tu nescis modo~: scies autem póstea. Dixit ei Petrus, Non lavábis michi pedes in etérnum. Respóndit ei Jesus, Si non lávero te~: non habébis partem mecum. Dixit ei Symon Petrus, Dómine non tantum pedes ⟨meos⟩\footnote{1882:64., \emph{Vulgate.}}~: sed ⟨et⟩\footnote{1882:64., \emph{Vulgate.}} manus et caput. Dixit ei Jesus, Qui lotus est non índiget nisi ut pedes lavet~: sed est mundus totus. Et vos mundi estis~: sed non omnes. Sciébat enim Jesus,\footnote{`Jesue' is not in \emph{Vulgate.}} quisnam esset qui tráderet eum~: proptérea dixit non estis mundi omnes. Postquam ergo lavit pedes eórum ⟨et⟩\footnote{\emph{Vulgate.}} accépit vestiménta sua~: et cum recubuísset íterum dixit eis, Scitis quid fécerim vobis~? Vos vocátis me Magíster et Dómine~: et benedícitis.\footnote{`bene dícitis', \emph{Vulgate.}} Sum étenim. Si ergo ⟨ego⟩\footnote{1882:64., \emph{Vulgate.}} lavi pedes vestros Dóminus et Magíster~: et vos debétis alter altérius pedes laváre. Exémplum enim dedi vobis~: ut quemádmodum ego feci vobis~: ita et vos faciátis.
\end{lesson}

\emph{¶ Deinde fiat sermo ad populum. Quo finito surgant duo predicti presbyteri~: et incipient a majoribus et unus lavet omnium pedes ex una parte, et alius ex altera parte~: et postmodum sibi invicem pedes lavent~: et interim cantentur hee sequens antiphone cum suis psalmis a toto choro interim cantent sedentes hanc antiphonam ut sequitur.\footnote{In 1519:61r. the following is added by hand `¶ Ablutio pedum.'}}

\gregorioscore{003688-mandatum-novum}

\emph{¶ Totus psalmus dicatur sine} Glória Patri. \emph{et post unumquemque versum psalmi, repetatur antiphona. Eodem modo dicantur omnes antiphone sequentes cum suis psalmis.}

Rylands-24:184; 1519:61r; GS:96.

\gregorioscore{}

\emph{⟨Cantetur psalmus usque in finem.}

\emph{Item alia ant.⟩\footnote{1882:65.}}

Rylands-24:184; GS:96; 1519:61v; AS:456; Ant-1520-S:43v; Brev-1531-S:87v.\footnote{In 1520-S:43v. the final `ejus' is set CDCCBB.BA. GS:96. has `retro sedens secus'; a single G is provided for `sedens.' In GS:96. `capíllis' is set FG.GFGE.D.}

\gregorioscore{}

\emph{⟨Et cantetur psalmus usque in finem.}

\emph{Item⟩\footnote{1882:65.} alia ant.}

Rylands-24:184; GS:96; 1519:62r.\footnote{1519:62r. has no flats. The flats appear in 1528:57v. and in Rylands-24:184. In Rylands-24:184. `immaculáti' is set A.A.A.B♭.A.}

\gregorioscore{}

\emph{⟨Et cantetur usque ad finem psalmi.}

\emph{Alia ant.⟩\footnote{1882:65.}}

Rylands-24:184; GS:97; 1519:62r.\footnote{Rylands-24:184. has no flat. In GS:97. and Rylands-24:184. the psalm is set to psalm-tone IV.i:

  \gregorioscore{}

  It may be that this earlier practice was at some time replaced by the officium tone instead.}

\gregorioscore{}

\emph{⟨He sequentes antiphone si necesse fuerit cantentur.}

Rylands-24:185; GS:97; 1555:unpaged.\footnote{In GS:97. `debétis' is set D.DE.D; `altérius' is set F.E.D.DED.}

\gregorioscore{}

\emph{Sequitur alia antiphona.}

Rylands-24:159; 1555:unpaged.\footnote{In 1555. `magis vos' is set CDCA.CB B. 1555. has no flats, and would therefore be in mode I transposed.}

\gregorioscore{}

\emph{Item alia antiphona.}

Rylands-24:159; GS:97; 1555:unpaged.\footnote{1555. has flats only at `hora' and at the beginning of `discípulorum.' GS has flats at `hora', `ad', and throughout `discipulórum.' The edition follows the flats as found in Rylands-24:159. In 1555. `diem' is set DGGG.GA; `sciens' is set CGACA.CG; `venit' is set A.AFGGFFFC. In Ry lands-24:159. `ut' is set DEF. In 1555. `mundo ad patrem' is set GA.GAGFFG ABGA EGGE.E; `surréxit' is set DEFGA.EGGE.E. In 1555. `líntheo' is set DFG.G.GFGGD. In 1555. the fourth syllable of `discipulórum' appears one note later.

  GS:97. has an additional A before the second last GF of the long melisma.}

\gregorioscore{}

\emph{Sequitur alia antiphona.}

Rylands-24:159; GS:97; 1555:unpaged.\footnote{1555. has `dixit ei.' 1555. begins thus:

  \gregorioscore{}

  In GS:97. `ei' appears to be set AC.ABCBA. In 1555. `pondens' is set B.CDEDE. The flat appears in Rylands-24:159. In 1555. `Dómine' is set C.C.ABCCBDCDBBG. GS:97. has `solum pedes sed.' In Rylands-24:159. the final melisma is as follows:

  \gregorioscore{}

  In 1555. the final melisma is as follows:

  \gregorioscore{}

  The edition follows GS:98. here.}

\gregorioscore{}⟩\footnote{1882:65.}

\emph{¶ Peracta ablutione pedum dictoque sermone accipiant potum charitatis, quibus rite peractis unus sacerdos dicat has preces.}

\emph{℣.} Suscépimus Deus misericórdiam tuam. \emph{℟.} In médio templi tui.

\emph{℣.} Tu mandásti. \emph{℟.} Mandáta tua custodíri nimis.

\emph{℣.} Ecce quam bonum est quam jocúndum. \emph{℟.} Habitáre fratres in unum.

\emph{℣.} Dómine exáudi oratiónem meam. \emph{℟.} Et clamor meus ad te véniat.

\emph{⟨℣.⟩} Dóminus vobíscum. \emph{⟨℟.⟩} Et cum spíritu tuo.

\emph{⟨℣.⟩} Orémus.

\begin{lesson}
\subsubsection{Oratio.}

\lettrine{A}{d}ésto quésumus Dómine offício servitútis nostre~: et quia tu pedes laváre dignátus es tuis discípulis~: ne despícias ópera mánuum tuárum que nobis retinénda mandásti~: et\footnote{`sed', 1882:66.} sicut hec exterióra\footnote{`exterióra hic', 1882:66.} abluúntur inquinaménta córporum sic a te ómnium nostrum interióra mundéntur peccáta~: quod ipse prestáre dignéris. Qui cum Patre et Spíritu Sancto vivis.
\end{lesson}

\emph{Si vero nullus assit clericus sermonem faciens ceteris ut ⟨prius⟩\footnote{1882:66.} peractis post ablutionem pedum dicantur preces predicte. Deinde legatur evangelium secundum Johannem ⟨scilicet⟩\footnote{1882:66.}} Amen amen dico vobis. \emph{sub tono lectionis a quodam dyacono in superpelliciis, fratribus interim caritatis potum\footnote{`\emph{potum caritatis}', 1882:66.} sumentibus et legatur ⟨usque ad illum locum⟩\footnote{1882:66.}} Súrgite eámus hinc.

\begin{lesson}
\subsubsection[⟨Evangelium⟩ secundum Johannem. ⟨xiij. 16. et xiv. capitulis.⟩]{⟨Evangelium⟩\footnote{1882:66.} secundum Johannem. ⟨xiij. 16. et xiv. capitulis.⟩\footnote{1882:66.}}

\lettrine{A}{m}en amen dico vobis~: non est servus major dómino suo~: neque apóstolus major eo qui misit illum. Si hec scitis~: beáti éritis si fecéritis ea. Non de ómnibus vobis dico. Ego scio quos elégerim.\footnote{`elégeri', 1519:62v.} Sed ut adimpleátur scriptúra, Qui mandúcat mecum\footnote{`meum', 1519:63r.} panem~: levábit contra me calcáneum suum. Amen amen dico\footnote{`Amodo dico vobis', 1555, \emph{Vulgate.}} vobis~: priúsquam fiat ut cum factum fúerit~: credátis quia ego sum.\footnote{`priúsquam fiat ut credátis cum factum fúerit~: quia ego sum.' 1519:63r.} Amen amen dico vobis~: quia qui áccipit si quem mísero, me áccipit. Qui autem me áccipit~: áccipit eum qui me misit. ⟨Et⟩\footnote{1519:63r.} cum hec dixísset Jesus turbátus est spíritu~: et protestátus est et dixit, Amen amen dico vobis~: quia unus ex vobis me tradet. Aspiciébant ergo discípuli ad ínvicem\footnote{`ad ínvicem discípuli', \emph{Vulgate}.}~: hesitántes de quo díceret. Erat autem\footnote{`ergo', \emph{Vulgate}.} recúmbens unus ex discípulis ejus in sinu Jesu~: quem diligébat Jesus. Innuit ergo discípulo\footnote{\emph{Vulgate} omits `discípulo.'} huic Symon Petrus~: et dixit ei, Quis est de quo dixit~? Itaque cum recubuísset ille supra pectus Jesu~: dixit ei, Dómine, quis est~? Respóndit Jesus, Ille est~: cui ego intínctum panem porréxero~: et cum intinxísset panem, dedit Jude Symónis Scarióthis. Et post bucéllam~: introívit in illum Sáthanas. Et dixit ei Jesus, Quod facis~: fac cítius. Hoc autem nemo scivit discumbéntium~: ad quid díxerit ei. Quidam autem putábant quia lóculos habébat Judas~: quod dixísset ei Jesus, Eme ea que opus sunt nobis\footnote{`nobis opus sunt', 1882:67.} ad diem festum~: aut egénis ut áliquid daret. Cum ergo accepísset ille bucéllam~: exívit contínuo. Erat autem nox. Cum ergo exísset dixit Jesus, Nunc clarificátus est Fílius Hóminis~: et Deus clarificávit\footnote{`clarificátus', 1519:63r.} est in eo. Si Deus clarificátus est in eo~: et Deus clarificávit eum in semetípso~: et contínuo clarificábit eum. Filióli~Gdhuc módicum vobíscum sum. Quéretis me~: et sicut dixi Judéis quo ego vado~: vos non potéstis veníre. Et vobis dico modo. Mandátum novum do vobis~: ut diligátis ínvicem, sicut ego\footnote{\emph{Vulgate} omits `ego'\emph{.}} diléxi vos ⟨ut et vos diligátis ínvicem⟩.\footnote{1882:67., \emph{Vulgate.}} In hoc agnóscent\footnote{`cognóscent', \emph{Vulgate}.} omnes quia discípuli mei estis~: si dilectiónem habuéritis ad ínvicem. Dixit ei Symon Petrus, Dómine~: quo vadis~? Respóndit Jesus, Quo ego vado non potes me modo sequi~: sequéris autem póstea. Dixit ei Petrus, Quare non possum te sequi modo~? Animam meam pro te ponam. Respóndit ⟨ei⟩\footnote{1882:67., \emph{Vulgate.}} Jesus, Animam tuam pro me pones~? Amen amen dico tibi~: non cantábit gallus, donec ter negábis me. Et ait discípulis suis,\footnote{`Et ait discípulis suis', not in \emph{Vulgate.}} Non turbétur cor vestrum neque fórmidet. Créditis in Deum~: et in me crédite. In domo Patris mei mansiónes multe sunt. Si quo minus\footnote{`quóminus', \emph{Vulgate}.} dixíssem vobis~: quia vado paráre vobis locum. Et si abíero et preparávero vobis locum~: íterum véniam et accípiam vos ad meípsum~: ut ubi ego sum, et vos sitis. Et quo ego vado scitis~: et viam scitis. Dixit ei Thomas, Dómine~: néscimus quo vadis. Et quómodo póssumus viam scire~? Dixit ei Jesus, Ego sum via, véritas, et vita. Nemo venit ad Patrem nisi per me. Si cognovissétis me~: et Patrem meum útique cognovissétis. Et ámodo cognoscétis eum~: et vidístis eum. Dixit ei Philíppus, Dómine~: osténde nobis Patrem~: et súfficit nobis. Dixit ei Jesus, Tanto témpore vobícum sum~: et non cognovítis me~? Philíppe qui videt me~: videt et Patrem ⟨meum⟩.\footnote{1882:68.} Et quómodo tu dicis, Osténde nobis Patrem~? Non credis quia ego in Patre~: et Pater in me est~? Verba que ego loquor vobis~: a meípso non loquor. Pater autem in me manens ipse facit ópera. Non créditis quia ego in Patre~: et Pater in me est~? Alíoquin propter ópera ipsa crédite. Amen amen dico vobis~: qui credit in me ópera que ego fácio~: et ipse fáciet. Et majóra horum fáciet~: quia ego ad Patrem vado. Et quocúmque petiéritis Patrem in nómine meo hoc fáciam~: ut glorificétur Pater in Fílio. Si quid petiéritis Patrem in nómine meo hoc faciam. Si dilígitis me~: mandáta mea serváte. Et ego rogábo Patrem, et álium Paráclytum dabit vobis~: ut máneat vobíscum in etérnum. Spíritum veritátis quem mundus non potest cápere~: quia non videt eum, nec scit eum. Vos autem cognoscétis eum~: quia apud vos manébit, et in vobis erit. Non relínquam vos órphanos~: véniam ad vos. Adhuc módicum~: et mundus me jam non vidébit.\footnote{`videt', \emph{Vulgate}.} Vos autem vidébitis me quia ego vivo~: et vos vivétis. In illo die vos cognoscétis quia ego sum in Patre meo~: et vos in me, et ego in vobis. Qui habet mandáta mea et servat ea~: ille est qui díligit me. Qui autem díligit me~: diligétur a Patre meo. Et ego díligam illum\footnote{`eum', \emph{Vulgate}.}~: et manifestábo ei meípsum. Dixit\footnote{`Dicit', \emph{Vulgate}.} ei Judas non ille Scarióthis, Dómine~: quid factum est quia manifestatúrus es nobis teípsum~: et non mundo~? Respóndit Jesus~: et dixit ei, Si quis díligit me~: sermónem meum servábit. Et Pater meus díliget eum~: et ad eum veniémus, et mansiónem apud eum faciémus. Qui non díligit me~: sermónes meos non servat. Et sermónem quem audístis, non est meus~: sed ejus qui misit me Patris. Hec locútus sum vobis~: apud vos manens. Paráclytus autem Spíritus Sanctus quem mittet Pater in nómine meo~: ille vos docébit ómnia, et súggeret vobis ómnia quecúmque díxero vobis. Pacem relínquo vobis~: pacem meam do vobis. Non quómodo mundus dat ego do vobis. Non turbétur cor vestrum neque fórmidet. Audístis quia ⟨ego⟩\footnote{\emph{Vulgate.}} dixi vobis, Vado et vénio ad vos. Si diligerétis me gauderétis útique~: quia vado ad Patrem~: quia Pater major me est. Et nunc dixi vobis priúsquam fiat~: ut cum factum fúerit credátis. Jam non multa loquar vobíscum. Venit enim princeps mundi hujus~: et in me non habet quicquam. Sed ut cognóscat mundus quia díligo Patrem~: et sicut mandátum dedit michi Pater, sic fácio. Súrgite, eámus hinc.
\end{lesson}

\emph{Et ita recedant euntes in ecclesiam~: et ibidem dicant completorium privatim.}

\chapter{¶ Feria sexta in die parasceves.}

\emph{Dicta hora nona accedat sacerdos ad altare indutus vestibus sacerdotalibus, et casula rubea cum dyacono et subdyacono, et ceteris ministris altaris~: qui omnes sint albis cum amictibus tantum sine paruris ⟨induti⟩\footnote{1882:69.} \&c.\footnote{`(\emph{i.e., ut in missali continetur})', 1882:69.}}

\emph{Finitis orationibus, exuat sacerdos casulam post passionem ⟨et illam super altare dimittat⟩\footnote{1517 ⟨1882:69.⟩} : et in sede sua juxta altare se ponat\footnote{`\emph{reponat}', 1882:69.} cum dyacono et subdyacono~: et interim alii duo presbyteri de superiori gradu nudis pedibus albis indutis absque paruris tenentes crucem coopertam super eorum brachia sollemniter inter eos retro magnum altare ex dextera parte retro dextrum cornu stantes cantent hunc versum.}

Grad-1508:100r; Rylands-24:192; GS:101; 1519:64v.\footnote{The flat at the final `Agyos' appears in Rylands-24:192. and Arsenal:87v. In Arsenal:87v. the third `Sanctus' is set GA.GAGF. Arsenal:87v. has a flat at `immortális.' Rylands-24:192. and GS:101. have `satis', not `nimis.'}

\gregorioscore{}

¶ \emph{⟨Finito versu,⟩ duo dyaconi de secunda forma ⟨indutis⟩ in cappis nigris stantes ad gradum ⟨chori⟩\footnote{1555, unpaged.} conversi ad altare simul dicant sic.\footnote{`\emph{habitu non mutato, sparsim stantes simul dicant}', 1882:69. 1517---Duo diaconi de ii. forma stantes ad gradum chori conversi ad altare simul dicant. ⟨1882:69.⟩ `\emph{simul incipiant hoc modo sequentur}', 1523:64v.}}

\gregorioscore{}

\emph{Chorus respondeat cum genuflexione osculando tribus vicibus in una responsione et resurgat cum canitur} Deus. \emph{Iterum genuflectat cum dicitur} Sanctus. \emph{et resurgat cum dicatur} Fortis. \emph{Tertio chorus genuflectat cum dicatur} Sanctus. \emph{et resurgat cum} Et immortális miserére ⟨nobis⟩.\footnote{1882:69.}

\gregorioscore{}

\emph{Hoc est ad quemlibet} Sanctus. \emph{genuflectere debet chorus. Sacerdotes vero tenentes crucem retro magnum altare et dyaconi ad gradum chori cantantes} Agyos. \emph{semper sint stantes}.\footnote{`\emph{sparsim}', 1882:70. 1544, \&c,---semper sint. ⟨1882:70.⟩}

\emph{Item sacerdotes loco non mutato simul cantent hanc antiphonam sequentem.\footnote{`\emph{modo quo sequitur}.' 1882:70. In 1882:70 `Quia edúxi' is labelled `\emph{Ant.}'}}

\gregorioscore{}

\emph{Dyaconi.} Agyos. \emph{Chorus.} Sanctus Deus.

\emph{Sacerdotes loco non mutato dicat.}

\gregorioscore{}

\emph{Dyaconi} Agyos. \emph{Chorus} Sanctus Deus. \emph{\&c.}

\emph{Deinde sacerdotes stantes juxta altare ex dextra parte discooperientes crucem simul cantent hanc sequentem antiphonam hoc modo.}

Grad-1508:100v; Rylands-24:193; GS:102; 1519:65v.\footnote{1519:65v. has no flat at `salus.' 1528:61r. has both flats. In Pro-1519:65v. `lignum' is set DEFE.EDC. In Rylands-24:193. `adorémus' is set FFE.G.GAGF.GF.}

\gregorioscore{}

\emph{Chorus cum genuflectione osculando terram vel formulas ⟨cantent antiphonam⟩.\footnote{1882:70.}}

Grad-1508--100v; Rylands-24:193; GS:102; 1519:66r.\footnote{In 1519:66r. this is identified as a Responsorium. In GS:102 `univérso' is set G.GFG.GAG.}

\gregorioscore{}

\emph{Et repetatur antiphona post unumquemque versum istius psalmi.}

Ut cognoscámus in terra viam tuam~: in ómnibus géntibus salutáre tuum.

Confiteántur tibi pópuli Deus~: confiteántur tibi pópuli omnes.

Leténtur et exúltent gentes quóniam júdicas pópulos in equitáte~: et gentes in terra dírigis.

Confiteántur tibi pópuli Deus~: confiteántur tibi pópuli omnes terra dedit fructum suum.

Benedícat nos Deus Deus noster benedícat nos Deus~: et métuant eum omnes fines terre.

\emph{⟨Totus psalmus dicatur⟩\footnote{1882:71.} sine} Glória Patri. \emph{a toto choro cum genuflexione continue et interim deportetur crux sollenniter super gradum ⟨tertium⟩\footnote{1882:71.} altaris juxta quam sedeant predicti ⟨duo⟩\footnote{1882:71.} sacerdotes, unus a dextris et alius a sinistris. Deinde procedant clerici ad crucem adorandam nudis\footnote{`\emph{nudatis}', 1882:71.} pedibus incipientes a majoribus. Finito psalmo cum antiphona cantetur hymnus a predictis ⟨duobus⟩\footnote{1882:71.} sacerdotibus interim sedentibus juxta crucem hoc modo hymnus sequens.}

Rylands-24:193; GS:102; Grad-1508:101r; 1519:66v.\footnote{In GS:102; Grad-1508:101r. and 1519:66v. the second line of each verse ends EF.E.D. The edition follows Rylands-24:193. here. In GS:102. `Dulce pondus sustinet.' is set D.DG GFF.DC DE.D.C. In Rylands-24:193. `Dulce' is set D.GA. In GS:102. only the first verse is set to music. In Rylands-24:193. `ligni ut sólveret' is set AAG.ED E F.E.D. In Verse 4. Grad-1508:101r. has `Ut medélam.' In Verse 5. Grad-1508:101r. sets `Quando' D.E. GS:102. includes `Amen.' after the final Verse. The Verses `Flecte ramos.' and `Sola digna.' indicated in Missal-1513:79r. do not appear in the noted sources. Their omission would suggest that they were not always included in the liturgy. London, British Library, MS Harley 3866. includes the texts of these verses.}

\gregorioscore{} \gregorioscore{}

\emph{Chorus idem repetat post unumquemque versum, interim sedendo cantent versum hoc modo.\footnote{`\emph{interim sedentes sacerdotes cantant versum}', 1882:71. 1517---Chorus idem repetat post unumquemque versum interim stando~: sacerdotes dicant versum. ⟨1882:71.⟩}}

\gregorioscore{}

\emph{His finitis deportetur crux per medium chori solenniter a duobus predictis sacerdotibus precedente ceroferario et reponatur ante aliquod altare ubi a populo adoretur, et interim cantetur hec sequens antiphona cum suo versu a tota choro interim sedente\footnote{1517---stando. ⟨1882:72.⟩} cantore incipiente ⟨hanc⟩\footnote{1882:72.} antiphonam ⟨sequentem⟩.\footnote{1882:72.}}

Rylands-24:193; GS:103; Grad-1508:101v; 1519:68r.\footnote{Grad-1508:101v. does not indicate any flats. GS:103. indicates flats only at `voce magna trádidit.' The other flats are found in Rylands-24:193. The flats also appear in Pro-1544:79v. In GS:103. `scissum' is set CDDCA.CD. In Rylands-24:193. the first `est' is set ED. In Grad-1508:101v. `enim' is set AGABAGAEFGF.FEFGFFE; `fúerat' is set FGAGF.FE.GFEFEDED; `ergo' is set GACC.BA; `sanguis' appears as AG.ABAAA; `aqua' is set EGG.GAD; `redemptiónem' is set D.E.C.GA.GAEGFEFEDED; the last note of the first syllable of `nostre' is E, not D. Pro-1544:80r. indicates `℣.' at `Apérto.' In Rylands-24:193. `non posse' is set Fe GA.G. In Pro-1519:68v. `aqua' is set DGG.GAD. In Grad-1508:102r. and Pro-1519:68v. `redemptiónem' appears to be set D.E.F.GA.AGEGFEFEDED. In Pro-1519:68v. `nostre' ends FFFEECEFD.D. In Rylands-24:193. `O' is set DFD. In GS:103. `póndere' is set ADCCDGa.ABAG.G. In Pro-1519:68v. Tartárea' is set FD.F.GF.F. In Pro-1915:68v. and Grad-1508:102r. `inférni' is set a third higher. GS:103. does not indicate the repetition.}

\gregorioscore{}

\emph{Revertendo ceroferariis ad vesbibulum precedant sacerdotes in superpelliceis discalceatis sine almico\footnote{Presumably and Amice `\emph{amictus}' or Almuce, `\emph{almucia}.'} deferentes Corpus Christi in pixide ad altare authenticum dum predicta antiphona cantatur. Adorata cruce et finita predicta antiphona cum suo ℣. predicti duo sacerdotes ea reverentia qua crucem asportaverunt usque ad summum altare iterum reportent~: tunc conveniant omnes clerici de choro ad altare et reinduat ibidem sacerdos casulam qua prius exuerat et accedat ad gradum altaris cum dyacono et subdyacono et dicant} Confíteor. Misereátur. \emph{et} Absolutiónem. \emph{cum precibus et oratione} Aufer a nobis. \emph{more solito.} Missale:1142.

\emph{Finitis vesperis exuat sacerdos casulum et sumens secum unum de prelatis in superpelliceis discalciati reponant crucem cum corpore dominico in sepulcrum incipiens ipse solus hoc ℟.} Estimátus sum. \emph{genuflectendo cum socio suo~: quo incepto statim surgat~: similiter fiat in responsorio} Sepúlto Dómino. \emph{Chorus totum ℟. prosequatur cum suo ℣. genuflectendo per totum tempus usque ad finem servitii. Responsoria ut sic.}

1519:69r; AS:233; Ant-1519:212v; Brev-1531:124v.\footnote{1519:69r. has no flats.}

\gregorioscore{}

\emph{Chorus prosequatur.}

\gregorioscore{}

\emph{Dum predictum ℟. canitur cum suo ℣. predicti duo sacerdotes thurificent sepulchrum quo facto et clauso hostio incipiat idem sacerdos sequens.}

\gregorioscore{007640-sepulto-domino-1}

\emph{Chorus prosequatur.}

\gregorioscore{007640-sepulto-domino-2}

\emph{⟨Item predicti duo⟩ sacerdotes dicat ⟨istas⟩\footnote{1882:73.} iij. antiphonas sequentes ⟨genuflectendo continue.⟩}

\gregorioscore{003265-in-pace-1}

\emph{Chorus prosequatur.}

\gregorioscore{003265-in-pace-2}

\emph{⟨Iterum⟩\footnote{1882:73.} sacerdotes.}

\gregorioscore{003264-in-pace-1}

\emph{Chorus prosequatur.}

\gregorioscore{003264-in-pace-2}

\emph{Item sacerdotes dicat sic.}

\gregorioscore{001175-caro-mea-1}

\emph{Chorus prosequatur.}

\gregorioscore{001175-caro-mea-2}

\emph{Ad istas iij. antiphonas genuflectent continue predicti ij. sacerdotes.}

\emph{His finitis ordine servato reinduat sacerdos casulam~: et eodem modo quo accesserit in principio servitii cum dyacono et subdyacono et ceteris ministris abscedat, dictis prius orationibus ad placitum. Secrete ab omnibus cum genuflexione omnibus aliis ad libitum recedentibus.}

\emph{Exinde continuo ardebit unus cereus ad minus ante sepulchrum usque ad processionem que fit in resurrectione Domini in die pasche, ita tamen quod dum psalmus} Benedíctus. \emph{cantetur \&c.~que sequuntur in sequenti nocte extinguatur~: similiter et extinguatur in vigilia pasche dum benedicitur novus ignis usque accendatur cereus paschalis ⟨triginta et sex pedes continens in longitudine⟩.\footnote{1517. ⟨}1882:73.⟩}

\chapter{¶ Sabbato in vigilia pasche.}

\emph{Congregatis clericis in choro~: dictaque hora ix. executor officii indutus vestibus sacerdotalibus cum cappa serica rubea~: dyaconus dalmatica et subdyaconus tunica induantur cum ceteris ministris altaris albis cum amictibus indutis sine lumine ⟨in⟩\footnote{1882:74.} cereis et sine cruce et sine igne in thuribulo et quidam de prima forma scilicet acolitus in superpellicio cereum extinctum de tribus candelis tortis in unum in yma parte conjuncti et insuper abinvicem divisis super quandam hastam deferens et processio precedat\footnote{`\emph{procedat}', 1882:74.} post portitorem aque benedicte per medium chori ad fontes et ad novum ignem benedicendum processionaliter eant~: choro sequente habitu non mutato excellentioribus precedentibus et ad columnam ex parte australi juxta fontem ubi sacerdos executor officii illius diei~: ignem benedicat qui accendatur ibidem videlicet inter ij. columnas. In eundo vero dicatur a toto choro alternatim sine nota iste psalmus sequens.}

\begin{lesson}
\subsubsection{⟨Psalmus xxvj.⟩}

\lettrine{D}{o}minus illuminátio mea~: et salus mea quem timébo.

Dóminus protéctor vite mee~: a quo trepidábo.

Dum apprópiant super me nocéntes~: ut edant carnes meas.

Qui tríbulant me\footnote{1882:74. In 1519:70v. the punctuation of the following verses differs from the Breviary.

  4. Qui tríbulant me~:* inimíci mei ipsi infirmáti sunt et cecidérunt.

  10. In petra exaltávit me et nunc exaltávit caput meum~:* super inimícos meos.

  11. Circuívi et immolávi in tabernáculo ejus hóstiam vociferatiónis cantábo~: * et psalmum dicam Dómino.} inimíci mei~: ipsi infirmáti sunt et cecidérunt.

Si consístant advérsum me castra~: non timébit cor meum.

Si exúrgat advérsum me prélium~: in hoc ego sperábo.

Unam pétii a Dómino hanc requíram~: ut inhábitem in domo Dómini ómnibus diébus vite mee.

Ut vidéam voluntátem Dómini~: et vísitem templum ejus.

Quóniam abscóndit me in tabernáculo suo in die malórum~: protéxit me in abscóndito tabernáculi sui.

In petra exaltávit me~: et nunc exaltávit caput meum super inimícos meos.

Circuívi et immolávi in tabernáculo ejus hóstiam vociferatiónis~: cantábo et psalmum dicam Dómino.

Exáudi Dómine vocem meam qua clamávi ad te~: miserére mei et exáudi me.

Tibi dixit cor meum exquisívit te fácies mea~: fáciem tuam Dómine requíram.

Ne avértas fáciem tuam a me~: ne declínes in ira a servo tuo.

Adjútor meus esto Dómine\footnote{1882:77. omits `Dómine.'} ne derelínquas me~: neque despícias me Deus\footnote{1882:77. omits `Deus.'} salutáris meus.

Quóniam pater meus et mater mea dereliquérunt me~: Dóminus autem assúmpsit me.

Legem pone michi Dómine in via tua~: et dírige me in sémita recta propter inimícos meos.

Ne tradidéris me in ánimas tribulántium me~: quóniam insurrexérunt in me testes iníqui, et mentíta est iníquitas sibi.

Credo vidére bona Dómini~: in terra vivéntium.

Expécta Dóminum viríliter age et confortétur cor tuum~: et sústine Dóminum.
\end{lesson}

\emph{Et sine} Glória Patri. \emph{neque} Sicut erat. \emph{\&c.}

\emph{¶ Hoc modo fiat statio ad ignem benedicendum~: sacerdos juxta ignem stet ad orientem conversus~: et ad sinistrum ejus stet dyaconus, subdyaconus vero ad sinistrum dyaconi, unus ceroferarius stet sacerdoti oppositus, ad dextram cujus stet puer ferens librum proximior sacerdoti, alius autem ceroferarius stet retro sacerdotem, ad dextrum cujus stet portitor aque proximior sacerdote et ultimo loco ultra omnes a parte occidentali~: stet portitor haste cum cereo ex alia parte ignis videlicet ex parte australi stet thuribularius ad accipiendum ignem in thuribulo post benedictionem omnibus istis ministris ad sacerdotem conversis, choro interim circumstante videlicet ex parte boreali~: ut patet ⟨ubi protrahitur⟩\footnote{1882:75.} in statione sequente hoc modo.}

\begin{figure}
\centering
\includegraphics{images/71r-statio-ignis.jpg}
\caption{¶ Statio dum benedicitur ignis in vigilia pasche.}
\end{figure}

\emph{¶ Sequatur benedictio ignis ⟨in vigilia pasche⟩\footnote{1882:76.} solenniter a sacerdote dicente.}

1519:71v.

\gregorioscore{}

\begin{lesson}
\subsubsection{Oratio.}

\lettrine{D}{o}mine Deus noster Pater omnípotens~: lumen indefíciens~: Cónditor ómnium lúminum, exáudi nos fámulos tuos et béne✠dic hunc ignem qui tua sanctificatióne atque\footnote{`ac', 1882:76.} benedictióne consecrátur, tu illúminas omnem hóminem veniéntem in hunc mundum~: illúmina consciéntias cordis nostri igne tue claritátis, ut tuo igne ígniti, tuo lúmine illumináti, expúlsis a córdibus nostris peccatórum ténebris ad vitam te illustránte perveníre mereámur etérnam~: et sicut illuminásti ignem Móysi fámulo tuo per colúmnam ignis ambulánti\footnote{`ambulántem', 1882:77.} in mari rubro~: ita illústra nostrum lumen ut candéla que de eo fúerit accénsa~: in hónore majestátis tue semper perséveret benedícta, ut quicúnque ex eo lúmine\footnote{`lumen ex eo', 1882:77.} portáverit sit illuminátus misericórdia grátie spiri-
\end{lesson}

1519:72v.\footnote{`spiritális', 1882:77.}

\gregorioscore{}

\emph{Omnes orationes dicuntur cum} Orémus. \emph{sub tono predicto. Hic aspergatur aqua benedicta super ignem. Deinde cantatur ut sequitur.}

1519:71v.

\gregorioscore{}

\begin{lesson}
\subsubsection[⟨Oratio.⟩]{⟨Oratio.⟩\footnote{1882:77.}}

\lettrine{D}{o}mine sancte Pater omnípotens etérne Deus bene✠dícere et sancti✠ficáre dignéris ignem istum\footnote{`istum ignem', 1882:77.} quem nos indígni per invocatiónem unigéniti Fílii tui Dómini nostri Jesu Christi benedícere presúmimus~: tu clementíssime Pater eum tua benedictióne sanctífica~: et ad proféctum humáni géneris perveníre concéde. Per eúndem Christum Dóminum nostrum. \emph{⟨℟.⟩} Amen.

Orémus.

\subsubsection[⟨Oratio.⟩]{⟨Oratio.⟩\footnote{1882:77.}}

\lettrine{C}{e}lésti lúmine quésumus Dómine semper hic et ubíque nos prevéni ut mystérium cujus nos partícipes esse voluísti~: et puro cernámus intúitu, et digno percipiámus efféctu. Per Dóminum nostrum.

\subsubsection{¶ Sequatur benedictio thimiamatis sive incensi sub tono predicti.}

\lettrine{E}{x}orcízo te immundíssime spíritus et omne phantásma inimíci in nómine Dei Patris omnipoténtis et in nómine Jesu Christi Fílii ejus et Spíritus Sancti~: ut éxeas et recédas ab hac creatúra thimiamátis \emph{sive} incénsi cum omni fallácia atque nequítia tua, ut sit hec creatúra sancti✠ficáta\footnote{`✠' is in 1882:77., but not in 1519:72r.} in nómine Dómini nostri Jesu Christi~: ut omnes gustántes, tangéntes, sive odorántes eam virtútem et auxílium percípiant Spíritus Sancti~: ita ut ubicúnque hoc incénsum \emph{vel} thimiáma fúerit~: íbidem nullátenus appropinquáre áudeas nec advérsa inférre presúmas, sed quicúnque spíritus immúnde es~: cum omni versútia tua procul inde fúgias atque discédas adjurátus per nomen et virtútem Dei Patris omnipoténtis et Fílii ejus Dómini nostri Jesu Christi qui ventúrus est ⟨cum (\emph{sic}) Spíritu Sancto⟩\footnote{1882:77.} judicáre vivos et mórtuos et te prevaricatórem et séculum per ignem.

Orémus.

\subsubsection[⟨Oratio.⟩]{⟨Oratio.⟩\footnote{1882:78.}}

\lettrine{E}{t}érnam ac justíssimam pietátem tuam deprecámur Dómine sanctíssime\footnote{`sancte', 1882:78.} Pater omnípotens etérne Deus ut bene✠dícere dignéris hanc thimiamátis \emph{vel} incénsi spéciem ut sit incénsum majestáti tue in odórem suavitátis accéptum sit a te spécies hec bene✠dícta, per invocatiónem sancti nóminis tui sanctificáta~: ita ut ubicúmque fumus ejus pervénerit~: extricétur et effugétur omne genus demoniórum sicut incénsum jécoris piscis quam\footnote{`quem', 1882:78.} Ráphael archángelus Thobíam fámulum tuum dócuit, cum ascéndit ad Sare liberatiónem. Per Christum Dóminum nostrum.

Orémus.

\subsubsection[⟨Oratio.⟩]{⟨Oratio.⟩\footnote{1882:78.}}

\lettrine{D}{e}scéndat bene✠díctio\footnote{`✠' is in 1882:78., but not in 1519:72v.} ⟨tua Dómine⟩\footnote{1882:78.} super hanc spéciem\footnote{1882:78. 1519:72v. has `hunc spíritum.'} incénsi, \emph{vel\footnote{`et', 1882:78.}} thimiamátis sicut in illo de quo David prophéta tuus cécinit dicens, Dirigátur orátio mea sicut incénsum in conspéctu tuo sit nobis odor consolatiónis suavitátis et grátie, ut fumo istíus\footnote{`isto', 1882:78.} effugétur omne fantásma inimíci mentis et córporis~: ut simus Pauli apóstoli voce bonus odor Deo~: effúgiant\footnote{`effúgiat', 1882:78.} a fácie incénsi hujus \emph{sive\footnote{1882:78; `et', 1519:72v.}} thimiamátis omnes\footnote{`omnis', 1882:78.} demónum incúrsus sicut pulvis a fácie venti, et sicut fumus a fácie ignis~: ⟨et⟩\footnote{1882:78.} presta hoc piíssime Pater boni odóris incénsum ad opus ecclésie tue ob causam religiónis júgiter permanére~: ut mýstica nobis significatióne spirituálium virtútum fragrans osténdit\footnote{`osténdat', 1882:78.} odor suavitátem. Tu\footnote{`tua', 1882:78.} ergo quésumus omnípotens\footnote{1882:78; `omnípotens quésumus', 1519:72v.} Deus imménse majestátis déxtera hanc creatúram benedícere ex diversárum rerum commixtióne conféctam\footnote{1882:78; `inféctam', 1519:72v.} dignáre~: ut in virtúte sancti tui nóminis, omnes immundórum spírituum fantásticos incúrsus effugáre, omnésque morbos réddita sanitáte expéllere, ut ubicúmque fumus arómatum ejus affláverit mirabíliter possit atque in odóre fragrantíssimo tibi perpétua suavitáte redolére. Per Dóminum nostrum Jesum Christum Fílium tuum. ⟨Qui tecum vivit et regnat Deus. Per ómina sécula seculórum. \emph{⟨℟.⟩} Amen.⟩
\end{lesson}

\emph{¶ Post benedictionum incensi ponatur de ipsis carbonibus in thuribulo cum incenso et incensetur novus ignis~: postea accendatur cereus super hastam solus de novo igne ceteris luminariis ⟨ecclesie prius extinctis et deferatur in processione ad locum ubi cereus paschalis benedicendus est⟩.\footnote{1882:78.} In redeundo duo\footnote{1882:78; `\emph{secundi}', 1519:72v.} clerici de secunda forma, in superpelliceis post sacerdotem incedentes cantent hos versus sequentes hoc modo.}

Rylands-24:195; 1519:62v; GS:104; Grad-1508:103v.\footnote{`rútuli', 1523:72v. In Rylands-24:195 `redde' is set CA.G. In Rylands-24:195. (verse 2) `Ne' is set G. In many sources the third phrase is varied in the last verse: `Regnum' is set C.D rather than C.DE. Only the 1555 edition and GS:104. conform this verse to the others. Arsenal:89v. has a different tune---see the Appendix.}

\gregorioscore{}

\emph{Chorus idem repetat post unumquemque versum ⟨ita videlicet ut dum dicti clerici cantent versum stent gradibus fixis, et dum chorus prosequatur primum versum procedant. Quod similiter est observandum dum canitur} Rex sanctórum angelórum. \emph{cum suis versibus⟩.\footnote{1882:79.}}

\gregorioscore{}

\emph{Chorus repetat} Invéntor ⟨rútili⟩.\footnote{1882:79.}

\emph{Clerici ⟨dicant⟩\footnote{1882:79.}}

\gregorioscore{}

\emph{Chorus repetat} Invéntor ⟨rútili⟩.

\emph{Clerici ⟨dicant⟩\footnote{1882:79.}}

\gregorioscore{}

\emph{¶ Chorus repetat} Invéntor rútili.

\emph{Clerici ⟨dicant⟩\footnote{1882:79.} versum.}

\gregorioscore{}

\emph{Deinde sequatur benedictio cerei paschalis ab ipso dyacono indutus ad processionem accepta ⟨prius⟩\footnote{1882:79.} benedictione ab executore officii ad borealem converso ad gradum presbyterii ceroferarii dyaconi assistentibus uno a dextris, reliquo a sinistris ad eum conversis cereis extinctis nisi cereus super hastam. Subdyaconus vero textum tenens stet ei oppositus juxta quem stet portitor haste ex una parte et ceroferarius ejusdem parvi cerei\footnote{`The \emph{parvus cereus} is explained by the Ordinal quoted by Dr.~Rock, iv. 246: \emph{Finita benedicitione ignis et incensi \ldots{} Cereus super hastam illuminetur, et alia candela incendatur, unde idem cereus super hastam si forte extinguatur possit reaccendi.}' ⟨1882:xv.⟩} ut patet ⟨in ubi protrahitur⟩\footnote{1882:80.} in statione ⟨sequente⟩\footnote{1882:80.} hoc modo.}

\begin{figure}
\centering
\includegraphics{images/74r-statio-cereus.jpg}
\caption{¶ Statio dum benedicitur cereus paschalis in vigilia pasche.}
\end{figure}

\emph{Ad fontes benedicendas ex altera eodem modo conversus et post dyaconum dyaconus, dyacono canente.}

1519:74r; GS:105; 1508:104r; Rylands-24:196; Missal-1513:80v.\footnote{In 1519:79r. `patre nostro papa \emph{N.} atque' is crossed out.

  (1.): In 1513:80v. `turba' is set C.B; `et' is set EEE; `regis' is set GA.AG. In Rylands-24:196. `regis' is set G.AAG. Rylands-24:196. has `íntonat'; `salutáris' ends GFEEF.FE.

  (2.): In 1508:104r. `fulgóribus' is set A.C.CB.AA. In GS:105. `fulgóribus' is set A.CC.A.A. In Rylands-24:196. `etérni' is set AC.C.C. In 1513:81r. `amisísse' is set DEFA.A.AC.BAG. Rylands-24:196. and GS:105. have `lustráta.' In 1508:104r. `illustráta' is set A.C.CB.A. In Rylands-24:196. `se' is set BCBAGAAGGE; `amisísse' is set DEFGA.A.ACBA.G.

  (3.): In 1513:81r. `mater ecclésie' is set D.CB C.CD.CB.A; `tanti' is set DEDC.C. In 1508:104v. `letétur' is set G.AC.CC. In 1513:81r. `fulgóribus' is set A.CCB.A.A. In 1513:81r. `vócibus' is set C.CBA.G; `aula resúltet' is set BAAG.GF EFG.GFEF.FE. In 1508:104v. `vócibus' is set CCB.A.G. In Rylands-24:196. `resúltet' is setGFEF.GAGFEF.FE.

  (4.): In 1508:104v. and Rylands-24:196. `tam' is set A. In GS:106. queso is set C.C. In 1513:81r. `misericórdiam' is set A.AB.GF.GA.A.A; `invocáte' is set G.FEFG.AGFEF.FE.

  (5.): In Rylands-24:196. `Ut' is set EFEFFEEDE. In 1513:81r. `méritis' is D.C.CBA; `dignátus' is set C.AGGA.BC. In 1513:81r. `cérei' is set DDE.DC.C; `implére' is set GGA.BC.FGA; `perfíciat' is set EFG.AGF.EF.FE. In 1508:104v. `perfíciat' is set AGFEF.G.GFEEF.F. In GS:106. `implére' is set G.GABC.BA.

  (6.): In 1508:104v. `Per' is set EFEFF. In GS:106. `per' is set EFEFEE. In 1513:81r. `Per' is set EFEF. In Rylands-24:196. `Per' is set EFEEFe. In 1508:104v. `nostrum' is set B.BCA. In 1513:81r. `Christum' is set C.CB; `Fílium' is set A.BC.BAG. `tuum qui tecum' ⟨in place of `suum qui cum eo'⟩, 1515. ⟨Dickinson:338† .⟩ In GS:106. `Fílium suum' is set BC.BA.AG A.A. In 1513:81r. `regnat Deus' is set G.AC C.CB; `Spíritus' is set B.G.E. 1513:81r, 1508:104v. and Rylands-24:196. have no flat. GS:106. omits Deus and its music. In GS:106. `unitáte' is set G.A.CCB.A. In 1513:81r. `Sancti' is set FA.A.

  (7.): In 1513:81v. `Habémus' is set A.AC.B. In GS:106. `sursum' is set GABCCDc.BAG. In 1513:81v. `Dómino' is set ABC.A.AABG. In 1508:105r. `Grátias' is set C.C.CBAGGA; `Deo' is GAC.CB. In 1508 `et justum est' appears to be set CB BC.AB B.

  (8.): In 1513:81v. `omnipoténtem' is set DCB.Cb.CB.BC.A. In 1508:105r. `omnipoténtem' is set D.C.B.C.CBBCA. In 1513:81v. `unigénitum' is set CB.BCA.AB.B.B. In 1513:81v. `cordis' is set CC.C; `ac' is set `A'; `afféctu' is set CBBC.AAB.B. In 1508:105r. `ac' is set C. GS:107. has `éffectu.' In 1513:81v. `ministério' is set A.B.C.A.ACBAG.

  (9.): In 1513:81v. `Qui pro' is set Cb CBAG. In GS:107. `nobis' is set AC.C. In 1508:105r. `etérno' is set C.B.C. In 1513:81v. `pio cruóre' is set CCA.B C.CB.BAG; `detérsit' is set GA.ACBAB.A. In 1508:105r. `pio cruóre' is set C.A A.C.CBBAG.

  (10.): In GS:107. `paschália' is set CBBCb.A.B.B. In 1508:105r. `occíditur' is set CB.BCb.AB.B. `postes' is set C.CBAG.

  (11.): In 1508:105r. `transíre' is set B.C.CBBAG.

  (12.): In 1513:82r. `Hec' is set CCB. In GS:108. `hec' is set CCCDCC. In 1513:82r. `peccatórum' is set C.C.D.CBBC; `colúmne' is set A.C.B. In 1508:105r. `peccatórum' is set CB.C.D.CBBCC (the text underlay is not clear).

  (13.): In 1508:105v. `credéntes' is set CB.BCAAB.B. In GS:107. `séculi' is set D.C.C. In 1513:82r. `segregátos' is set C.CBBC.AB.B. In 1508:205b. `segregátos' is set C.CBCB.AB.B; `calígine' is set A.C.C.C; `peccatórum' ends on A; `reddit' is set C.CBAG. In Rylands-24:197. `calígine' is set A.C.B.B.

  (14.): In 1513:82r. `Hec nox' is set thus:

  \gregorioscore{}

  In 1508:105v. `nox' appears 5 notes earlier. In 1513:82r. `qua' is set C. In 1508:105v. `mortis' is set AB.A. In GS:108. `victor' is CCB.BAG. In GS:108. `vínculis' is set C.B.CBBCA.

  (15.): In 1508:105v. `Nichil' is set C.CCBBAG; `prófuit' is set AB.B.B. In 1513:82r. `nisi' is set C.CBA. In 1513:82r. `rédimi' is set B.BCA.AG.

  (16.): In 1513:82r. `mira' is set D.CBC In Rylands-24 `O mira' is set thus:

  \gregorioscore{}

  (17.): In 1508:105v. `inestimábilis' is set E.FEF.D.DCB.BD.DCBCCB. In GS:108. `inestimábilis is set E.FEF.D.DCB.CB.B. In 1513:82r. 'caritátis' is set C.CB.AB.BA; `ut' is set C. In 1508:105v. `Fílium' is set C.CBA.ACBAG (or CCB.A.ACBAG).

  (18.): In 1513:82r. `certe' is set C.CBA; `peccátum' is set CBBC.AB.B; `Christi morte' is set C.CB.AB.BAG.

  (19.): 1513:82v. `'tantum' is set C.CBA. In GS:108. `O felix' appears to be set CC CDEDCBC.B.

  In 1508:105v. `culpa' is set AAB.B. In Rylands-24:197. `méruit' is set C.CB.A. In 1513:82v. `redemptórem' is set GA.A.ACBCD.CBABA. In GS:108. 'habére is set B.CCB.BAG.

  (20.) In 1508:105v. `O certe' is set thus:

  \gregorioscore{}

  In 1513:82v. `et is set CBBC. In GS:108. 'et' is set CBBCb. In 1513:82v. `ab' is set A. In 1508:105v. `ínferis' is set CCb.AA.ABG. In GS `ínferis' is set CA.AABG. In GS:108 `ab' is set B.

  (21.): In 1513:82v. `Hec' is set thus:

  \gregorioscore{}

  In GS:108. `Hec' is set thus:

  \gregorioscore{}

  In 1513:82v. `illuminátio' is set D.C.B.C.B.CBBC.

  (22.): 1513:82v. `sanctificátio' is set D.C.B.C.B.CBBC; the first `fugat' is set CC.A. In GS:108. `fugat scélera' is set C.BABG A.B.B. In 1513:82v. `culpas' is set C.BABA; `innocéntiam' is set D.CB.C.B.CBBC. In 1513:82v. the second `fugat' is set DC.CB; `concórdiam' is set A.C.CB.AG.

  (23.): In 1508:106r. `noctis' is set DDCB.C. In Rylands-24:197. `noctis' is set E.DCBC. In 1513:82v. `súscipe' is set CBBACDCCABC.DCB. In 1513:82v. `incénsi hujus' appears before the rubrics. In 1513:82v. `vespertínum' is set GA.ACB.BCA.A. 1508:106r. has `iniénsi' for `incénsi.'

  (24.): In 1508:106r. `tibi' is set CDC.CBB. In 1513:82v. `oblatióne' solénni per ministrórum' is set thus:

  \gregorioscore{}

  In GS:109. `solénni' is set DCB.CCB.B. In 1513:82v. `opéribus' is set B.C.CB.AG; `sacrosáncta' is set C.BAB.C.B. In 1508:106r. `opéribus' is set B.CCB.A.AG. GS:109. omits `sacro-', but includes the music. In Rylands-24:197. `sacrosáncta' is set B.BAAB.C.B.

  (25.): In 1513:83r. `Sed jam' is set CCCBABCCDC CAGGA; `colúmne' is set A.A.C; `hujus' is set C.DCB; `precónia' is set C.B.CB.BC; `quam' is set BBB. In 1508:106r. `quam' is set BBB. In 1508:106v. `ignis' is set CCb.BAG. In GS:109. `ascéndit' is set GA.ACBBC.A.

  (26.): In 1513:83r. `mutuáti' is set BC.C.CB.C; `lúminis' is set DE.C.B; `detriménta' is set C.B.AC.B; `liquántibus' is set B.C.C.AG.

  (27.): In 1508:106v. `Alitur' is set B.CB.A. 1508:106v. has `liquéntibus' set B.CC.CA.G. In GS:109. `liquántibus' is set B.CCb.A.G. In Rylands-24:198. `Alitur liquántibus' is set C.CCB.A B.CB.CA.G.

  In 1513:83r. `lámpadis' is set C.CB.AG; `apes' is set C.A.

  (28.): In 1513:83r. `O beáta' is set thus:

  \gregorioscore{}

  In GS:109. `O' is set as follows:

  \gregorioscore{}

  In 1508:106v. `mater' is set CBA.G. `O' omits the last 6 notes of the melisma. In 1508:106v. the text-underlay at `beáta' is misaligned. In 1513:83r. `expoliávit' is set A.C.B.C.B; `Egýptios' is set CB.BC.AB.B. In 1508:106v. `Egíptios' is set CB.BCAAB.B.B. In GS:109. Egýptios' is set AG.AB.B.B. In GS:109. `in' is set CB.

  (29.): In 1513:83r. `Dómine' is set DCBC.CBA.B; `honórem' is set E.EGFFGE.E. In GS:109. `Dómine' is set CCB.AB.B. In 1513:83r. `consecrátus' is set DCBCD.DE.DCB.B; `ad' is set BBC; `calíginem' is set A.C.C.B. In GS:109. `consecrátus' is set BAGA.DDEDEC.B.B. In Rylands-24:198. `consecrátus is set DEDC.DEDC.B.B; 'calíginem' is set A.C.A.CCb. In 1513:83r. `indefíciens' is set C.CA.CA.CEEF.D; `persevéret' is set C.CB.ABCCBAGA.AG. In 1508:106v. `persevéret' is set CCb.ABCCBAG.GA.A.

  (30.): In 1513:83v. `odórem' is set BC.DE.E; `suavitátis' is set E.EF.D.DC.BC; `acceptúrus' is set DE.DEC.B.B; `lumináribus' is set C.A.C.A.ACBAG. `accéptus; accénsus in 97. 26.' Dickinson:342† .

  In 1508:106v. the text underlay of `suavitátis' is unclear.

  (31.): In 1513:83v. `ejus lúcifer' is set AB.C CB.AB.B. In 1508:106v. `ejus' is set ABC.CB.

  (32.): In 1513:83v. `Ille' is set DEDDc.CB. In 1513:83v. `nescit' is set C.BA.

  (33.): In 1508:106v. `Ille' is set CCBABCCDCCB.BA. In Rylands-24:198. `Ille' is set CCCBACCDCC.BA. In 1513:83v. `humáno' is set C.BAA.B; `Precámur' is set CCBABCDCCAG.A.C.

  (34.): In 1513:83v. `Dómine' is set DCBC.CBA.B; `tuos' is set AB.C. In 1508:107r. `ergo' is set CDE.EF. In Rylands-24:198. `Dómine' is set DCBCCB.AB.B. In 1513:83v. `devotíssimum' is set C.C.D.CCB.AG; `una cum patre' is set B.BC D CCb.AB. In 1508:107r. `una' is set B.BCD. In 1513:83v. `nostro papa' is set C.BA.BGAB.B; the first `\emph{N.}' is set A.AB.AB; the following two are set A.AB.AB.B; `atque rege' is set BAB.C CBB.G. In 1513:83v. `patre nostro pape \emph{N.} atque' is stricken out. In 1508:107r. each `\emph{N.}' is set A.AB.AB.B. In GS:110. the first two `\emph{N.}' are set A.ABG.AB.B; the third is set A.ABa.AB.B. In Rylands-24:198. the first `\emph{N.}' is set AABa.AAAB.B, and the second and third are set AAABa.AB.B. In 1513:83v. `témporum' is set D.DE.CB. In Rylands-24:198. `necnon' has no music; `et epíscopo' is set A CBC.D.C.CCB. In 1513:83v. `paschálibus' is set E.EEF.D.CBAG. In Pro-1519:79r. `patre nostro papa \emph{N.} atque' is crossed out. If a queen were on the throne the following would presumably be sung:

  \gregorioscore{}

  (35.): In 1513:83v. `ímperas' is set BDED.Dc.CB. In 1513:83v. `gloriáris' is set DE.E.GFFG.E; `solus' is set CC.C; `altíssimus' is set CB.C.E.CBC. In GS:110. `altíssimus is set CBC.D.EDCBC.CB. In Rylands-24:198. `gloriáris' is set D.DE.EFFG.FE. In 1513:83v. `Jesu' is set CBBCBCDCBAACBAG; `cum Sancto Spíritu' is set AACb ABCEEFED.DC BCD.DCBC.CB; `in glória' is set B GAC.C.C. In 1508:107r. `Spíritu' is set D.DCBCCB.B. In GS:110. `Christe is set GACBC.A. In Rylands-24:198. `Spíritu' set D.EDCBC.CB. In GS:110. `Patris' is set BCBCBAGACBBC.A. In Rylands-24:198. `Patris' is set BCBCCAAGACBCCb.A. In 1513:84r. `Amen' is set ADDCBCDEFEDCBCDCBAAB.B.}

\gregorioscore{}

\emph{Hic dyaconus ponat incensum in cereum in modum crucis ⟨cantando hoc quod sequitur hoc modo⟩.\footnote{1882:82.}}

\gregorioscore{}

\emph{Hic accendatur cereus paschalis de novo igne, nec extinguatur ante completorium diei sequentis. Hic accendantur cetera luminaria per ecclesiam.}

\gregorioscore{}

\emph{Finita benedictione cerei paschalis sacerdos completurus officium indutus casula ad altare authenticum assumpta~: cum ministris suis ad altare accedat confessione jam non dicta, sed tantum} Pater noster. \emph{et osculando altare cum suis ministris eat sessum. Accedat cereus super hastam~: minister vero qui alium cereum defert ad sinistrum cornu altaris stet super gradum ⟨ad⟩\footnote{1882:83.} australe conversus quousque finiatur septiformis letania, postea legantur lectiones sine titulo a dignioribus personis, \&c.} Missale:731.

\emph{Quibus peractis sequatur septiformis letania, que in medio chori a septem pueris in superpelliceis dicatur et interim exuat sacerdos casulam et super altare reponat, et sumat cappam rubeam adhuc stando ante altare donec cantetur letania sequens, \&c.}

\emph{Sequitur letania.}

1519:79v; Ant-1519-P:172r; Missal-1531-P:48v.

\gregorioscore{}

Sancta virgo vírginum.

Ora pro nobis.

Sancte Míchael. Ora.

Sancte Gábriel. Ora.

Sancto Ráphael. Ora.

Omnes sancti ángeli et archángeli Dei. Oráte pro nobis.

Sancte Johánnes baptísta. Ora.

Omnes sancti patriárche et prophéte. Oráte.

Sancte Petre. Ora.

Sancte Andréa. Ora.

Sancte Johánnes. Ora.

Sancte Jacóbe. Ora.

Sancte Philíppe. Ora.

Sancte Bartholomée. Ora.

Sancte Mathée. Ora.

Omnes sancti apóstoli et evangelíste. Oráte ⟨pro nobis⟩.\footnote{1882:83.}

Sancte Stéphane. Ora.

Sancte Line. Ora.

Sancte Clete. Ora.

Sancte Laurénti. Ora.

Sancte Vincénti. Ora.

Sancte Sixte. Ora.

Sancte Dionýsi cum sociis tuis.

Oráte\footnote{Usually `Ora pro nobis.'} ⟨pro nobis⟩.\footnote{1882:83.}

Omnes sancti mártyres. Orate.

Sancte Silvéster. Ora.

Sancte Gregóri. Ora.

Sancte Hylári. Ora.

Sancte Martíne. Ora.

Sancte Remígi. Ora.

Sancte Audoéne. Ora.

Sancte Augustíne. Ora.

Omnes sancti confessóres.

Oráte ⟨pro nobis⟩.\footnote{1882:83.}

Sancta María Magdaléna. Ora.

Sancta Felícitas. Ora.

Sancta Perpétua. Ora.

Sancta Agatha. Ora.

Sancta Agnes. Ora.

Sancta Cecília. Ora.

Sancta Scolástica. Ora.

\gregorioscore{}

\emph{¶ Si episcopus presens fuerit indutus cappa serica stet in sede sua dum predicta letania canitur.}

\emph{Finita hac letania statim incipiatur quinta partita letania que a v. dyacono similiter in medio chori in superpellicio de ij. forma dicatur et finiatur sub tono predicto.}

\emph{Cumque perventum fuerit ad hanc prolationem,} Sancta María. \emph{statim eat processio ad fontes benedicendas hoc ordine. In primis accolitus crucem ferens alba et tunica indutus~: post eum vero duo ceroferarii in albis cum amictibus deinde thuribularius in simili habitu, post eum vero duo pueri in superpelliceis pariter incedentes unus ferens librum, alius a dextris ejus ferens cereum ad fontes benedicendas~: deinde duo dyaconi de secunda forma albis cum amictis induti, pariter incedentes unus ferens oleum, alius a dextris ejus ferens chrisma~: deinde subdyaconus tunica~: et deinde dyaconus in dalmatica~: deinde sacerdos in cappa serica rubea, clero itaque sequente habitu non mutato, ex australi latere ecclesie procedendo ad fontes veniant, predicti dyaconi letaniam canentibus de singulis ordinibus, v. in medio clericorum de secunda forma post executorem officii hoc modo.}

\begin{lesson}
\subsubsection{⟨¶ Sequitur letania.⟩}

Kyrieléyson. Christeléyson. Christe audi nos.

Sancta María. Ora.

Sancta Dei génitrix. Ora.

Sancta virgo vírginum. Ora.

Sancte Míchael. Ora.

Sancte Gábriel. Ora.

Sancte Ráphael. Ora.

Omnes sancti ángeli et archángeli Dei. Oráte.

Sancte Johánnes baptísta. Ora.

Omnes sancti patriárche et prophéte. Oráte ⟨pro nobis⟩.\footnote{1882:85.}

Sancte Paule. Ora.

Sancte Jacóbe. Ora.

Sancte Thoma. Ora.

Sancte Symon. Ora.

Sancte Thadée. Ora.

Omnes sancti apóstoli et evangelíste. Oráte.

Sancte Clemens. Ora.

Sancte Cornéli. Ora.

Sancte Cypriáne. Ora.

Sancte Sebastiáne. Ora.

Sancte Mauríci cum sóciis tuis.

Oráte\footnote{`Ora', 1882:85; usually `Ora pro nobis.'} ⟨pro nobis⟩.\footnote{1882:85.}

Onmes sancti mártyres. Oráte.

Sancte Benedícte. Ora.

Sancte Nicoláe. Ora.

Sancte Germáne. Ora.

Sancte Románe. Ora.

Sancte Aldélme. Ora.

Sancte Augustíne. Ora.\footnote{This line does not appear in 1555 (unpaged).}

Omnes sancti confessóres.

Oráte ⟨pro nobis⟩.\footnote{1882:85.}

Sancta Lúcia. Ora.

Sancta Petronílla. Ora.

Sancta Katherína. Ora.

Sancta Christína. Ora.

Sancta Brigída. Ora.

Omnes sancte vírgines. Oráte.

Omnes sancti. Oráte.
\end{lesson}

\emph{In his duabus letaniis non dicatur} Pater de celis. \emph{⟨neque⟩\footnote{1882:85.}} Fili Redémptor mundi Deus. \emph{neque} Spíritus Sancte Deus. \emph{neque} Sancta Trínitas unus Deus.\footnote{In 1882:85. this rubric appears at the head of the second Litany.}

¶ \emph{⟨Item⟩ Gelasius papa\footnote{In 1519:81r. `papa' is crossed out.} ostendit dicens quia ipse qui Pater et Filius et Spiritus Sanctus una persona in Trinitate,\footnote{`\emph{Unitate}', 1519:81r.} et tres persone in Unitate et in sepulchro se custodiri mittitur\footnote{`\emph{promittitur}', 1882:86.} omnino de\footnote{`\emph{mittitur omnino dicit}', 1519:81r.} adhuc surrexerat a mortuis qui voluit prophetiam implere, sed jacuit in sepulchro usque ad tertium diem quod bene iste predicte iiij. clausule in his letaniis possunt pretermittere.\footnote{`The first paragraph of the rubric, the quotation from Gelasius, I have not been able to correct.' 1882:xv.}}

\emph{¶ Hoc modo fiat statio ad fontes ex parte occidentali donec percantetur letania scilicet ad gradum fontis ex parte occidentali stet sacerdos, retro quem stent quinque dyaconi letaniam cantantes~: deinde ad alium gradum fontis ex parte orientali puer librum ferens, deinde dyaconus, deinde subdyaconus, deinde oleum et crisma~: deinde portitor cerei fontis, deinde thuribularius, deinde oleum~: ceroferarius, exinde duo accoliti crucem ferentes omnibus ad orientem conversis.}

\emph{Executor officii conversus ad orientem fontes benedicendo, assistat minister juxta ad fontem circumstantibus ordinate scilicet a dextris juxta sacerdotem stet dyaconus, subdiaconus vero a sinistris, qui fert chrisma stet juxta dyaconum, qui vero fert crucem stet sacerdoti opposito ad eum conversus, juxta quem eodem modo stent ceroferarii ij. post ceroferarium et thuribularium, qui vero fert cereum inter dyaconum et chrisma~: puer autem ferens librum stet inter subdyaconum et oleum ⟨ut patet in pictura sequente⟩.\footnote{1882:86.} Episcopus tamen si presens fuerit a tergo canentium letaniam ut in aliis processionibus semper in fine ultimum locum tenet.}

\begin{figure}
\centering
\includegraphics{images/81v-statio-fontes.jpg}
\caption{¶ Statio dum cantatur letania ad fontes in vigilia pasche.}
\end{figure}

\emph{Deinde executor officii ad fontes dicat sic.}

1519:81v.

\gregorioscore{}

\begin{lesson}
\lettrine{O}{m}nípotens sempitérne Deus adésto magne pietátis mystériis, adésto sacraméntis et ad recreándos novos pópulos quos tibi fons baptismátis párturit Spíritum adoptiónis emítte~: ut quod nostre humilitátis geréndum est, ministério tue virtútis impleátur
\end{lesson}

\emph{⟨Prosequatur ut sequitur.⟩}

1519:82r.

\gregorioscore{}

1519:82r.

\gregorioscore{}

\emph{Hic dividat aquam manu sua in modum crucis dicenda ⟨hoc quod sequitur hoc modo⟩\footnote{1882:88.} sic.}

\gregorioscore{}

\emph{Hic sacerdos ejiciat sua manu aquam de fonte in modum crucis in iiij. partes ⟨dicendo hoc modo⟩.\footnote{1882:89.}}

\gregorioscore{}

\emph{Hic mutet vocem suam quasi legendo} Hec nobis precépta servántibus tu Deus omnípotens et\footnote{not in 1882:89.} clemens~: adésto tu benígnus aspíra.

\emph{Hic aspiret in fontem ter in modum crucis.} Tu has símplices aquas tuo ore benedícito ut preter naturálem emundatiónem quam lavándis possunt adhibére corpóribus sint étiam purificándis méntibus efficáces.

\emph{Hic stillet de cereo in fontem in modum crucis postea dicat.}

\gregorioscore{}

\emph{Hic mittat cereum in medio\footnote{`\emph{infra}', 1882:89.} fontis crucem faciens, ⟨et postea prosequatur⟩.\footnote{1523:85r. `\emph{dicendo hoc quod sequitur hoc modo}', 1882:89. 1519:85r. has after faciens `\emph{\&c.}'}}

\gregorioscore{}

\emph{Hic extrahatur cereus a\footnote{`\emph{extra}', 1882:89.} fontes dicendo ⟨sic⟩.\footnote{1882:89.}}

\gregorioscore{}Per Dóminum nostrum Jesum Christum Fílium tuum~: qui tecum vivit et regnat in unitáte Spíritus Sancti Deus. Per ómnia sécula seculórum. \emph{⟨℟.⟩} Amen.

\emph{¶ Consecratis fontibus non infundetur oleum neque crisma nisi fuerit aliquis baptizandus. Completo fontium ministerio tres clerici de superiori gradu in cappis sericis videlicet duo in cappis rubeis et tertius in cappa alba\footnote{`\emph{alma}', 1519:85v.} in medio processionis in eundo cantent hanc letaniam simul~: ita quod primus ℣. cantetur a predictis clericis antequam procedat processio hoc modo sequente.}

Rylands-24:207; 1519:85v; Grad-1508:108r; GS:114; Missal-1513:85v.\footnote{1508:108v. shows no indication of the end of the refrain or the beginning of the first verse. In Rylands-24:207. `mundum' is set Dc.BA. In Arsenal:94r. `mundum' is DCB.AG. It is common that the first neume of the second line of each verse be simply E, and the penultimate neume of each verse be simply A. In (1.): in 1508:108v. `angélici' is A.AB.A.G. In Arsenal-94r. (verse 1) `nobis' is set AG.G. In 1508:. `et' is set E. In (4.): GS:114. has `vos precámur.' In (5.): GS:114. has `Cujus.' Verses 6--7. appear in Rylands-24:207 and Arsenal:94r. but not in 1508:108v. or GS:114. In the printed processionals the seventh syllable of each verse is set AA.}

Mode III. mundum notes together.

\gregorioscore{}

\emph{Chorus idem repetat post unumquemque versum.}

\gregorioscore{}

\emph{Deinde incipiatur missa sine regimine chori immediate cantore incipiente.} Missale:743.

\chapter{¶ In die pasche.}

\emph{Ante matutinas\footnote{1528, \&c.---Ante missam. ⟨1882:91.⟩ Also 1519:86v. `Ante matutinas' appears in the Breviary.} et ante campanarum pulsationem conveniant clerici ad ecclesiam et accendantur omnia luminaria per ecclesiam. ij. excellentiores cum ceroferariis et thuribulario et clero circumstante ad sepulcrum accedant, et incensato prius sepulchro cum magna veneratione statim post thurificationem videlicet genuflectendo corpus Domini privatim super altare deponant. Iterum accipiant crucem de sepulchro et incipiat excellentior persona} Christus resúrgens. \emph{cum quo eat processio ⟨per ostium⟩\footnote{1882:92.} presbyterii australe incedendo per medium chori et regrediens cum predicta cruce de sepulchro assumpta, inter duos sacerdotes predictos, super eorum brachia venerabiliter portata cum thuribulario et ceroferariis precedentibus per ostium presbyterii boreale, exeundo ad unum altare ex parte boreali ecclesie choro sequente habitu non mutato excellentioribus precedentibus, corpore vero Dominico super altare in pixide dimisso in subthesaurarii custodia, qui illud in predicta pixide in tabernaculo dependat, ut patet in ⟨ista⟩\footnote{1882:92.} statione sequenti ⟨precedente⟩.\footnote{1882:92.}}

\begin{figure}
\centering
\includegraphics{images/87r-matutinas-cum-cruce.png}
\caption{¶ Statio et ordo processionis in die pasche ante matutinas cum cruce.}
\end{figure}

\emph{Et tunc pulsentur omnes campane in classicum, et cantetur antiphona.}

1519:87r; AS:241; Ant-1519:214r; Brev-1531:125v.\footnote{1519:87r. has no flats. In 1519:87v. `Judéi' is set CD.EFD.D.}

\gregorioscore{}

\emph{Chorus respondeat ⟨hoc modo⟩\footnote{1882:92.} sic ⟨quo sequitur⟩.\footnote{1882:92.}}

\gregorioscore{}

\emph{Finita antiphona cum sua versu a toto choro dicat\footnote{`\emph{incipiat}', 1882:92.} excellentior persona in ipsa statione conversus ad altare versiculum.}

\emph{⟨℣.⟩} Surréxit Dóminus de sepúlchro.

\emph{⟨℟.} Qui pro nobis pepéndit in ligno.⟩\footnote{1882:92.}

Orémus.

\begin{lesson}
\subsubsection{⟨Oratio.⟩}

\lettrine{D}{e}us qui pro nobis Fílium tuum crucis patíbulum subíre voluísti~: ut inimíci a nobis expélleres potestátem, concéde nobis fámulis tuis~: ut in resurrectiónis ejus gáudiis semper vivámus. Per eúndem Christum Dóminum nostrum. \emph{⟨℟.⟩} Amen. \emph{Nec precedat nec subsequatur} Dóminus vobíscum.
\end{lesson}

\emph{¶ Finita oratione omnes cum gaudio genuflectant ibidem, et ipsam crucem adorent.\footnote{1882:92. 1519:88r. has `\emph{odorent}.'} In primis\footnote{`\emph{imprimis}', 1882:92.} digniores persone et secrete sine processione in chorum redeant. His itaque gestis discooperiantur omnes cruces per ecclesiam et omnes ymagines~: et pulsentur campane ad matutinas more solito. Hac die ad matutinas nulla fiat processio ad crucem sicut in ceteris diebus.}

\emph{¶ In die pasche et\footnote{`\emph{atque}', 1882:93.} in omnibus domincis diebus ab hinc ad festum Trinitatis dicetur hec antiphona in aspersione aque benedicte cantore incipiente hoc modo.}

1502:2r; 1530:4r; 1519:88r; GS:116; Grad-1508:110r; Missal-1513:86r.\footnote{In the processionals `salvi' is st GBD.C. ⟨In 1508:110r. `pervénit' is set Ba.AC.C; `salvi' is set GCCD.C; at `dicent' the letter `e' is missing; the first `allelúya' is set AG.G.GAABa.A. GS:116. gives only beginning of Psalm-Verse, plus the word `Amen' along with the Psalm-Tone ending.⟩}

\gregorioscore{}

Vidi aquam. Glória Patri. \emph{Et omnes ℣. et orationes ut supra in prima dominica adventus Domini.} 7.

\emph{Hac die hora vj. cantata et aqua benedicta aspersa ordinetur processio ad gradum chori sicut in die natalis Domini cum aque bajulo tribus crucibus, duobus ceroferariis et duobus thuribulariis. Subdyacono et dyacono utroque codicem evangelii deferentes, et eat processio per medium chori et ecclesie, circueundo ecclesiam et claustrum~: tres clerici de superiore gradu prius in choro incipiant prosam modo sequente.}

Rylands-24:209; GS:116; 1519:88v.\footnote{In Rylands-24:209. and GS:134. `inférnum' is set FA.AAG.ED. GS:166. breaks off in the middle of verse 3. In verse 1. GS:116. has `mundi'; in verse 2. GS:116. has `dantque creatóri', set FD.EF GF.D.C.E; in verse 3. GS:116. has `Pollícitam; 1882:93. has 'Sollicitam sed \ldots{} sepúlte Deus'; 1528:82r. has `Solícitam te \ldots{} sepúlte Deus'; in verse 8. 1882:93. has `Redde diem, qui nos'; in verse 10. 1882:93. has `Trístia cessérunt'; 1528:83r. has `expavíque.' Verse 1: in GS:116. and Rylands-24:209. `grátia' is set G.F.D. Verse 2: in Rylands-24:209. `Deus ecce per' is set FD.E D.G G. GS:116. and Rylands-24:209. have `Dantque Creatóri'; this is set GD.EF GF.D.C.E. Verse 3: in 1519:88v. and 1528:82r. `redde fidem precor alma' is set F.A F.D E.E DG.F. 1528:82r. has `preco.' GS:116. appears to begin `Pollícitam.' In Rylands-24:209. `redde fidem' is set G.F G.A. Verse 5: in Rylands-24:209. `claudúntur' is F.D.E. In the printed processionals `tegat inclúsum' is st EF.GF D.C.E. Here the edition follows Rylands-24:209. Verse 7: in the printed processionals `Intrans mortis iter' is set FD.EF GF.D C.E. Here the edition follows Rylands-24:209. Verse 8: in the printed processionals `ut' is set D. Here the edition follows Rylands-24:209. In Rylands-24:209. `Redde diem quem nos' is set FD.E F.GF DC E. Verse 9: in the printed processionals `de' is set D. Here the edition follows Rylands-24:209. In the printed processionals `séquitur liber' is set EF.GF.D C.E. Here the edition follows Rylands-24:209.}

\gregorioscore{}

\emph{Chorus idem repetat post unumquemque versum cantore incipiente} ⟨Salve festa dies⟩.\footnote{1882:93.} \emph{Quotienscunque dicitur} Salve festa dies. \emph{percantetur primus ℣. a predictis clericis in medio chori antequam procedat processio.}

\gregorioscore{}

\emph{In revertendo usque ad crucem in ecclesia per idem ostium qua egressa est dicatur hec antiphona cantore incipiente ⟨hoc modo⟩.\footnote{1882:94.}}

Rylands-24:210; 1519:89v.\footnote{1519:89v. has no flats. Flats appear in 1528:83r. and in Rylands-24:210. In 1519:90r. `vobis quia illum quem quéri' appears a third lower. 1528:83v, 1530:106v, and 1544:106v. have `propter vos' in the verse. In the printed processionals `terróre' is set C.DCCACECD.D. In the printed processionals `Crucifíxum' is set G.GA.AB.A.}

\gregorioscore{}

\emph{Tres clerici de superiori gradu in cappis sericis ad populum conversi in pulpito dicant ⟨hunc⟩\footnote{1882:94.} versum ⟨modo quo sequitur⟩.\footnote{1882:94.}}

\gregorioscore{}

\emph{In introitu chori cantor incipiat antiphonam} Christus resúrgens. \emph{cum suo versu ⟨et percantetur⟩\footnote{1882:94.} a toto choro ut supra.} 200.

\emph{℣.} Surréxit Dóminus de sepúlchro.

\emph{℟.} Qui pro ⟨nobis pepéndit in ligno⟩.

\emph{⟨℣.⟩} ⟨Orémus.⟩\footnote{1901:88.}

\begin{lesson}
\subsubsection{Oratio.}

\lettrine{D}{e}us qui hodiérna die per Unigénitum tuum eternitátis nobis áditum devícta morte reserásti, vota nostra que preveniéndo aspíras~: étiam adjuvándo proséquere. Per eúndem Christum ⟨Dóminum nostrum⟩. \emph{⟨℟.⟩} ⟨Amen⟩.\footnote{1882:94.}
\end{lesson}

\emph{¶ In die pasche ad vesperas fiat processio ad fontes per ostium australe presbyterii cum oleo et chrismate, ordinata processione cum cruce et ceroferariis et thuribulariis, exinde oleum et crisma a duobus dyaconis de secunda forma qui induti sint albis, deinde puer librum ferens superpelliceo induto~: deinde executor officii~: post illum rectores secundarii~: deinde rectores principales. Nulla vero die per hanc hebdomadam precedat cereus paschalis processioni nec subsequatur secundum usum Sarum ecclesie ⟨nec⟩\footnote{1882:94.} ad vesperas nec ad matutinas.}

\emph{Rectores chori in eundo et redeundo incipiant in primis incipiant antiphonam ⟨in chorum⟩\footnote{1882:94.} ⟨incipiant antiphonam sequentem hoc modo.⟩ ⟨Imprimis vero incipiant antiphonam hoc modo, antequam procedat processio.⟩}

Rylands-24:211; 1519:90v; AS:239; Ant-1519:218v; Missal-1531:127v.\footnote{1519:91r. omits the doubling of F at the mediation throughout. In 1519:91r. the verse `Quis sicut' is set thus~:

  \gregorioscore{}

  In 1519 `érigens páuperem' is set F.F.EE C.C.D; `pópuli' is set F.E.E; `filiórum' is set F.F.F.E.}

\gregorioscore{}

\emph{Chorus ⟨totam⟩\footnote{1882:94.} prosequatur antiphonam antequam processio procedat sic.}

\gregorioscore{}

\emph{Qua finita rectores ex parte chori incipiant psalmum sequentem sic.\footnote{`\emph{hoc modo}', 1882:94.}} \gregorioscore{}

\emph{Et percantetur ab illa parte hoc modo ut sequitur hic.}

\gregorioscore{}

\emph{Hic procedat processio~: deinde dicatur alius versus ex alia parte chori hoc modo ⟨quo sequitur⟩.\footnote{1882:95.}}

\gregorioscore{}

\emph{⟨Et⟩\footnote{1882:95.} sic dicatur totus psalmus cum} Glória Patri. \emph{et} Sicut erat. \emph{repetatur j.} allelúya. \emph{post suum versiculum semel dicendo non alternando ⟨sed modo superius notato⟩\footnote{1882:95.} ut patet.\footnote{`\emph{ut prius}', 1882:95.}}

\gregorioscore{}

\emph{Quo finito reincipiatur antiphona a rectoribus chori, et percantetur a toto choro.}

\emph{Hoc modo fiat statio ad fontes. In primis\footnote{`\emph{Imprimis}', 1882:95.} cruciferarius, deinde ceroferarius, deinde thuribularius, deinde oleum et crisma~: deinde rectores secundarii, post ipsos vero tres pueri cantant} Allelúya. \emph{Ps.} Laudáte púeri.

\emph{Deinde ad gradum fontis orientalem puer ferens librum~: deinde ad gradum fontis occidentalem executor officii. Post illum vero rectores principales. Ad fontes thurificandos thuribularius ad sacerdotem accedat quo facto redeat ⟨thurificandus⟩\footnote{1528:84v.} ad stationem suam, similiter ad versum et ad orationem dicendum, accedant ceroferarii ad sacerdotem, et dicta oratione resumant locum suum. Eodem vero ordine fiat consequens statio ante crucem exceptis rectoribus secundarii, qui stabunt proximi post sacerdotem~: et exceptis tribus pueris qui cantaverunt} Allelúya. \emph{Sacerdos vero in fine psalmi} In éxitu Israel de Egípto. \emph{accedat ante cruciferarium ad thurificandum crucifixum~: quo facto redeat ad locum suum ubi\footnote{`\emph{et ibi}', 1882:96.} dicat versiculum et orationem de cruce~: et hoc modo faciat sacerdos per totam hebdomadam~: ut patet in pictura vel in statione ⟨sequente⟩\footnote{1882:96.} hoc modo.}

\emph{¶ Sequitur statio.}

\begin{figure}
\centering
\includegraphics{images/92r-statio-fontes.jpg}
\caption{⟨¶ Hoc modo fiat statio ad fontes in die pasche, ut patet in pictura, et est procendum tali modo ut sequitur.⟩}
\end{figure}


\emph{¶ Deinde tres clerici\footnote{`\emph{pueri}', 1882:96.} in ipsa statione ante fontes conversi ad altare in superpelliceis simul cantent post psalmum} Laudate púeri Dóminum. \emph{hoc modo ut sequitur.}

Rylands-24:212; 1519:92v; AS:239; Ant-1519:219v; Brev-1531:127v.\footnote{1519:92v. has no flat. In 1519:92v. `Allelúya' appears thus at the beginning~:

  \gregorioscore{}

  and thus at the end~:

  \gregorioscore{}

  In 1519:92v. `nomen' is set EFG.GDG.}

\gregorioscore{}

\emph{Chorus repetatur.\footnote{`\emph{Chorus finiat}', 1882:96.}}

\gregorioscore{}

\emph{Pueri ⟨dicant hunc versum sequentem⟩.\footnote{1528:85v.}}

\gregorioscore{}

\emph{Chorus sic respondeat ⟨ut sequitur⟩.\footnote{1528:85v.}}

\gregorioscore{}

\emph{⟨Item⟩\footnote{1882:97.} pueri dicant sic ⟨ut sequitur⟩.}\footnote{1528:85v. At this point AS:239. includes a second verse:

  \gregorioscore{}}

\gregorioscore{}

\emph{Non ulterius dicatur post repetitionem} Allelúya. \emph{sine neuma.}

\emph{Incensatis prius fontibus dicat sacerdos}

\emph{℣.} Surréxit Dóminus de sepúlchro.

\emph{℟.} Qui pro nobis pepéndit in ligno allelúya.

\emph{⟨℣.⟩} ⟨Orémus.⟩\footnote{1882:97.}

\begin{lesson}
\subsubsection{Oratio.}

\lettrine{P}{r}esta quésumus omnípotens Deus~: ut qui resurrectiónis domínice sollénnia cólimus~: ereptiónis nostre letíciam suscípere mereámur. Per eúndem Christum Dóminum nostrum. \emph{⟨Chorus respondent} Amen.⟩\footnote{1882:97.}
\end{lesson}

\emph{Nec procedat nec subsequatur} Dóminus vobíscum.

\emph{Deinde in eundo ad crucem ab omnibus rectoribus chori incipiatur antiphona sic. ⟨hoc modo.\footnote{`\emph{sequens sic dicendo}', 1882:97.⟩}}

Rylands-24:212; 1519:92v; AS:239; Ant-1519:219v; Brev-1531:127v.\footnote{It should be noted that the text setting of the first `Allelúya' differs from those that follow. Rylands-24:212. gives only a single `Allelúya', at the beginning; it is set E.GAGFE.DE.E. Perhaps in earlier days the first Allelúya' would have been treated like the incipit of an antiphon, and sung by the cantor only. Rylands-24:212. gives only the first verse of the psalm. In verse 1: Rylands-24:212. `de Egypto' is set E E.D.C. In 1519:103r. in verse 15. `gútture' is set F.F.E; verse 19. has `speravérunt'; in verse 21. `Israel' is set D.D.C; in verse 23. `a Dómino' is st D D.C.C; in verse 27. `Dómino' is set D.D.C; in the `Glória Patri' `Fílio' is set D.D.C.}

\gregorioscore{}

\emph{Chorus prosequatur ⟨sic⟩.\footnote{1882:97.}}

\gregorioscore{}

\emph{Que licet brevis sic terminetur a choro. Statim rectores ex parte decani ad chorum conversi simul incipiant ⟨hunc⟩\footnote{1882:97.} psalmum ⟨sequentem hoc modo⟩.\footnote{1882:97.}}

\gregorioscore{}

\emph{Chorus ex parte predicta ℣. prosequatur ⟨ut sequitur⟩.\footnote{`\emph{hoc modo}', 1882:97.}}

\gregorioscore{}

\emph{Hinc procedat processio~: alius ℣. ex alia parte chori dicens sic.}

\gregorioscore{}

\emph{Et hoc modo dicatur totus psalmus cum} Glória Patri. \emph{et} Sicut erat. \emph{⟨cum⟩\footnote{1882:97.} uno} Allelúya. \emph{tantum\footnote{`\emph{tamen}', 1882:97.} post unumquemque versum \&c.}

\gregorioscore{}

\emph{Hic thurificet sacerdos crucifixum.}

\gregorioscore{}

\emph{Quo finito dicat sacerdos ℣.} Dícite in natiónibus.

\emph{℟.} Quia Dóminus regnávit a ligno, allelúya.

\emph{⟨℣.⟩} ⟨Orémus.⟩\footnote{1882:99.}

\begin{lesson}
\subsubsection{Oratio.}

\lettrine{D}{e}us qui pro nobis Fílium tuum crucis patíbulum subíre voluísti~: ut inimíci a nobis expélleres potestátem concéde nobis fámulis tuis ut in resurrectiónis ejus gáudiis semper vivámus. Per eúndem Christum Dóminum nostrum. \emph{⟨℟.⟩} Amen.
\end{lesson}

\emph{Nec precedat nec subsequatur} Dóminus vobíscum.

\emph{In introitu chori dicatur antiphona de sancta Maria. Ant.} Alma redemptóris. \emph{terminata cum} allelúya. 293.

\emph{℣.} Sancta Dei génitrix ⟨virgo semper María.

\emph{℟.} Intercéde pro nobis. \emph{et cetera.⟩}.\footnote{1882:99.} 300.

\emph{⟨℣.⟩} ⟨Orémus.⟩\footnote{1882:99.}

\begin{lesson}
\subsubsection{Oratio.}

\lettrine{G}{r}átiam tuam quésumus Dómine méntibus nostris infúnde~: ut qui ángelo nunciánte Christi Fílii tui incarnatiónem cognóvimus~: per passiónem ejus et crucem ad resurrectiónis glóriam perducámur. Per eúndem ⟨Dóminum nostrum Jesum Christum Fílium. \emph{et cetera⟩.\footnote{G1882:99.}}
\end{lesson}

\chapter{¶ ⟨Feria secunda.⟩}

\emph{Feria ij. et quotidie per hanc hebdomadam fiat processio ad matutinas per medium chori ante crucifixum ad ostium occidentale~: cum accolito deferente crucem in superpelliceo, ceroferariis et thuribulario ⟨albis indutis⟩,\footnote{1882:99.} cantando antiphonam} Christus resúrgens. 201. \emph{Hac die dicatur ℣.} Dicant nunc. \emph{a duobus de superiori gradu in superpelliceis, ante introitum chori ad clerum conversi. Sequentibus vero duobus diebus a duobus clericis de ij. forma loco et habitu supradictis. Tamen in v. et vj. feria et sabbato dicatur sine ℣. Finito ℣. incensato prius crucifixo dicat sacerdos ℣.} Dícite in natiónibus. 220. \emph{Oratio.} Deus qui pro nobis. 220.

\emph{In introitu ⟨chori⟩\footnote{1528:88v.} de sancta Maria antiphona.} Ave regína celórum. 292. \emph{℣.} Post partum. Brev:⟨192⟩. \emph{Oratio.} Grátiam tuam. 220.

\emph{Ad vesperas processio fiat ad fontes eodem modo et ordine quo in die pasche ad vesperas cum oleo et chrismate cum cruce et ceroferariis et thuribulario cantando antiphonam} Sedit Angelus. 207. \emph{sine ℣. rectoribus incipientibus~: excepto quod hac die pueri non cantent} Allelúya. \emph{in statione ante fontes nec in eundo post psalmum} Laudáte púeri ⟨Dóminum⟩.\footnote{1882:100.} \emph{nec ante crucem psalmum} In éxitu ⟨Israel⟩.\footnote{1882:100.} \emph{sed finito antiphona ⟨predicta⟩\footnote{1882:100.} incensatis prius fontibus, dicat sacerdos}

\emph{℣.} Surréxit Dóminus de sepúlchro.

\emph{℟.} Qui pro nobis. \emph{⟨et cetera⟩.\footnote{1882:100.}} 214.

\emph{⟨℣.⟩} Orémus.

\begin{lesson}
\subsubsection{Oratio.}

\lettrine{C}{o}ncéde quésumus omnípotens Deus ut festa paschália que venerándo cólimus~: étiam vivéndo teneámus. Per Christum.
\end{lesson}

\emph{Deinde usque ad crucem antiphona} Christus resúrgens. 201. \emph{sine ℣. qua dicta incensato prius crucifixo dicat sacerdos versum} Dícite in natiónibus. 220. \emph{Oratio.} Deus qui ⟨pro⟩\footnote{1882:100.} nobis. 220.

\emph{In introitu chori de sancta Maria antiphona.} Anima mea liquefácta est. 299.

\emph{℣.} ⟨Sancta Dei génitrix.⟩\footnote{1882:100.} \emph{et oratio} ⟨Grátiam tuam.⟩\footnote{1882:100.} \emph{ut supra ad vesperas in die pasche.} 220.

\emph{Hic ordo servetur de processionibus in eundo et ⟨in⟩\footnote{1882:100.} introitu chori quotidie ad vesperas usque ad sabbatum cum propriis orationibus ad fontes et diversis antiphonis in introitu chori.}

\chapter{¶ Feria iij.}

\emph{Fiat processio ad matutinas ut supra.} 221.

\emph{In redeundo de sancta Maria antiphona} Beáta Dei génitrix. 291. \emph{℣. et oratio ut supra.} 220.

\emph{¶ Feria iij.\footnote{`\emph{Feria ij.}' 1519:96r.} ad vesperas ad fontes.}

\begin{lesson}
\subsubsection{Oratio.}

\lettrine{P}{r}esta quésumus omnípotens Deus~: ut per hec festa paschália que cólimus~: devóti semper in tua laude vivámus. Per Christum Dóminum nostrum.
\end{lesson}

\emph{In redeundo antiphona} Descéndi ⟨in hortum⟩. 300. \emph{⟨Versus et oratio ut supra}.⟩\footnote{1882:100.} 220.

\chapter{¶ Feria iiij.}

\emph{Ad matutinas fiat processio ut supra.} 221.

\emph{In redeundo de sancta Maria antiphona} Speciósa. 294. \emph{Versus et oratio ut supra.} 220.

\emph{Ad vespere ad fontes.}

\begin{lesson}
\subsubsection{Oratio.}

\lettrine{C}{o}ncéde quésumus omnípotens Deus~: ut qui festa paschília ágimus~: celéstibus desidériis accénsi, fontem vite sitiámus\footnote{`sentiámus', 1882:101.} Dóminum nostrum Jesum Christum Fílium tuum.\footnote{1519:96v. adds `Qui tecum.' which seems contradictory to the rubric that follows.} \emph{Non dicatur ulterius.}
\end{lesson}

\emph{In redeundo\footnote{`\emph{eundo}', 1519:96v.} de sancta Maria antiphona} Alma redemptóris mater.\footnote{1882:101.} 293. \emph{℣. et oratio ut supra.} 220.

\chapter{¶ Feria v.}

\emph{Ad matutinas fiat processio ut supra.} 221.

\emph{In redeundo antiphona} Ave regína. 292. \emph{℣. et oratio ut supra.} 220.

\emph{⟨Ad vesperas⟩\footnote{1882:101.} ad fontes.}

\begin{lesson}
\subsubsection{Oratio.}

\lettrine{D}{a} quésumus omnípotens Deus~: ut ecclésia tua et suórum firmitáte membrórum, et nova semper fecunditáte letétur. Per Christum Dóminum nostrum. \emph{⟨℟.⟩} Amen.
\end{lesson}

\emph{In revertendo de sancta Maria antiphona} Anima mea. 299. \emph{℣. et oratio ut supra.} 220.

\chapter{¶ Feria vj.}

\emph{Ad matutinas\footnote{`\emph{missam}', 1519:96v.} processio.} 221.

\emph{In revertendo antiphona} Beáta Dei génitrix. 291. \emph{℣. et oratio ut supra}. 220.

\emph{Ad vesperas ad fontes.}

\begin{lesson}
\subsubsection{Oratio.}

\lettrine{A}{d}ésto quésumus Dómine famílie tue et dignánter impénde~: ut quibus grátiam fidéi contulísti et corónam largiáris etérnam. Per Christum Dóminum nostrum. \emph{⟨℟.⟩} Amen.
\end{lesson}

\emph{In introitu chori ⟨dicatur⟩\footnote{1528:89r.} de sancta Maria antiphona} Descéndi ⟨in ortum meum⟩. 300. \emph{℣. et oratio ut supra.} 220.

\chapter{Sabbato.}

\emph{Ad matutinas fiat processio ut supra.} 221.

\emph{In revertendo antiphona} Speciósa. 294. \emph{℣. et oratio ut supra.} 220.

\emph{Ad vesperas non fiat processio ad fontes cum oleo et crismate sicut in precedentibus diebus~: sed tantum eat processio ante crucem, et exeat processio per medium chori cum ceroferariis et thuribulario ⟨albis indutis⟩\footnote{1882:101.} sine cruce, et puer librum deferens ante sacerdotem in superpelliceo~: sacerdos autem in simili habitu cum cappa serica choro sequente in superpellicio cantando antiphonam} Christus resúrgens. 201. \emph{Rectoribus chori in medio processionis in cappis sericis incipientes. Finita antiphona duo clerici de superiori gradu in superpelliciis ante ostium chori ad populum conversi, dicant versum} Dicant nunc ⟨Judéi⟩.\footnote{1882:102.} \emph{Deinde dicat sacerdos ⟨post⟩\footnote{1882:102.} thurificationem crucifixi ℣.} Dícite in natiónibus. 220. \emph{Oratio.} Deus qui pro nobis. 220.

\emph{In introitu chori dicatur una de predictis antiphonis per ordinem de sancta Maria ut supra.} ⟨Descéndi in ortum⟩. 300. \emph{℣.} Sancta Dei génitrix. 220. \emph{Oratio.} Grátiam tuam. 220.

\emph{Et hoc in omnibus sabbatis omnium ebdomadarum observetur ad vesperas in eundo et in introitu chori usque ad ascensionem Domini~: sive de dominica fit servitium sive de festo sanctorum~: nisi in festo sancte crucis ad utrasque vesperas quod si in sabbato contigerit nulla fiat processio tunc ad secundas vesperas~: et nisi quod in ceteris sabbatis dicatur versus} Dicant nunc ⟨Judéi⟩.\footnote{1882:102.} \emph{a duobus de secunda forma, loco et habitu observato~: ad clerum conversis. Proxima tamen dominica ante ascensionem Domini dicatur ℣.} Dicant nunc ⟨Judéi⟩.\footnote{1882:102.} \emph{a duobus de superiori gradu, loco et habitu predicto.}

\emph{Similiter fiat omnibus duplicibus festis in dominicis tempore predicto contingentibus. Preterea rectores chori precedant in capis nigris\footnote{`\emph{habitu non mutato}', 1882:102.} in omnibus sabbatis sequentibus nisi duplex festum fuerit.}

\chapter{¶ Dominica in octavis pasche.}

\emph{Eat processio in cappis sericis sicut in omnibus festis duplicibus~: que in dominicis contingunt per medium chori, circueundo chorum et claustrum sicut in die nativitatis}

\emph{Domini~: cantando antiphonam} Sedit ángelus. 207. \emph{Tres clerici de superiori gradu dicant ℣.} Crucifíxum in carne. \emph{in pulpito sicut in die pasche.}

\emph{In introitu chori antiphona.} Christus resúrgens. 201. \emph{cum suo versu a toto choro.}

\emph{℣.} Surréxit Dóminus. 209.

\emph{Oratio.} Deus qui per Unigénitum. 209.

\chapter{¶ ⟨Dominica prima post octavas pasche.⟩}

\emph{Dominica prima post octavas pasche, et omnibus dominicis usque ad proximam dominicam ante ascensionem Domini quando de dominica agitur ad processionem in eundo cantor incipiat sequentem antiphonam hoc modo.}

Rylands-24:223; GS:125; 1519:97v.\footnote{1519:97v. has `latus sáncea mórtuus.' 1519:97v. has no flat at `fecérunt' or at `redémptio.' In 1882:103 `fel dedérunt et in latus láncea'is omitted. 1528:90r. has all the flats. In many printed processionals 'coronátus' is set FE.FGFE.D.D. In Rylands-24:223. `resurréxi' is set A.A.G.FE. In many printed processionals `est' is set D. In the verse, in Rylands-23:223. `die' is set D.E.}

\gregorioscore{}

\emph{Duo clerici de secunda forma in superpelliciis ante introitum chori conversi ad populum dicant versum.}

\gregorioscore{}

\emph{In introitu chori antiphona} Christus resúrgens. \emph{sine versu.} 201.

\emph{℣.} Surréxit Dóminus de sepúlchro. 209.

\emph{Oratio.} Deus qui per Unigénitum. 209.

\emph{Eodem modo fiat processio omnibus diebus dominicis usque ad dominicam ante ascesionem Domini quando de dominica agitur. Quando ultimo fit servitium de}

\emph{dominica ante ascensionem Domini ad processionem antiphona} Sedit ángelus. 207. \emph{Tres clerici in pulpito de superiori gradu ad populum conversi in superpelliciis dicant ℣.} Crucifíxum.

\emph{In introitu chori antiphona.} Christus resúrgens. 201. \emph{cum suo versu a toto choro.}

\emph{℣.} Surréxit Dóminus de sepúlchro. 209.

\emph{Oratio.} Deus qui per Unigénitum. 209.

\chapter{¶ Feria ij. in rogationibus.}

\emph{Si vacaverit~: post sextam dicatur missa ⟨dominicalis scilicet⟩\footnote{1882:103.}} Vocem jocunditátis. \emph{ut supra in dominica.\footnote{`\emph{ut infra.}' 1882:103.}} 248. \emph{: nona cantata et peracta~: omnibus que ad processionem pertinent ad gradum chori ordinetur processio cum aque bajulo cum cappa sua nigra cum cruce~: ceroferariis in albis et thuribularius~: deinde capsule reliquiarum deferantur a duobus dyaconis de secunda forma habitu non mutato~: post hec dyaconus et subdyaconus cum sacerdote~: omnibus albis indutis procedant, et processione per medium chori et ecclesie ⟨incedant⟩\footnote{1882:104.} et exeat processio per ostium ecclesie occidentale et per portam claustri\footnote{'Should probably run \emph{per portam} clausi \emph{borealem. Clausum} and \emph{claustrum} are confused in the Consuetudinary. ⟨1882:xv.⟩} borealem~: ad aliquam ecclesiam in civitate, cantando antiphonas sequentes. Preterea in principio processionis deferatur dracho tribus vexillis rubeis precedentibus, ij. loco leo, iij. loco cetera vexilla~: deinde sequatur processio suo ordine eodem modo et habitu sicut dicto~: preter capsulam reliquiarum~: ita tamen quod ⟨sacerdos⟩\footnote{1882:104.} absque cappa serica incedat ⟨ut patet in statio sequens⟩.\footnote{1528:90v.}}

\begin{figure}
\centering
\includegraphics{images/98v-ordo-rogationibus.jpg}
\caption{¶ Ordo processionis in secunda feria in rogationibus.}
\end{figure}

\emph{¶ Hec antiphona sequens dicatur a toto choro in stallis ⟨antequam exeat processio, cantore incipiente⟩.\footnote{1882:105.}}

1519:99r; Rylands-24:229.

\gregorioscore{}

\emph{Non dicatur nisi primus versus sed statim sequatur} Glória Patri. 426. \emph{Deinde repetatur antiphona} Exúrge Dómine. \emph{et sic post ceteras antiphonas fiat que sequuntur. His dictis fiat processio cantore incipiente ⟨antiphonam sequentem⟩\footnote{1882:105.} hoc modo ⟨ut sequitur⟩.\footnote{1528:91v.}}

1519:99r; Rylands-24:229.\footnote{In Rylands-24:229. the final syllable of `sanctificáte' is set DCCDAGBAAG; the final syllable of `benedícite' is set GAGFE; the second syllable of `allelúya' is set GCGAGGFGACCEDDBAGBCBABAG. In 1519:99r. `pace' is set CD.CDCCD. In many printed processionals `benedícite' is set CDC.CB.AG.A.GABAGA.GAGGF. In 1525:105r. the melisma of `allelúya' begins CBC. (Compare WO-F-160:113r.).}

\gregorioscore{}

Glória Patri. 429. \emph{Repetatur antiphona.}

\emph{Alia antiphona.}

1519:99v; Rylands-24:229.\footnote{In Rylands-24:229. `monte' is set GAC.CCCA. 1519:99v. has no flat. In 1519:99v. the final syllable of `protéctio' is set GF. In Rylands-24:229. `monte' is set GAC.CCCA. In the printed processionals the second `et' is set F; `salvábitur' is set AC.C.C.CCDBCD; `allelúya' is set Ag.GCGAGGFGACCDCCBABCBABAG.GACAB.AG.}

\gregorioscore{}

\emph{⟨Et repetatur antiphona.⟩\footnote{1882:105.}}

\emph{⟨Alia antiphona.⟩}

1519:100r; Rylands-24:229.\footnote{The flat appears in Rylands-24:229. In Rylands-24:229. `Hierusalém' is set F.FE.DE.E.CEGGFDCDFGFFE. In the printed processionals `allelúya' is set thus:

  \gregorioscore{}

  In 1519:100r. the psalm tone ending is IV.i. (ending on E). The edition uses the ending provided in Rylands-24:229. which agrees with patterns of the \emph{Tonale}.}

\gregorioscore{}

\emph{⟨cum} Glória Patri. \emph{et} Sicut erat.⟩\footnote{1882:105. This could be sung as in a psalm, i.e.~in two verses, or as in an officium, i.e.~a single verse, as follows:

  \gregorioscore{}}

\emph{Si necesse fuerit ad pluviam postulandam cantentur hec sequens antiphonam, sinautem non dicitur.}

1519:100r; Rylands-24:229.\footnote{In the printed processionals `Dómine' is set D.A.CDED; `plúviam' is set FF.DE.DDC; `terre' is set CECD.A. In Bodleian Library MS. Rawl. liturg. d.~4:96v. `Dómine' is set D.A.CDE. In 1519:100r. `plúviam' is set FF.DEDC.C.}

\gregorioscore{}

⟨Glória Patri. Sicut erat. 426. \emph{Et repetatur antiphona.⟩\footnote{1882:105.}}

\emph{⟨Alia antiphona.⟩}

1519:100v; 1525:106v.\footnote{In Bodleian Library MS. Rawl. Liturg. d.~4:96v. `géntium' is set F.FGF.F; `nobis' is set AAG.AGAFGAG. In 1525:106v. `géntium' is set EF.F.EGFF; `dare' is set A.GA. In most printed processionals `nisi tu' is set FD.FE FG. Here the edition follows 1525:106v. (and WO-F-160:231). In most printed processionals `nobis' is set AAG.GAGF. In 1555:unpaged. `nobis' is set AAG.FAGFGAG. The edition follows 1525:106v.}

\gregorioscore{}

⟨Glória Patri. Sicut erat. 427. \emph{Et repetatur antiphona.⟩\footnote{1882:106.}}

\emph{⟨Alia antiphona.⟩}

1519:101r; 1525:107r.\footnote{In most printed processionals the underlay differs at `plúviam \ldots{} dedísti' thusly:

  \gregorioscore{}

  In 1555:unpaged. `super' is set DCC.A.}

\gregorioscore{}

⟨Glória Patri. Sicut erat. 427.

\emph{Alia antiphona.⟩\footnote{1882:106.}}

1519:101v; 1525:107v.\footnote{Bodleian Library MS. Rawl. Liturg. d.~4:97v. omits the flat at `plantávit', but includes the flat at `allelúya.' In 1525:107v. `rúgiunt' is set F.F.FFF; `miserére' is set A.ACA.BC.C. In 1519:101v.'plúviam' is set AGA.FD.EFEF. The flat at `allelúya' appears in 1525:107v.}

\gregorioscore{}

\emph{Ad ⟨hanc⟩\footnote{1882:106.} antiphonam non sequatur psalmus.}

\emph{¶ Pro serenitate aeris ⟨dicuntur he sequentes antiphone⟩.\footnote{1882:106.}}

1519:102r; Rylands-24:230.\footnote{In the printed processionals `Inundavérunt' is set D.D.D.DFGFF.DEFEC; `cápita' is set FGAG.GFFED.DCDEDCD; `tuum' is set AC.AB♭A; `fáciem' is set GA.FED.DG. In Rylands-24:230. `lacu' is set GD.DCBA. Bodleian Library MS. Rawl. Liturg. d.~4:98r. has no flats. In Rylands-24:230. `a' is set D. In Rylands-24:230. the division of the psalm verse is at `aque.'}

\gregorioscore{}

⟨Glória Patri. Sicut erat. 426. \emph{Et repetatur antiphona.}

\emph{Alia antiphona.⟩\footnote{1882:106.}}

1519:102r.\footnote{In 1519:102r. `neque' is set BCED.DEC; `líbera' is set GCC.CD.CD. The edition follows 1508:unpaged and Bodleian Library MS. Rawl. Liturg. d.~4:98v. here.}

\gregorioscore{}

⟨Glória Patri. Sicut erat. 428.⟩\footnote{1882:106.}

\emph{¶ Contra mortalitatem hominum tempore belli dicatur hec antiphona.}

1519:102v; Rylands-24:230.\footnote{In Rylands-24:230. `mortis' is set ACGA.GF. In the printed processionals `plebem' is set EFG.GA.}

\gregorioscore{}

\emph{⟨cum} Gloria Patri. 427.⟩\footnote{1882:107.}

\emph{⟨Contra hostum impugnacionem.}
% TODO: chk

\gregorioscore{003400-invocantes-dominum}

\emph{Pro quacumque tribulacione.}

\gregorioscore{004219-parce-domine}⟩\footnote{Rylands-24:230.}

\emph{His dictis dicant septem. psalmos penitentiales quantum\footnote{`\emph{si tantum}', 1882:107.} restat iter~: cum} Glória Patri. \emph{et} Sicut erat. \emph{et antiphona} Ne reminiscáris. \emph{Require hos psalmos cum antiphona in quarta feria in capite jejunii\footnote{`\emph{in fine libri}', 1882:107.}} 56. \emph{: et dicatur cum hac letania et ℣. et oratione sequente~: et hec omnia sine nota dicantur~: ita tamen quod quicquid dicat sacerdos de letania, chorus idem repetat plene et integre usque ad prolationem} Peccatóres te rogámus audi nos. \emph{tunc enim post} Ut pacem nobis dones. \emph{Chorus respondent,} Te rogámus audi nos. \emph{Et ⟨sic⟩\footnote{1882:107.} post singulos ℣. usque ad} Fili Dei. Te rogámus ⟨audi nos⟩.\footnote{1882:107.}

\begin{lesson}
\subsubsection[⟨Sequitur⟩ Letania.]{⟨Sequitur⟩\footnote{1528:95v.} Letania.}

\lettrine{K}{y}rieléyson. Christeléyson. Christe audi nos.

Pater de celis Deus. Miserére nobis.

Fili Redémptor mundi Deus. Miserére nobis.

Spíritus Sancte Deus. Miserére nobis.

Sancta Trínitas unus Deus. Miserére nobis.

Sancta María. ora.

Sancta Dei génitrix. ora.

Sancta virgo vírginum. ora.

Sancte Míchael. ora.

Sancte Gábriel. ora.

Sancte Ráphael. ora.

Omnes sancti ángeli et archángeli Dei.\footnote{1882:107. omits `Dei.'} Oráte pra nobis.

Omnes sancti beatórum spirítuum órdines. Oráte pro.

Sancte Johánnes baptísta. ora.

Omnes sancti patriárche et prophéte.

Oráte pro nobis.

Sancte Petre. ora.

Sancte Paule. ora.

Sancte Andréa. ora.

Sancte Johánnes. ora.

Sancte Jacóbe. ora.

Sancte Thoma. ora.

Sancte Philíppe. ora.

Sancte Jacóbe. ora.

Sancte Mathée. ora.

Sancte Bartholomée. ora.

Sancte Simon. ora.

Sancte Thadée. ora.

Sancte Mathía. ora.

Sancte Bárnaba. ora.

Sancte Marce. ora.

Sancte Luca. ora.

Omnes sancti apóstoli et evangelíste. Oráte ⟨pro nobis⟩.\footnote{1882:108.}

Omnes sancti discípuli Dómini et innocéntes. Oráte pro nobis.

Sancte Stéphane. ora.

Sancto Line. ora.

Saucte Clete. ora.

Sancte Clemens. ora.

Sancte Fabiáne. ora.

Sancte Sebastiáne. ora.

Sancte Cosma. ora.

Sancto Damiáne. ora.

Sancte Prime. ora.

Sancte Feliciáne. ora.

Sancto Dionísi cum sóciis tuis. Oráte\footnote{`Ora.' 1882:108.} ⟨pro nobis⟩.\footnote{1528:95v.}

Sancte Victor cum sóciis tuis. Oráte.\footnote{`Ora.' 1882:108.}

Omnes sancti mártyres.

Oráte ⟨pro nobis⟩.

Sancte Silvéster. ora.

Sancte Leo. ora.

Sancte Hierónime. ora.

Sancte Augustíne. ora.

Sancte Ysidóre. ora.

Sancte Juliáne. ora.

Sancte Medárde. ora.

Sancte Gildárde. ora.

Sancte Albíne. ora.

Sancte Eusébi. ora.

Sancte Svvythún. ora.

Sancte Biríne. ora.

Omnes sancti confessóres. oráte.

Omnes sancti monáchi et heremíte. Oráte pro nobis.

Sancta María Magdaléna. ora.

Sancta María Egýptia. ora.

Sancta Margaréta. ora.

Sancta Scolástica. ora.

Sancta Petronílla. ora.

Sancta Genovéfa. ora.

Sancta Praxédis. ora.

Sancta Sothéris. ora.

Sancta Prisca. ora.

Sancta Tecla. ora.

Sancta Affra. ora.

Sancta Edítha. ora.

Omnes sancte vírgines. oráte.

Omnes sancti. Oráte pro.

Propítius esto. Parce nobis Dómine.

Ab omni malo. Líbera nos Dómine.

Ab insídiis diáboli. Líbera nos.

A damnatióne perpétua. Líbera.

Ab imminéntibus peccatórum nostrórum perículis. Líbera.

Ab infestatiónibus demónum. Líbera.

A spíritu fornicatiónis. Líbera.

Ab appetítu inánis glórie. Líbera.

Ab ⟨omni⟩\footnote{1882:109; 1531:49r.} immundítia mentis et córporis. Líbera.

Ab ira et ódio et omni mala voluntáte. Líbera.

Ab immúndis cogitatiónibus. Líbera.

A cecitáte cordis. Líbera.

A fúlgure et tempestáte. Líbera.

A subitánea et improvísa morte. Líbera.

Per mystérium sancte incarnatiónis tue. Líbera.

Per nativitátem tuam. Líbera.

Per circumcisiónem tuam. Líbera.

Per baptísmum tuum. Líbera.

Per jejúnium tuum. Líbera.

Per crucem et passiónem tuam. Líbera.

Per pretiósam mortem tuam. Líbera.

Per gloriósam resurrectiónem tuam. Líbera.

Per admirábilem ascensiónem tuam. Líbera.

Per grátiam Sancti Spíritus Paráclyti. Líbera.

In hora mortis. Succúrre nobis Dómine.\footnote{`In hora mortis \ldots{} Domine' usually does not appear in the Roman Breviaries (however it is to be found in the Pontificale Romanum Distributum Clementis 8 (Rome:1738) Vol 1:166.). The sources are not uniform in either punctuation or capitalization, ranging from `In hora mortis succúrre nobis Dómine.' through to `In hora mortis. Succúrre nobis Dómine.' No indication of the musical setting is provided in the sources.}

In die judícii. Líbera.

Peccatóres. Te rogámus audi nos.

Ut pacem nobis dones. Te rogámus audi nos.

Ut misericórdia et píetas tua nos custódiat. Te rogámus.

Ut ecclésiam tuam cathólicam régere et defensáre dignéris. Te rogámus.

Ut domnum apostólicum et omnes gradus ecclésie in sancta\footnote{`sanctam', 1519:104r.} religióne conserváre dignéris. Te rogámus.

Ut regibus et princípibus nostris pacem et veram concórdiam atque victóriam donáre dignéris. Te rogámus.

Ut epíscopos et abbátes nostros in sancta religióne conserváre dignéris. Te rogámus.

Ut congregatiónes ómnium sanctórum morum in tuo sancto servítio conserváre dignéris. Te rogámus.

Ut cunctum pópulum Christiánum precióso sánguine tuo redémptum conserváre dignéris. Te rogámus.

Ut ómnibus benefactóribus nostris sempitérna bona retríbuas. Te rogámus.

Ut ánimas nostras et paréntum nostrórum ab etérna damnatióne erípias. Te rogámus.

Ut fructus terre dare et conserváre dignéris. Te rogámus.

Ut óculos misericórdie tue super nos redúcere dignéris. Te rogámus.

Ut obséquium servitútis nostre rationábile fácias. Te rogámus.

Ut mentes nostras ad celéstia desidéria érigas. Te rogámus.

Ut misérias páuperum et captivórum intúeri et releváre dignéris.

Te rogámus audi nos.

Ut ómnibus fidélibus defúnctis réquiem etérnam dones. Te rogámus.

Ut nos exaudíre dignéris. Te rogámus.

Fili Dei. Te rogámus.

Agnus Dei qui tollis peccáta mundi. Exáudi nos Dómine.

Agnus Dei qui tollis peccáta mundi. Parce nobis Dómine.

Agnus Dei qui tollis peccáta mundi. Miserére nobis.

Kyrieléyson. Christeléyson. Kyrieléyson.

Pater noster.

Et ne nos indúcas ⟨in tentatiónem⟩.\footnote{1882:111.} Sed líbera nos ⟨a malo⟩.\footnote{1882:111.}

Osténde nobis Dómine misericórdiam tuam. Et\footnote{1882:111. omits `Et.'} salutáre tuum da nobis.

Peccávimus cum pátribus nostris. Injúste égimus ⟨iniquitátem fécimus⟩.\footnote{1882:111.}

Dómine non secúndum peccáta ⟨nostra fácias nobis⟩.\footnote{1882:111.} Neque secúndum ⟨iniquitátes nostras retríbuas nobis⟩.\footnote{1882:111.}

Orémus pro omni gradu ⟨ecclésie⟩.\footnote{1882:111.} Sacerdótes tui ⟨induántur justítiam et sancti tui exúltent⟩.\footnote{1882:111.}

Pro frátribus ⟨et soróribus nostris⟩.\footnote{1882:111.} Salvos fac servos tuos ⟨et ancíllas tuas Deus meus sperántes in te.⟩\footnote{1882:111.}

Pro cuncto pópulo Christiáno. Salvum fac pópulum ⟨tuum Dómine et bénedic hereditáti tue~: et rege eos et extólle illos usque in etérnum⟩.\footnote{1882:111.}

Dómine fiat pax ⟨in virtúte tua⟩.\footnote{1882:111.} Et habundántia ⟨in túrribus tuis⟩.\footnote{1882:111.}

Anime famulórum famularúmque tuárum requiéscant in pace. Amen.

Dómine exáudi oratiónem meam. Et clamor meus ad te veniat.

Dóminus vobíscum. Et cum spíritu tuo.

Orémus.

\subsubsection{Oratio.}

\lettrine{D}{e}us cui próprium est miseréri semper et párcere súscipe deprecatiónem nostram, et\footnote{`ut', 1882:111.} quos delictórum cáthena constríngit~: miserátio tue piétatis absólvat. Per Christum Dóminum nostrum.

\subsubsection[⟨Oratio.⟩]{⟨Oratio.⟩\footnote{1882:111.}}

\lettrine{O}{m}nípotens sempitérne Deus~: qui facis mirabília magna solus, preténde super fámulos tuos pontífices~: et super cunctas congregatiónes illis commíssas spíritum grátie salutáris~: et ut in veritáte tibi compláceant, perpétuum eis rorem tue benedictiónis infúnde.

\subsubsection[⟨Oratio.⟩]{⟨Oratio.⟩\footnote{1882:111.}}

\lettrine{D}{e}us qui charitátis dona per grátiam Sancti Spíritus tuórum córdibus fidélium infúndis~: da fámulis et famulábus tuis frátribus et soróribus nostris pro quibus tuam deprecámur cleméntiam salútem mentis et córporis~: ut te tota virtúte díligant~: et que tibi plácita sunt tota dilectióne perfíciant.

\subsubsection[⟨Alia oratio.⟩]{⟨Alia oratio.⟩\footnote{1882:112.}}

\lettrine{D}{e}us a quo sancta desidéria recta consília ⟨et justa sunt ópera, da servis tuis illam quam mundas dare non potest pacem~: ut et corda nostra mandátis tuis dédita, et hóstium subláta formídire~: témpora sint tua protectióne tranquílla⟩.\footnote{1882:112. 1519:105r. has `\emph{\&c.}'}

\lettrine{I}{n}effábilem misericórdiam tuam nobis quésumus Dómine cleménter osténde~: ut simul nos ⟨et⟩\footnote{I1882:112.} a peccátis ⟨ómnibus⟩\footnote{1882:112.} éxuas et a penis quas pro his merémur benígnus erípias.

\lettrine{F}{i}délium Deus ómnium Cónditor et Redémptor animábus ⟨famulórum famularúmque tuárum remissiónem cunctórum tribus peccatórum~: ut indulgéntiam quam semper optavérunt piis supplicatiónibus consequántur⟩.\footnote{1882:112. 1519:105r. has `\emph{\&c.}'}

\lettrine{P}{i}etáte tua quésumus Dómine nostrórum solve víncula ómnium delictórum~: et intercedénte beáta et gloriósa sempérque vírgine Dei genitríce María~: cum ómnibus sanctis tuis~: dómnum papam,\footnote{In Gough Missals 75 `papam' has been deleted.} reges et príncipes, epíscopos et abbátes nostros, et omnem plebem illis commíssam, nosque fámulos tuos, atque loca nostra in omni sanctitáte custódi omnésque consanguinitáte ac familiaritáte, vel confessióne et oratióne nobis junctos~: seu omnem pópulum cathólicum a vítiis purga virtútibus illústra, pacem et salútem nobis tríbue, hostes visíbiles et invisíbles rémove~: famem et pestem repélle~: amícis et inimícis nostris veram charitátem~: atque infírmis sanitátem largíre~: et iter famulórum tuórum in salútis tue prosperitáte dispóne~: et ómnibus fidélibus vivis ac defúnctis in terra vivéntium vitam et réquiem etérnam concéde. Per eúndem Christum Dóminum nostrum. \emph{⟨℟.⟩} ⟨Amen.⟩\footnote{1882:112.}
\end{lesson}

\emph{Et si transierit processio per ecclesiam dicatur responsorium cum suo ℣. antiphona cum versiculum et oratione de sancto ⟨de⟩\footnote{1882:112.} quo est ecclesia illa~: ita quod ad januas cimiterii vel citius inchoetur non ibi processio circa cimiteriam sed directe in ecclesia. Cum autem pervenerit in ecclesia ubi fieri debet statio~: et finita antiphona vel responsorio, dicat sacerdos ℣. et orationem de sancto de quo est ecclesia illa, deinde sequantur preces in prostratione cum dicitur.} Kyrieléyson. Christeléyson. Kyrieléyson. Pater noster. \emph{\&c.~ut prenotatum est in processione feriali quadragesime.} 82.

\emph{¶ His finitis incipiatur missa de jejunio, ⟨cantore incipiente⟩\footnote{1528:97v.} hoc modo ⟨ut sequitur⟩.\footnote{1528:97v.}}

1519:105v; GS:132; Grad-1508:128r; Missal-1513:97r.\footnote{In 1519:105v. part of the music for the first `allelúya' is missing because of the deletion of `papam' on the recto. In Grad-1508:128v. `sancto suo' is set A.AGAG GEF.FE. In Grad-1508:128v. the second `ejus' is set GFFEE.ED. GS:132. has no flat. GS:132. gives the beginning of the Psalm-Verse followed by the Psalm-Tone ending together with the word `Amen.'}

\gregorioscore{} \gregorioscore{}

\begin{lesson}
\subsubsection{Oratio.}

\lettrine{P}{r}esta quésumus omnípotens Deus ut ⟨qui⟩\footnote{1882:113.} in afflictióne nostra de tua pietáte confídimus~: contra ómnia advérsa\footnote{`advérsa ómina', 1882:113.} tua semper protectióne muniámur. Per Dóminum ⟨nostrum Jesum Christum Fílium tuum. \emph{et cetera.⟩\footnote{1882:113.}}
\end{lesson}

\emph{Secunda oratio de sancto de quo est ecclesia illa, iij. de omnibus sanctis ⟨scilicet⟩\footnote{1882:113.} Oratio.} Concéde quésumus omnípotens Deus. \emph{⟨\&c.~ut supra⟩.\footnote{1882:113.}} 222.

\begin{lesson}
\subsubsection[Lectio epistole beati Jacobi apostoli. ⟨ultimo capitulo⟩. (v. 16--20.)]{Lectio epistole beati Jacobi apostoli. ⟨ultimo capitulo⟩.\footnote{1882:113.} (v. 16--20.)}

\lettrine{C}{h}aríssimi. Confitémini alterútrum peccáta vestra~: et oráte pro ínvicem ut salvémini. Multum enim valet~: deprecátio justi assídua. Helýas homo erat símilis nobis passíbilis~: et oratióne orávit ut non plúeret super terram et non pluit annos tres et menses sex. Et rursum orávit, et celum dedit plúviam~: et terra dedit fructum suum. Si quis autem ex vobis erráverit a veritáte et convérterit quis eum, scire debet quóniam qui convérti fécerit peccatórem ab erróre vite sue salvábit ánimam ejus a morte. Et óperit multitúdinem peccatórum.
\end{lesson}

1519:106r; GS:132; Grad-1508:128v; Missal-1513:97r.\footnote{In Grad-1508:128v. `Allelúya.' is set F.G.AC.CCCDBCAG; the final note of the jubilus is A; `confitémini' is set G.G.C.CC.CCAG; `misericórdia' is set AF.G.A.ACC.C.AGG; the final melisma is BABDBAGABCAAG. GS:132. omits the last four notes. 1519:106r. `Dómino' is set BG.ACBC.C; the second syllable of `ejus' appears 2 notes later.}

\gregorioscore{}

\emph{⟨Diaconus legat evangelium sic dicendo.} Sequéntia sancti evangélii secúndum Lucam. \emph{Chorus respondeat} Glória tibi Dómine.⟩\footnote{1882:113.}

\begin{lesson}
\subsubsection{¶ Evangelium secundum Lucam. (xj. 5--13.)}

\lettrine{I}{n} illo témpore. Dixit Jesus discípulis suis, Quis vestrum habébit amícum, et ibit ad illum média nocte~: et dicet illi, Amíce~: commóda michi tres panes quóniam amícus meus venit de via ad me~: et non hábeo quod ponam ante illum. Et ille deíntus respóndens dicat, Noli michi moléstus esse~: jam óstium clausum est~: et púeri mei mecum sunt in cúbili~: non possum súrgere et dare tibi. Et ille si perseveráverit pulsans~: dico vobis et si non dabit illi surgens eo quod amícus ejus sit, propter improbitátem tamen ejus surget~: et dabit illi quotquot habet necessários~: et ego vobis dico, Petíte et dábitur vobis. Quérite et inveniétis. Pulsáte~: et aperiétur vobis. Omnis enim qui petit áccipit~: et qui querit invénit~: et pulsánti aperiétur. Quis autem ex vobis patrem petit panem, nunquid lápidem dabit illi~? Aut piscem~: nunquid pro pisce serpéntem dabit illi~? Aut si petíerit ovum~: nunquid pórriget illi scorpiónem~? Si ergo vos cum sitis mali~: nostis bona data dare fíliis vestris~: quanto magis Pater vester de celo~: dabit Spíritum bonum peténtibus se~?
\end{lesson}

1519:106v; GS:132; Grad-1508:128v; Missal-1513:97v.\footnote{GS:132. has a flat signature until `fáceret.' In GS:132. and Grad-1508:128v. `multórum' is set G.GACB(♭)C.GA; the edition follows Rylands-24:231. GS:132. has `salvum.' 1519. has no flat at `fáceret.' In Grad-1508:128v. the text underlay at `fáceret' is unclear; the second syllable of `persequéntibus' is set GCDB♭CA; `persequéntibus' has a flat throughout; `ánimam' is set AG.CDC.CCCA. In Rylands-24:231. `salvam' is set ACCC.A. Grad-In 1508:128v. `allelúya' is set thus:

  \gregorioscore{}}

\gregorioscore{}

\emph{¶ Hic fiat sermo ad populum si placet.}

\emph{⟨Postea cantetur communio sequens.⟩\footnote{1882:114.}}

1519:107r; GS:133; Grad-1508:129r; Missal-1513:97v.\footnote{In Grad-1508:129r. `aperiétur' is set F.F.FE.DC.DFEF. `et pulsánti', Missal-1513:97v; Dickinson:409.}

\gregorioscore{}

\emph{¶ Feria iij. dicatur magna missa de sancta Maria~: nisi\footnote{1882:114. `\emph{ubi}', 1519:107r.} aliquod festum cum regimine chori contigerit, et tunc missa} Salus pópuli. Missale:⟨194⟩. \emph{dicatur in processione ubi fit statio et in vigilia ascensionis Domini dicatur missa de pace in processione. Tamen si festum alicujus sancti cum regimine chori in ij. feria contigerit tunc dicatur missa dominicalis in iij. feria, vel in vigilia ascensionis Domini in processione ubi fit statio. Similiter quando festum cum regimine chori in vigilia ascensionis Domini contigerit tunc dicatur missa de vigilia ubi fit statio et semper in his predictis tribus diebus dicatur missa in choro post sextam et post illam missam dicatur ⟨hora⟩\footnote{1882:114.} nona~: et tunc eat processio ⟨ad aliquam ecclesiam ut supra dictum est⟩.\footnote{1882:114.}}

\chapter{¶ Missa dominicalis. Officium.}

1519:107v; GS:130; Grad-1508:127v; Missal-1513:96v.\footnote{In 1515:107v. the firt `allelúya' is set AC.AG.GEFEE.ED. 1519:107v. has no flat at `allelúya.' In 1519:107v. `laudi ejus' is set A.A B.GA. In Grad-1508:127v. `annunciáte' is set AC.C.CCCA.A.GAGGEFE. Grad-1508:127v. has `annunciáte usque'; `annunciáte' is set FE.FF.DGE.EG.GA. Grad-In 1508:127v. the second `allelúya' is set AC.AG.GEFEE.ED; the final allelúya' begins DEGA.AGC etc. GS:130. has no flat. GS:130. gives the beginning of the Psalm-Verse together with the Psalm=Tone ending and the word `Amen.'}

\gregorioscore{}

\begin{lesson}
\subsubsection{Oratio.}

\lettrine{D}{e}us, a quo cuncta bona procédunt, ⟨largíre supplícibus tuis, ut cogitémus, te inspiránte, que recta sunt, et te gubernánte eádem faciámus. Per Dóminum.⟩\footnote{1882:115.}
\end{lesson}

\emph{Secunda oratio de sancto de quo est ecclesia illa.}

\emph{⟨Tertia de omnibus sanctis scilicet,} Concéde, quésumus, omnípotens Deus.⟩\footnote{1882:115.} 222.

\begin{lesson}
\subsubsection[Lectio epistole beati Jacobi, apostoli ⟨primo capitulo⟩. (j. 22--27.)]{Lectio epistole beati Jacobi, apostoli ⟨primo capitulo⟩.\footnote{1882:115.} (j. 22--27.)}

\lettrine{C}{h}aríssimi, Estóte factóres verbi et non auditóres tantum, falléntes nosmetípsos~: quia si quis audítor est verbi et non factor~: hic comparátur viro consideránti vultum nativitátis sue in spéculo. Considerávit enim se et ábiit~: et statim oblítus est qualis fúerit. Qui autem prospéxerit in lege perfécte libertátis et permánserit in ea~: non audítor obliviósus factus sed factor óperis~: hic beátus in facto suo erit. Si quis autem putat se religiósum esse non refrénans linguam suam sed sedúcens cor suum, hujus vana est relígio. Relígio munda et immaculáta apud Deum et Patrem hec est~: visitáre pupíllos et víduas in tribulatióne eórum. Et immaculátum se custodíre ab hoc século.
\end{lesson}

1519:108r; GS:130; Grad-1508:127v; Missal-1513:96v.\footnote{The flat appears in GS:130. In Rylands-24:228. `nómine' is set E.G.GA. No flat appears in Grad-1508:127v. In Grad-1508:127v. `accipiétis' is set as follows:

  \gregorioscore{}

  GS:131. omits last 16 notes of the final melisma.}

\gregorioscore{}

\emph{⟨Evangelium.} Sequéntia sancti evangélii secúndum Johánem, décimo sexto capítulo. Glória tibi Dómine.⟩\footnote{1882:115.}

\begin{lesson}
\subsubsection{¶ Secundum Johannem. (xvj. 23--30.)}

\lettrine{I}{n} illo témpore. Dixit Jesus discípulis suis, Amen amen dico vobis~: si quid petiéritis Patrem in nómine meo~: dabit vobis. Usque modo non petístis quicquam in nómine meo. Pétite et accipiétis~: ut gáudium vestrum sit plenum. Hec in provérbiis locútus sum vobis. Venit hora cum jam non in provérbiis loquar vobis~: sed palam de Patre meo annunciábo vobis. Illo die in nómine meo petétis. Et nunc\footnote{`non', \emph{Vulgate,} 1555.} dico vobis, quia ego rogábo Patrem de vobis. Ille enim\footnote{`Ipse enim', \emph{Vulgate,} 1555.} Pater amat vos quia vos me amástis et credidístis quia ego a Deo exívi. Exívi a Patre et veni in mundum~: íterum relínquo mundum~: et vado ad Patrem. Dicunt ei discípuli ejus, Ecce nunc palam lóqueris~: et provérbium nullum dicis. Nunc scimus quia scis ómnia~: et non opus est tibi ut quis te intérroget. In hoc crédimus~: quia a Deo exísti.
\end{lesson}

1519:108v; GS:131; Grad-1508:128r; Missal-1513:97r.\footnote{In 1519:108v. `posuit' is set F.FFAG.GAFFDED; `pedes' is set D.CFF; `amóvit' is set F.FFDFC.AC. In Grad-1508:128r. `pósuit' is set FF.FAGGAFF.FDED.}

\gregorioscore{}

\emph{⟨Postea cantetur communio sequens.⟩}

1519:108v; GS:131; Grad-1508:128r; Missal-1513:97r.⟩\footnote{In 1519:108v. `Dómino' is set FA.FE.FGF; `die' is set FE.E. `et benedícite', Missal-1513:97r; Dickinson:406. Grad-1508:128r. has no flat.}

\gregorioscore{}

\chapter{Missa de vigilia ascensionis Domini.}

1519:109r; GS:133, 148; Grad-1508:129r, 152r; Missal-1513:97v.\footnote{In 1519:109r. `Omnes gentes' is set FFFDCDFAGA.GF AGA.GFGFD; `voce exultatiónis' is set F.FFF; FF.DF.C.FF.FAGAFGGF.F; 1519 the first `allelúya' is set D.FF.FGF.FEFDDC. 1519:109r. has no flat. In Grad-1508:129r. `Omnes gentes' is set FFFEDDFAGA.GF AGA.GFGFE; `voce exultatiónis' is set FF.FF F.FCF.D.FF.FAGAFG.GFF. In GS:133. the final `allelúya' begins on C. GS:133. has no flat. GS:133. gives the incipit of the Psalm verse followed by the Psalm tone ending together with the word `Amen.' In Grad-1508:152r. `voce' is set FF.FF; the fourth syllable of `exultatiónis' appears one note earlier. Grad-1508:152r. has no flat in the Psalm verse. GS:133. gives the beginning of the Psalm verse followed by the Psalm tone ending together with the word `Amen.'}

\gregorioscore{}

\begin{lesson}
\subsubsection{Oratio.}

\lettrine{P}{r}esta quésumus omnípotens ⟨Pater, ut nostras mentis inténtio, quo Unigénitus Fílius tuus Dóminus noster ventúre solennitátis gloriósus auctor ingréssus est, semper inténdat~; et quo fide pergit conversatióne pervéniat. Per.
\end{lesson}

\emph{Secunda oratio de sancto, de quo est ecclcsia illa. Tertia de omnibus sanctis.} Concéde, quésumus, omnípotens Deus. \emph{ut supra.⟩\footnote{1882:116.}}

\begin{lesson}
\subsubsection{Lectio Actuum Apostolorum iv. ⟨32--35.⟩}

\lettrine{I}{n} diébus illis. Multitúdinis autem credéntium erat cor unum et ánima una. Nec quisquam ⟨eórum⟩\footnote{1882:116; 1513:97v.} que possidebat áliquid suum esse dicébat~: sed erant illis ómnia commúnia. Et virtúte magna reddébant apóstoli testimónium resurrectiónis Jesu Christi Dómini nostri~: et grátia magna erat in ómnibus illis. Neque enim quisquam egens erat inter illos. Quotquot enim possessóres agrórum aut domórum erant~: vendéntes afferébant précia eórum que vendébant~: et ponébant ante pedes apostolórum. Dividebátur autem síngulis~: prout cuíque opus erat.
\end{lesson}

1519:109v; GS:133; Grad-1508:129r; Missal-1513:97v.\footnote{In GS:133. `gentes' is set thus:

  \gregorioscore{}

  In Grad-1508:129r. `pláudite' is set thus:

  \gregorioscore{}

  In 1519:109v. the syllable `-ya.' in `Allelúya' comes two notes earlier; `pláudite' is set thus~:

  \gregorioscore{};

  in `mánibus' the syllable `-ni-' appears two notes earlier. `in voce' is set thus~:

  \gregorioscore{}

  Grad-1508:129r. has no flats. 1519:109v. has no flats. In Grad-1508:129r. the second syllable of `mánibus' comes two notes later. In Grad-1508:129r. the second syllable of `voce' appears five notes later. GS:133. omits the last 35 notes of the final melisma.}

\gregorioscore{}

\begin{lesson}
\subsubsection{Evangelium secundum Johannem. ⟨xvij. 1.-11.⟩}

\lettrine{I}{n} illo témpore. Sublevátis Jesus óculis in celum dixit, Pater venit hora~: clarífica Fílium tuum, ut et Fílius tuus claríficet te. Sicut dedísti ei potestátem omnis carnis~: ut omne quod dedísti ei det eis vitam etérnam. Hec est autem vita etérna~: ut cognóscant te solum verum Deum~: et quem misísti Jesum Christum. Ego te clarificávi super terram~: opus consummávi quod dedísti michi ut fáciam. Et nunc clarífica me tu Pater apud temetípsum~: claritáte quam hábui priúsquam mundus esset apud te. Magnifestávi\footnote{`Magnificávi', 1519:110r; 1523:110r. The later editions have `Manifestávi.'} nomen tuum homínibus quos dedísti michi de mundo. Tui erant, et michi eos dedísti et sermónem tuum servavérunt. Nunc cognovérunt quia ómnia que dedísti michi abs te sunt~: quia verba que dedísti michi dedi eis. Et ipsi accepérunt, et cognovérunt vere quia misísti~: credidérunt quia a te exívi.\footnote{1554:115v. and 1555 have the \emph{Vulgate}~: `et cognovérunt vere quia a te exívi, et credidérunt quia tu me misísti.'} Ego pro eis rogo. Non pro mundo rogo~: sed pro his quos dedísti michi~: quia tui sunt. Et ómnia mea tua sunt, et tua mea sunt~: et clarificátus sum in eis. Et jam non sum in mundo~: et hii in mundo sunt et ego ad te vénio.
\end{lesson}

1519:110r; GS:133; Grad-1508:129v; Missal-1513:98r.\footnote{1519:110r. has no flats; `admirámini' is set F.F.FG.GFFD.D'; `celum' is set thus~:

  \gregorioscore{};

  Jesus is set G.FEGFDEC; `est' is set F; `ascendéntem' is set thus~:

  \gregorioscore{};

  `celum' is set GEDFEFD.DFDDC.

  In GS:133. `Galiléi' is set F.GAG.AFAG.GAFEFEDED. In Grad-1508:129v. `aspiciéntes' begins DF. In GS:133. `celum' is set thus:

  \gregorioscore{}

  In Grad-1508:129v. `Jesus' is set G.FEGFDFC. In GS:133. `Jesus' is set G.FEGFDEC. Here the edition follows Rylands-24:233. In Grad-1508:129v. the first syllable of `assúmptus' is set C. In GS:133. the second `celum' appears to be set FFFDDDGFFDFC.C. Grad-In 1508:129v. the second `celum' appears to be set FFDDDGFFDFC.C. Here the edition follows Rylands-24:233.

  In Grad-1508:129v. `ascendéntem' is set thus:

  \gregorioscore{}

  In Grad-1508:129v. the last `celum' is set GFDFEFD.DFDDC.}

\gregorioscore{}

1519:110v; GS:134; Grad-1508:129v; Missal-1513:98r.\footnote{1519:110v. has no flat. In 1519:110v. `eis' is set ACBG.AGABA; `servábam' is set F.FG.G. In Grad-1508:129v. `eis' is set G.ABA. In GS:134. `vénio' is set CBC.CACG.G. GS:134. has no flats. The flat at `serves' appears in Rylands-24:233.}

\gregorioscore{}

\emph{¶ Cantata missa statim si duplex festum illa die contigerit tres clerici de superiori gradu in medio processionis cantent letaniam in revertendo habitu non mututo. Si duplex festum non fuerit dicatur a duobus clericis de secunda forma, et sic eat processio ad portam claustri orientalem et intrent chorum per ostium occidentale~: et dicatur letania usque ud prolationem} Sancta María ⟨quésumus almum⟩.\footnote{1882:117.} \emph{antequam exeat processio ad gradum altaris.}

\emph{Prima letania.}

Rylands-24:231; 1519:111r.

\gregorioscore{}

\emph{Chorus idem.}

\emph{Clerici.}

\emph{⟨Chorus idem repetat post unumquemque versum.}

\emph{Hic procedat processio et clerici predicti prosequantur.⟩\footnote{1882:117.}}

\gregorioscore{}

\emph{Chorus} Kýrie\emph{.}

\emph{Clerici ⟨cantent versum⟩}.\footnote{1882:118.}

\gregorioscore{}

\emph{Chorus} Kýrie\emph{.}

\emph{Clerici ⟨dicant versum⟩.\footnote{1882:118.}}

\gregorioscore{}

\emph{Chorus ut supra}. ⟨Kýrie.⟩\footnote{1882:118.}

\emph{Et clerici predicti prosequantur de ceteris ordinibus quantum sufficit iter usque ad gradum chori in ecclesia propria~: ita tamen quod in hac letania et iiij. non dicatur versus sic,} Omnes sancti ángeli, Omnes sancti apóstoli. \emph{sed ita pronuntietur} Omnis chorus angelórum et archangelórum \emph{vel} apostolórum \emph{vel} evangelistárum. \emph{Similiter de ceteris ordinibus.}

\emph{Ad gradum chori finiatur sic~:} Omnis chorus sanctórum.

\emph{Deinde sequatur versus et oratio de omnibus sanctis.}

\emph{Quando vero dicatur} Kýrie qui pretióso.

\emph{¶ In die sancti Marce evangeliste ⟨tunc⟩\footnote{1882:118.} in secunda feria in rogationibus dicatur ista secunda letania sic.}

\subsubsection[⟨Secunda letania.⟩]{⟨Secunda letania.⟩\footnote{1882:118.}}

Rylands-24:232; 1519:111v.\footnote{1519:111v. has no flats. 1554:117v. has some of the flats. In 1519:111r. `Christeléyson' is set A.AGF.G.DC. Some printed processionals have:

  \gregorioscore{}}

\gregorioscore{}

\emph{Chorus idem repetat post unumquemque ℣.}

\emph{Clerici dicant versum.} \gregorioscore{}

\emph{⟨Item⟩\footnote{1882:118.} chorus ⟨repetat⟩\footnote{1882:118.}} Kýrie. \emph{⟨et cetera⟩.\footnote{1882:118.}}

\emph{Clerici ⟨cantent versum⟩.\footnote{1882:118.}} \gregorioscore{}

\emph{⟨Iterum⟩\footnote{1882:118.} chorus ⟨repetat⟩\footnote{1882:118.}} Kýrie.

\emph{Clerici ⟨cantent versum⟩.\footnote{1882:118.}}

\gregorioscore{}

\emph{Chorus} Kýrie ⟨eléison. \emph{et cetera.⟩\footnote{1882:118.}}

\emph{Clerici ⟨dicant versum⟩.\footnote{1882:118.}}

\gregorioscore{}

\emph{Chorus ⟨repetat⟩\footnote{1882:118.}} Kýrie.

\emph{⟨Postea⟩\footnote{1882:118.} clerici ⟨dicant versum⟩.\footnote{1882:118.}} \gregorioscore{}

\emph{Chorus} Kýrie.

\emph{Hic procedat processio et clerici predicti procedant de ceteris ordinibus, quantum sufficit iter.}

\emph{Item clerici.} \gregorioscore{}

\emph{⟨Item⟩\footnote{1882:118.} chorus} Kýrie.

\emph{⟨Postea⟩\footnote{1882:119.} clerici ⟨dicant versum⟩.\footnote{1882:119.}}

\gregorioscore{}

\emph{⟨Iterum⟩\footnote{1882:119.} chorus} Kýrie.

\emph{⟨Postea⟩\footnote{1882:119.} clerici ⟨dicant versum⟩.\footnote{1882:119.}} \gregorioscore{}

\emph{⟨Postea⟩\footnote{1882:119.} chorus ⟨repetat⟩\footnote{1882:119.}} Kýrie.

\emph{⟨Item⟩\footnote{1882:119.} clerici ⟨versum⟩.\footnote{1882:119.}} \gregorioscore{}

\emph{⟨Item⟩\footnote{1882:119.} chorus ⟨repetat⟩\footnote{1882:119.}} Kýrie.

\emph{⟨Postea⟩\footnote{1882:119.} clerici ⟨dicant versum⟩.\footnote{1882:119.}} \gregorioscore{}

\emph{⟨Item⟩\footnote{1882:119.} chorus ⟨repetat⟩\footnote{1882:119.}} Kýrie.

\emph{⟨Deinde⟩\footnote{1882:119.} clerici ⟨dicant versum⟩.\footnote{1882:119.}} \gregorioscore{}

\emph{⟨Deinde⟩\footnote{1882:119.} chorus ⟨repetat⟩\footnote{1882:119.}} Kýrie.

\emph{⟨Postea⟩\footnote{1882:119.} clerici ⟨dicant versum⟩.\footnote{1882:119.}}

\gregorioscore{}

\emph{⟨Postea⟩\footnote{1882:119.} chorus ⟨repetat⟩\footnote{1882:119.}} Kýrie.

\emph{⟨Item⟩\footnote{1882:119.} clerici ⟨versum⟩.\footnote{1882:119.}}

\gregorioscore{}

\emph{⟨Deinde⟩\footnote{1882:119.} chorus ⟨repetat⟩\footnote{1882:119.}} Kýrie ⟨eleison. \emph{et cetera.}

\emph{Iterum⟩\footnote{1882:119.} clerici ⟨dicant versum⟩}.\footnote{1882:119.}

\gregorioscore{}

\emph{⟨Chorus repetat} Kýrie.⟩\footnote{1882:119.}

\emph{Et sic de apostolis et de ceteris ordinibus usque ad} Omnes Sancti. \emph{Sequatur ℣. et oratio de omnibus sanctis ⟨ut supra⟩.}

\chapter{¶ Tertia letania.}

Rylands-24:232; 1519:112v.

\gregorioscore{}

\emph{Chorus idem ⟨repetat post unumquemque versum⟩.\footnote{1882:119.}}

\emph{⟨Postea⟩\footnote{1882:119.} clerici ⟨cantant hunc versum⟩.\footnote{1882:119.}}

\gregorioscore{}

\emph{⟨Item⟩\footnote{1882:119.} chorus ⟨repetat⟩\footnote{1882:119.}} Kýrie.

\emph{⟨Deinde⟩\footnote{1882:119.} clerici ⟨versum⟩.\footnote{1882:119.}} \gregorioscore{}

\emph{⟨Iterum⟩\footnote{1882:119.} chorus ⟨repetat⟩\footnote{1882:119.}} Kýrie.

\emph{⟨Postea⟩\footnote{1882:119.} clerici ⟨versum⟩.\footnote{1882:119.}}

\gregorioscore{}

\emph{Chorus} Kýrie.

\emph{Et clerici predicti prosequantur de ceteris ordinibus quantum sufficit ⟨iter⟩\footnote{1882:119.} ⟨usque ad gradum chori⟩,\footnote{1517. ⟨1882:119.⟩} usque ad} Omnes Sancti. \emph{⟨et cetera.⟩\footnote{1882:119. Rylands-24:232. includes another litany here, Rex Kyrie. See the appendix.}}

\chapter[¶ Quarto letania.]{¶ Quarto letania.\footnote{In 1508. this litany has the page heading `Letania in tempore belli.'}}

Rylands-24:232; 1519:112v.\footnote{In some printed processionals `Kyrie' is set G.A.GFE; `Dómine' is set G.A.GFE.}

\gregorioscore{}

\emph{Chorus idem ⟨repetat⟩.\footnote{1882:119.}}

\emph{⟨Postea⟩\footnote{1882:119.} clerici ⟨dicant versum⟩.\footnote{1882:119.}}

\gregorioscore{}

\emph{Chorus idem ⟨repetat⟩.\footnote{1882:120.}}

\emph{⟨Postea⟩\footnote{1882:120.} clerici ⟨dicant versum⟩.\footnote{1882:120.}}

\gregorioscore{}

\emph{Chorus idem ⟨repetat⟩.\footnote{1882:120.}}

\emph{⟨Postea⟩\footnote{1882:120.} clerici ⟨dicant versum⟩.\footnote{1882:120.}}

\gregorioscore{}

\emph{Chorus} Kýrie.

\emph{⟨Item⟩\footnote{1882:120.} clerici.}

\gregorioscore{}

\emph{Chorus ⟨repetat} Dómine⟩\footnote{1882:120.} miserére.

\emph{⟨Item⟩\footnote{1882:120.} clerici ⟨dicant versum⟩.\footnote{1882:120.}}

\gregorioscore{}

\emph{Chorus ⟨repetat⟩\footnote{1882:120.}} Miserére.

\emph{⟨Postea⟩\footnote{1882:120.} clerici ⟨dicant versum⟩.\footnote{1882:120.}}

\gregorioscore{}

\emph{Chorus ⟨repetat⟩\footnote{1882:120.}} Kýrie.

\emph{Clerici ⟨dicant versum⟩.\footnote{1882:120.}}

\gregorioscore{}

\emph{⟨Item⟩\footnote{1882:120.} chorus} Dómine ⟨miserére⟩.\footnote{1882:120.}

\emph{⟨Deinde⟩\footnote{1882:120.} clerici ⟨versum⟩.\footnote{1882:120.}}

\gregorioscore{}

\emph{⟨Iterum⟩\footnote{1882:120.} chorus} Miserére.

\emph{Clerici ⟨dicant versum⟩.\footnote{1882:120.}}

\gregorioscore{}

\emph{⟨Item⟩\footnote{1882:120.} chorus ⟨repetat⟩\footnote{1882:120.}} Kýrie.

\emph{⟨Deinde⟩\footnote{1882:120.} clerici ⟨dicant versum⟩.\footnote{1882:120.}}

\gregorioscore{}

\emph{Hic modus servetur per omnes ordines usque ad} Omnis chorus sanctórum oret. \emph{Si necesse fuerit ℣. sequentes dicantur a predictis clericis tempore belli.}

\emph{Versus.}

\gregorioscore{}

\emph{Chorus idem ⟨repetat⟩.\footnote{1882:120.}}

\emph{⟨Deinde⟩\footnote{1882:120.} clerici ⟨dicant versum⟩.\footnote{1882:120.}}

\gregorioscore{}

\emph{Chorus idem ⟨repetat⟩.\footnote{1882:120.}}

\emph{Clerici ⟨versum⟩.\footnote{1882:120.}}

\gregorioscore{}

\emph{Chorus idem ⟨repetat⟩.\footnote{1882:120.}}

\emph{⟨Deinde⟩\footnote{1882:120.} clerici ⟨dicant versum⟩.\footnote{1882:120.}}

\gregorioscore{}

\emph{Chorus idem ⟨repetat⟩.\footnote{1882:120.}}

\emph{⟨Iterim⟩\footnote{1882:120.} clerici ⟨dicant versum⟩.\footnote{1882:120.}}

\gregorioscore{}

\emph{Chorus idem ⟨repetat⟩.\footnote{1882:120.}}

\emph{⟨Postea⟩\footnote{1882:120.} clerici ⟨dicant versum⟩.\footnote{1882:120.}}

\gregorioscore{}

\emph{Chorus idem ⟨repetat⟩}.\footnote{1882:120.}

\emph{⟨Deinde⟩\footnote{1882:120.} clerici ⟨dicant versum⟩}.\footnote{1882:120.}

\gregorioscore{}

\emph{Chorus idem ⟨repetat⟩}.\footnote{1882:120.}

\emph{⟨Iterum⟩\footnote{1882:121.} clerici ⟨dicant versum⟩}.\footnote{1882:121.}

\gregorioscore{}

\emph{¶ Chorus idem repetat.}

\emph{Finita aliqua letania in his tribus diebus precedentibus dicat sacerdos versum} Vox letície et exultatiónis. \emph{Resp.} In tabernáculis justórum allelúya.

\begin{lesson}
\subsubsection{Oratio.}

\lettrine{P}{r}esta quésumus omnípotens Deus ut in resurrectióne Dómini nostri Jesu Christi Fílii tui cum ómnibus sanctis percipiámus veráciter portiónem. Qui tecum vivit et regnat.
\end{lesson}

\emph{Tamen in vigilia ascensionis finita aliqua letania dicat sacerdos loco predicto ℣.} Letámini in Dómino ⟨et exultáte justi. \emph{Resp.} Et gloriámini omnes recti corde.⟩

\begin{lesson}
\subsubsection{Oratio.}

\lettrine{I}{n}firmitátem nostram quésumus Dómine propícius réspice et mala ómnia que juste mereámur ómnium sanctórum tuórum intercessióne avérte. Per Christum Dómimim nostrum. \emph{⟨℟.⟩} ⟨Amen.⟩\footnote{1882:121.}
\end{lesson}

\emph{¶ Feria iij. et iv. fiat processio in eundo et redeundo eodem modo et ordine quo prediximus~: excepto quod in vigilia ascensionis retrocedat dracho~: videlicet proximo loco ante crucem tam in eundo quam in redeundo.}

\chapter{¶ In die ascensionis Domini.}

\emph{Ordinetur processio sicut in die pasche~: excepto quod hac die vexilla processionis procedant primo videlicet leo\footnote{`\emph{loco}', 1519:114r; 1523:114r; 1528:105v; 1554:120v. 1508, unpaged, 1530:135r. and 1555, unpaged, have `\emph{leo}.'} : deinde minora vexilla per ordinem~: ultimo loco draconis, deinde inter subdyaconum et thuribularium duo de secunda forma capsulam reliquiarum in cappis sericis deferant~: ipse quoque dyaconus ⟨in⟩\footnote{1882:121.} eundo reliquias deferat, pro dispositione sacriste. Preterea hac die procedat processio per ostium chori et ecclesie ⟨et⟩\footnote{1882:121.} exiet per ostium occidentale circumeundo extrinsecus totam ecclesiam et atrium intrando per portam juxta cimiterium canonicorum circumeundo claustrum et rediet in ecclesiam per idem ostium quo egressa est. Hac itaque processio ad gradum chori prius ordinata ut patet in ⟨ista⟩\footnote{1882:121.} statione sequenti.}

\begin{figure}
\centering
\includegraphics{images/114v-ordo-ascensionis.jpg}
\caption[¶ ⟨Statio et⟩ ordo processionis in die ascensionis ⟨Domini⟩ ante missam.]{¶ ⟨Statio et⟩\footnote{1882:122.} ordo processionis in die ascensionis ⟨Domini⟩\footnote{1882:122.} ante missam.}
\end{figure}

\emph{Tres clerici de superiori gradu in medio processionis in cappis sericis dicant hanc prosam sequentem.}

Rylands-24:234; 1519:114v; GS:134.\footnote{In 1519:115r. `astra' is set G.E. At ℣. 8: 1519:115v. has `venit.' 1882:123. has `vernet' with the note `Sic MS. Harl. 2942; Bodl. MS.---vernat ; Edd.---venit.' At ℣. 3: GS:134. has `fraude', not `fronde.' At ℣. 7: GS:135. has `ferens', not `feras.' GS:135. omits ℣℣. 9--11. Rylands-24:234. omits ℣℣. 7--11.

  In Rylands-24:234. `celum' is set AAG.ED. Verse 1: in Rylands-24:234. `grátia' is set G.D.F. It appears that a D has been erased below the G. Verse 2: in Rylands-24:234. `rútilant'' is set F.D.G. Verse 4: in Rylands-24:234. `oppréssis' is set F.D.E. Verse 5: in Rylands-24:234. `inférni' is set F.D.E. Verse 6: in Rylands-24:234. `cárcere' is set G.FE.D.}

\gregorioscore{}

\emph{Chorus idem repetat post unumquemque versum.}

\emph{Clerici ⟨dicant⟩\footnote{1882:122.} ℣.}

\gregorioscore{}

\emph{Per idem ostium quo egressa est ⟨processio⟩\footnote{1882:123.} regredietur usque ad crucem in ecclesia cantando ⟨hoc sequens⟩\footnote{1882:123.} ℟. In revertendo cantore incipiat.}

1519:116r; AS:269; Ant-1519:250v; Missal-1531:146r.\footnote{In 1519:116r. `admirámini' is set lower, using a B♭ clef: D.GAB♭.B♭.AFAB♭AB♭.AG; 1519:116r. has no flat at `celum.' In 1519:116r. `illos' is set DED.D; `véstibus' is set DF.ED.CD.}

\gregorioscore{}

\emph{¶ In introitu chori dicatur aliud responsorium hoc modo ⟨quo sequitur⟩.\footnote{1882:123.}}

1519:116v; AS:266; Ant-1519:248r; Missal-1531:145v.\footnote{1519:116v. has flats only at `† Spíritum' and `allelúya.'}

\gregorioscore{}

\emph{℣.} Ascéndit Deus in jubilatióne.

\emph{Resp.} Dóminus in voce tube ⟨allelúya⟩.\footnote{1882:123.}

\begin{lesson}
\subsubsection{Oratio.}

\lettrine{C}{o}ncéde quésumus omnípotens Deus~: ut qui hodiérna die Unigénitum tuum redemptórem nostrum ad celos ascendísse crédimus~: ipsi quoque mente in celéstibus habitémus. Per eúndem Christum Dóminum nostrum. ⟨Amen.⟩\footnote{1882:124.}
\end{lesson}

\chapter{¶ Dominica infra octavas ascensionis.}

\emph{Ad processionem. ℟.} Viri Galiléi. 271.

\emph{In introitu chori. ℟.} Non conturbétur. 272.

\emph{℣. et oratio ut supra in die ascensionis.}

\chapter{¶ In vigilia penthecostes.}

\emph{Cantentur letanie et fiat benedictio fontium sicut in vigilia pasche eodem modo et ordine, tam in eundo quam in redeundo et in statione.}

\emph{In redeundo dicatur} Rex sanctórum. 196.

\chapter{¶ In die penthecostes.}

\emph{Ante iij. aspergatur aqua benedicta~: et post aspersionem aque ordinetur processio sine vexillis et sine reliquiis~: circumeundo ecclesiam et claustrum sicut in die pasche~: et cantetur prosa sequens in medio processionis a tribus clericis de superiori gradu in cappis sericis ⟨qui⟩\footnote{1882:124.} dicant hoc modo.}

Rylands-24:240; 1519:117r.\footnote{In 1519:117r. `celo' appears to be set AAGG.DC. In verse 2: in Rylands-24:240. `solem celi órbita' is set FD.E D.G G.F.D. In verse 4: in Rylands-24:240. `prodúcunt' is set F.D.E. In verse 5: Rylands-24:240. has `gaudet'; in Rylands-24:240. `bene vernáles' is set E.F GF.DC.E. In verse 7: note: `al.~cuncta.' 1882:125.}

\gregorioscore{}

\emph{Chorus idem repetat post unumquemque versum.}

\emph{Clerici.}

\gregorioscore{}

\emph{In redeundo per idem ostium quo egressa est regrediatur ⟨processio⟩\footnote{1882:125.} usque ad crucem cantando hoc ℟. sequens hoc modo.}

1519:118r; AS:279; Ant-1519:259v, 265v; Brev-1531:152v.\footnote{1519:118r. has a flat only at `penetrávit.'}

\gregorioscore{}

\emph{In introitu chori dicatur hec ⟨sequens⟩\footnote{1882:125.} antiphona ⟨hoc modo quo sequitur⟩.\footnote{1882:125.}}

1519:118v; AS:281; Ant-1519:261v; Brev-1531:153r.

\gregorioscore{}

\emph{℣.} Loquebántur váriis linguis apóstoli.

\emph{℟.} Magnália Dei allelúya.

\begin{lesson}
\subsubsection{Oratio.}

\lettrine{D}{e}us, qui hodiérna die corda fidélium Sancti Spíritus illustratióne docuísti~: da nobis in eódem Spíritu recta sápere~: et de ejus semper consolatióne gaudére. Per Christum Dóminum ⟨nostrum. Amen.⟩\footnote{1882:125.}
\end{lesson}

\emph{¶ Et nota quod hac die cantetur hora iij. et vj. a toto choro in cappis sericis et non alias per totum annum.}

\chapter{¶ In die sancte Trinitatis.}

\emph{Ordinetur ⟨processio⟩\footnote{1882:125.} sicut in die natalis Domini. Sexta\footnote{Note: `1517---Tertia.' 1882:125.} cantata eat processio per medium chori circumeundo claustrum cantore incipiente ⟨hoc responsorium sequens⟩\footnote{1882:125.} hoc modo.}

1519:119r; AS:292; Ant-1520:6r; Brev-1531:158v.\footnote{1519:119r. has no flats.}

\gregorioscore{}

\emph{In introitu chori dicatur istud responsorium ⟨sequens hoc modo quo sequitur⟩.\footnote{1882:126.}}

1519:119v; AS:290; Ant-1520:5r; Brev-1531:158r.\footnote{1519:119v. has no flats.}

\gregorioscore{}

\emph{In tempore paschali hoc modo dicatur.} \gregorioscore{}

\emph{℣.} Sit nomen Dómini benedíctum.

\emph{℟.} Ex hoc nunc ⟨et usque in séculum⟩.\footnote{1882:126.}

\emph{⟨℣.} Orémus.⟩

\begin{lesson}
\subsubsection{Oratio.}

\lettrine{O}{m}nípotens sempitérne Deus qui dedísti fámulis tuis in confessióne vere fídei etérne Trinitátis glóriam agnóscere et in poténtia majestátis adoráre Unitátem~: quésumus ut ejúsdem fídei firmitáte ab ómnibus semper muniámur advérsis. Qui vivis et regnas in unitáte.\footnote{1554. has `In qua vivis. \emph{\&c.}' 1882:126. has `In qua vivis.' The edition follows the Breviary and Missal here.}
\end{lesson}

\chapter{¶ In festo Corporis Christi.}

\emph{Ante missam procedat processio per medium chori et ecclesie, exiens per ostium occidentale circumeundo ecclesiam et atrium ut in die ascensionis Domini ordinata prius processione ad gradum chori cum sacerdote alba et cappa serica induto~: Corpus Christi in tabernaculo deferente sub quodam pallio de serico ⟨quod⟩\footnote{1882:126.} super iiij. hastas deferatur, cum cereis illuminatis a clericis in superpelliceis. Tres clerici de superiori gradu in medio processionis in cappis sericis cantent prosam et cantetur ℣. antequam exeat processio~: chorus idem repetat post unumquemque versiculum, clerici vero sequentur.}

1519:120r.\footnote{In ℣. 1: `MS. Harl. 2945---ipse cibus', 1882:126; in ℣. 4: `Harl. MS. 2945---Fit caro de pane, de vino sanguis, inane Ne cor deficiat, hic cibus efficiat.' `Edd.---memorandum; Bodl.---manducanda.' 1882:127; in ℣. 5, `Ita MS. Harl. 2945; Edd.---Hic cibus, hic potus facit, quoque numine totus; MS. Bodl.---Hic cibus, hic potus facit munimine totus.' 1882:127; 1882:127. has `fáciat'; in ℣. 12, `huic cibus', 1882:127; in ℣. 13, `ille cibus', 1882:127; in ℣. 14, `sit cor', 1882:127; in ℣. 15: `Ita MS. Harl. 2945 ; Edd.---Quod te sanctificet Christus qiu te benedicet.' 1882:127; in ℣. 16: `Ita MS. Harl. 2945 ; Edd.---Cui.' 1882:127. In 1525:125r. in verse 12 `criménque' is set E.DG.F; in verse 13 `teguménta' is set E.DG.F D; in verse 15 'quod te is set E.DG.F.}

\gregorioscore{}

\emph{In revertendo usque ad crucem dicatur ⟨hoc responsorium modo quo sequitur⟩.\footnote{1882:127.}}

1519:122r; Ant-1520:11r; Brev-1531:160v.\footnote{1519:122r. has no flats. In 1519:122r. `cibi' is set BDCBCA.BA.}

\gregorioscore{}

\emph{Si necesse fuerit dicatur istum versiculum.}

\gregorioscore{}

\emph{In introitu chori dicatur ista antiphona.}

1519:122v; Ant-1520:15r; Brev-1531:161v.\footnote{1519:122v. has no flats.}

\gregorioscore{}

\emph{℣.} Panem de celo prestitísti eis.

\emph{℟.} Omne delectaméntum ⟨in se habéntem⟩.\footnote{1882:128.}

\emph{⟨℣.} Orémus.⟩

\begin{lesson}
\subsubsection{Oratio.}

\lettrine{D}{e}us qui nobis sub sacraménto mirábili passiónis tue memóriam reliquísti tríbue quésumus~: ita nos córporis et sánguinis tui sacra mistéria venerári~: ut redemptiónis tue fructum in nobis júgiter sentiámus. Qui vivis et regnas.
\end{lesson}

\chapter{¶ Dominica infra octavas.}

\emph{Ubi habentur cum regimine chori\footnote{`1517, \&c.---\emph{ubi octave habentur cum regimine chori}', 1882:128.} ad processionem. ℟.} Respéxit Helýas. 285.

\emph{In introitu chori de sancta Maria.\footnote{`1517---In introitu chori, ut supra in prima die, nisi historia inchoata fuerit, sicut in processione ante missam. Coram ecclesia dicitur antiphona Adoremus. Et in introitu chori antiphona Ave regina celorum de sancta Maria.' 1882:128.}} 291.

\lettrine{I}{}\emph{N sabbatis per estatem scilicet a Trinitate usque ad adventum Domini~: ad vesperas post omnes memorias eat\footnote{\emph{I}'\emph{et}', 1519:122v. and most other editions. 1530:144v. and 1555. have `\emph{eat}.'} processiones~: ante crucem de quocunque fit servitium per medium chori ubi\footnote{`\emph{nisi}', 1555.} duplex festum fuerit, ordinata prius processione ad gradum chori cum duobus ceroferariis, albis tantum indutis, et thuribulario in simili habitu sine cruce, deinde puer librum ferens ante sacerdotem in superpelliceo deinde executor officii in simili habitu cum cappa serica, post eum vero duo rectores\footnote{1882:128. has `\emph{cantores}', with the note: `1517 and later edd.---rectores.'} in medio processioni in simili habitu antiphonam in eundo et in introitu incipientes, choro sequente habitu non mutato.}

\chapter{⟨Ante crucem in sabbatis post Trinitatem.⟩}

\begin{figure}
\centering
% \includegraphics{images/1v-statio-aqua.png}
\caption{¶ Statio ad vesperas ante crucem in sabbatis per estatem.\footnote{1519. is missing folio 123 (r. and v.).}}
\end{figure}

\emph{¶ Tunc cantent unam istarum antiphonarum per ordinem.}

1523:123v; AS:424; Ant-1519-S:70v; Brev-1531-S:47v.\footnote{1523:123v. has a B♭ signature throughout. In 1523:123v. `et' is set CB♭ADB♭B♭A.}

\gregorioscore{}

\emph{Et finiatur cum} Allelúya. \emph{qandocunque dicatur} Allelúya.

1519:123v; AS:428; Ant-1519-S:73v; Brev-1531-S:48v.

\gregorioscore{}

\emph{⟨Sequitur alia antiphona.⟩\footnote{1882:130.}}

1519:124r; Ant-1520:17v; Brev-1531:171v.\footnote{1519:124r. has `inmici' for 'inimíci.}

\gregorioscore{}

\emph{Statio post thurificationem crucifixi.}

\emph{℣.} Adorámus te Christe et benedícimus tibi.

\emph{℟.} Quia per sanctam crucem tuam ⟨redemísti mundum⟩.\footnote{1882:130.}

\emph{⟨℣.} Orémus.⟩

\begin{lesson}
\subsubsection{Oratio.}

\lettrine{D}{e}us qui Unigéniti Fílii tui Dómini nostri Jesu Christi precióso sánguine vivífice crucis vexíllum sanctificáre voluísti~: concéde quésumus eos qui ejúsdem sancte crucis gáudent honóre, tua quoque ubíque protectióne gaudére. Per eúndem Christum.
\end{lesson}

\emph{In introitu chori dicatur una istarum antiphonarum sequentium per ordinem tam in processione in sabbatis quam ad processionem ante missam in dominicis pro dispositione cantoris.\footnote{GS:141. and Rylands-24:251. list two further antiphons: O gloriósa. (see Brev:⟨524⟩) and Sancta María. (see Brev.:⟨512⟩.}}

1519:124v; AS:519; Ant-1520-S:105r; Brev-1531-S:130r.\footnote{In the Antiphonal the `Allelúya' is different.}

\gregorioscore{}

\emph{In tempore paschali.}

\gregorioscore{}

1519:124v; AS:529; Ant-1519-C:49r; Ant-1520-C:49r; Brev-1531:171v.

\gregorioscore{}

\emph{In tempore paschali ⟨ita finiatur⟩.\footnote{1882:130.}}

\gregorioscore{}

1519:125r; AS:529; Ant-1519-C:48v; Ant-1520-C:48v; Brev-1531:171v.

\gregorioscore{}

\emph{In tempore paschali ⟨finiatur cum⟩.}

\gregorioscore{}

1519:125v; Ant-1519-C:50r; Ant-1520-C:50r; Brev-1531:171v.

\gregorioscore{}

\emph{In tempore paschali ⟨finiatur cum⟩.}

\gregorioscore{}

1519:125v; AS:528; Ant-1520-S:111v; Brev-1531-S:132v.

\gregorioscore{}

\emph{⟨Postea sequitur alia antiphona⟩.\footnote{1555. unpaged.}}

1519:126r; AS:528; Ant-1520-S:112r; Brev-1531-S:132v.

\gregorioscore{}

\emph{Infra assumptionem et nativitatem beate Marie tam ad processionem ad vesperas in sabbatis quam ante missam in dominicis dicatur una istarum antiphonarum.}

1519:126v; AS:490; Ant-1520-S:89v; Brev-1531-S:115v.

\gregorioscore{}

1519:127r; AS:492; Ant-1520-S:91r; Brev-1531-S:105v.

\gregorioscore{}

1519:127v; AS:491; Ant-1520-S:90r; Brev-1531-S:115v. \gregorioscore{}

\emph{In tempore paschali.}

\gregorioscore{}

1519:128r; AS:492; Ant-1520-S:90v; Brev-1531-S:115v.

\gregorioscore{}

\emph{In tempore paschali.}

\gregorioscore{}

\emph{℣.} Sancta Dei génitrix ⟨virgo semper Maria⟩.\footnote{1882:132.}

\emph{℟.} Intercéde ⟨pro nobis ad Dóminum Deum nostrum⟩.\footnote{1882:132.}

\emph{Ad processionem ante missam. ℣.} Post partum ⟨virgo⟩.\footnote{1882:132.} \emph{℟.} Dei génitrix. 44. \emph{et semper cum hac oratione.}

\emph{⟨℣.} Orémus.⟩

\begin{lesson}
\subsubsection{Oratio.}

\lettrine{C}{o}ncéde quésumus omnípotens et miséricors Deus fragilitáti nostre presídium~: ut qui sancte Dei genitrícis et vírginis Maríe commemoratiónem ágimus~: intercessiónis ejus auxílio a nostris iniquitátibus resurgámus. Per eúndem Dóminum. \emph{vel} ⟨Per⟩\footnote{1882:132.} eúndem Christum ⟨Dóminum nostrum⟩.\footnote{1882:132.}
\end{lesson}

\emph{¶ Sabbato vero infra octavas assumptionis et nativitatis beate Marie dicatur in introitu chori de omnibus sanctis ant.} Salvátor mundi. 315. \emph{cum versu} Letámini in Dómino. 316. \emph{et cum oratione} Infirmitátem. 316.

\chapter[¶ Omnibus dominicis ab octavis sancte Trinitatis.]{¶ Omnibus dominicis ab octavis sancte Trinitatis usque ad adventum Domini.}

\emph{Quando fit plenum servitium de dominica ad processionem ante missam dicatur unum istorum responsoriorum sequentium per ordinem de Trinitate quando ad matutinas exeat processio et redeat ut in dominica j. adventus ut supra dictum est.}

1519:128v; AS:288; Ant-1520:3v; Brev-1531:157v.\footnote{1519:128v. has no flats. In 1519:128v. `Patri' is set A.AG.}

\gregorioscore{}

\emph{⟨Item sequitur aliud responsorium.⟩\footnote{1882:132.}}

1519:128v; AS:288; Ant-1520:3v; Brev-1531:157v.\footnote{In 1519:128v. `majestátis' is set FE.FG.GFFEDC.F.}

\gregorioscore{}

6249a. \gregorioscore{}

\emph{⟨Postea sequitur aliud responsorium.⟩\footnote{1882:132.}}

1519:129r; AS:289; Ant-1520:4r; Brev-1531:158r.

\gregorioscore{} 7498.

\gregorioscore{} 7498a.

\gregorioscore{}

\emph{⟨Item sequitur aliud responsorium.⟩\footnote{1882:133.}}

1519:129v; AS:289; Ant-1520:4v; Brev-1531:158r.\footnote{1519:129v. has no flats. In 1519:129v. `Magnus' is set EDCDE.E; `magna' is set GA.BGGAG; the final four notes of `est' are AFDF; in the ℣. `Glória Patri', `et' is set G.}

\gregorioscore{} 7117.

\gregorioscore{}

\emph{⟨Item sequitur aliud responsorium.⟩\footnote{1882:133.}}

1519:130r; AS:290; Ant-1520:4v; Brev-1531:158r.\footnote{1519:130r. has a flat oly at `proli.' In 1519:130r. the second `Da' is set AB.}

6777. \gregorioscore{}

\emph{⟨Postea sequitur aliud responsorium.⟩\footnote{1882:133.}}

1519:130v.

\gregorioscore{}

\emph{⟨Postea sequitur aliud responsorium.⟩\footnote{1882:133.}}

\emph{¶ In festo Sancte Trinitatis.}

1519:130v; AS:291; Ant-1520:5v; Brev-1531:158v.\footnote{1519:130v. has no flats.}

7764. \gregorioscore{}

7764a. \gregorioscore{}

\emph{⟨Deinde sequitur aliud responsorium.⟩\footnote{1882:133.}}

1519:131r; AS:292; Ant-1520:6r; Brev-1531:158v.

6239.

\gregorioscore{} 6239a.

\emph{⟨Item aliud responsorium} Summe Trinitáti. \emph{et cetera.⟩\footnote{1882:133.} Require ⟨ut supra⟩\footnote{1882:133.} in festo Sancte Trinitatis.} 277.

\emph{Et semper tunc post ℟. ante crucem in ecclesia dicatur una istarum antiphonarum de cruce et si de aliquo festo fit processio, licet ad vesperas in sabbato processio ante crucem facta non fuerit nisi tantum aliquod festum duplex in ipsa dominica contigerit.}

1519:131v; AS:533; Ant-1520-S:117r; Brev-1531-S:137v.

\gregorioscore{}

\emph{⟨Item sequitur alia antiphona.⟩\footnote{1882:134.}}

1519:131v; AS:532; Ant-1520-S:117r; Brev-1531-S:137v.\footnote{In 1528:122r, 1530:154v. and 1554:141r. the music appears a fifth lower. In AS:532. the music is a fourth lower, in Mode VII.}

\gregorioscore{}

\emph{℣.} Hoc signum crucis erit in celo.

\emph{℟.} Cum Dóminus ad judicándum vénerit.

\emph{⟨℣.⟩} Orémus.

\begin{lesson}
\subsubsection{Oratio.}

\lettrine{A}{d}ésto nobis Dómine Deus noster et quos sancte crucis letári facis honóre~: ejus quoque perpétuis defénde subsídiis. Per Christum Dóminum nostrum. ⟨Amen.⟩\footnote{1882:134.}
\end{lesson}

\emph{Deinde dicantur preces consuete ut supra in dominica prima adventus Domini. Quibus dictis, dicatur in introitu chori una ex istis per ordinem de sancta Maria antiphona.} Beáta Dei génitrix. 291. Ave regína celórum. 292. Alma redemptóris mater. 293. Speciósa facta est. 294. Ibo michi ad. 295. Quam pulchra. 295. \emph{Infra ⟨octavis⟩ assumptionem beate Marie ⟨usque ad nativitatem ejusdem⟩\footnote{1882:134.} dicatur una istarum antiphonarum.} Tota pulchra es. 296. Ascéndit Christus. 298. Anima mea. 299. Descéndi in hortum. 300. \emph{℣.} Post partum. 44. \emph{Et semper cum hac oratione ⟨scilicet⟩\footnote{1882:134.}} Concéde quésumus miséricors Deus. \emph{ut supra} 50. \emph{: que terminetur sic,} Per eúndem ⟨Dóminum. \emph{vel} eúndem Christum⟩.\footnote{1882:134.}

\chapter{¶ In dedicatione ecclesie.}

\emph{Eat processio sicut in die penthecostes circumeundo ecclesiam et claustrum, tres clerici de superiori gradu in medio processionis in cappis sericis dicant prosam sequentem hoc modo.}

1519:132r; GS: 173; Rylands-24:287.\footnote{In ℣. 2: Rylands-24:287. has `Hic iníent'; GS:174. has `Hic íneunt'; 1882:134. has `Hinc et eunt.' In ℣. 3: Rylands-24:287. has `nos spirituáliter \ldots{} si.' GS:174. and 1882:135. have `si.' In ℣. 5: Rylands-24:287. and 1882:135. have `Dat Rex.' In ℣. 6: GS:174. has `turris si pede', set GF.A F DE; Rylands-24:288. has `prepóte'; 1882:125. has `pérpete'; GS:174. has `pérpete.' In ℣. 8: Rylands-24:287. has `conscéndere solam \ldots{} vite'; GS:174. has `vite. At the conclusion of ℣. 8. GS:174. gives the incipit `Salve festa dies.'

  In Rylands-24:287. `Qua sponso sponsa' is set F ED.FA AAG.ED.}

\gregorioscore{}

\emph{Chorus idem repetat post unumquemque versum.}

\gregorioscore{}

\emph{¶ In introitu chori dicatur hoc responsorium ⟨sequens⟩.\footnote{1882:135.}}

1519:132v; AS:pl. p; Ant-1520:58v; Brev-1531:215v.\footnote{In 1519:132v. the first `est' is set CDEDF. In 1519:132v. and the other editions no repeat is indicated at `vere', but a repeat is indicated at `et'; yet the indicated repetition cue at the end is `† Vere.' In 1519:132v. `in' is set CD. In 1519:133r. `somno' is set thus~:

  \gregorioscore{}}

\gregorioscore{}

\emph{In tempore paschali ⟨dicitur⟩.\footnote{1882:135.}}

\gregorioscore{}

\emph{¶ Hac die non dicatur} Glória Patri.

\emph{℣.} Beáti qui hábitant in domo tua Dómine.

\emph{℟.} In sécula seculórum laudábunt te.

\emph{⟨℣.} Orémus.⟩

\begin{lesson}
\subsubsection{Oratio.}

\lettrine{D}{e}us qui nobis per síngulos annos hujus sancti templi tui consecratiónis réparas diem~: et sacris semper mystériis represéntas incólumes, exáudi preces pópuli tui et presta~: ut quisquis hoc templum benefícia petitúrus ingréditur~: cuncta se impetrásse letétur. Per Christum Dóminum.
\end{lesson}

\emph{¶ Dominica vero infra octavas\footnote{`\emph{et in octava}, MS.' 1882:135.} si dominica fuerit ad processionem Resp.} Terríbilis. 312. \emph{cum versu} Cumque evigilásset. \emph{ut supra et cum} Glória Patri. \emph{et ⟨cum⟩\footnote{1882:135.}} ‡Et ego.\footnote{1882:135.

  \gregorioscore{}} \emph{In paschali tempore cum} Allelúya.

\emph{In introitu chori de sancta Maria.} 309.

\chapter{¶ In vigilia sancti Andree apostoli.}

\emph{Si hoc festum infra adventum Domini vel extra contigerit~: fiat processio ad j. vesperas post omnes memorias que habentur ad altare sancti Andree si habeatur, cum ceroferariis et thuribulario a puero librum deferente ante sacerdotem sine cruce, choro sequente habitu non mutato cantando ℟. rectoribus incipientibus cum suo versu, itaque rectores chori incipiant ℟. hoc modo.}

1519:133v; AS:350; Ant-1519-S:6r; Brev-1531-S:3v.\footnote{In 1519:133v. `iste est' is set G.GBAGA AFFE. 1519:133v. has no flat.}

7899. \gregorioscore{}

\emph{Non dicatur} Glória Patri.

\emph{Dum ℣. canitur thurificet sacerdos altare illud deinde ymaginem sancti Andree et postea dicat ℣.} In omnem ⟨terram exívit sonus eórum⟩.\footnote{1882:136.}

\emph{℟.} Et in ⟨fines orbis terre verba eórum⟩.\footnote{1882:136.}

\emph{⟨℣.} Orémus.⟩

\begin{lesson}
\subsubsection{Oratio.}

\lettrine{M}{a}jestátem tuam quésumus Dómine supplíciter exorámus~: ut sicut ecclésie tue beátus Andréas éxtitit predicátor ⟨et rector⟩\footnote{1882:136.}~: ita apud te sit pro nobis perpétuus intercéssor. Per Christum Dóminum nostrum.
\end{lesson}

\emph{Nec precedat nec subsequatur} Dóminus vobíscum. \emph{sed tantum cum} Orémus. \emph{ante orationem.}

\emph{Cum vero oratio} Majestátem. \emph{dicatur ad j. veperas tunc ad processionem dicatur ista oratio.}

\begin{lesson}
\lettrine{Q}{u}ésumus omnípotens Deus~: ut beátus Andréas apóstolus tuum pro nobis implóret auxílium~: ut ⟨a nostris reátibus absolúti⟩\footnote{1555, unpaged.} a cunctis étiam perículis eruámur. Per Christum.
\end{lesson}

\emph{In redeundo extra adventum de sancta Maria. Infra adventum vero dicatur ista antiphona de omnibus sanctis.}

1519:134r; AS:543, 579; Ant-1520-S:155r; Brev-1531-S:79r.\footnote{1519:134r. has no flats. In 1519:134r. `sanctórum' is set C.AB.A; `semper' is set DF.DC.}

\gregorioscore{}

\emph{℣.} Letámini in Dómino et exultáte justi.

\emph{℟.} Et glorámini ⟨omnes recti corde⟩.\footnote{1882:136.}

\emph{⟨℣.} Orémus.⟩

\begin{lesson}
\subsubsection{Oratio.}

\lettrine{I}{n}firmitátem nostram quésumus Dómine propícius réspice, et mala ómnia que juste merémur ómnium sanctórum tuórum intercessióne avérte. Per Dóminum ⟨nostrum⟩.\footnote{1882:136.}
\end{lesson}

\emph{¶ Et eat processio in omni festo sanctorum ad j. vesperas post memorias que habentur semel per annum quorum altaria sunt in ecclesia cum propriis responsoriis vel de communi cum ℣. et oratione de sanctis~: ita tamen quod quando oratio de vigilia dicitur ad primas vesperas ut supra diximus~: tunc ad processionem dicatur oratio de die. Similiter quando oratio de festo dicatur ad j. vesperas tunc ad proccessionem dicatur oratio de communi~: nisi quando de festo propria habeatur~: ut in festo sancti Nicolai et consimilibus. Et semper in revertendo dicatur de sancta Maria~: nisi quando de ea ad vesperas prius fuerit memoria tunc enim dicatur de omnibus sanctis. In revertendo ut supra notatum est.}

\chapter{¶ In die sancti Nicolai.}

\emph{Ad j. vesperas post memorias dictas eat processio ad altare ejusdem cum ceroferariis et thuribulario, ⟨et⟩\footnote{1882:137.} puero librum deferente ante sacerdotem, sine cruce, choro sequente habitu non mutato, cantando ⟨responsorium sequens cum suo versu sine prosa⟩.\footnote{1882:137.}}

1519:134v; AS:359; Ant-1519-S:14r; Brev-1531-S:7r.\footnote{1519:134v. has no flats.}

\gregorioscore{}

\emph{Cum suo ℣. sed sine prosa in hac processione.}

\gregorioscore{}

\emph{Non dicatur} Glória Patri.

\emph{⟨Dum⟩\footnote{1882:137.} ℣. canitur thurificet sacerdos altare, deinde ymaginem sancti Nicolai et postea dicat ℣. sic.} Ora pro nobis, beáte Nícholáe.

\emph{℟.} Ut digni efficiámur ⟨promissiónibus Christi⟩.\footnote{1882:137.}

\emph{⟨℣.} Orémus.⟩

\begin{lesson}
\subsubsection{Oratio.}

\lettrine{D}{e}us bonitátis auctor et bonórum dispensátor~: concéde propítius~: ut qui beáti Nicolái confessóris tui atque pontíficis sollémnia venerámur~: ejus patrocínio atque suffrágio, majestátis tue propitiatiónem consequámur. Per Christum Dóminum.
\end{lesson}

\emph{In redeundo ut supra. ℣.} Letámini in Dómino. 316. \emph{Oratio.} Infirmitátem nostram. 316.

\chapter{¶ In conceptione beate Marie virginis.}

\emph{Omnia fiant sicut in nativitate} 369. \emph{verbis tamen mutatis in} Conceptióne.

\emph{Si festum alicujus sancti ix. lectionum ab octavis epiphanie usque ad lxx. vel ab octavis pasche usque ad penthecostes\footnote{`\emph{ascensionem}', 1555, unpaged. 1882:138. has `\emph{Ascensionem}' with the note `1517, \&c.---Pentecosten.'} et a} Deus ómnium. \emph{usque ad adventum Domini in dominica contigerit, fiat servitium de ipso festo cum ix. responsorio communis historie vel cum aliquo responsorio de historia ipsius festi si propria habeantur. In revertendo scilicet in introitu chori dicatur aliqua antiphona vel ℟. de sancta Maria cum versu et oratione que tempori conveniant. Tamen si fuerit festum majus duplex alicujus sancti, tam in eundo et in introitu chori totum fiat de sancto. Dominica vero infra octavas assumptionis et nativitatis beate Marie si dominica fuerit, in eundo dicatur de sancta Maria, et in introitu chori de omnibus sanctis antiphona} Salvátor mundi. 315. \emph{℣.} Letámini in Dómino. 316. \emph{Oratio.} Infirmitátem nostram. 316. \emph{Quando vero antiphona} Salvátor mundi. \emph{dicatur ad vesperas tunc ad processionem ante missam dicatur hec antiphona sequens hoc modo.}

1519:135v; AS:543; Ant-1520-S:40v; Brev-1531-S:79r.\footnote{In 1519:135v. `beáta' is set GBCD.D.GA.}

\gregorioscore{}

\emph{cum versiculo et oratione predictis de omnibus sanctis ⟨ut supra⟩.\footnote{1882:138.}} 316.

\chapter{¶ Sanctorum Fabiani et Sebastiani martyrum.}

\emph{Si in dominica fuerit ad processionem ⟨dicatur responsorium sequens⟩.\footnote{1882:138.}}

1519:135v; AS:370; Ant-1519-S:23r; Brev-1531-S-S:15r.\footnote{`pax in homínibus' \emph{Chevallon.} ⟨SB:83.⟩ SB:83 has `pax homínibus.' AS:368 has no indication of the second repeat `‡Et in terra.'}

\gregorioscore{} 6647.

\gregorioscore{}

6647a. \gregorioscore{}

\emph{⟨℣.⟩} Glória Patri ⟨et Fílio et Spirítui Sancto⟩.\footnote{1882:138.} 425. ‡Et in terra.

\emph{⟨Post versum, si necesse fuerit, quotienscunque dicitur responsorium ad processionem, dicitur} Glória Patri. \emph{per totum annum extra passionem Domini, nisi cum dicitur responsorium in introitu chori.⟩\footnote{1882:138.}}

\emph{In introitu chori de sancta Maria.} 50.

\chapter{¶ Sancti Agnetis virginis et martyris.}

\emph{Si dominica fuerit ad processionem ⟨dicatur responsorium sequens⟩.\footnote{1882:139.}}

1519:136r; AS:373; Ant-1519-S:26r; Brev-1531-S:15v.\footnote{1519:136r. has no flats. 1519:136v. has `† Et intra.' 1528:127r. has `† Et tanquam.'}

\gregorioscore{} 6955.

\gregorioscore{} 6955b.

\gregorioscore{}

\emph{⟨℣.⟩} Glória. 425. † Et tanquam.

\emph{In introitu chori de sancta Maria ⟨cum versu et oratione que tempori conveniunt⟩.\footnote{1882:139.}} 50.

\chapter[¶ ⟨In die⟩ sancti Vincentii.]{¶ ⟨In die⟩\footnote{1882:139.} sancti Vincentii.}

\emph{Si dominica fuerit ante lxx. fiat processio sicut in j. dominica adventus. Ad processionem dicatur ⟨hoc⟩\footnote{1882:139.} ℟. sequens.}

1519:136v; AS:384; Ant-1519-S:33v; Brev-1531-S:19r.\footnote{1519:136v. has no flats. In 1519:136v. `-ciósus martyr Vincén-' is a third lower, as if the clef is wrong. In 1519:136v. `resolútus' is set ED.CD.DEFD.D; at the ℣. `de' is set GAF.}

sar0658. \gregorioscore{}

\emph{In introitu chori de sancta Maria.} 50.

\chapter{¶ In conversione sancti Pauli.}

\emph{Si dominica fuerit ad processionem dicatur ℟. sequens.}

1519:137v; AS:393; Ant-1519-S:40v; Brev-1531-S21r.\footnote{1519:137v. has no flats. In 1519:137v. `hódie' is set FG.DEFEDCD.CDC; in the V. the second `et' is set G.}

\gregorioscore{} 6272.

\gregorioscore{}

\emph{⟨℣.⟩} Glória Patri ⟨et Fílio et Spirítui⟩.\footnote{1882:139.} 424. † Quia.

\emph{In introitu chori de sancta Maria.} 50.

\chapter{In purificatione beate Marie.}

\emph{Cantata hora sexta fiat benedictio luminis solemniter a pontifice vel a sacerdote cappa serica induta cum aliis indumentis sacerdotalibus~: et super supremum gradum altaris converso ad australem\footnote{1882:139 has `\emph{orientem}' with the note `1517, \&c.---ad australe.'} sic incipiente.}

\subsubsection{¶ Sequitur statio.}

\begin{figure}
\centering
\includegraphics{images/138r-statio-candele.jpg}
\caption[¶ Statio dum benedicuntur candele ⟨luminaria⟩ in die purificationis ⟨beate Marie⟩.]{¶ Statio dum benedicuntur candele ⟨luminaria⟩\footnote{1882:140.} in die purificationis ⟨beate Marie⟩.\footnote{1882:140.}}
\end{figure}

1519:138r; Grad-1508-S:378; Missal-1513-S:10v.

\gregorioscore{}

\begin{lesson}
\lettrine{B}{e}ne✠dic Dómine Jesu Christe hanc ✠ creatúram cere supplicántibus nobis et infúnde ei per virtútem sancte crucis bene⟨✠⟩\footnote{B1555, unpaged.}dictiónem celéstem~: ut qui eam ad repellándas ténebras humáno úsui tribuísti, talem signáculo sancte crucis tue fortitúdinem et benedictiónem accípiat~: ut quibuscúnque locis accénsa sive appósita fúerit, discédat diábolus et contremíscat et fúgiat pállidus cum ómnibus minístris suis de habitatiónibus illis nec presúmat ámplius inquietáre \emph{et finiatur ⟨benedictio⟩\footnote{1882:140.} sic}

\gregorioscore{}
\end{lesson}

\emph{Et dicantur omnes orationes cum} Orémus. \emph{sub eodem tono. Non dicatur} Dóminus vobíscum. \emph{nisi ante primum orationem tantum.}

Orémus.

\begin{lesson}
\subsubsection{Oratio.}

\lettrine{D}{o}mine sancte Pater omnípotens etérne Deus qui ómnia ex níhilo creásti et jussu tuo per ópera apum hunc liquórem ad perfectiónem cereórum veníre fecísti~: ut\footnote{`et', 1555, unpaged.} qui hodiérna die petitiónem justi Symeónis implésti~: te humíliter deprecámur, ut has cándelas ad usum hóminum ⟨et⟩\footnote{1555, unpaged.} sanitátem córporum et animárum preparátas~: sive in terra sive in aquis per invocatiónem sanctíssimi nóminis tui et per intercessiónem sancte Marie ⟨semper⟩\footnote{1882:141.} vírginis~: cujus hódie festa devóte celebrántur et per preces ómnium sanctórum tuórum bene✠dícere et sanctifi⟨✠⟩\footnote{1555, unpaged.}cáre dignéris~: ut et hujus plebis tue que illas honorífice in mánibus portáre desíderat, teque laudándo exsultáno exáudias voces de celo sancto tuo et de sede majestátis tue, et propítius sis ómnibus clamántibus ad te quos redemísti pretióso sánguine Fílii tui. Qui tecum et cum Spíritu Sancto vivit et gloriátur Deus. Per ómnia sécula seculórum. \emph{⟨℣.⟩} Amen.

Orémus.

\subsubsection{Oratio.}

\lettrine{O}{m}nípotens sempitérne Deus qui hodiérna die Unigénitum tuum ulnis sancti Symeónis in templo sancto tuo suscipiéndum presentári voluísti~: tuam súpplices deprecámur cleméntiam ut hos céreos, quos nos fámuli tui in tui nóminis magnificéntia suscipiéntes gestáre cúpimus luce accénsos bene✠dícere et sancti✠ficáre atque lúmine supérne benedictiónis accéndere dignéris quátinus eos tibi Dómino Deo nostro offeréndo digni et sancto igne dulcíssime tue claritátis succéndi in templo sancto glórie
\end{lesson}

1519:139r; Missal-1513-S:11r.\footnote{At the fifth line from the end of the following prayer, 1555, unpaged has `gubernáre ut portum.'}

\gregorioscore{}

\emph{Hic mutet vocem suam quasi legendo sic.} Qui tecum vivit et regnat in unitáte Spíritus Sancti Deus. Per ómnia sécula seculórum amen.

\emph{Hic aspergantur candele aqua benedicta, et thurificet. Deinde sequantur ⟨he due orationes super aliquo cereo accenso cum⟩\footnote{1882:142.}}

1519:140v.

\gregorioscore{}

\begin{lesson}
\subsubsection{Oratio.}

\lettrine{D}{o}mine sancte Pater omnípotens lumen indefíciens qui ea cónditor ómnium lúminum~: béne✠dic hoc lumen tuis fidélibus in honórem nóminis tui portándum~: quátenus a te sanctificáti atque benedícti lúminis tue claritátis accendámur et illuminémur~: et propítius concédere dignéris, ut velúti eódem igne quondam illuminásti Móysen fámulum tuum~: ita illúmines corda nostra et sensus nostros, quátinus ad visiónem etérne claritátis perveníre mereámur. Per Christum ⟨Dóminum⟩.\footnote{1882:142.}

\emph{⟨Alia oratio cum} Orémus.⟩\footnote{1882:143.}

\lettrine{O}{m}nípotens sempitérne Deus~: qui Unigénitum tuum ante témpora de te génitum, sed temporáliter de María vírgine incarnátum lumen verum et indefíciens ad repelléndas humáni géneris ténebras et ad incendéndum lumen fidéi et veritátis misísti in mundum~: concéde propítius~: ut sicut extérius corporáli ita et\footnote{`étiam', 1882:143.} intérius luce spirituáli irradiári mereámur. Per eúndem Christum ⟨Dóminum nostrum⟩.\footnote{1882:143.}
\end{lesson}

\emph{¶ His dictis sacerdos cum suis ministris descendat ad gradum ⟨chori⟩\footnote{1882:143.} et si episcopus presens fuerit officium exsequatur et in sedem suam se recipiat. Deinde accendantur candele et distribuantur cantore incipiente hanc antiphonam sequentem hoc modo.}

Rylands-24:329; 1519:141r; GS:180; Grad-1508-S; Missal-1513-S:12r.

\gregorioscore{}

\emph{Repetatur antiphona post unumquemque versum.}

Quia vidérunt óculi mei~: salutáre tuum.

Quod parásti~: ante fáciem ómnium populórum.

Lumen ad revelatiónem géntium~: et glóriam plebis tue Israel.

Glória Patri.

Sicut erat.

\emph{Si necesse fuerit, reincipiatur psalmus cum antiphona. Deinde eat processio sicut in die nativitatis Domini singuli clerici cum cereis ardentibus in manibus suis. Adjecto quod unus sacristarum\footnote{`1517---altaristarum.' 1882:143.} in superpellicio post thuribularium ante subdyaconum deferat cereum cum aliis benedictum usui benedictionis fontium in vigiliis pasche et penthecostes specialiter reservatur, cantore incipiente ⟨antiphonam⟩.\footnote{1882:143.}}

Rylands-24:329; GS:180; pl. i.; 1519:141r.\footnote{1519:141r. has `qui in ténebris.' GS:180. omits the text `tu sénior, setting `letáre' as follows:

  \gregorioscore{}

  Missal-1530:167r. has `vitam et resurrectiónem.' GS:pl. i. begins with `-nebris sunt.' The flats appear in Rylands-24:329. In GS:pl. i. the second syllable of `letáre' comes 2 notes earlier. In the printed processionals `letáre' is set FGA.B♭AAF.GA. In GS:180. `juste' is set FF.D.}

\gregorioscore{}

\emph{⟨Postea sequitur alia antiphona.⟩\footnote{1882:143.}}

Rylands-24:329; GS:180; pl. i.; 1519:141v.\footnote{In 1519:141v. the music for the second syllable of `Salvatórem' is illegible. The music is provided from 1530:167v. In GS:180. and pl. i. and in Rylands-24:329. `porta' is set CD.C. In Rylands-24:329. `quem' is set A.}

\gregorioscore{}

\emph{Postea sequitur Responsorium.\footnote{`\emph{Antiphonam}', 1519:142r.}}

GS:pl. i.; Rylands-24:329; 1519:142r.\footnote{In 1519:142r. and Rylands-24:329. this is labelled `Antiphona.' In Missal-1530:167v. it is labelled `℟.' In some printed processionals `mortem' is set B♭C.AG. Some printed processionals have b-flat throughout; in these cases `accépit' is set DDFCD.CB♭G. In some printed processionals `benedíxit' is set G.GB♭.B♭A.GAB♭GF. In GS:pl. i. the responsory appears a tone higher, with no flats, while the verse remains at the same pitch:

  \gregorioscore{}}

\gregorioscore{}

\emph{Tres clerici seniores in pulpito conversi ad populum in cappis sericis simul cantant hunc versum.}

\gregorioscore{}

\emph{In introitu chori dicatur hoc ℟.} ⟨s\emph{equens, sine versu et cum versu pro dispositione precentoris⟩.\footnote{1882:144.}}

1519:142v; AS:395, 401; Ant-1519-S:43v, 47v; Brev-1531-S:24r.\footnote{1519:142v. has no flat. In 1519:143r. `ignára' is set GA.ABAGF.AB. In AS:396. the second repeat is to `‡Que se nescit.'}

\gregorioscore{}

\emph{℣.} Suscépimus Deus misericórdiam tuam.

\emph{℟.} In médio templi ⟨tui⟩.

\emph{⟨℣.} Orémus.⟩

\begin{lesson}
\subsubsection{Oratio.}

\lettrine{E}{x}áudi Dómine plebem tuam et quem intrínsecus ánnua tríbuis devotióne venerári~: interveniénte beáta Dei genitríce sempérque vírgine María intérius asséqui grátie tue luce concéde. Per eúndem Christum ⟨Dóminum nostrum. Amen⟩.
\end{lesson}

\chapter{¶ Sancte Agathe virginis.}

\emph{Si dominica fuerit ad processionem ℟.} Regnum mundi. 399.

\emph{In introitu chori de sancta Maria.} 50.

\chapter{¶ In annuciatione beate Marie.}

\emph{In quacumque feria contigerit eat processio sicut in festis majoribus duplicibus cantando ℟.}

1519:143v; AS:419; Ant-1519-S:64v; Brev-1531-S:42v.\footnote{1519:143v. omits the T.P. idication. It is found in 1555, unpaged.}

\gregorioscore{}

\emph{⟨In introitu chori cantetur hec antiphona sequens~:}

1555:unpaged; AS:415; Ant-1519-S:61v; Brev-1531-S:40v.

3340. \gregorioscore{}

\emph{Tempore paschali.}

\gregorioscore{}

\emph{℣.} Roráte celi désuper.

\emph{℟.} Et nubes pluant justum~: aperiátur terra, et gérminet salvatórem.

\emph{⟨℣.} Orémus.⟩

\begin{lesson}
\subsubsection{Oratio.}

\lettrine{D}{e}us, qui de beáte Maríe vírginis útero Verbum tuum ángelo nuntiánte carnem suscípere voluísti~; presta supplícibus tuis, ut qui vere eam genitrícem Dei crédimus, ejus apud te intercessiónibus adjuvémur. Per eúndem Christum.⟩\footnote{1555, unpaged.}
\end{lesson}

\emph{In revertendo de omnibus sanctis ant.} Salvátor mundi. 315. \emph{℣.} Letámini in Dómino. 316. \emph{Oratio.} Infirmitátem nostram. 316.

\emph{In tempore paschali} Christus resúrgens. \emph{cum suo ℣. a toto choro.} 201. \emph{℣.} Surréxit Dóminus de sepúlchro. 209. \emph{Oratio.} Deus qui per Unigénitum. 209.

\chapter{¶ Sancti Georgii martyris.}

\emph{Si in dominica fuerit ad processionem responsorium.} Fílie Hierúsalem. 392.

\emph{In introitu chori de sancta Maria.} 291. \emph{Quare predictum ℟. in communi unius martyris tempore paschali.}

\chapter{¶ Sancti Marci evangeliste.}

\emph{Si dominica fuerit ad processionem responsorium.} Cándidi facti sunt. \emph{require in communi unius apostoli paschali tempore. In introitu chori de sancta Maria. Et tunc nihil fiat de jejunio nec de processione que solet fieri ipso die propter dominicam illo anno. Si hoc festum in aliqua feria post octavas pasche contigerit tunc fiat processio post magnam missam de sancto Marco. Hoc modo ordinetur processio ad gradum chori sicut in simplicibus dominicis dum hora ix. canitur~: qua cantata incipiat cantor antiphonam} Exúrge Dómine. 228. \emph{et cantetur a toto choro loco nec habitu mutato, in stallis antequam exeat processio. Deinde sequantur omnes antiphone per ordinem sicut in feria secunda rogationum et exeat processio eodem modo et ordine quo in ⟨predicta⟩\footnote{1882:146.} secunda feria rogationum sed absque leone et drachone ad aliquam ecclesiam in urbe vel in suburbio, et dicatur missa dominicalis vel missa de pace} Missale:⟨258⟩. \emph{vel} Salus pópuli. Missale:⟨194⟩. \emph{pro dispositione cantoris vel pro quacunque necessitate cum memoria de festo loci et de omnibus sanctis et cum uno} Allelúya. \emph{tantum.}

\emph{Cantata missa tres clerici de superiori gradu habitu non mutato dicant hanc letaniam} Kyrie⟨léyson⟩ qui precióso. 256. \emph{in redeundo ad ecclesiam propriam et finiatur ad gradum chori deinde dicat sacerdos ℣.} Vox letície et exultatiónis.\footnote{`1517---Letámini in Dómino.' 1882:146.} 266. \emph{Oratio.} Presta quésumus omnípotens Deus ut in.\footnote{`1517---Infirmitatem.' 1882:146.} 266.

\chapter{¶ Sancti Vitalis martyris.}

\emph{Si dominica fuerit ad processionem resp.} Fílie Hierúsalem. \emph{⟨in communi unius martyris tempore paschali.⟩\footnote{1882:146.}} 392.

\emph{In introitu chori de sancta Maria.} 291.

\chapter{¶ Apostolorum Philippi et Jacobi.}

\emph{Si dominica fuerit ad processionem} Cándidi facti. \emph{⟨quere in communi unius apostoli tempore paschali.⟩\footnote{1882:146.}} 391.

\emph{In introitu chori de sancta Maria.} 291.

\chapter{¶ In inventione sancte crucis.}

\emph{Ad j. vesperas post memorias eat processio ordinata cum ceroferariis et thuribulario ante crucem per medium chori sine cruce cantando antiphonam sequens rectoribus incipientibus.}

1519:144v; AS:424; Ant-1519-S:70v; Brev-1531-S:47v.\footnote{1519:144v. has no flats.}

\gregorioscore{}

\emph{Et finiatur cum} Allelúya. \emph{quacumque dicatur cum} Allelúya.

\emph{Dum antiphona canitur thurificet sacerdos crucem~: et postea dicat ℣.} Dícite in natiónibus. 220. \emph{Oratio.} Deus qui pro nobis Fílium tuum. \emph{ut supra in die pasche ad vesperas.} 220.

\emph{In introitu chori de sancta Maria. ℣.} Sancta Dei génitrix. 220. \emph{Oratio.} Grátiam tuam. \emph{⟨ut supra in die pasche ad vesperas.⟩\footnote{1882:146.}} 220.

\emph{Si hoc festum post diem ascensionis Domini celebretur tunc oratio} Concéde quésumus miséricors Deus. \emph{loco} Grátiam tuam. \emph{dicetur.} 220.

\emph{Si hoc festum in sabbato contigerit nulla fiat processio\footnote{`MS.---ad secudas vesperas.' 1882:147.} nec ulla memoria nisi de sequente dominica tantum.}

\emph{Si vero in die ascensionis evenerit differatur in crastinum ⟨et⟩\footnote{1882:147.} ad secundas vesperas de ascensione fiat sollemnis memoria de cruce procedendo ante crucem ⟨cum cruce⟩\footnote{1882:147.} cum hac antiphona~:} O crux gloriósa. 341. \emph{et non dicatur} O crux benedícta. \emph{propter nimiam suam brevitatem.}

\emph{℣.} Hoc signum crucis erit in celo.

\emph{⟨Resp.} Cum Dóminus.⟩\footnote{1882:147.} 308.

\emph{⟨℣.} Orémus.⟩

\begin{lesson}
\subsubsection{Oratio.}

\lettrine{D}{e}us qui in preclára salutífere crucis inventióne passiónis tue mirácula suscitásti~: concéde ut vitális ligni précio etérne vite suffrágia consequámur. Qui vivis et regnas cum Deo Patre.
\end{lesson}

\emph{In introitu chori de sancta Maria.} 291.

\emph{¶ In die ejusdem si dominica fuerit ad processionem dicitur hoc responsorium.}

1519:145r; AS:425; ant-1519-S:71v; Brev-153-S:48r.\footnote{In 1519-S:71v. the first clef is an F-clef. The guide at the end of the line suggests that the clef should be a C-clef.}

\gregorioscore{}

\emph{In introitu chori de sancta Maria.} 291.

\chapter{¶ Sancti Johannis ante portam Latinam.}

\emph{Si dominica fuerit ad processionem ℟.} Cándidi facti sunt. 391.

\emph{In introitu chori de sancta Maria.} 291.

\chapter[¶ ⟨In die⟩ sancti Dunstani episcopi et confessoris.]{¶ ⟨In die⟩\footnote{1882:147.} sancti Dunstani episcopi et confessoris.}

\emph{Si dominica fuerit ad processionem ℟.} Fílie Hierúsalem. 392.

\emph{In introitu chori de sancta Maria.} 291. \emph{vel} 309.

\chapter{¶ In vigilia sancti Johannis baptiste.}

\emph{Post vesperas fiat processio ad altare sancti Johannis cum ceroferariis, thuribulario et puero librum deferente ante sacerdotem sine cruce cantando sequentem ℟. rectore incipiente.}

1519:145v; AS:431; Ant-1520-S:5r; Brev-1531-S:58v.\footnote{1519:145v. has no accidentals. `1517--- Rectores simul cantent versum.' 1882:148.}

\gregorioscore{}

\emph{⟨Sine} Glória Patri.⟩\footnote{1555, unpaged.}

\emph{Dum versus canitur thurificet sacerdos altare, deinde imaginem sancti Johannis~: et postea dicat ℣.} Glória et honóre coronásti eum Dómine. 363.

\begin{lesson}
\subsubsection{Oratio.}

\lettrine{S}{a}ncti Johánnis baptíste mártyris tui Dómine quésumus veneránda festívitas salutáris auxílii nobis prestet efféctum. Per Christum Dóminum nostrum. Amen.
\end{lesson}

\emph{In redeundo de sancta Maria.} 309.

\chapter[⟨In die sancti Johannis baptiste.⟩]{⟨In die sancti Johannis baptiste.⟩\footnote{1882:148.}}

\emph{In die ejusdem si dominica fuerit ⟨dicitur hoc responsorium⟩.\footnote{1882:148.}}

1519:146r; AS:434; Ant-1520-S:7r; Brev-1531-S:59v.\footnote{1519:146r.has no flats.}

\gregorioscore{}

\emph{⟨℣.⟩} Glória Patri. 424. † Qui viam.

\emph{In introitu chori de sancta Maria.} 309.

\chapter{¶ In festo sanctorum Johannis et Pauli.}

\emph{⟨Si dominica fuerit.⟩\footnote{1882:148.}}

\emph{Ad primas vesperas in sabbato fiat processio ante crucem, nisi aliqua dominica omnino oporteat differri~: tunc memoria de dominica et de Trinitate non fiat, nec processio ⟨In die eorum ante missam si dominica fuerit ad processionem⟩ ℟.} In circúitu tuo. \emph{quere in communi plurimorum martyrum.} 396.

\emph{In introitu chori de sancta Maria.} 309.

\chapter{¶ In die apostolorum Petri et Pauli.}

\emph{Si dominica fuerit ad processionem ante missam dicatur istud ℟. ⟨sequens hoc modo.⟩\footnote{1882:149.}}

1519:146v; AS:446; Ant-1520-S:16v; Brev-1531-S:64r.\footnote{1519:146v. has no flats.}

\gregorioscore{}

\emph{In introitu chori de sancta Maria.} 309.

\chapter{¶ Commemoratio sancti Pauli apostoli.}

\emph{Si dominica fuerit ad processionem dicatur hoc responsorium.}

1519:147r; AS:393; Ant-1519-S:40v; Ant-1520-S:21r; Brev-1531-S:65v.\footnote{1519:147r. has no flats.}

\gregorioscore{} \gregorioscore{}

\emph{℣.} Glória Patri. 420. ‡Qui te ⟨elégit⟩.\footnote{SB-S:386.}

\emph{In introitu chori de sancta Maria.} 309.

\chapter{¶ In festo visitationis Marie.}

\emph{Ad processionem}.

1519:147v.\footnote{In ℣. 10: `quidem tu', 1882:150.}

\gregorioscore{}

\emph{In redeundo de omnibus sanctis} Salvátor mundi. \emph{tantum.} 315. \emph{℣.} Letámini in Dómino. 316. \emph{Oratio.} Infirmitátem. 316.

\chapter{¶ In octava apostolorum Petri et Pauli.}

\emph{Si dominica fuerit ad processionem resp.} Cives apostolórum. 393.

\emph{In introitu chori de sancta Maria.} 309.

\chapter[¶ Translatio sancti Thome martyris.]{¶ Translatio sancti Thome martyris.\footnote{In 1519:148v. this procession is crossed out. In 1523. the folio is missing.}}

\emph{Si dominica fuerit ad processionem ℟.} Jacet granum. \emph{require in festo ejusdem ebdomade natalis Domini} 38. \emph{: et dicetur sine prosa.}

\emph{In introitu chori de sancta Maria.} 309.

\chapter{¶ In festo reliquiarum.}

\emph{Ad processionem ℟.} Concéde nobis. \emph{℣.} Adjúvent. \emph{require in festo omnium sanctorum.} 381. \emph{Finito responsorio cum suo ℣. et} Glória Patri. \emph{si tantum restat iter fiat statio in ecclesia, ibique legantur nomina reliquiarum in lingua materna~: et interim abluantur ibi reliquie, choro sequente. Quibus finitis processio more solito in chorum redeat cantore incipiente antiphonam} Salvátor mundi. 315. \emph{cum versu} Letámini in Dómino. 316. \emph{Oratio.} Infirmitátem nostram. 316.

\chapter{¶ In festo sancte Margarete virginis et martyris.}

\emph{Ad j. vesperas fiat processio ad altare ejusdem cantando ℟. sequentem.}

1519:149r; AS:665; Ant-1519-C:43r; Brev-1531-P:81r.\footnote{1519:149r. has no flats. In 1519:149r. `tui' is set CDE.C.}

\gregorioscore{} 6142a.

\gregorioscore{}

\emph{℣.} Ora pro nobis beáta Margaréta.

\emph{℟.} Ut digni ⟨efficiámur promissiónibus Christi⟩.\footnote{1882:151.}

\begin{lesson}
\subsubsection{Oratio.}

\lettrine{E}{x}áudi nos Deus salutáris noster ut sicut de beáte Margaréte vírginis tue et mártyris festivitáte gaudémus ita pie devotiónis erudiámur afféctu. Per christum Dóminum. ⟨Qui tecum.⟩\footnote{1882:151.}
\end{lesson}

\emph{In introitu chori de sancta Maria.}

\emph{In die ejusdem ⟨si dominica fuerit⟩\footnote{1882:151.} ad processionem ⟨ante missam dicitur⟩\footnote{1882:151.} ℟.} Regnum mundi. 399.

\emph{In introitu chori de sancta Maria.} 309.

\chapter{¶ In festo sancte Marie Magdalene.}

\emph{Ad j. vesperas eat processio ad altare ejusdem cantando ℟.} Regnum mundi. 399.

\emph{℣.} Dimíssa sunt ei peccáta multa.

\emph{⟨℟.} Quóniam diléxit multum.⟩\footnote{1882:151.}

\emph{⟨℣.} Orémus.⟩

\begin{lesson}
\subsubsection{Oratio.}

\lettrine{S}{a}cratíssimam Dómine beáte Maríe Magdaléne qua celos súbiit sollennitátem recenséntes supplíciter rogámus cleméntiam tuam~: ut qui devotiónis ejus recólimus insígnia ipsíus mereámur compótes fíeri glórie. Per Christum.
\end{lesson}

\emph{In redeundo de sancta Maria.} 309.

\emph{¶ In die ejusdem si dominica fuerit, ad processionem ante missam ⟨dicitur responsorium⟩}.\footnote{1882:151.}

1519:149v; AS:462; Ant-1520-S:46v; Brev-1531-S:89v.\footnote{1519:149v. has a flat only at `Glória Patri.' In 1519:149v. `éthere' is set FAAFGFD.CD.D.}

\gregorioscore{}

\emph{¶ In introitu chori de sancta Maria.} 309.

\chapter{¶ In festo sancte Anne.}

\emph{Si dominica fuerit ad processionem ℟.}

1519:150r; Ant-1520-S:52v; Brev-1531-S:94v.\footnote{1519:150r. has a flats only at `letos', `cunctis', and `et.' In 1519:150r. `remissionem' is set A.AGADAA.BCA.GA.A. In 1519:150r. `fac' is set AC; `post' is set BAG.}

\gregorioscore{}

\emph{In redeundo de sancta Maria.} 309.

\chapter{¶ Ad vincula Petri.}

\emph{Si fuerit dominica fuerit ad processionem.}

1519:150v; AS:464, 469; Ant-1520-S:57v; Brev-1531-S:97v.

\gregorioscore{}

\emph{In introitu chori dicatur aliqua antiphona de sancta Maria.} 309.

\chapter{¶ In festo transfigurationis Domini.}

\emph{Ad processionem.}

1519:151r.\footnote{This ℟. has the same text as the ninth ℟. at Matins of the Tranfiguration, but different music.}

\gregorioscore{}

\emph{In redeundo ⟨dicatur⟩\footnote{1882:152.} de sancta Maria.} 309.

\chapter{¶ In festo nominis Jesu.}

\emph{Ad processionem.}

1519:151v.\footnote{In ℣. 1. 1882:152. has `veréntur'; in ℣. 3 1882:153. has `impédit et dirimit atque fugat' with the note `Edd.---Hostes impédiat, dírimit, atque fugit.' For ℣. 4: `The couplet would have a possible meaning if \emph{vacuum} or some equivalent were read for \emph{gratum} ; \emph{portans precordia vite} seems a paraphrase of \emph{cordiale} in the Sequence \emph{Dulcis Jesus Nazarenus}, on which this extremely obscure prose is closely based, and \emph{vacuum eo} is the exact opposite of \emph{arca cordis inseratum}.' ⟨1882:xv.⟩; in ℣. 5 1882:153. has `salútiem'; in ℣. 6 1882:153. has `légitur'; in ℣. 7 1882:153. has `letítia.'}

\gregorioscore{}

\emph{⟨Chorus idem repetat post unumquemque versum. Clerici versus.⟩\footnote{1555, unpaged.}}

\gregorioscore{}

\emph{In redeundo de sancta Maria.} 309.

\chapter{¶ In festo sancti Laurentii.}

\emph{Ad primas vesperas eat processio ad altare ejusdem cantando.}

1519:153r; AS:485; Ant-1520-S:85v; Brev-1531-S:110v.

\gregorioscore{}

\emph{⟨℣.⟩} Glória Patri. 420. † Et ideo.

\emph{℣.} Glória et honóre coronásti eum Dómine.

\emph{℟.} Et constituísti ⟨eum super ópera mánuum tuárum⟩.\footnote{1882:153.}

\emph{⟨℣.} Orémus.⟩

\begin{lesson}
\subsubsection{Oratio.}

\lettrine{D}{a} nobis quésumus omnípotens Deus~: viciórum nostrórum flammas extínguere, qui beáto Lauréntio tribuísti tormentórum suórum incéndia superáre. Per Christum Dóminum nostrum. \emph{⟨℟.⟩} Amen.
\end{lesson}

\emph{In introitu chori de sancta Maria.} 309.

\emph{In die ejusdem si dominica fuerit ad processionem ℟.} Méruit esse. \emph{ut supra ad vesperas.}

\emph{In introitu chori de sancta Maria.} 309.

\chapter[¶ ⟨In die⟩ sancti Ypolyti cum sociis suis martyribus.]{¶ ⟨In die⟩\footnote{1882:154.} sancti Ypolyti cum sociis suis martyribus.}

\emph{Si hoc festum in dominica contigerit totum servitium fiat de festo et memoria de dominica ei de Trinitate et processio ante crucem ad j. vesperas.}

\emph{In die eorundem si dominica fuerit ad processionem.} In circúitu tuo. \emph{quere in communi plurimorum martyrum.} 396.

\emph{In introitu chori de sancta Maria.} 309.

\chapter{¶ In assumptione beate Marie.}

\emph{Quacunque die contigerit ad processionem.}

1519:153v; AS:498, 572; Ant-1519:100v; Ant-1520-S:94v; Brev-1531-S:117v.\footnote{1519:153v. has no flats. At the repeat after ℣. Glória. `Quia.' \emph{Chevallon.} ⟨SB-S:698.⟩ 1520-S:94v. omits `\emph{V.} Glória. † Quia.' SB-S:98 has at the repeat after ℣. `Glória. Christus.' The settings in AS, for Assumption and for All Saints, have neither the `Allelúya' nor the final word of the ℣. `commemoratiónem.' In AS:498. the last syllable of `justície' is set AGDEFGEDEDDC.}

\gregorioscore{}

\emph{⟨℣.⟩} Glória Patri. 424. † Quia ext te.

\emph{⟨Sequitur⟩\footnote{1882:154.} aliud responsorium.}

1519:154r; AS:496; Ant-1520-S:93v; Brev-1531:117r.\footnote{In 1519:154e. `concíves' is set G.GEG.EDG.}

\gregorioscore{}

\emph{⟨℣.⟩} Glória Patri. 426. ‡Consórtes et concíves.

\emph{In introitu chori dicatur ista antiphona.}

1519:154r; AS:492; Ant-1520-S:91r; Brev-1531-S:105v.\footnote{In 1519:154r. the final note of `immortalitátis' is the first note of `locum.'}

\gregorioscore{}

\emph{℣.} Exaltáta es sancta Dei génitrix.

\emph{℟.} Super choros angelórum ad celéstia regna.

\emph{⟨℣.} Orémus.⟩

\begin{lesson}
\subsubsection{Oratio.}

\lettrine{V}{e}neránda nobis Dómine hujus diéi festívitas opem cónferat sempitérnam~: in qua sancta Dei génitrix mortem súbiit temporálem, nec tamen mortis néxibus déprimi pótuit que Fílium tuum Dóminum nostrum de se génuit incarnátum. Qui tecum ⟨vivit⟩.\footnote{V1882:154.}
\end{lesson}

\emph{Si dies dominica infra octavos assumptionis contigerit processio fiat ante crucem nisi aliqua dominica omnino differatur propter prolixitatem temporis. Quando dicitur de omnibus sanctis in revertendo ad vesperas semper dicitur ista antiphona.} Salvátor mundi. 315. \emph{℣.} Letámini. 316. \emph{Oratio.} Infirmitátem. 316.

\emph{Eodem die fiat processio ante missam.}

1519:155r; AS:500; Ant-1520-S:96r; Brev-1531-S:118r.\footnote{1519:155r. has no flats.}

\gregorioscore{}

\emph{In introitu chori de omnibus sanctis antiphona.} Te gloriósus. 319. \emph{quando vero antiphona} Salvátor mundi. \emph{dicitur ad vesperas tunc in crastino ante missam in introitu chori dicitur antiphona} Te gloriósus. \emph{quando de omnibus sanctis dicitur in revertendo et semper cum hoc ℣.} Letámini. 316. \emph{et cum oratione} Infirmitátem. 316.

\emph{¶ In octava assumptionis si dominica fuerit ad processionem omnia fiant sicut in dominica infra octavas ejusdem ⟨ut⟩\footnote{1882:155.} supra notatum est.} 367.

\chapter{¶ In decollatione sancti Johannis baptiste.}

\emph{Si in dominica fuerit ad processionem ℟.} Perceptúrus. \emph{quere in communi unius martyris.} 390.

\emph{In introitu chori de sancta Maria.} 309.

\chapter{¶ In die nativitatis beate Marie.}

\emph{Quacunque die contigerit ⟨ad processionem⟩.\footnote{1882:155.}}

1519:155v; AS:524; Ant-1520-S:109v; Brev-1531-S:132r.\footnote{1519:155v. has no flats. In 1519:155v. `maris' is set GFDFFG.DF.}

\gregorioscore{}

\emph{⟨Sequitur⟩\footnote{1882:155.} aliud ⟨resp.⟩\footnote{1882:155.}}

1519:155v; AS:523; Ant-1520-S:108v; Brev-1531-S:131v.\footnote{In 1519.155v. `génuit' is set GAEDF.AGC.AGC.}

\gregorioscore{}

\emph{In introitu chori ⟨cantetur hec sequens antiphona⟩.\footnote{1882:156. `\emph{In redeundo ℟}. Styrps Jesse. \emph{vel} Natívitas tua.' Rylands-24:370.}}

1519:156r; AS:519; Ant-1520-S:106r; Brev-1531:130r.\footnote{1519:156r. has no flats.}

\gregorioscore{}

\emph{℣.} Sancta Dei génitix virgo semper María. 300.

\begin{lesson}
\subsubsection{Oratio.}

\lettrine{S}{u}pplicatiónem servórum tuórum Deus miserátor exáudi~: ut qui in nativitáte Dei genitrícis Maríe congregámur, ejus intercessiónibus a te de instántibus perículis eruámur. Per eúndem Christum.
\end{lesson}

\emph{¶ Si dies dominica infra octavas ante festum exaltationis sancte crucis contigerit~: fiat processio ante crucem. In introitu chori de omnibus sanctis ut supra in alio festo.}

\emph{¶ Dominica infra octavas beate nativitatis Marie cum extra festum exaltationis sancte crucis evenerit. Ad processionem ante missam ℟.} Ad nutum. 366. \emph{In introitu chori de omnibus sanctis ut in octava beate Marie.}

\chapter{¶ In exaltatione sancte crucis.}

\emph{Ad primas vesperas eat processio ante crucem cantando antiphonam.} O crux gloriósa. \emph{ut supra in alio festo.} 341. \emph{℣.} Dícite in natiónibus. 220. \emph{Oratio.} Deus qui Unigéniti. \emph{ut supra in historie} Deus ómnium. \emph{ad primas vesperas.} 291.

\emph{In introitu chori dicatur ista antiphona} Beáta Dei génitrix. 291. \emph{℣.} Elégit eam. Brev:36. \emph{Oratio.} Supplicatiónem servórum. \emph{ut supra in nativitate beate Marie.} 372.

\emph{In die ante missam si dominica fuerit ad processionem.}

1519:156v; AS:427; Ant-1520-S:118r; Brev-1531-S:139r.\footnote{1519:156v. has no flats. In 1519:156v. `cleménter' is set D.Dc.F; `nobis' is set AFAGAFEDEFD.D.}

\gregorioscore{}

\emph{¶ In introitu chori de sancta Maria.} 309.

\emph{In octava beate Marie si dominica fuerit ad processionem omnia fiant sicut in dominica infra octavas.} 372.

\chapter{¶ In festo sancti Matthei apostoli et evangeliste.}

\emph{Si dominica fuerit ad processionem ⟨cantetur hoc responsorium sequens⟩.\footnote{1882:157.}}

1519:157r; AS:549; Ant-1520-S:125r; Brev-1531-S:143v.\footnote{1519:157r. has no flats. In 1519:157r. juxta is set AAGEFED.EFE.}

\gregorioscore{}

\emph{⟨℣.⟩} Glória Patri. 424. ‡Illuc.

\emph{In introitu chori de sancta Maria.} 309.

\chapter{¶ In festo sancti Michaelis archangeli.}

\emph{Ad primas vesperas fiat processio ad altare ⟨ejusdem⟩\footnote{1882:157.} cantando responsorium.}

1519:157v; AS:555; Ant-1520-S:129v; Brev-1531-S:147v.\footnote{1519:157v. has no flats. The editions maintain the cue `‡Et' even though there is no ℣. Glória Patri' or final repetition, as there is at matins.}

\gregorioscore{}

\emph{Non dicatur ⟨℣.⟩} Glória Patri.

\emph{℣.} Stetit ángelus ⟨juxta aram templi.

\emph{℟.} Habens thuríbulum áureum in manu sua⟩.\footnote{1531-S:146r.}

\begin{lesson}
\subsubsection{Oratio.}

\lettrine{B}{e}áti archángeli tui Michaélis\footnote{BIn some Processionals this prayer begins `Beátus Michaélis archángeli.'} interventióne suffúlti te Dómine súpplices deprecámur~: ut quos honóre proséquimur, contingámus et mente. Per Christum.
\end{lesson}

\emph{In redeundo de sancta Maria.} 309.

\emph{In die ante missam si dominica fuerit ad processionem ⟨dicitur hoc⟩.\footnote{1882:157.}}

1519:158r; AS:556; Ant-1520-S:130v; Brev-1531-S:146r.\footnote{1519:158r. has a flat only at `omnis.'}

\gregorioscore{}

\emph{In introitu chori de sancta Maria.} 309.

\chapter[¶ ⟨In die⟩ sancti Dionysii sociorumque ejus.]{¶ ⟨In die⟩\footnote{1882:157.} sancti Dionysii sociorumque ejus.}

\emph{Si dominica fuerit ad processionem responsorium.}

1519:158v; AS:564; Ant-1520-S:139r; Brev-1531:152v.\footnote{1519:158v. has no flats. In 1519:158v. `Beátus' is set C.D.DEFEDDC; `intulérunt' is set FEGA.FE.DCEFGED.}

\gregorioscore{}

\emph{⟨℣.⟩} Glória Patri. 420. ‡Beátas.

\emph{In introitu chori de sancta Maria.} 309.

\chapter{¶ In translatione sancti Edwardi regis et confessoris.}

\emph{Eat processio ad primas vesperas ad altare ejusdem cantando responsorium.}

1519:159r; AS:653; Ant-1519-C:29v; Brev-1531-P:76v.\footnote{In 1519:159r. `rogántes' is set CD.D.BE.}

7580. \gregorioscore{}

\emph{¶ Dum versus canitur thurificet sacerdos altare, deinde imaginem sancti Edvvardi regis~: deinde ⟨dicat⟩\footnote{1882:158.}}

\emph{℣.} Ora pro nobis beáte Edvvárde.

\emph{℟.} Ut digni efficiámur promissiónibus Christi.

\emph{⟨℣.} Orémus.⟩

\begin{lesson}
\subsubsection{Oratio.}

\lettrine{D}{e}us qui unigénitum tuum Dóminum nostrum Jesum Christum gloriosíssimo regi Edvvárdo in forma visíbili demonstrásti~: tríbue quésumus ut ejus méritis et précibus ad etérnam ipsíus Dómini nostri Jesu Christi visiónem pertíngere mereámur. Qui tecum vivit et regnat.
\end{lesson}

\emph{In redeundo ⟨de⟩ sancta Maria.} 309.

\chapter{¶ Sancti Michaelis in monte Tumba.}

\emph{Si dominica fuerit ad processionem responsorium.}

1519:159v; AS:567; Ant-1520-S:131r; Brev-1531-S:148r.\footnote{In 1519:159v. `Dómine' is set DFGA.G.G.}

\gregorioscore{}

\emph{⟨℣.⟩} Glória Patri. 426. ‡Contingámus.

\emph{In introitu chori de sancta Maria.} 309.

\chapter{¶ Sancti Luce evangeliste.}

\emph{Si dominica fuerit ad processionem ℟.} Cum ambulárent animália. \emph{ut supra in festo sancti Matthei ⟨apostoli⟩.\footnote{1882:158.}} 373.

\chapter{¶ In die omnium sanctorum.}

\emph{Qacunque die contigerit ad processionem ⟨cantetur istud responsorium sequens⟩.\footnote{1882:159.}}

1519:160v; AS:542, 577; Ant-1520-S:153r; Brev-1531:165v.\footnote{1519:160v. has no flats.}

\gregorioscore{}

\emph{In introitu chori antiphona} Salvátor mundi. \emph{require in festo Sancti Andree apostoli ad j. vesperas.} 315.

\emph{℣.} Letámini in Dómino ⟨et exultáte justi.

\emph{Resp.} Et gloriámini.⟩\footnote{1882:159.} 316.

\emph{⟨℣.} Orémus.⟩

\begin{lesson}
\subsubsection{Oratio.}

\lettrine{O}{m}nípotens sempitérne Deus qui nos ómnium sanctórum tuórum mérita sub una tribuísti sollennitáte venerári~: quésumus ut desiderátam nobis tue propitiatiónis abundántiam multiplicátis intercessiónibus largiáris. Per Christum Dóminum nostrum.
\end{lesson}

\chapter{¶ In festo sancti Martini.}

\emph{Ad primas vesperas eat processio ad altare ejusdem cantando ℟.}

1519:161r; AS:592; Ant-1520-S:161r; Brev-1531:173r.\footnote{1519:161r. has no flats.}

\gregorioscore{}

\emph{⟨℣.⟩} Glória. 424. ‡Celum.

\emph{Dum versus canitur thurificet sacerdos altare deinde ymaginem sancti Martini, postea dicat sacerdos}

\emph{℣.} Ora pro nobis beáte Martíne.

\emph{℟.} Ut digni efficiámur ⟨promissiónibus Christi.⟩\footnote{1882:159.}

\emph{⟨℣.} Orémus.⟩

\begin{lesson}
\subsubsection{Oratio.}

\lettrine{D}{e}us qui bonitátis auctor es et bonórum dispensátor concéde propítius~: ut beáti Martíni confessóris tui atque pontíficis suffrágio, majestátis tue propitiatiónem consequámur. Per Christum.
\end{lesson}

\emph{In introitu chori de sancta Maria.} 309.

\emph{In die ejusdem ante missam si dominica fuerit ad processionem responsorium.} Martínus Abrahe. \emph{ut supra.} 383.

\emph{In introitu chori de sancta Maria.} 309.

\chapter{¶ In festo sancti Bricii.}

\emph{Si dominica fuerit totum fiat de festo~: et ad primas vesperas processio fiat ante crucem.}

\chapter{¶ In festo sancti Edmundi episcopi et confessoris.}

\emph{Fiat processio ad j. vesperas ad altare ejusdem cantando ℟.} Sancte Edmúnde. \emph{cum ℣.} O sancte. \emph{sicut in translatione sancti Edvvardi regis et confessoris.} 378. \emph{Dum ℣. canitur thurificet sacerdos altare deinde ymaginem sancti Edmundi~: deinde dicat sacerdos versiculum.} Ora pro nobis sancte Edmúnde. \emph{℟.} Ut digni efficiámur ⟨promissiónibus Christi⟩.\footnote{1882:160.}

\emph{⟨℣.} Orémus.⟩

\begin{lesson}
\subsubsection{Oratio.}

\lettrine{P}{l}enam in nobis etérne Salvátor tue virtútis operáre médelam~: ut qui preclára beáti Edmúndi confessóris tui atque pontíficis mérita venerémur ipsíus adjúti suffrágiis, a cunctis animárum nostrórum languóribus liberémur. Per Christum.
\end{lesson}

\emph{In introitu chori de sancta Maria.} 309.

\chapter{¶ De sancto Edmundo rege et martyre.}

\emph{Fiat processio ad altare ejusdem ad j. vesperas in eundo et redeundo sicut in festo sancti Edmundi episcopi et confessoris cantando ℟.} Beátus vir qui suffert. \emph{in communi unius martyris.} 395.

\emph{℣.} Ora pro nobis beáte Edmúnde.

\emph{℟.} Ut digni efficiámur promissiónibus Christi.

\emph{⟨℣.⟩} Orémus.

\begin{lesson}
\subsubsection{Oratio.}

\lettrine{P}{r}esta quésumus omnípotens Deus~: ut qui beáti Edmúndi regis et mértyris tui natalítia cólimus~: intercessióne ejus in tui nóminis amóre roborémur. Per Christum.
\end{lesson}

\emph{In introitu chori de sancta Maria.} 309.

\chapter{¶ Sancte Cecilie virginis et martyris.}

\emph{Si dominica fuerit ad processionem.}

1519:162r; AS:pl. S; Ant-1520-S:169v; Brev-1531:185r.

\gregorioscore{}

\emph{⟨℣.⟩} Glória Patri. 421. ‡Abjícite.

\emph{In introitu chori de sancta Maria.} 309.

\chapter[¶ Sancti Clementis pape et martyris.]{¶ Sancti Clementis pape\footnote{In 1519:162v. `\emph{pape}' is erased.} et martyris.}

\emph{Si dominica fuerit ad processionem.}

1519:162v; AS:pl. S; Ant-1520-S:171r; Brev-1531-S:187r.\footnote{`abundantia' \emph{Chevallon.} ⟨SB:1094.⟩ `abundántia', 1519:162v. 1519:162v. has a flat only at `ad' and throughout the ℣. In 1519:162v. `confessóribus' is set F.AC.CD.CCB.CBA. In 1555, unpaged, the repeat is to `‡Ut de benefíciis.'}

\gregorioscore{}

\emph{⟨℣.⟩} Glória Patri. 421. † Ut ⟨confessóribus⟩.\footnote{1882:161.}

\emph{In introitu chori de sancta Maria.} 309.

\chapter[¶ ⟨In festo⟩ De sancte Katharine ⟨virginis⟩.]{¶ ⟨In festo⟩\footnote{1882:161.} De sancte Katharine ⟨virginis⟩.\footnote{1882:161.}}

\emph{Ad primas vesperas fiat processio ad altare ejusdem cum ceroferariis et thuribulario cantando.}

1519:163r; AS:pl. Y; Ant-1520-S:176r; Brev-1531-S:189r.

\gregorioscore{}

\emph{Non dicatur} Glória Patri.

\emph{Dum ℣. canitur thurificet sacerdos altare deinde imaginem beate Katharine~: et postea dicat ⟨sacerdos⟩\footnote{1882:161.} ℣.} Ora pro nobis beáta Katharína.

\emph{℟.} Ut digni ⟨efficiámur promissiónibus Christi⟩.\footnote{1882:161.}

\emph{⟨℣.} Orémus.⟩

\begin{lesson}
\subsubsection{Oratio.}

\lettrine{D}{e}us qui dedísti legem Moysi in summitáte montis Sýnay et in eódem loco per sanctos ángelos tuos corpus beáte Katharíne vírginis et mártyris tue mirabíliter collocásti~: tríbue quésumus, ut ejus mentis et intercessiónibus ad montem qui Christus est valeámus perveníre. Per Christum Dóminum nostrum.
\end{lesson}

\emph{In introitu chori de sancta Maria.} 309.

\emph{¶ In die ejusdem si dominica fuerit ad processionem ⟨dicitur hoc resp⟩.\footnote{1882:162.}}

1519:163v; ⟨AS:pl. W; Ant-1520-S:173v; Brev-1531-S:189r.\footnote{1519:163v. has no flats.}

\gregorioscore{}

\emph{In introitu chori de sancta Maria.} 309.

\chapter{¶ In natali unius apostoli ⟨sive⟩ evangeliste in paschali tempore.}

\emph{Si dominica fuerit ad processionem ⟨dicatur istud responsorium sequens⟩.\footnote{1882:162.}}

1519:163v; AS:plate H.; Ant-1519-C:1r; Brev-1531-P:66r.

\gregorioscore{} 6263.

\gregorioscore{}

\emph{⟨℣.⟩} Glória Patri. 424. ‡Allelúya.

\emph{In introitu chori de sancta Maria.} 291.

\chapter{In natali unius martyris sive confessoris paschali tempore.}

\emph{Si dominica fuerit ad processionem.}

1519:164r; AS:pl. L; Ant-1519-C:3v; Brev-1531-P:66v.

\gregorioscore{}

\emph{⟨℣.⟩} Glória Patri. 424. ‡Allelúya.

\emph{In introitu chori de sancta Maria.} 291.

\chapter{¶ In natali unius apostoli sive plurimorum apostolorum extra tempus paschale.}

\emph{Si dominica fuerit ad processionem.}

1519:164v; AS:plate Q; Ant-1519-C:9r; Brev-1531-P:67v.

\gregorioscore{} 6289.

\gregorioscore{} 6289a.

\gregorioscore{}

\emph{⟨℣.⟩} Glória Patri. 425. † Et liberáre.

\emph{In introitu chori de sancta Maria.} 309.

\chapter{¶ In natali unius martyris decollati quando communis historia dicitur.}

\emph{Si dominica fuerit ad processionem.}

1519:165r; AS:639; Ant-1519-C:15v; Brev-1531-P:70v.\footnote{1519:165r. has no flats. In 1519:165r. `gaudens' is set D.CEE; `vívere' is set FGAGFG.FE.E.}

\gregorioscore{}

\emph{In introitu chori de sancta Maria.} 309.

\chapter{¶ De uno martyre non decollato quando communis historia dicitur.}

\emph{Si dominica fuerit ad processionem.}

1519:165v; AS:638; Ant-1519-C:16r; Brev-1531-P:70v.\footnote{AS:639. does not indicate `℣. Glória. † Quam repromísit.'}

\gregorioscore{}

6232a. \gregorioscore{}

\emph{⟨℣.⟩} Glória Patri. 425. † Quam.

\emph{In introitu chori de sancta Maria.} 309.

\chapter{¶ In nativitate plurimorum martyrum quando communis historia dicitur.}

\emph{Si dominica fuerit ad processionem ⟨dicatur istud responsorium sequens⟩.\footnote{1882:163.}}

1519:166r; AS:648; Ant-1519-C:24r; Brev-1531-P:74v.

6891. \gregorioscore{}

\emph{⟨℣.⟩} Glória Patri. 424. ‡Ibi.

\emph{In introitu chori de sancta Maria.} 309.

\chapter[¶ In natali unius confessoris et pontificis ⟨sive doctoris⟩.]{¶ In natali unius confessoris et pontificis ⟨sive doctoris⟩.\footnote{1882:163.}}

\emph{Si dominica fuerit ad processionem.}

1519:166v; AS;655; Ant-1519-C:31r; Brev-1531-P:77r.\footnote{In 1519:166v. `\emph{N.}' is set GBG.A.A.}

\gregorioscore{}

\emph{In introitu chori de sancta Maria.} 309.

\chapter{¶ In nativitate unius confessoris et doctoris sive abbatis quando historia communis dicitur.}

\emph{Si dominica fuerit ad processionem ℟.} Miles Christi. \emph{ut supra.} 395.

\emph{In introitu chori de sancta Maria.} 305.

\chapter{¶ In natali plurimorum confessorum.}

\emph{Si dominica fuerit ad processionem ⟨dicitur⟩ responsorium.\footnote{1882:164.}}

1519:166v; AS:576, 661; Ant-1519-C:38v; Brev-1531-P:80r.⟩\footnote{In 1519:166v. `precíncti' ends CDE.D; `homínibus' is set CD.D.DEFD.C. In 1519:167r. a repeat is indicated by the capital at `† Et.' as at matins of many confessors in the Breviary, but the cue for the repetition is `‡Quando.'}

\gregorioscore{} 7675a.

\gregorioscore{}

\emph{⟨℣.} Glória Patri. 424. ‡Quando.⟩\footnote{1555, unpaged.

  \gregorioscore{}}

\emph{In introitu chori de sancta Maria.} 309.

\chapter{¶ In nativitate unius virginis et martyris quando communis historia dicitur.}

\emph{Si dominica fuerit ad processionem.}

1519:167r; AS:666; Ant-1519-C:44r; Brev-1531-P:81v.

\gregorioscore{}

\emph{In tempore paschali in processione contra reginam suscpiendam finiatur sic.}

7524a. \gregorioscore{}

\emph{In tempore paschali post} Glória Patri. \emph{repetatur a choro} ‡Allelúya.

\emph{In introitu chori ⟨dicatur aliqua antiphona⟩\footnote{1882:164.} de sancta Maria.} 309.

\chapter[⟨¶ De processionibus causa necessitatis factis.⟩]{⟨¶ De processionibus causa necessitatis factis.⟩\footnote{1882:164.}}

\lettrine{F}{}\emph{Iunt autem quedam processiones causu necessitatis vel tribulationis scilicet pro serenitate aeri vel ad pluviam postulandam, vel contra mortalitatem hominum vel in tempore belli, vel pro pace ecclesie vel pro aliqua tribulatione que eodem modo et ordine ordinantur per omnia quo in processionibus ferialibus xl.}

\emph{Procedant autem per medium chori ad unum altare ejusdem ecclesie vel ad aliquam ecclesiam in urbe vel in suburbio~: eodem modo et ordine per omnia quo in ij. feria in rogationibus sed absque drachone et leone et vexillis que non deferantur, choro sequente habitu non mutato cantando antiphonam vel ℟. ut patet inferius~: et cum vij. psalmis penitentialibus si tantum restat iter, cum letania et collectis.}

\chapter{¶ Pro serenitate aeris.}

\emph{Ant.} Inundavérunt. \emph{Ps.} Salvum me fac. \emph{require post primum v. psalmi statim dicatur} Glória Patri. \emph{et} Sicut erat. \emph{Deinde repetatur antiphona.} 234.

\emph{Alia antiphona.} Non nos demérgat. \emph{Ps.} Exáudi\footnote{`Exáudi nos', 1555, unpaged.} Dómine. Glória Patri. Sicut erat. \emph{Deinde repetatur antiphona ⟨ut supra feria secunda rogationum⟩}.\footnote{1882:165.} 235.

\chapter{¶ Ad pluviam postulandam.}

\emph{Ant.} Dómine Rex ⟨Deus⟩ Abraham. \emph{Ps.} Exúrgat Deus. Glória Patri. Sicut erat. \emph{⟨Deinde⟩\footnote{1882:165.} repetatur antiphona. Similiter fiat post omnes alias ⟨antiphonas⟩}.\footnote{1882:165.} 231.

\emph{Alia ant.} Numquid est in ydólis. \emph{Ps.} Exúrgat Deus. 231.

\emph{Ant.} Exáudi Dómine. \emph{Ps.} Plúviam voluntáriam.\footnote{In 1882:165. the order is confused: `\emph{Ant.} Plúviam voluatáriam. \emph{Ps.} Exáudi Dómine.'} 232.

\emph{⟨Ant.} Réspice, Dómine. \emph{Non sequitur psalmus. Require has antiphonas feria secunda rogationum.⟩\footnote{1882:165.}} 234.

\chapter{¶ Contra mortalitatem hominum tempore belli.}

\emph{Ant.} Líbera Dómine. \emph{Ps.} Dómine Dóminus noster. Glória Patri. Sicut. \emph{Repetatur antiphona ⟨ut supra⟩\footnote{1882:165.} require has antiphonas feria ij. rogationum.} 236.

\chapter{¶ Pro pace petenda.}

\emph{Dicitur unum vel plura de ℟. subscriptis per ordinem in eundo quantum sufficit iter.}

\emph{℟.} Afflícti pro peccátis. \emph{require in processionibus ferialibus xl.} 76.

1519:168r; AS:321; Ant-1520:40r; Brev-1531:194r.\footnote{In 1520:40r. `convérte' is set CD.D.D.}

\gregorioscore{}

\emph{In tempore paschali.}

\gregorioscore{}

\emph{⟨Sequitur aliud responsorium.⟩\footnote{1882:165.}}

1519:168v; AS:321; Ant-1520:40r; Brev-1531:194v.

\gregorioscore{}

\emph{⟨Postea sequitur aliud responsorium.⟩\footnote{1882:166.}}

1519:169r; AS:324; Ant-1520:43v; Brev-1531:196v.\footnote{1519:169r. has no flats.}

6687. \gregorioscore{}

6687a. \gregorioscore{}

1519:169r; AS:325; Ant-1520:44r; Brev-1531:196v.\footnote{`pátentia', 1519:169r.}

\gregorioscore{} 7793.

\gregorioscore{}

\emph{In tempore paschali.}

\gregorioscore{} 7793a.

\gregorioscore{}

1519:169v; AS:320; Ant-1520:39r; Brev-1531:192v.\footnote{1519:169v. has no flat.}

\gregorioscore{}

\emph{In tempore paschali.}

\gregorioscore{}

\emph{¶ Si transierit processio ad aliquam ecclesiam in urbe vel in suburbio ubi fieri debet statio dicitur antiphona vel ℟. de sancto de quo est ecclesia illa~: ita quod ad januas cymiterii vel citius inchoetur. Finita antiphona vel ℟. statim dicat sacerdos ad gradum chori ℣. et orationem de sancto ejusdem ecclesie sine} Dóminus vobíscum. \emph{sed tantum cum} Orémus. \emph{qua dicta clerici eodem modo et ordine quo in processione ordinantur se prosternant ita quod sacerdos ad gradum altaris cum diacono a dextris et subdiacono a sinistris suam faciant prostrationem cum dicitur} Kyrieléyson. Christeléyson. Kyrieléyson. \emph{sine nota sequatur} Pater noster. \emph{cetera omnia sicut predictum est in processione feriali xl.} 80. \emph{His dictis incipiatur ⟨missa⟩\footnote{1882:166.} ex sua causa processione precedente a cantore.}

\emph{¶ Missa pro fratribus et sororibus. Officium.} Salus pópuli. \emph{Epistola} Hec dicit. \emph{⟨et cetera⟩.\footnote{1882:166.}} Missale:⟨194⟩.

\emph{¶ Post missam duo clerici de secunda forma habitu non mutato dicant hanc letaniam que terminetur ab eisdem ad gradum chori in ecclesia propria hoc modo sequente.}

1519:170r; 1519-P:172r; Brev-1531-P:48v.\footnote{The processionals do not indicate the repetition of the petitions. In 1519:170r. `de celis' is set A C.A. 1519:170r. omits `Spíritus sancte Deus~: Miserére nobis.'; In 1519:170r. `unus' is set C.C.}

\gregorioscore{}

\emph{Cetera ut supra} 83. \emph{usque ad}

\gregorioscore{}

\emph{Deinde statim sequatur.}

\gregorioscore{}

\emph{Chorus idem repetat.}

\emph{Clerici prosequantur sic versum.\footnote{In 1519:170v. `Ab' is set D.}}

\gregorioscore{}

\emph{Chorus idem repetat et sic de singulis ℣. ut in secunda feria rogationum.}

\gregorioscore{}

\emph{Chorus idem.}

\emph{Clerici.}

\gregorioscore{}

\emph{Chorus respondent sic.\footnote{1519:170v. omits `audi nos.'}}

\gregorioscore{}

\emph{Clerici.}

\gregorioscore{}

\emph{Chorus respondeat.}

\gregorioscore{}

\emph{Et sic de singulis ℣. ut supra in ij. feria rogationum.}

\emph{Clerici versum.}

\gregorioscore{}

\emph{Chorus idem versus.}

\emph{Clerici.\footnote{1519:171r. omits the flats.}}

\gregorioscore{}

\emph{Chorus idem versus.}

\emph{Clerici.}

\gregorioscore{}

\emph{Chorus idem versus.}

\emph{Clerici.}

\gregorioscore{}

\emph{Chorus idem.}

\emph{Clerici.}

\gregorioscore{}

\emph{Chorus idem.}

\emph{Clerici.}

\gregorioscore{}

\emph{Chorus idem.}

\emph{Clerici.\footnote{1519:171r. omits the flat.}}

\gregorioscore{}

\emph{Deinde dicat sacerdos ℣.} Letámini in Dómino. 312.

\begin{lesson}
\subsubsection{Oratio.}

\lettrine{I}{n}firmitátem nostram quésumus propícius réspice~: et mala ómnia que juste merémur~: ómnium sanctórum tuórum intercessióne avérte. Per Dóminum.
\end{lesson}

\emph{⟨require in festo sancti Andree.⟩\footnote{1882:167.}} 316.

\chapter{¶ Cum corpus defuncti portatur ad ecclesiam.}

\emph{Dicitur hec antiphona.}

1519:171v.

\gregorioscore{}

\emph{Repetatur antiphona. Deinde dicatur iste psalmus} De profúndis ⟨clamávi⟩.\footnote{1882:168.} 62. \emph{et post unumquemque versum repetatur antiphona. Deinde si necesse fuerit dicatur iste psalmus} ⟨In éxitu.⟩\footnote{1882:168.} Brev:⟨318⟩. \emph{in ordine.}

\emph{In introitu vel citius ⟨sequens⟩\footnote{1882:168.} ℟. inchoetur.}

1519:171v; AS:583; Ant-1519-P:188v; Brev-1531-P:53r.

\gregorioscore{}

7091g. \gregorioscore{}

\emph{Et dicitur cum uno versu tantum.}

\emph{In introitu chori dicatur sequens antiphona, et aspergatur corpus aqua benedicta et nunquam debet corpus alicujus defuncti portari circa cimiterium sed directe in ecclesiam et sequatur ⟨alia antiphona hoc modo quo sequitur⟩.\footnote{1882:168.}}

1519:172r.

\gregorioscore{}

\emph{Repetatur antiphona et statim dicatur quod sequitur.}

1519:172r.

\gregorioscore{}

\emph{Tunc sacerdos aspergat corpus cum aqua benedicta et thurificet rogans} Oráte pro ánima \emph{N.} et pro animábus fidélium defunctórum.

\emph{⟨℣.⟩} Pater noster. Brev:⟨5⟩.

\emph{⟨℣.⟩} Et ne nos. \emph{⟨℟.⟩} Sed líbera.

\emph{⟨℣.⟩} A porta ínferi. \emph{⟨℟.⟩} Erue Dómine ánimas eórum.

\emph{⟨℣.⟩} Non intres in judícium servo tuo. \emph{⟨℟.⟩} Quia non justificábitur ⟨in conspéctu tuo omnis vivens.⟩\footnote{1882:168.}

\emph{⟨℣.⟩} Dóminus vobíscum. \emph{⟨℟.⟩} Et cum spíritu tuo.

\emph{⟨℣.⟩} Orémus.

\begin{lesson}
\subsubsection{Oratio.}

\lettrine{S}{u}scipe Dómine servum tuum \emph{vel} ancíllam tuam \emph{N.} in bonum habitáculum~: et da ei réquiem in regno celésti Hierúsalem, ut in sinu Abrahe patriárche tui collocátus resurrectiónis diem prestolétur, et inter resurgéntes ad glóriam resúrgat~: et cum benedíctis ad déxteram Dei veniéntibus véniat~: et cum possidéntibus vitam etérnam possídeat. Per Christum.
\end{lesson}

\emph{Tunc sic.} Anima ejus et ánime ómnium fidélium defunctórum per misericórdiam Dei in pace requiéscant. Amen. Pater noster.

\chapter[⟨¶ Processio causa venerationis.⟩]{⟨¶ Processio causa venerationis.⟩\footnote{1882:169.}}

\lettrine{F}{}\emph{Iunt autem quedam processiones causa venerationis ad suscipiendum archiepiscopum, proprium episcopum, legatum vel cardinalem, regem vel reginam~: que eodem modo ordinantur, et habitu quo in die nativitatis Domini~: procedant autem per medium chori et ecclesie per ostium occidentale usque ad locum destinatum, ibique ad personam suscipiendum disponantur in processione. Duo excellentiores persone in cappis sericis~: et ipsam personam reverenter in revertendo suscipiant. Post thurificationem et aque benedicte aspersionem cantore incipiat ⟨responsorium⟩\footnote{1882:169.} scilicet contra archiepiscopum~: proprium episcopum, legatum vel cardinalem~: dicatur hoc ℟.} Summe Trinitáti. \emph{ut in festo Sancte Trinitatis.} 277. \emph{Contra regem dicatur ℟.} Honor virtus. \emph{⟨ut supra in festo Sancte Trinitatis⟩}.\footnote{1882:169.} 279. \emph{Contra reginam dicatur hoc ℟.} Regnum mundi. \emph{Require in communi unius virginis} 395. \emph{: eadem quoque via qua accesserunt usque ad gradum altaris accedant.}

\emph{Finito responsorio cum suo ℣. a toto choro sequatur} Kyrieléyson. Christeléyson. Kyrieléyson. \emph{sine nota.} Pater noster. Brev:⟨5⟩. \emph{Deinde super archiepiscopum~: episcopum proprium~: legatum vel cardinalem cum prosternant se in oratione ad gradum altaris, dicat sacerdos in cappa serica ℣. et orationes}

\emph{⟨℣.⟩} Et ne nos ⟨indúcas in tentatiónem⟩. \emph{⟨℟.⟩} Sed líbera ⟨nos a malo⟩.

\emph{⟨℣.⟩} Salvum fac servum tuum Dómine. \emph{⟨℟.⟩} Deus meus sperántem in te.

\emph{⟨℣.⟩} Mitte ei Dómine auxílium de sancto. \emph{⟨℟.⟩} Et de Syon túere eum.

\emph{⟨℣.⟩} Nichil profíciat inimícus in eo. \emph{⟨℟.⟩} Et fílius iniquitátis non appónat nocére ei.

\emph{⟨℣.⟩} Esto ei Dómine turris fortitúdinis. \emph{⟨℟.⟩} A fácie inimíci.

\emph{⟨℣.⟩} Dómine exáudi ⟨oratiónem meam⟩. \emph{⟨℟.⟩} Et clamor ⟨meus at de véniat⟩.

\emph{⟨℣.⟩} Dóminus vobíscum. \emph{⟨℟.} Et cum spíritu tuo.⟩

\emph{⟨℣.⟩} Orémus.

\begin{lesson}
\subsubsection{Oratio.}

\lettrine{C}{o}ncéde quésumus Dómine, fámulo tuo \emph{N.} metropolitáno nostro \emph{vel} epíscopo \emph{vel} preláto ut predicándo et exercéndo que recta sunt exémplo bonórum óperum ánimas suórum ínstruat subditórum~: et etérne remuneratiónis mercédem a te piíssimo pastóre percípiat. Per Christum.
\end{lesson}

\emph{¶ Super regem vel reginam in prostratione ad gradum altaris dicat sacerdos in cappa serica ℣.} Et ne ⟨nos indúcas in tentatiónem⟩ \emph{⟨℟.⟩} Sed líbera ⟨nos a malo⟩.

\emph{⟨℣.⟩} Osténde nobis ⟨Dómine misericórdiam tuam⟩.\footnote{1882:170.} \emph{⟨℟.⟩} Et salutáre tuum ⟨da nobis⟩.\footnote{1882:170.}

\emph{⟨℣.⟩} ⟨Dómine⟩\footnote{1882:170.} salvum fac regem \emph{vel} ancíllam tuam. \emph{⟨℟.⟩} Et exáudi nos in die qua invocavérimus te.

\emph{⟨℣.⟩} Mitte ei Dómine auxílium de sancto. \emph{⟨℟.⟩} Et de Syon tuére eum.

\emph{⟨℣.⟩} Nichil profíciat inimícus in eo. \emph{⟨℟.⟩} Et fílius iniquitátis non appónat nocére ei.

\emph{⟨℣.⟩} Dómine Deus virtútum convérte nos. \emph{⟨℟.⟩} Et osténde fáciem tuam et salvi érimus.

\emph{⟨℣.⟩} Dómine exáudi ⟨oratiónem meam. \emph{℟.} Et clamor meus ad te véniat.⟩

\emph{⟨℣.⟩} Dóminus vobíscum. \emph{⟨℟.} Et cum spíritu tuo.⟩

\emph{⟨℣.⟩} Orémus.

\begin{lesson}
\subsubsection{Oratio.}

\lettrine{D}{e}us in cujus manu corda sunt regum qui es humílium consolátor et fidélium fortitúdo~: et protéctor ómnium in te sperántium, da regi nostro \emph{vel} regíne nostre populóque Christiáno triúmphum virtútis tue sciénter excólere ut per te semper reparéntur ad véniam. Per Christum.
\end{lesson}

\chapter[⟨¶ Antiphone de beata Maria.⟩]{⟨¶ Antiphone de beata Maria.⟩\footnote{1882:170.}}

1519:173r.

\gregorioscore{}

1519:174v.

\gregorioscore{}

1519:175r; AS: 55; Ant-1519:63r, 97v; Brev-1531:31v.⟩

\gregorioscore{}

1519:175r.

\gregorioscore{}

\emph{℣.} Ave María grátia plena Dóminus tecum.

\emph{℟.} Benedícta ⟨tu in muliéribus.⟩\footnote{1882:171.}

\emph{⟨℣.⟩} ⟨Orémus.⟩\footnote{1882:171.}

\begin{lesson}
\subsubsection{Oratio.}

\lettrine{O}{m}nípotens sempitérne Deus qui glorióse vírginis et matris Maríe corpus et ánimam~: ut dignum Fílii tui habitáculum éffici mererétur Spíritu Sancto cooperánte mirabíliter preparásti~: da ut cujus commemoratióne letámur ejus pia intercessióne ab instántibus malis et subitánea morte et improvísa liberémur. Per eúndem ⟨Christum Dóminum nostrum.⟩\footnote{1882:171.}
\end{lesson}

\emph{⟨Postea sequitur responsorium.⟩\footnote{1882:171. This is in fact an Antiphon with Verse.}}

1519:175r.

\gregorioscore{}

1519:175v.

\gregorioscore{}

1519:175v; AS:518; Ant-1520-S:105r; Brev-1531-S:7v.

\gregorioscore{}\footnote{1508. and 1555. include here five proses, which are not properly part of the Processional but of the Antiphonale (\emph{O morum doctor egregie}, \emph{Sospitati dedit}, \emph{Inviolata integra et casta}, \emph{Crux fidelis}, and \emph{Eterne virgo memorie},) followed by the seven penitential psalms.}

\begin{verse}
⟨Quis satis enumeret, quantam pressoria nobis\\*
Conferat ars cunctis utilitatis opem\\*
Tempore sub modico facili commissa papiro\\*
Et precio parvo grandia lector habes\\*
Innumeris mendis scriptorum ignavia quondam\\*
Tedia quam multis grandia sepe dabat\\*
Sed cum pressores patuerunt arte periti\\*
Palladis occulte tunc rediere faces\\*
Ergo tibi hoc munus Sarensis mater habeto\\*
Quod tibi divinis proderit officiis.\\*
Sane hoc pressorum digessit in arte magister\\*
Martinus Morin incola Rothomagi.⟩\footnote{1508:unpaged; 1555:unpaged.}
\end{verse}

\begin{centering}
⟨¶ Processionale cum bonis notulis et bonis li-\\
gaturis~: atque cum stationibus picturatis infra ap-\\
positis Secundum usum insignis ad preclare\\
ecclesie Sarum~: noviter correctum, ac rursus\\
emendatum, per Christophorum Endoviensis.\\
Antwerpie excusum. Impensis\\
honesti viri Francisci Byrk-\\
man. Anno ab incarna-\\
tione Domini. 1523.\\
Die vero 16.\\
Martii.\\
Laus soli Deo.

✠

¶ Venales reperiuntur Londoniis in cimiterio\\
sancti Pauli, a Francisco Byrckman,\\
vel as suis servitoribus.⟩\footnote{1523:76r, 76v.}
\end{centering}

\chapter{Versi responsoria.}

Primus modus.

ST:x.

\gregorioscore{}

Secundus modus.

ST:xvij.

\gregorioscore{}

Tertius modus.

ST:xxj.

\gregorioscore{}

Quartus modus.

ST:xxxj.

\gregorioscore{}

Quintus modus.

ST:xxxv.

\gregorioscore{}

Sextus modus.

ST:xliij.

\gregorioscore{}

%TODO: add flat?

\gregorioscore{}

Septimus modus.

ST:liij.

\gregorioscore{}

Octavus modus.

ST:lxv.

\gregorioscore{}

\chapter{Ad Officium.}

\gregorioscore{}

\backmatter

\printendnotes

% \end{document}
